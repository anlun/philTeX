\documentclass[12pt]{article}
\usepackage[left=2cm,right=2cm, top=2cm,bottom=2cm,bindingoffset=0cm]{geometry}
\usepackage{fontspec}
\usepackage{polyglossia}
\setdefaultlanguage{russian}
\setmainfont[Mapping=tex-text]{CMU Serif}
\usepackage[colorlinks=true,urlcolor=black,linkcolor=black,filecolor=black,citecolor=black]{hyperref}

\begin{document}

\tableofcontents


\newpage
\section{Предмет философии в ее истории}
Слово  «философия» --- греческого  происхождения  и  буквально  означает  «любовь  к  мудрости».
Философия
представляет собой систему взглядов на окружающую нас действительность, систему наиболее общих понятий
о мире и месте в нем человека. С момента своего возникновения она стремилась выяснить, что представляет
собой мир как единое целое, понять природу самого человека, определить,
какое место занимает он в обществе,
может ли его разум проникнуть в тайны мироздания, познать и обратить на благо людей могущественные силы
природы. Философия таким образом, ставит самые общие и вместе с тем очень важные, коренные вопросы,
определяющие  подход  человека  к  самым  разнообразным  областям  жизни  и  знания. 
На  все  эти  вопросы
философы давали самые различные, и даже взаимоисключающие ответы.

Предмет философии. Философия --- наука, которая изучает наиболее общие законы развития природы, общества
и познания. Философия рассматривает наиболее важные связи в системе «мир-человек»
Назначение философии --- поиск удела человека, обеспечение его бытия в причудливом мире,
а в конечном счете
в  возвышении  человека,  в  обеспечении  его  совершенствования.  Общую  структуру  философского  знания
составляют четыре основных раздела:
\begin{itemize}
  \item онтология (учение о бытие),
  \item гносеология (учение о познании),
  \item человек,
  \item общество.
\end{itemize}
Специфика философской мудрости состоит в ее нацеленности на смыслы максимально всеохватной (общей,
предельной, целостной) и вместе с тем фундаментальной значимости.


\newpage
\section{Специфика философского знания}
Основная специфика философского знания заключается в его двойственности, так как оно:
\begin{itemize}
  \item имеет очень много общего с научным знанием: предмет, методы, логико-понятийный аппарат;
  \item однако не является научным знанием в чистом виде.
\end{itemize}
Главное отличие философии от всех иных наук заключается в том, что философия является теоретическим
мировоззрением, предельным обобщением ранее накопленных человечеством знаний.
Предмет философии шире предмета исследования любой отдельной науки, философия обобщает, интегрирует
иные науки, но не поглощает их, не включает в себя все научное знание, не стоит над ним.
Можно выделить следующие особенности философского знания:
\begin{itemize}
\item имеет сложную структуру (включает онтологию, гносеологию, логику и т. д.);
\item носит предельно общий, теоретический характер;
\item содержит базовые, основополагающие идеи и понятия, которые лежат в основе иных наук;
\item во многом субъективно — несет в себе отпечаток личности и мировоззрения отдельных философов;
\item  является  совокупностью  объективного  знания  и  ценностей,  нравственных  идеалов  своего  времени,
испытывает на себе влияние эпохи;
\item изучает не только предмет познания, но и механизм самого познания;
\item  имеет  качество  рефлексии  —  обращенности  мысли  на  саму  себя  (то  есть  знание  обращено  как  на  мир
предметов, так и само на себя);
\item испытывает на себе сильное влияние доктрин, вырабатываемых прежними философами;
\item в то же время динамично — постоянно развивается и обновляется;
\item опирается на категории — предельно общие понятия;
\item неисчерпаемо по своей сути;
\item  ограничено  познавательными  способностями  человека  (познающего  субъекта),  имеет  неразрешимые,
"извечные"  проблемы  (происхождение  бытия,  первичность  материи  или  сознания,  происхождение  жизни,
бессмертие души, наличие либо отсутствие Бога, его влияние на мир), которые на сегодняшний день не могут
быть достоверно разрешены логическим путем.
\end{itemize}

\newpage
\section{Структура философского знания и его функции}
Философия, как особый вид знания сформировался в 7-6 в. до н.э. в Древней Греции, Индии и Китае. Термин
``философия" встречается впервые у Пифагора и переводится как ``любовь к мудрости”.
Философия --- "учение об общих
принципах бытия и познания, об отношении человека к миру; наука о всеобщих законах развития природы,
общества и мышления." Философия --- венец культуры, высшая мудрость (Панфилософизм, Гегель).
Кьеркегор:
Обычному человеку с его тревогами и бедами нечего ждать от философии. ``Я предъявляю к философии вполне
законные  требования --- что  делать  человеку?
Как  жить?.. Молчание  философии  является  в  данном  случае
уничтожающим  доводом  против  неё  самой".
От  религии  философия  отличается  тем,  что  рационально
обосновывает  свои  принципы. Розанов: ``Философия  объясняет,  религия  решает  проблемы  (проблему
преодоления  смерти)." Наука --- частное  знание,
философия  общее.  В  науке  существует  целый  корпус  не
подвергающихся  сомнению  положений,  имеется  явное  поступательное  развитие.  Результаты  научной
деятельности  безличны.  Философия  не  удовлетворяет  принципам  верификации  и  фальсификации.
Рассел: ``Философия --- размышление о предметах, знание о которых еще не возможно. <...>
Если человек ищет ответ на вопрос
как  жить,  и  хочет  обоснованного  ответа  то  он  обращается  к  философии.  Но  ответа  не  находит."


Мировоззренческая философия. Методологическая функция. 

Марксистско-ленинская версия:  философия --- есть  общая  методология  познания  и  практического  преобразования  объективного  мира.
``Философия содействует приросту научных знаний и создаёт предпосылки для научных открытий".
Соловьёв: “Что делает философия для человечества, какие блага ему даёт от каких зол избавляет?” 
Философия “освобождала человеческую личность от внешнего насилия, и давала ей внутреннее содержание”.
Философия служит, соответственно, и материальному и божественному началу. Но “Эта двойственная сила и
этот двойственный  процесс,  составляя  сущность философии,  вместе с тем и составляют сущность самого
человека”. Поэтому окончательный ответ Соловьева на вопрос, что же делает философия: “Она делает человека
человеком”.
Внутри философии сформировались такие философские дисциплины:
\begin{itemize}
  \item Онтология --- учение о бытии или первоначалах всего сущего;
  \item Гносеология (эпистемология) --- теория  познания,  занимающаяся  исследованием  природы  познания,  его
структуры, выясняющая условия его достоверности и истинности;
\item Логика --- наука о формах правильного, т. е. последовательного, связного, доказательного мышления;
\item Этика --- учение о морали или нравственности;
\item Эстетика --- учение о прекрасном, о природе искусства.
\end{itemize}
  Социальная  философия  исследует  общество  как  систему  надиндивидуальных  форм,  связей  и  отношений,
которые человек создаёт своей деятельностью. С ней непосредственно связана философия истории, которая
исследует  смысл,  закономерности  и  основные  направления  исторического  процесса;  философская
антропология,  которая  выясняет  сущность  человека  как  личности;  политическая  философия  (и  философия
права), которая выясняет природу власти и государства. Философия науки изучает строение научного знания,
механизмы  и  формы  его  развития.  Философия  религии  осмысляет  природу  и  функции  религии.  Всегда
предстаёт  либо  как  философское  религиоведение  либо  как  философская  теология.  Также  выделяются
философия культуры, философия техники, философия экономики, философия творчества, философия любви. 
Принципиально важной областью философии является история философии.

\newpage
\section{Плюрализм философского знания и его предпосылки}
Плюрализм --- философская мировоззренческая позиция, согласно которой существует множество независимых
и несводимых друг к другу начал или видов бытия. 
Плюрализм (от лат. pluralis, множественный) --- философская позиция, согласно которой существует множество
различных равноправных,  независимых и несводимых друг к другу форм знания и методологий познания
(эпистемологический  плюрализм), либо  форм  бытия  (онтологический  плюрализм).  Плюрализм  занимает
оппонирующую позицию по отношению к монизму. 
Термин  «плюрализм»  был  введён  в  начале  XVIII  в.  Христианом  Вольфом,  последователем  Лейбница  для
описания  учений,  противостоящих  теории  монад  Лейбница,  в  первую  очередь  различных  разновидностей
дуализма.
В конце XIX-XX веке плюрализм получил распространение и развитие как в андроцентрических философских
конценциях,  абсолютизирующих  уникальность  личного  опыта  (персонализм,  экзистенциализм),  так  и  в
эпистемологии (прагматизм Уильяма Джеймса, философия науки Карла Поппера и, особенно, теоретический
плюрализм его последователя Пауля Фейерабенда).
Эпистемологический плюрализм как методологическкий подход в науке, подчёркивая субъективность знания и
примат  воли  в  процессе  познания  (Джеймс),  историческую  (Поппер)  и  социальную  (Фейерабенд)
обусловленность знания, критикует классическую научную методологию и является одной из посылок ряда
антисциентистских течений.
Причины:
\begin{itemize}
  \item многообразие действительности, которую постигает философия (Гегель);
  \item историческая обусловленность философии;
  \item принципиальная неполнота и фрагментарность знаний о действительности;
  \item люди все разные.
\end{itemize}

\newpage
\section{Понятие мировоззрения. Типы мировоззрения и их характерные особенности}
Мировоззрение, миропонимание --- взгляд на мир и положение человека в этом мире, оценка и характеристика
взаимоотношений человека и мира. Мировоззрение формируется веками и продолжает формироваться, поэтому
в ходе развития мировоззрения нужно выделить различные этапы,
т.е. характеризовать мировоззрение как историческое. Исторические типы
мировоззрения: мифологический, религиозный, научный, философский. 
Мировоззрение   исторически  конкретно,  оно  вырастает  на  почве  культуры и вместе с ней претерпевает
изменения. Мировоззрение каждой эпохи реализуется во множестве групповых и индивидуальных  вариантах.
Мировоззрение  как
система включает в себя знания (имеющих своей опорой истину), а наряду с этим и ценности.
Мировоззрение выработано
не только разумом, но и чувствами. Это значит, что мировоззрение состоит как бы из двух частей:
интеллектуальной и эмоциональный.  Эмоциональная сторона  мировоззрения  представлена  мироощущением  и  мировосприятием.  Интеллектуальная ---  миропониманием.
Соотношение интеллектуальной и эмоциональной стороны мировоззрения зависит от эпохи,
от самого индивидуума. Так же бывает разная окраска
понимания мира, что выражается в чувствах. Второй уровень мировоззрения --- миропонимание, опирающееся прежде всего
на знания, хотя миропонимание и мироощущение не даны просто так рядом друг с другом: они,
как правило едины. Мировоззрение включает в
свою структуру уверенность и веру. МЗ делится на жизненно-повседневное и теоретическое. 
Жизненно-повседневное складывается повседневно. Важно, нужно учитывать, что она страдает
\begin{enumerate}
\item недостаточной широтой;
\item своеобразным переплетением 
положений и установок с примитивными, мистическими предрассудками;
\item большой эмоциональностью.
\end{enumerate}
Эти
минусы преодолеваются на теоретическом уровне мировоззрения. Это философский уровень мировоззрения,
когда человек подходит к миру с позиции разума, действует, опираясь на логику, обосновывая свои выводы и
утверждения.
Философии как особому типу мировоззрения предшествовали мифологический и религиозный типы мировоззрения.
Миф, как особая форма сознания и мировоззрения, представляет собой своеобразный сплав знаний, хотя и
весьма ограниченных, религиозных верований и различных видов искусств.
Дальнейшее развитие миропонимания пошло по двум линиям: по линии религии и по линии философии.
Религия ---  форма мировоззрения, в которой освоение мира осуществляется через его удвоение на земной,
(естественный) и потусторонний (сверхъестественный). При этом в отличии от науки, тоже создающей свой
второй  мир  в  виде  научной  картины  природы,  второй  мир  религии  основан  не  на  знании,
а  на  вере  в
сверхъестественные силы и их главенствующую роль в мире, в жизни людей. Религиозная вера --- это особое
состояние сознания, отличное от уверенности ученого, которая базируется на рациональных основах.
Общее, что роднит философию и религию, это решение мировоззренческих проблем, но пути и подходы решения
этих проблем у них сильно отличаются.

\newpage
\section{Мифология как первичная форма мировоззрения}
Мифологическое мировоззрение было древнейшей формой познания мира, космоса, общества и человека. Миф
по  необходимости  возник  из  потребности  индивида,  его  семьи,  рода  и  социума  в  целом,  в  осознании
окружающей природной и социальной стихии, сущности человека и передачи их единства через различные
символические системы. В мифологических системах человек и социум, как правило, не выделяют себя из
окружающего  мира.  Космос,  природа,  общество  и  человек ---  различные  проявления одного и того же
божественного  закона,  передаваемого  через  символическую  или  символико-мифологическую  системы.
Природа,  общество  и  человек  слиты  в  единое  целое,  неразрывное  и  единое,  однако  сами  они  внутри
неоднородны и уже авторитарны, авторитаризм общества перенесён на всю природу.
Мифологическое  сознание  мыслит  символами:  каждый  образ,  бог,  культурный  герой,  действующее  лицо
обозначает стоящее за ним явление или понятие. Это возможно потому, что в мифологическом мировоззрении
существует  постоянная  и  неразрывная  связь  между  «однотипными»  явлениями  и  объектами  в  социуме,
личности, природе и космосе.
Важнейшим аспектом Традиционной культуры и мифологического мировоззрения является и то, что мифы
изначально  живут  в  своём,  особенном  времени --- времени «первоначала», «первотворения», к которому
неприложимы  линейные  представления  о  течении  времени.  Подобное  отношение  к  времени  хорошо
прослеживается  в  народных,  в  частности русских  сказках,  где время  действия  определяется как «давным-давно», «в стародавние времена» и т. д.
Кроме того, миф, особенно на начальных стадиях своего развития (в долитературном виде), мыслит образами,
живёт эмоциями, ему чужды доводы современной формальной логики. При этом он объясняет мир, исходя из
ежедневной  практики.  Данный  парадокс  объясним  тем,  что  социум,  где  преобладает  мифологическое
мировоззрение, напрямую соотносит особенности своего восприятия мира с реальным миром, индивидуальные
психические процессы с природными и социальными явлениями, зачастую не делая различия между причиной
и следствием, а часто меняя их местами.
По Традиционному мифологическому мировоззрению её последователь способен подняться до уровня бога, а
значит для человека, рода и социума миф о странствиях и подвигах культурного героя, в большинстве случаев,
читай «бога», был практически полезен и являлся руководством к действию.
Особенности, характерные для мифологической формы мировоззрения:
Антропоморфность --- рассмотрение  явлений  природы  по  аналогии  с  человеком.  Явлениям  природы
приписываются все те свойства, которые есть у человека: ощущения, реакции на негативные факторы, желания,
страдание и т. п. 
Дескриптативность --- стремление к объяснению событий, явлений в форме описательного рассказа, сказания,
легенды; среди действующих фигур --- герои и боги в виде особых людей. 
Синкретизм (слитность, нерасчленённость) объективного и субъективного миров, что в значительной степени
объясняется антропоморфностью, пронизывающей все стороны этой формы мировоззрения. 
Связь  с  магией  свойственна  более  зрелому  первобытно-общинному  сознанию  и  выражается  в  действиях
колдунов, шаманов и других людей, вооружённых зачатками научных знаний о теле человека, о животных и
растениях. Наличие магического элемента в составе данной формы мировоззрения позволяет отвергнуть точку
зрения, будто это мировоззрение не связано с практикой, а является лишь пассивно-созерцательным. 
Апелляция  к  прецеденту  в  объяснении  событий,  определяющих  современный  порядок  вещей.  Например:
"человек стал смертным потому, что гонец (часто какое-то животное) неправильно передал волю божества",
"человек начал использовать огонь потому, что тот был украден у богов Прометеем" и т. д. 
Антиисторичность.  Время  не  понимается  как  процесс  прогрессивного  развития.  В  лучшем  случае  оно
допускается как обращённое вспять: движение от золотого века к серебрянному и медному, что само по себе
выражает желание видеть мир статичным, постоянно воспроизводящимся в том же самом виде. 

\newpage
\section{Генезис философии как переход от мифологического мышления к рациональному}
Нужно отметить, что генезис философии является проблемой для самой философии, развиваясь, она постоянно
сталкивается с проблемой собственного возникновения, ибо, только решив ее, философия сможет в полной мере
осознать свою сущность. 
Существуют  три  основополагающие  теории  возникновения  философии:  мифогенная,  гносеогенная  и  та,
которую, по ее характеру, можно назвать концепцией "качественного скачка". Любая из этих точек зрения, если
признать  только  ее  единственно  верной,  ограничена.  Философия,  ее  категориальный  аппарат  во  многом
наследует мифологемы и философское мировоззрение во многом вырастает из мифа, особенно у таких ранних
философов, как Ферекид, Эмпедокл и, даже, Платон. Вместе с тем, философское мировоззрение радикально
противоречит мифической картине мира. Поэтому нельзя говорить, что сущность философии вытекает из мифа:
она настолько же из него вытекает, насколько от него отталкивается в своем становлении. Миф у Ферекида и
даже у Гомера уже не тот миф, которому безусловно поклоняются и которому безусловно верят, --- это уже
рефлектированный миф, с которого снят ореол святости. 
Если же говорить о происхождении философии из науки, то, как уже выяснено, это невозможно. Вместе с тем,
невозможно, наверное, и развитие философии без науки, как и без мифа. Поэтому, отвечая на вопрос,
происходит ли философия из мифа или из накопления научного знания, невозможно дать односложный ответ
``да" или ``нет", здесь отношения более сложные и замысловатые.
Ясно одно: без них бы философия не возникла,
и с ними она также могла бы не возникнуть. Так, мы имеем примеры в иных культурах, когда существовала
развитая  мифология  и  был достаточный  уровень научного знания, однако  философия  так и не  состоялась
(Египет, Вавилон и др. культуры). Получается, что наука и миф являются необходимыми, но недостаточными
предпосылками возникновения философии. Не хватает еще чего-то, что делает философию именно философией
и что невозможно найти ни в науке, ни в мифе. 
Третья концепция генезиса философии, концепция ``качественного скачка", подчеркивает именно эту специфику
философского знания как его сущностное отличие от иных типов знания. Поясним: эта концепция не отрицает
важность  для  становления  философии  мифа  и  науки,  а  также  развития  социальных,  экономических,
политических  и  иных  связей.  Действительно,  без  совокупности  всех  этих  необходимых  предпосылок
философия  никак  бы  не  могла  появиться,  однако  весь  этот  перечень  предпосылок,  сколь  бы  длинным  и
объемным он ни был, не способен автоматически породить философию. В современном мире, как казалось бы,
всех этих предпосылок куда больше, и даже новых мифов создается больше, чем было старых у древних греков,
однако  Древняя  Греция  породила  такую  плеяду  оригинальных  философов,  которых  в  XX  веке  можно
пересчитать по пальцам одной руки. 
Концепция "качественного скачка" интересна, оригинальна и даже нова тем, что останавливает свое внимание
именно на внутреннем существе философского знания, которое несводимо ни к каким предпосылкам, которое
ниоткуда невозможно вывести, как только из самой философии. Философия  породила  саму  себя  в  акте
духовного познания и самопознания, она --- совершенно  суверенная,  автономная  область  знания,  знания
предвечного, ибо его неспособна постичь ни мифология, ни наука, ни религия. Философия возникает сразу и
как бы вдруг, разорвав все путы мифологического, научного, поэтического, эпического и какого угодно иного
вида  знания.  Но  хотя  философия  возникает  вдруг  и  сразу,  она  не  сразу  и  не  вдруг  приходит  к  своему
самосознанию, она не сразу обнаруживает себя в своей первозданной чистоте. В истории античной философии
она как бы идет от себя к себе, выражаясь терминологией Гегеля, от в-себе-бытия к для-себя-бытию.

\newpage
\section{Основные концепции античной натурфилософии}
Первые философские представления о природе вещества и происхождении его свойств зародились практически
одновременно в разных цивилизациях около VII века до н.э. В Китае это были Конфуций и Лао Цзы, в Индии ---
Будда, в Персии --- Зороастр (Заратустра), в Греции --- философы т.н. Милетской школы.
Все эти натурфилософские учения имеют общие черты:
\begin{enumerate}
  \item Космологический подход. Учение о природе вещества и его свойств является частью учения
    о мироздании в целом, причем свойства вещества с необходимостью следуют из свойств Вселенной.
\item Дуализм. Важнейшим  элементом  любого  натурфилософского  учения  является  существование  пар
противоположных мировых начал (Ян --- Инь, светлое --- тёмное, активное --- пассивное,
любовь --- ненависть и т.п.).
\end{enumerate}
Отличительная особенность греческой натурфилософии --- её в значительной степени светский (нерелигиозный)
характер. В греческой натурфилософии можно выделить два течения, выделившиеся по способу ответа на
вопрос о делимости материи: континуализм и атомизм.
Континуализм исходит из предположения, что материя непрерывна и делима до бесконечности; любая сколь
угодно малая часть материи тождественна тому телу, делением которого она получена.
Атомизм  утверждает, что материя дискретна и состоит из множества неделимых частичек --- атомов, ---
движущихся в пустоте.
Несмотря на принципиальные различия континуализма и атомизма в объяснении разнообразия веществ,  все
античные  натурфилософские  школы  имели  известную  общность  подхода:  в  любом  случае  многообразие
свойств считалось случайным проявление субстанции – неких абсолютных начал, хотя бы и дискретных.
В качестве основных черт натурфилософии можно отметить следующее:
\begin{enumerate}
  \item Умозрительность. Всякая античная натурфилософская концепция представляет собой абстракцию (порой
гениальную), лишённую каких-либо эмпирических основ. Чувственные данные всегда используются лишь как
иллюстрация для умозаключений;
\item Дедукция (рассуждение от общего к частному). Всякая античная натурфилософская концепция
  претендует на
всеобщее объяснение устройства Вселенной; свойства вещества логически вытекают из свойств Вселенной;
\item Выбор первоматерии (субстанции) в качестве объекта изучения.
\end{enumerate}

\newpage
\section{Антропологический поворот в античной философии (софисты и Сократ)}
К досократикам причисляют софистов, которые, однако, осуществили в философии антропологический поворот
---  от  исследования  природы  они  обратились  к  изучению  человека  как  общественного  существа.
Время
деятельности  софистов,  которые  были  первыми  “учителями  мудрости”  для  всех  желающих,  разоблачали
традиционные мифические представления и поставили под вопрос традиционные моральные и религиозные
нормы,  сделав  их  предметом  сознательного  и  критического  отношения,  называют  также  “греческим
просвещением”. Софисты открыли принципы относительности и субъективности всякого знания о мире (так
называемый философский релятивизм), т. е. его изменчивости и зависимости от человека как субъекта, от его
способностей и потребностей: человек есть мера всех вещей. Иначе говоря, первоначалом всего, “архэ”, надо
признать человека, личность, его чувства и его ум. В философии софистов впервые находит своё выражение и
обоснование характерный для западной цивилизации принцип индивидуализма. 

Сократ --- основоположник этики, т. е. философской теории морали, а также диалектики как такого искусства
ведения беседы, диалога, благодаря которому через столкновение противоположных мнений и противоречия
достигается общее понимание сути вещей, истина. Подвергнув критике релятивизм софистов, он искал общие
определения нравственных понятий, которые имеют силу для всех людей как разумных существ. Главный
предмет  всех  бесед  Сократа,  описанных  Платоном  и  Ксенофонтом, --- разумная  жизнь,  добродетель,
благо человека. 
Сократ обнаружил в своих беседах с людьми, что хотя сами они убеждены, что знают, в чем заключается их
благо и добродетель, в действительности они располагают лишь кажущимся знанием, которое не выдерживает
испытания “логосом” (разумом)  в  свободном  диалоге.  Он  придумал  определенный  метод  для  достижения
истинного, надежного знания о благе человека – повивально-иронический диалог. Метод Сократа вытекает из
полного доверия к “логосу”, разуму. Он был убеждён в том, что человек, как существо разумное, не должен
подчиняться ничему, кроме разума, который есть наилучшее, божественное во мне. Неосмысленная, неразумная
жизнь не имеет никакой ценности. Сократовское отождествление добродетели и знания называется этическим
рационализмом. Разум --- и только он один --- есть источник и мерило нравственности. Разум может и должен
подчинить себе жизнь. В этом и заключается назначение человека и высшее благо для него. 

\newpage
\section{Проблемы бытия и познания в философии Платона и Аристотеля}
\subsection{Платон}
Платон --- основоположник идеалистического направления философии.
Платон  является  основателем  идеализма.  Главными  положениями  его  идеалистического  учения  являются
следующие:
\begin{itemize}
  \item материальные вещи изменчивы, непостоянны и со временем прекращают свое существование;
  \item окружающий мир, «мир вещей», также временен и изменчив и в действительности не существует как
самостоятельная субстанция;
\item реально существуют лишь чистые (бестелесные) идеи (эйдосы);
\item чистые (бестелесные) идеи истинны, вечны и постоянны;
\item любая существующая вещь является всего лишь материальным отображением первоначальной идеи (эйдоса)
данной вещи (например, кони рождаются и умирают, но они лишь являются воплощением идеи коня, которая
вечна и неизменна и т.д.);
\item весь мир является отображением чистых идей (эйдосов).
\end{itemize}
«Триада» Платона --- «единое», «ум», «душа».
Также Платон выдвигает философское учение о триаде, согласно которому все сущее состоит  из  трех
субстанций:
\begin{itemize}
  \item «Единое»:
является основой всякого бытия;
не имеет никаких признаков (ни начала, ни конца, ни частей, ни целостности, ни формы, ни содержания, и т.д.);
есть ничто;
выше всякого бытия, выше всякого мышления, выше всякого ощущения;
первоначало всего – всех идей, всех вещей, всей явлений, всех свойств (как всего хорошего с точки зрения
человека, так и всего плохого).
\item «Ум»:
происходит от «единого»; разделен с «единым»; противоположен «единому»; является сущностью всех вещей;
есть обобщение всего живого на земле.
\item «Душа»:
подвижная субстанция, которая объединяет и связывает <<единое --- ничто» и «ум --- все живое»,
а также связывает
между собой все вещи и все явления; также согласно Платону душа может быть мировой и душой отдельного
человека; при гилозоическом (одушевленном) подходе душу могут иметь также вещи и неживая природа;
душа человека (вещи) есть часть мировой души;
душа бессмертна;
при смерти человека умирает только тело, душа же, ответив в подземном царстве за свои земные поступки,
приобретает новую телесную оболочку;
постоянство души, смена телесных форм --- естественный закон Космоса.
\end{itemize}

Касаясь гносеологии (учения о познании), Платон исходит из созданной им идеалистической картины мира:
поскольку материальный мир является всего лишь отображением «мира идей», то предметом познания должны
стать прежде всего «чистые идеи»; «чистые идеи» невозможно познать с помощью чувственного познания
(такой тип познания дает не достоверное знание, а лишь мнение --- «докса»); высшей духовной деятельностью
могут  заниматься  только люди подготовленные --- образованные  интеллектуалы,  философы,  следовательно,
только они способны увидеть и осознать «чистые идеи».

\subsection{Аристотель}
Аристотель (384-322 до н.э.).
Из семьи придворного врача. Был слушателем ``Aкадемии" Платона до смерти основателя
в течение 20 лет. ``Платон мне друг, но истина дороже."
Критиковал платонизм: сущность вещей в самих вещах, а не
в идеях. Странствует. В 50 лет возвращается в Афины.
Открывает философскую школу ликей (рядом с храмом Аполлона
Ликейского). Стал учителем А.Македонского. После смерти Александра обвинен в богохульстве. Покидает Афины.
Создатель самой обширной научной системы из существовавщих в античности.
Создал вместе со своими учениками новое
научное направление, систематизировал науку, определил предмет и методы отдельных наук.
``О философии", ``Диалог о
счастье", ``О небе", ``Политика", ``Экономика", ``Поэтика", ``Риторика", ``Метафизика". 
Для объяснения того, что существует, Аристотель принимал четыре причины:
\begin{itemize}
  \item сущность и суть бытия, в силу которой всякая вещь такова, какова она есть (формальная причина);
  \item материя и подлежащее (субстрат) --- то, из чего что-либо возникает (материальная причина);
  \item движущая причина, начало движения;
  \item  целевая причина --- то, ради чего что-либо осуществляется.
\end{itemize}

О предмете философии: если предметом физики являются материальные сущности,
то философия имеет право на самостоятельное
существование, если в ней есть элементы нематериальных причин, сверхчувствительные,
неподвижные, вечные сущности.
Это наиболее божественная наука в двух смыслах:
\begin{enumerate}
  \item Владеть ею пристало скорее богу, чем человеку;
  \item Ее предметом являются божественные предметы.
\end{enumerate}
Поэтому Аристотель называет свою философию теологией, учением о боге.
Дал определение понятиям сущего и сущности.
Сейчас: Сущее --- то, что существует, что можно наблюдать. Сущность --- то, что нельзя ощутить,
а только понять
как внутренние связи (Солнечное  затмение --- сущее, понимание его --- сущность). 
Основа  любого  бытия ---  проматерия,
перводвиг. всего --- бог. Материя то, из чего вещь возникла. Вещь существует независимо от идеи, материя
первична. Материя --- сама по себе не обозначается ни как определенное по существу, ни как определенное по
количеству, ни как обладающее каким-либо иным св-вом сущего. Такая материя включает и проматерию из
провеществ (вода, воздух, земля, огонь).
Познание: 1-чувств -овладение единичным 2-рацион-овлад. общим. 
Путь познания; ощущение -> представление -> опыт -> искусство -> наука.

Об обществе и государстве: "Политика".
Человек --- общественное существо. Рабовладение --- естественное состояние организации
общества. С рождением каждому определено быть или рабом, или рабовладельцем.
Общество свободных людей:
\begin{enumerate}
  \item Очень богаты. Противоественный способ приобретения богатства --- плохо; 
\item Средний класс;
\item Крайне бедные.
\end{enumerate}
Формы государства: хорошие (монархия, аристократия), плохие (тирания, олигаргия, демократия).
Задача --- предотвращение чрезмерного накопления богатства.

\newpage
\section{Античная философия эпохи эллинизма (эпикурейцы, стоики, скептики)}
Эллинистическая философия --- греко-римская философия в период от начала походов Александра Великого
до овладения
римлян Египтом, до Августина и в более позднюю эпоху --- до конца Древнего мира (середина VI века н.э.). 
Экономический и политический
упадок Греции, закат роли полиса отражаются в греческой философии. Усилия, направленные на
познание объективного мира (философия Аристотеля), активное участие в политической жизни,
которое проявилось у греческих философов,
постепенно  замещаются  индивидуализмом,  и  морализированием  либо  скептицизмом  и  агностицизмом.  Со
временем интерес к философскому мышлению вообще резко падает. Приходит период мистики,
религиозно-философского синкретизма, христианской философии. 

\textbf{Скептицизм} (конец IV века до н.э.). Основатель --- Пиррон.
Как и Сократ излагал свои идеи устно. Скепсис имел место в греческой философии
и раньше. В элленисическую эпоху складываются его принципы, ибо скепсис определяется
не методическими установками в
невозможности  дальнейшего  познания,  а  отказом  от  возможности  дойти  до  истины.  Доводы  против
правильности как чувственных восприятий, так и познаний мысли скептики объединили в десять тезисов,
тропов: 
\begin{enumerate}
  \item подвергает сомнению положения о действительности развития физиологической структуры животных; 
  \item подчеркивает индивидуальные различия людей с точки зрения физиологии и психики; 
  \item о различии чувственных органов, в которых вещи вызывают разные ощущения и т.д.
\end{enumerate}
В общем, они делали
акцент на субъективизме познания человека.
Исходя из принципа ``ничего не утверждать", подкрепленного тропами,
скептики отвергали любые попытки познания мира в котором всё взаимосвязано и
подчиняется единым законам. 

\textbf{Кинизм}. Циники (Антисфен, Диоген Синопский и др.) стремились не сколько к построению законченной
теории бытия и познания, сколько к отработке и эксперементальной
проверке на себе определенного образа жизни. Киники с
вызовом именовали себя “гражданами мира” и обязывались жить в любом обществе не по его законам, а по своим
собственным, с готовностью приемля статус нищих, юродивых. Положение не только крайне бедственное, но и
унизительное избирается ими как наилучшее. Киники хотели быть нагими и одинокими, социальные связи
и культурные навыки --- мнимость. Все виды духовной и физической бедности предпочтительнее богатства. 

\textbf{Эпикуризм}. Эпикуризм --- учение и образ жизни, исходящие из идей Эпикура и
его последователей, отдающих не
задумываясь  предпочтение  материальным радостям  жизни.  Видимо,  наиболее  выдающимся  мыслителем
эллинистического периода был Эпикур. Главное произведение: ``Правило" (канон),
``О природе" и т.д. Учение Демокрита
Эпикур не принимает пассивно, но поправляет его, дополняет и развивает. Если Демокрит характеризует
атомы по
величине, форме и положению в пространстве, то Эпикур им приписывает еще одно свойство ---
тяжесть. Вместе с
Демокритом он признает, что атомы, движутся в пустоте. Эпикур допускает и признает закономерным и
определяет
отклонение  от  прямолинейного движения.  Эпикурово  понимание  случайности  не  исключает,  причинного
объяснения.  У  человека  есть  свобода  выбора,  а  не  все  предопределено.  В  учении  о  душе  Эпикур  отстаивает
материалистические взгляды.
Согласно Эпикуру, душа --- это не нечто бестелесное, а структура атомов, тончайшая
материя, рассеянная по всему организму. Отсюда вытекает и отрицание бессмертия души. В области теории
познания Эпикур --- сенсуалист. В основе всякого познания лежат ощущения, которые возникают при отделении
отражений  от  объективно существующих  предметов  и  проникают  в  наши  органы  чувств.  Таким  образом,
основной  предпосылкой  всякого  познания  является  сущ-е
объективной  реальности  и  ее  познаваемость  с помощью чувств. 

\textbf{Стоицизм}. В конце IV века до н. э. в Греции формируется стоицизм, который в эллинистическом, а также в более
позднем римском периоде становится одним из самых распространенных философских течений.
Его основателем является Зенон.
Трактат  "О  человеческой  природе".  Стоики  часто  сравнивали  философию  с  человеческим организмом.  Логику  они  считали
скелетом, этику --- мышцами, а физику --- душой. Более определенную форму стоическому мышлению придает
Хрисипп. Он  превращает стоическую  философию  в  обширную  систему. 
Стоики  характеризовали философию  как ``упражнение  в
мудрости". Орудием философии, ее основной частью, они считали логику. Она учит обращаться понятиями,
образовывать  суждения  и  умозаключения.  Без  нее  нельзя  понять  ни  физику,  ни  этику,  которая  является
центральной частью стоической философии. В онтологии стоики признают два основных принципа:
материальный принцип
(материал), который считается основой, и духовный принцип --- логос (бог),
который проникает через всю материю
и образует конкретные единичные  вещи.  Стоики,  в отличие от Аристотеля, сущностью  считали материальный
принцип (хотя, так же как и он, признавали материю пассивным, а логос (бог) --- активным принципом).
Понятие
бога в стоической философии можно охарактеризовать как пантеистическое. Логос, согласно их взглядам, пропитывает всю
природу, проявляется везде в мире. Он является законом необходимости, провидением. Понятие бога сообщает всей
их концепции бытия детерминистский, вплоть до фатализма, характер, который пронизывает и их этику. В области
теории  познания  стоики  представляют  по  преимуществу  античную  форму  сенсуализма.  Стоики  упрощают
аристотелевскую  систему  категорий --- четыре  основными  категориями:  субстанция  (сущность),  количество,
определенное  качество  и  отношение,  согласно  определенному  качеству.  С  помощью  данных  категорий
постигается  действительность.  Центром  и  носителем  познания  является  душа.  Она  понимается  как  нечто  телесное,
материальное. Иногда ее обозначают как пневма (соединение воздуха и огня). Ее централь часть, в которой
локализируется способность к мышлению и вообще все то, что можно определить в нынешних терминах как
психическую деятельность, стоики называют разумом (гегемоником). Разум связывает человека со всем миром.
Индивидуальный разум является частью мирового разума. Хотя стоики считают основой всякого познания чувства,
большое внимание они уделяют и проблемам мышления. 

\newpage
\section{Характерные черты и этапы средневековой христианской философии}
Раннее  средневековье  характеризуется  становлением  Христианской  догматики  в  условиях  формирования
европейского  государства  в  результате  падения  Римской  империи.  В  условиях  жесткого  диктата  церкви  и
господства власти философия была объявлена служанкой богословия, которая должна была использовать свой
рациональный  аппарат  для  подтверждения  догматов  христианства.  Эта  философия  получила  название
"схоластики" (опиралась на формальную логику Аристотеля).
Еще в V веке (христианство уже господствующая религия в Греции и Риме) было сильно влияние философии
неоплатонизма,  враждебного  христианству. (Нехристианские  философские  школы  были  закрыты  по  декрету
императора  Юстиана  в  529г.)  При  этом  одни  христианские  идеологи  склонялись  к  отрицанию,  другие  к 
использованию  учений  философии идеалистов  древности.  Так  возникла  литература апологетов (защитников)
христианства,  а  за  ней  возникает  патристика --- сочинения отцов 
церкви,  писателей,  заложивших  основы философии христианства.
Со II века Греческие апологеты обращались к императорам, преследовавшим христианство. Они стремились
доказать,  что  христианство  поднимает  такие  вопросы,  которые  ставила  и  предшествующая  греческая
философия, но дает более совершенное их разрешение. Видный апологет --- Тертуллиан (из Карфагена, II в.)
--- существует
непримиримое  разногласие  между  религией,  божественным  откровением,  священным  писанием  и 
человеческой
мудростью. Не создав философских систем, апологеты, однако, наметили круг вопросов,
которые стали основными для христианской философии (о
боге,  о  сотворении  мира,  о  природе  человека  и  его  целях).  Апологетика  использует  логические  доводы,
обращенные к разуму, для доказательства бытия бога, бессмертия души, разбирает доводы, обращенные против
религии  и  отдельных  догматов.  Противоречие  в  том,  что  будучи  рациональной  по  форме,  апологетика
иррациональна по содержанию, т. е. обращаясь к разуму, говорит о непостижимости разумом религиозных
догматов.
Склонна к софистике и догматизму.
Античная философия космоцентрична, философия средневековья --- теоцентрична (основная проблема
--- проблема
христианского бога).

Христианство появилось примерно в середине 1 века и стимулировало развитие
\textbf{средневековой философии}.
Этапы развития средневековой философии:
\begin{enumerate}
  \item Этап патристики (II-VIII века, конец этапа --- деятельность Боэция I схоласта)
  \item Этап становления схоластики (VII–XII века --- Боэций, Эриуген, П. Абеяр)
  \item Расцвет схоластики(XIII век --- Бэкон, Альберт Великий, Фома Аквинский)
\end{enumerate}

\textbf{Блаженный Августин}. Все сущее, поскольку оно существует и именно потому что существует,
есть благо. Зло ---
не субстанция, а недостаток, порок и повреждение формы, небытие. Напротив, благо есть субстанция, «форма»
со всеми ее элементами: видом, мерой, числом, порядком. Бог есть источник бытия, чистая форма, источник
блага. Поддержание бытия мира есть постоянное творение его Богом вновь. Если бы творческая сила Бога
прекратилась, мир вернулся бы в небытие. Мир один, последовательность миров --- игра воображения.
В мировом порядке
всякая вещь имеет свое место. Материя также имеет свое место в строе целого.
Душа --- нематериальная субстанция, отличная от тела, а не простое свойство тела. Она бессмертна.
Вечность и время: Мир ограничен в пространстве, а бытие его ограничено во времени. Время и пространство
существуют только в мире и с миром. Начало творения мира --- начало времени. Время есть мера движения и
изменения. Августин пришел к гениальной идее: ни прошедшее, ни будущее не имеют реального существования
---
действительное  существование  присуще  только  настоящему.  Нет  никакого  «пред  тем»,  никакого  «потом».
Прошедшее обязано своим существованием нашей памяти, а будущее --- нашей надежде. «В вечном нет ни
приходящего, ни будущего, ибо что проходит, то перестает существовать, а что будет, то еще не начало быть.»

\textbf{Фома  Аквинский}  (Аквинат).  Систематизатор  ортодоксальной  схоластики,  основатель  томизма 
---  одного  из
направлений схоластики. Исходным принципом учения является божественное откровение: человеку для своего спасения
нужно знать такое, что ускользает от его разума, через божественное откровение.
Разграничил области философии и теологии: предметом философии является «истины разума», а второй --- «истины откровения».
Не все «истины откровения» доступны рациональному доказательству. Религиозная истина не может быть
уязвима со стороны философии, в
жизненном, практически-нравственном отношении любовь к Богу важнее познания Бога.
Исходя из учения Аристотеля, Фома рассматривал Бога как первопричину и конечную цель всего сущего,
как «чистую
форму», «чистую актуальность». Сущность всего телесного --- в единстве формы и материи. Они суть реальные
сверхчувственные внутренние принципы, образующие всякую реальную вещь, все телесное вообще.
Материя ---
только воспреемница сменяющих друг друга форм, «чистая потенциальность», ибо лишь благодаря форме вещь
является вещью определенного рода и вида. Кроме того, форма --- целевая причина образования вещи.
Индивидуальность человека --- это личностное единство души и тела, именно душа обладает животворящей
силой  человеческого  организма.  Душа  нематериальна,  самосуща,  уникальна  и  бессмертна ---
субстанция,
обретающая полноту лишь в единстве с телом, которое  тоже участвует в духовно-душевной деятельности
человека.
Основной принцип познания, по Аквинату, --- реальное существование всеобщего. В споре об универсалиях (см.
ниже) отстаивал позиции умеренного реализма, т.е. всеобщее существует трояко: «до вещей» (в разуме Бога как
идеи будущих идей, как вечные идеальные прообразы сущего), «в вещах», получив конкретное осуществление, и
«после вещей» --- в мышлении человека после абстрагирования и обобщения. Человеку присуще две способности
познания: чувство и интеллект. Познание начинается с чувствительного опыта под действием
внешних объектов. Но
воспринимается не все бытие объекта, а лишь то в нем, что уподобляется субъекту. При вхождении в душу
познающего познаваемое теряет свою материальность и может войти в нее лишь как образ. Вещь существует вне
нас во всем своем бытии и внутри нас как образ.
Истина --- соответствие интеллекта и вещи. При этом понятия, образуемые интеллектом, истинны в той мере, в
какой они соответствуют своим понятиям, предшествующим в интеллекте Бога.

\textbf{Универсалии}.  Одна  из  особенностей  средневековой  философии  проявилась  в  споре  между  реалистами  (realis  –
вещественный, действительный) и номиналистами (nomen – имя, наименование) о природе универсалий, т.е. о
природе общих понятий. Реалисты (Эриугенаи, Фома Аквинский), основываясь на положении Аристотеля о том,
что общее существует в неразрывной связи с единичным, являясь с его формой, сформулировали концепцию о
трех видах существования универсалий. Универсалии существуют трояко (см. выше про Фому). Такое решение 
вопроса носит в истории философии название «умеренного реализма» в отличие от «крайнего реализма», по которому
общее существует только вне вещей. Крайний реализм не мог быть принят ортодоксальной церковью именно из-за
того, что материя была частично оправдана христианством как одна из двух природ Иисуса Христа.
Номиналисты  (в  частности, Росцелин)  довели  идею  отрицания  объективного  существования  общего  до
логического конца, считая, что универсалии существуют лишь в человеческом разуме, в мышлении, т.е. они
отрицали  не  только  наличие  общего  в  конкретной  единичной  вещи,  но  и  его  существование  «до  вещи».
Универсалии суть только имена вещей, и существование их сводится лишь к колебаниям голоса. Существует
только индивидуальное, и только оно может быть предметом познания.

\newpage
\section{Особенности философии эпохи Возрождения}
1.  Философией  эпохи  Возрождения  называется  совокупность  философских  направлений,  возникших  и
развивавшихся  в  Европе  в  XIV  —  XVII  вв.,  которые  объединяла  антицерковная  и  антисхоластическая
направленность,  устремленность  к  человеку,  вера  в  его  великий  физический  и  духовный  потенциал,
жизнеутверждающий и оптимистический характер.
Предпосылками возникновения философии и культуры эпохи Возрождения были:
• совершенствование орудий труда и производственных отношений;
• кризис феодализма;
• развитие ремесла и торговли;
• усиление городов, превращение их в торгово-ремесленные, военные, культурные и политические центры,
независимые от феодалов и Церкви;
• укрепление, централизация европейских государств, усиление светской власти;
• появление первых парламентов;
• отставание от жизни, кризис Церкви и схоластической (церковной) философии;
• повышение уровня образованности в Европе в целом;
• великие географические открытия (Колумба, Васко да Гамы, Магеллана);
•  научно-технические  открытия  (изобретение  пороха,  огнестрельного  оружия,  станков,  доменных  печей,
микроскопа, телескопа, книгопечатания, открытия в области медицины и астрономии, иные научно-технические
достижения).
2. Основными направлениями философии эпохи Возрождения являлись:
• гуманистическое (XIV - XV вв., представители: Данте Алигьери, Франческо Петрарка, Лоренцо Валли и др.) -в центр внимания ставило человека, воспевало его достоинство, величие и могущество, иронизировало над
догматами Церкви;
• неоплатоническое (сер. XV - XVI вв.), представители которого - Николай Кузанский, Пико делла Мирандола,
Парацельс и др. - развивали учение Платона, пытались познать природу, Космос и человека с точки зрения
идеализма;
• натурфилософское (XVI - нач. XVII вв), к которому принадлежали Николай Коперник, Джордано Бруно,
Галилео Галилей и др., пытавшиеся развенчать ряд положений учения Церкви о Боге, Вселенной, Космосе и
основах мироздания, опираясь на астрономические и научные открытия;
• реформационное (XVI - XVII вв.), представители которого -Мартин Лютер, Томас Монцер, Жан Кальвин,
Джон  Усенлиф,  Эразм  Роттердамский  и  др.  -  стремились  коренным  образом  пересмотреть  церковную
идеологию и взаимоотношение между верующими и Церковью;
• политическое (XV - XV] вв., Николо Макиавелли) - изучало проблемы управления государством, поведение
правителей;
• утопическо-социалистическое (XV - XVII вв., представители -Томас Мор, Томмазо Кампанелла и др.) - искало
идеально-фантастические  формы  построения  общества  и  государства,  основанные  на  отсутствии  частной
собственности и всеобщем уравнении, тотальном регулировании со стороны государственной власти.
3. К характерным чертам философии эпохи Возрождения относятся: 
• антропоцентризм и гуманизм — преобладание интереса к человеку, вера в его безграничные возможности и
достоинство;
• оппозиционность к Церкви и церковной идеологии (то есть отрицание не самой религии, Бога, а организации,
сделавшей себя посредником между Богом и верующими, а также застывшей догматической, обслуживающей
интересы Церкви философии — схоластики);
• перемещение основного интереса от формы идеи к ее содержанию;
•  принципиально  новое,  научно-материалистическое  понимание  окружающего  мира  (шарообразности,  а  не
плоскости  Земли,  вращения  Земли  вокруг  Солнца,  а  не  наоборот,  бесконечности  Вселенной,  новые
анатомические знания и т. д.);
• большой интерес к социальным проблемам, обществу и государству;
• торжество индивидуализма;
• широкое распространение идеи социального равенства

\newpage
\section{Традиция рационализма в философии Нового времени}
Рассмотрение данного вопроса необходимо начать именно с самого яркого представителя рационалистического
направления – Рене Декарта. Несколько слов о его биографии. Он в восемь лет уходит на учебу в иезуитский
колледж Ла-Флеш. Здесь он получил основы образования. В ряде жизнеописаний Декарта указывается, что
сухое, педантичное обучение его не удовлетворяло. Отрицательное отношение к схоластическому пониманию
науки и философии проявилось у него, однако, позже, когда он как военный побывал в значительной части
Европы. В 1621 г. он уходит с военной службы и путешествует. Посетил Германию, Польшу, Швейцарию,
Италию и некоторое время жил во Франции. Наиболее интенсивно предавался исследованиям во время своего
сравнительно долгого пребывания в Голландии в 1629—1644 гг. В этот период он пишет большинство своих
работ. Годы 1644—1649 были наполнены стремлением отстоять, и не только теоретически, взгляды и идеи,
содержащиеся, в частности, в «Размышлениях о первой философии» и в «Началах философии». В 1643 г. в
Утрехте, а в 1647 г. в Лейдене (где сравнительно долго жил Декарт) было запрещено распространение его
воззрений, а его труды были сожжены. В этот период Декарт вновь несколько раз посещает Париж и думает
даже о возвращении во Францию. Однако затем он принимает приглашение шведской королевы Христины и
уезжает в Стокгольм, где вскоре умирает от простуды.
Наиболее  выдающиеся  из  его  философских  трудов  —  это  работы,  посвященные  (как  и  у  Бэкона)
методологической  проблематике.  К  ним  принадлежат,  прежде  всего,  «Правила  для  руководства  разума»,
написанные в 1628—1629 гг., в которых Декарт излагает методологию научного познания. С этой работой
связано и вышедшее в 1637 г. как введение к его трактату о геометрии «Рассуждение о методе». В 1640— 1641
гг.  Декарт  пишет  «Размышления  о  первой  философии»,  в  которых  вновь  возвращается  к  определен ным
аспектам своей новой методологии и одновременно придает ей более глубокое философское обоснование. В
1643 г. выходит его труд «Начала философии», в котором полно изложены его философские воззрения.
Естествознание XVI—XVII столетий еще не формулирует эти новые принципы познания (по крайней мере без
соответствующей степени общности). Оно скорее реализует их непосредственно в процессе овладения своим
предметом. Если философия  Бэкона является  предвестником  нового (его  философия скорее  симпатизирует
естествознанию Нового времени, чем создает для него философское обоснование), то в философии Декарта уже
закладываются  основания  (достаточно  общие)  новой  теории  света,  в  которой  не  только  обобщены,  но  и
философски  разработаны  и  оценены  все  полученные  к  тому  времени  результаты  нового  естествознания.
Поэтому философия Декарта представляет собой новый, цельный и рационально обоснованный образ мира, не
только соответствующий актуальному состоянию естествознания, но и полностью определяющий направление
его развития. Одновременно она вносит  и основополагающие  изменения в развитие  самого философского
мышления, новую ориентацию в философии.
Первую и исходную определенность всякой философии Декарт видит в определенности сознания — мышления.
Требование, что должно исходить лишь из мышления как такового, Декарт выражает словами: «во всем должно
сомневаться» - это абсолютное начало. Таким образом, первым условием философии он делает само отвер-жение всех определений.
Декартово сомнение и «отвержение всех определений» исходит, однако, не из предпосылки о принципиальной
невозможности существования этих определений. Это не скепсис, с которым мы встречались, например, в
античной философии. Принцип Декарта, согласно которому во всем следует сомневаться, выдвигает сомнение
не как цель, но лишь как средство.
Первичную достоверность Бэкон находит в чувственной очевидности, в эмпирическом, смысловом познании.
Для Декарта, однако, чувственная очевидность как основа, принцип достоверности познания неприемлема.
Декарт  ставит  вопрос  о  постижении  достоверности  самой  по  себе,  достоверности,  которая  должна  быть
исходной предпосылкой и поэтому сама не может опираться на другие предпосылки. Такую достоверность он
находит в мыслящем Я — в сознании, в его внутренней сознательной очевидности. «Если мы отбросим и
провозгласим ложным все, в чем можно каким-либо способом сомневаться, то легко предположить, что нет
бога, неба, тела, но нельзя сказать, что не существуем мы, которые таким образом - мыслим. Ибо является
противоестественным полагать, что то, что мыслит, не существует. А поэтому факт, выраженный словами: «я
мыслю, значит, существую» (cogito ergo sum), является наипервейшим из всех и наидостовернейшим из тех,
которые перед каждым, кто правильно философствует, предстанут».
С  проблематикой  познания  в  философии  Декарта тесно  связан  вопрос  о  способе конкретного достижения
наиболее истинного, т. е. наиболее достоверного, познания. Тем самым мы подходим к одной из важнейших
частей философского наследия Декарта — к рассуждениям о методе.
В «Рассуждении о методе» Декарт говорит, что его «умыслом не является учить здесь методу, которому каждый
должен следовать, чтобы правильно вести свой разум, но лишь только показать, каким способом я стремился
вести свой разум».
Правила, которых он придерживается и которые на основе своего опыта полагает важнейшими, он формулирует
следующим образом:
не  принимать  никогда любую  вещь за  истинную, если ты ее не познал как истинную с очевидностью;
избегать всякой поспешности и заинтересованности; не включать в свои суждения ничего, кроме того, что
предстало как ясное и видимее перед моим духом, чтобы не было никакой возможности сомневаться в этом;
разделить каждый из вопросов; которые следует изучить, на столько частей, сколько необходимо, чтобы эти 
вопросы лучше разрешить;
свои идеи располагать в надлежащей последовательности,    начиная    с    предметов    наипростейших и
наилегче познаваемых, продвигаться медленно, как бы со ступени на ступень, к знанию наиболее
сложных, предполагая порядок даже среди тех, которые естественно не следуют друг за другом;
совершать везде такие полные расчеты и такие полные обзоры, чтобы быть уверенным в том, что ты ничего не
обошел.
Правила  Декарта,  как  и  все  его  «Рассуждения  о  методе»,  имели  исключительное  значение  для  развития
философии и науки Нового времени.
Рационализм Декарта нашел много продолжателей. К наиболее выдающимся из тех, кто существенным образом
способствовал обогащению и развитию философской мысли, принадлежат, в частности, Б. Спиноза и Г. В.
Лейбниц.
Спиноза всю жизнь прожил в Голландии. И хотя, как уже говорилось, Голландия в то время была страной
прогрессивной,  он  и  здесь  не  избежал  жестокого  преследования  со  стороны  как  еврейских,  так  и
протестантских и католических религиозных кругов.
Основные  идеи  философии  Спинозы  изложены  в  его  главном  и  основном  труде  –  «Этике».  Рассуждения,
содержащиеся в «Этике», разделены на пять основных частей (о Боге, о природе и происхождении мысли, о
происхождении  и  аффекте,  о  человеческой   несвободе,   или   о   силе   аффектов,  о  силе  разума,  или  о
человеческой свободе). Как раз интерес для нашей темы представляет собой последняя, пятая часть «Этики».
Мышление  трактовалось  как  своего  рода  самосознание  природы.  Отсюда  принцип  познаваемости  мира  и
глубокий вывод: порядок и связь идей те же, что порядок и связь вещей. И те, и другие суть только следствия
божественной сущности: любить то, что не знает начала и не имеет конца, — значит любить Бога. Человек
может  лишь  постигнуть  ход  мирового  процесса,  чтобы  сообразовать  с  ним  свою  жизнь,  свои  желания  и
поступки. Мышление тем совершеннее, чем шире круг вещей, с которыми человек вступает в контакт, т.е. чем
активнее субъект. Мера совершенства мышления определяется мерой его согласия с общими законами природы,
а подлинными правилами мышления являются верно познанные общие формы и законы мира. Понимать вещь
—  значит  видеть  за  ее  индивидуальностью  универсальный  элемент,  идти  от  модуса  к  субстанции.  Разум
стремится постичь в природе внутреннюю гармонию причин и следствий. Эта гармония постижима, когда
разум, не довольствуясь непосредственными наблюдениями, исходит из всей совокупности впечатлений.
Готфрид  Вильгельм  Лейбниц  представляет  определенное  завершение  европейского  философского  рацио-нализма. В философии Лейбница отразились почти все философские импульсы его времени. Он был подробно
знаком  не только с  философией Декарта,  Спинозы,  Бойля, но  и с  философскими трудами представителей
эмпирически ориентированной английской философии, в частности с работами Дж. Локка, и давал им оценку.
Так,  он  был  не  согласен  с  концепцией  картезианского  дуализма,  отвергал  определенные  элементы  кар-тезианской теории познания, в частности тезис о врожденных идеях, возражал и против спинозовской единой
субстанции (бога), которая является и всем, и субстанцией.
Лейбниц не создал ни одного философского труда, в котором он представил бы или логически разработал свою
философскую систему. Его воззрения разбросаны по разным статьям и письмам.
Ядро философской системы Лейбница составляет учение о «монадах» — монадология. Монада — основное
понятие системы — характеризуется как простая, неделимая субстанция. Лейбниц отвергает учение Спинозы о
единой  субстанции,  которое,  по  его  представлениям,  вело  к  тому,  что  из  мира  исключаются  движение,
активность. Он утверждал, что субстанций бесконечное множество. Они, согласно его воззрениям, являются
носителями силы, имеют духовный характер.
Вопрос гармонии — важнейший в философии Лейбница. Она является неким внутренним порядком всего мира
монад и представляет собой принцип, преодолевающий изолированность монад.
Следующая характерная черта монад заключается в том, что каждая монада имеет собственную определенность
(является носителем определенных качеств), которой она отличается от всех остальных. В этой связи Лейбниц
формулирует и свой известный принцип тождества. Если бы две монады были полностью одинаковы, они были
бы тождественны, т. е. неразличимы.
По  степени  развития  он  различает  монады  трех  видов.  Низшая  форма,  или  монады  нижайшей  степени,
характеризуется «перцепцией» (пассивной способностью восприятия). Они способны образовывать неясные
представления. Монады высшей степени уже способны иметь ощущения и опирающиеся на них более ясные
представления. Эти монады Лейбниц определяет как монады-души. Монады наибольшей степени развития
способны к апперцепции (наделены сознанием). Их Лейбниц определяет как монады-духи.
Монады сами не имеют никаких пространственных (или каких-либо физических) характеристик, они, таким
образом,  не  являются  чувственно  постижимыми.  Мы  можем  их  постичь  лишь  разумом.  Чувственно  вос-принимаемые тела, т. е. соединения монад, различаются согласно тому, из каких монад они состоят. Тела,
содержащие лишь монады низшей степени развития (т. е. тела, в которых не содержатся монады, способ ные к
сознанию или ощущениям),-это тела физические (т. е. предметы неживой природы). Тела, в которых монады
способны к ощущениям и представлениям (содержат монады-души), являются биологическими объектами.
Человек  представляет  собой  такую  совокупность  монад,  в  которой  организующую  роль  играют  монады,
наделенные  сознанием.  Образование  совокупностей  монад  не  является  случайным.  Оно  определено
«предустановленной гармонией». При этом, однако, в каждой из монад потенциально заключена возможность
развития. Этим Лейбниц объясняет тот факт, что все монады постоянно изменяются, развиваются и при этом их 
развитие не «подвержено влиянию извне».
Отношение Лейбница к основным идеям сенсуалистской концепции познания более внимательное и осто-рожное, чем, например, отношение к ней Спинозы. Он не отвергает чувственного познания или роли опыта в
процессе познания. Он принимает главный тезис сенсуализма «ничего нет в разуме, что не прошло бы раньше
через чувства», но он дополняет его следующим положением — «кроме самого разума», т. е. врожденных
способностей к мышлению и образованию понятий или идей.
Чувственное  познание  выступает,  таким  образом,  как  определенная  низшая  ступень  или  предпосылка
рационального  познания.  Разумное,  рациональное  познание  раскрывает  действительное,  необходимое  и
существенное в мире, тогда как чувственное познание постигает лишь случайное и эмпирическое.
Философское мышление Лейбница представляет собой вершину европейской рационалистической философии.
Итак,  подводя  итог  вышесказанному,  мы  выделяем  следующие  характерные  черты  рационализма
новоевропейской философии:
- через достоверность мысли мыслящее существо идет к познанию окружающего мира;
- рационализм требует ясности и непротиворечивости мышления;
-  на  начальной  стадии  развития  неприятие  чувственного  метода  постижения  истины  как  такового,  с
последующим признанием, что он необходим как низшая ступень познания.


\newpage
\section{Традиция эмпиризма в философии Нового времени}
Европейскую философию 17 века условно принято называть философией Нового времени. Данный период
отличается неравномерностью социального развития. Так, например, а Англии происходит буржуазная
революция (1640–1688). Франция переживает период рассвета абсолютизма, а Италия вследствие победы
контрреформации оказывается надолго отброшенной с переднего края общественного развития. Общее
движение от феодализма к капитализму носило противоречивый характер и часто принимало драматические
формы. Расхождение между силой власти, права и денег приводит к тому, что сами жизненные условия для
человека становятся случайными.
В силу всего  перечисленного,  философия  Нового  времени  не  является тематически  и  содержательно
однородной, она представлена различными национальными школами и персоналиями. Но, несмотря на
все различия, сущность философских устремлений у всех одна: доказать, что между фактическим и
логическим положением дел существует принципиальное тождество. По вопросу о том, как реализуется
это тождество, существую две философские традиции: эмпиризм и рационализм.
Успешное  освоение  природы   в   рамках капиталистического способа производства было немыслимо без
развития наук о природе, а претворение в жизнь новых социально-политических идеалов предполагало
иную, по сравнению с теоцентризмом, модель человеческого участия в переустройстве мира. Новое время
входило  в  жизнь  и  развивалось под лозунгами  свободы,  равенства,  активности  индивида.

\subsection {Ф.Бэкон}
Главным
орудием реализации этих лозунгов явилось рациональное знание. Один из классиков философии Нового
времени,  Ф.Бэкон,  выразил  это  в  ставшем   знаменитым утверждении: «Знание есть сила, и  тот,  кто
овладеет знанием, тот будет могущественным».
Сын высокопоставленного английского сановника, Бэкон и сам стал со временем государственным канцлером
Англии. Главный  труд  Бэкона  – «Новый Органон»(1620).  Название это показывает, что  Бэкон сознательно
противопоставлял свое понимание науки и ее метода тому пониманию, на которое опирался «Органон» (свод
логических работ) Аристотеля. Другим важным сочинением Бэкона была утопия «Новая Атлантида».
В согласии с передовыми умами своего века Бэкон провозгласил высшей задачей познания завоевание природы
и усовершенствование человеческой жизни. Но это — практическое в последнем счете — назначение науки не
может, по Бэкону, означать, будто всякое научное исследование должно быть ограничено соображениями о его
возможной непосредственной пользе. Знание — сила, но действительной силой оно может стать, только если
оно истинно, основывается на выяснении истинных причин происходящих в природе явлений. Лишь та наука
способна побеждать природу и властвовать над ней, которая сама «повинуется» природе, т. е. руководится
познанием ее законов.
Поэтому Бэкон различает два вида опытов: 1) «плодоносные» и 2) «светоносные». Плодоносными он называет
опыты, цель которых — принесение непосредственной пользы человеку, светоносными — те, цель которых
не   непосредственная   польза,  а познание законов явлений и свойств вещей. Недостоверность известного
доселе знания обусловлена, по Бэкону, ненадежностью умозрительной формы умозаключения и доказательства.
Как  уже  говорилось,  творчество  Бэкона  характеризуется  определенным  подходом  к  методу  человеческого
познания и мышления. Исходным моментом любой познавательной деятельности являются для него прежде
всего  чувства.  Поэтому  его  часто  называют  основателем  эмпиризма  —  направления,  которое  строит  свои
гносеологические  посылки  преимущественно  на  чувственном  познании  и  опыте.  Принятие  этих
гносеологических  посылок  характерно  и  для  большинства  других  представителей  английской  философии
Нового времени. Основной принцип этой философской ориентации в области теории познания выражен в тезисе:
«Нет ничего в разуме, что бы до этого не прошло через чувства».
Понятия добываются обычно путем слишком поспешного и недостаточно обоснованного обобщения. Поэтому
первым условием реформы науки, прогресса знания является усовершенствование методов обобщения, образования понятий.
Так как процесс обобщения есть индукция, то логическим основанием реформы науки должна 
быть новая теория индукции.
До Бэкона философы, писавшие об индукции, обращали понимание главным образом на те случаи или
факты, которые подтверждают доказываемые или обобщаемые положения. Бэкон подчеркнул значение тех
случаев,   которые   опровергают обобщение, противоречат ему. Это так называемые негативные инстанции.
Уже  один-единственный  такой  случай  способен  полностью  или  по  крайней  мере  частично  опровергнуть
поспешное обобщение. По Бэкону, пренебрежение к отрицательным инстанциям есть главная причина ошибок,
суеверий и предрассудков.
Бэкон выставляет новую логику: «Моя логика, однако, существенно отличается от традиционной логики тремя
вещами: самой своей целью, способом доказательства и тем, где она начинает свое исследование. Целью моей
науки не является изобретение аргументов, но различные искусства; не вещи, что согласны с принципами, но
сами принципы; не некоторые правдоподобные отношения и упорядочения, но прямое изображение и описание
тел». Как видно, свою логику он подчиняет той же цели, что и философию. Основным рабочим методом своей
логики Бэкон считает индукцию. В этом он видит гарантию от недостатков не только в логике, но и во всем
познании  вообще.  Характеризует  он  ее  так:  «Под  индукцией  я  понимаю  форму  доказательства,  которая
присматривается к чувствам, стремится постичь естественный характер вещей, стремится к делам и почти с
ними сливается».
Бэкон,  однако,  останавливается  на  данном  состоянии  разработки  и  существующем  способе  использования
индуктивного подхода. Он отвергает ту индукцию, которая, как он говорит, осуществляется простым
перечислением. Такая индукция «ведет к неопределенному заключению, она подвержена опасностям, которые
ей угрожают со стороны противоположных случаев, если она обращает внимание лишь на то, что ей привычно,
и не приходит ни к какому выводу». Поэтому он подчеркивает необходимость переработки или, точнее говоря,
разработки индуктивного метода.
Условием реформы науки должно быть также очищение разума от заблуждений. Бэкон различает четыре вида
заблуждений, или препятствий, на пути познания — четыре вида   «идолов»   (ложных  образов)   или
призраков. Это — «идолы рода», «идолы пещеры», «идолы площади» и «идолы театра».
«Идолы рода» — препятствия, обусловленные общей для всех людей природой. Человек судит о природе по
аналогии с собственными свойствами. Отсюда возникает телеологическое представление о природе, ошибки,
проистекающие  из несовершенства   человеческих   чувств   под влиянием различных желаний, влечений.
«Идолы пещеры» — ошибки, которые присущи не всему человеческому роду, а только некоторым группам
людей вследствие субъективных предпочтений, симпатий, антипатий ученых: одни больше видят различия
между предметами, другие — их сходства; одни склонны верить в непогрешимый авторитет древности, другие,
наоборот, отдают предпочтение только новому. «Идолы  площади» — препятствия, возникающие вследствие
общения между людьми посредством слов. Во многих случаях значения слов были установлены не на
основе познания сущности предмета; а на основании совершенно случайного впечатления от этого предмета.
«Идолы театра» — препятствия, порождаемые в науке некритически усвоенными, ложными мнениями. «Идолы
театра» не врождены нашему уму, они возникают вследствие подчинения ума ошибочным воззрениям.
Этим Бэкон существенно способствовал формированию философского мышления Нового времени. И хотя
его  эмпиризм  был  исторически  и  гносеологически  ограничен,  а  с  точки  зрения  последующего  развития
познания его можно по многим  направлениям критиковать, он в свое время сыграл весьма положительную
роль.

\subsection{Д.Локк}
Продолжая  тему  развития  эмпиризма,  нельзя  не  упомянуть  представителя  английской  философской  линии
данного направления – Джона Локка. Главное внимание он уделяет проблематике познания. Уже в первой части
его  «Опыта  о  человеческом  разуме»  встречается  идея,  суть  которой  состоит  в  том,  что  предпосылкой
исследования всех разнообразнейших проблем является изучение способностей нашего собственного познания,
т. е. выяснение того, что оно может достичь, каковы его границы, а также каким образом оно получает знания о
внешнем мире.
Для  философских  и  гносеологических  воззрений  Локка  характерным  является  подчеркивание  чувственно
постигаемой эмпирии. Гегель (который, естественно, такой способ философствования оценивал не слишком
высоко)  подчеркивал,  что  «Локк  указал  на  то,  что  общее,  и  мышление  вообще,  покоится  на  чувственно
воспринимаемом сущем,  и  указал, что общее  и  истину  мы  получаем  из опыта».  В  признании приоритета
чувственного познания Локк близок эмпиризму Бэкона.
Локк  не  исключает  в  целом  роль  разума  (как  иногда  упрощенно  представляется),  но  признает  за  ним  в
«достижении истины» еще меньше простора, чем Т. Гоббс. В сущности роль разума он ограничивает лишь,
говоря нынешними терминами, простыми эмпирическими суждениями.
Философию Локка можно характеризовать как учение, которое прямо направлено против рационализма Декарта
(и  не  только  против  Декарта,  но  и  во  многом  против  систем  Спинозы  и  Лейбница).  Локк  отрицает
существование «врожденных идей», которые играли такую важную роль в теории познания Декарта, и концепцию
«врожденных принципов» Лейбница, которые представляли собственно некую естественную потенцию
понимания идей.
Человеческая мысль (душа), согласно Локку, лишена всяких врожденных идей, понятий, принципов либо еще
чего-нибудь  подобного.  Он  считает  душу  чистым  листом  бумаги  (tabula  rasa).  Лишь  опыт  (посредством
чувственного познания) этот чистый лист заполняет письменами.
Локк понимает опыт прежде всего как воздействие предметов окружающего мира на нас, наши чувственные 
органы. Поэтому для него ощущение является основой всякого познания. Однако в соответствии с одним из
своих  основных  тезисов  о  необходимости  изучения  способностей  и  границ  человеческого  познания  он
обращает внимание и на исследование собственно процесса познания, на деятельность мысли (души). Опыт,
который мы приобретаем при этом, он определяет как «внутренний» в отличие от опыта, обретенного при
посредстве восприятия чувственного мира. Идеи, возникшие на основе внешнего опыта (т. е. опосредованные
чувственными восприятиями), он называет чувственными (sensations); идеи, которые берут свое происхождение
из  внутреннего  опыта,  он  определяет  как  возникшие  «рефлексии».  Однако  опыт  —  как  внешний,  так  и
внутренний — непосредственно ведет лишь к возникновению простых (simple) идей. Для того, чтобы наша
мысль (душа) получила общие идеи, необходимо размышление. Размышление не является сущностью души
(мысли), но лишь ее свойством.
Размышление,  в  понимании  Локка,  является  процессом,  в  котором  из  простых  (элементарных)  идей
(полученных на основе внешнего и внутреннего опыта) возникают новые идеи, которые не могут появиться
непосредственно на основе чувств или рефлексии. Сюда относятся такие общие понятия, как пространство,
время и т. д.

Таким образом, мы рассмотрели основные черты эмпиризма философии Нового времени, которые заключаются
в следующем:
\begin{itemize}
\item исключительная значимость и необходимость наблюдений и опыта в обнаружении истины;
\item путем, ведущим к знанию, является наблюдение, анализ, сравнение, эксперимент;
\item исключительно все знания черпаются из опыта, ощущений.
\end{itemize}

\newpage
\section{Трансцендентальная философия Канта: структура познания и априоризм}
Главным философским произведением Канта является «Критика чистого разума». Исходной проблемой для
Канта  является  вопрос  «Как  возможно  чистое  знание?»  Прежде  всего,  это  касается  возможности  чистой
математики и чистого естествознания («чистый» означает «неэмпирический», то есть такой к которому не
примешивается  ощущение).  Указанный  вопрос  Кант  формулировал  в  несколько  неудачных  терминах
различения аналитических и синтетических суждений — «Как возможны синтетические суждения априори?»
Под  «синтетическими»  суждениями  Кант  понимал  суждения  с  приращением  содержания,  по  сравнению  с
содержанием  входящих в  суждение понятий, которые  отличал от  аналитических суждений, раскрывающих
смысл самих понятий. Термин «априори» означает «вне опыта», в противоположность термину «апостериори»
— «из опыта». Кант, вслед за Юмом, соглашается, что если наше познание начинается с опыта, то его связь —
всеобщность и необходимость не из него. Однако, если Юм из этого делает скептический вывод о том, что связь
опыта является всего лишь привычкой, то Кант эту связь относит к необходимой априорной деятельности
сознания. Выявлением этой деятельности сознания в опыте Кант называет трансцендентальным исследованием.
Вот как об этом пишет сам Кант: «Я называю трансцендентальным всякое познание, занимающееся не столько
предметами, сколько видами нашего познания предметов, поскольку это познание должно быть возможным
априори». Кант называет свою философию «критической» в противоположность догматической философии,
которая оставляет нерешенным вопрос возможности знания. Кант, по его словам, совершает Коперниканский
переворот в философии, тем, что первый указывает, что для обоснования возможности знания следует признать,
что не наши познавательные способности должны сообразовываться с миром, а мир должен сообразоваться с
нашими способностями, чтобы вообще могло состояться познание. Иначе говоря, наше сознание не просто
пассивно постигает мир как он есть на самом деле (догматизм), как бы это можно было доказать и обосновать?
Но  скорее,  наоборот,  мир  сообразуется  с  возможностями  нашего  познания,  а  именно:  сознание  является
активным  участником  становления  самого  мира,  данного  нам  в  опыте.  Опыт  по  сути  есть  синтез  того
содержания, материи, которое дается миром (ощущений) и той субъективной формы, в которой эти ощущения
постигаются сознанием. Единое синтетическое целое материи и формы Кант и называет опытом, который по
необходимости становится чем-то только субъективным. Именно поэтому Кант различает мир как он есть сам
по себе (то есть вне деятельности формирования сознания) — вещи-в-себе и мир как он дан в явлении, то есть в
опыте. В опыте выделяется два уровня формообразования (активности) сознания:
\begin{enumerate}
\item Субъективные формы
чувства — пространство и время. В созерцании, чувства (материя) постигаются нами в формах пространства и
времени, и тем самым опыт чувства становится чем-то необходимым и всеобщим. Это чувственный синтез.

\item
Категории рассудка, благодаря которому связываются созерцания. Это рассудочный синтез.
\end{enumerate}
Основой всякого
синтеза  является,  согласно  Канту,  самосознание  —  единство  апперцепции  (Лейбницевский  термин).  В
«Критике» много места уделяется тому, как понятия рассудка подводятся под представления. Здесь решающую
роль играет воображение и рассудочный категориальный схематизм. Наконец, описав эмпирическое применение
рассудка, Кант задается вопросом возможности чистого применения рассудка, которое он называет разумом.
Здесь возникает новый вопрос: «Как возможна метафизика?». В результате исследования чистого разума Кант
доказывает, что разум не может иметь конститутивного значения, то есть основывать на самом себе чистое
знание, которое должно было бы составить чистую метафизику, поскольку «запутывается» в паралогизмах и
неразрешимых антиномиях (противоречиях, каждое из утверждений которого одинаково обосновано), но только
регулятивное значение — как систему принципов, которым должно удовлетворять всякое знание. Собственно,
всякая будущая метафизика, согласно Канту, должна принимать во внимание его выводы.

Кант выделяет следующие категории рассудка:
\begin{enumerate}
\item Категории количества
    \begin{enumerate}
        \item Единство
        \item Множество
        \item Цельность
    \end{enumerate}
    
\item Категории качества
    \begin{enumerate}
        \item Реальность
        \item Отрицание
        \item Ограничение
    \end{enumerate}
\item Отношения
    \begin{enumerate}
        \item Субстанция и принадлежность
        \item Причина и следствие
        \item Взаимодействия
    \end{enumerate}
\item Категории модальности
    \begin{enumerate}
        \item Возможность и невозможность
        \item Существование и несуществование
        \item Предопределённость и случайность
    \end{enumerate}
\end{enumerate}

Знание даётся путём синтеза категорий и наблюдений. Кант впервые показал, что наше знание о мире не
является  пассивным  отображением  реальности,  а  является  результатом  активной  творческой  деятельности
сознания.

\newpage
\section{Этическое учение Канта}
Этическое учение Канта изложено в «Критике практического разума». Этика Канта основана на принципе «как
если бы». Бога и свободу невозможно доказать, но надо жить как если бы они были. Практический разум — это
совесть,  руководящая  нашими  поступками  посредством  максим  (ситуативные  мотивы)  и  императивов
(общезначимые правила). Императивы бывают двух видов: категорические и гипотетические. Категорический
императив — требует соблюдения долга. Гипотетический императив — требует, чтобы наши действия были
полезны. Существует две формулировки категорического императива:
\begin{itemize}
\item «Поступай всегда так, чтобы максима (принцип) твоего поведения могла стать всеобщим законом (поступай
так, как ты бы мог пожелать, чтобы поступали все)»;
\item «Относись к человечеству в своем лице (так же, как и в лице всякого другого) всегда только как к цели и
никогда – как к средству».
\end{itemize}
В этическом учении человек рассматривается с двух точек зрения:
\begin{itemize}
\item Человек как явление;
\item Человек как вещь в себе.
\end{itemize}
Поведение  первого детерминировано исключительно  внешними факторами  и  подчиняется  гипотетическому
императиву. Второй — категорическому императиву — высшему априорному моральному принципу. Таким
образом, поведение может определяться практическими интересами и моральными принципами. Возникает два
тенденции: стремление к счастью (удовлетворению некоторых материальных потребностей) и стремление к
добродетели. Эти стремления могут противоречить друг другу и возникает «антиномия практического разума».

\newpage
\section{Система объективного идеализма и диалектический метод Гегеля}
Высшей  ступени  своего  развития  диалектика  в  идеалистической  форме  достигла  в  философии  Гегеля
(1770-1831), который был великим представителем объективного идеализма. Гегелевская система объективного
идеализма состоит из трех основных частей.
В первой части своей системы - в "Науке логики" - Гегель изображает мировой дух (называемый им здесь
"абсолютной идеей") таким, каким он был до возникновения природы, т.е. признает дух первичным.
Идеалистическое учение о природе изложено им во второй части системы - в "Философии природы". Природу
Гегель как идеалист считает вторичной, производной от абсолютной идеи.
Гегелевская идеалистическая теория общественной жизни составляет третью часть его системы - "Философию
духа". Здесь абсолютная идея становиться по Гегелю "абсолютным духом".
Таким образом, система взглядов Гегеля носила ярко выраженный идеалистический характер. Существенная
позитивная особенность идеалистической философии Гегеля состоит в том, что абсолютная идея, абсолютный
дух  рассматривается  им  в  движении,  в  развитии.  Учение  Гегеля  о  развитии  составляет  ядро  гегелевской
идеалистической диалектики и целиком направлено против метафизики. Особенное значение в диалектическом
методе  Гегеля  имели  три  принципа  развития,  понимаемые  им  как  движение  понятий,  а  именно:  переход
количества в качество, противоречие как источник развития и отрицание отрицания.
В этих трех принципах, хотя и в идеалистической форме, Гегель вскрыл всеобщие законы развития. Впервые в
истории философии Гегель учил, что источником развития являются противоречия, присущие явлениям. Мысль
Гегеля о внутренней противоречивости развития была драгоценным приобретением философии.
Выступая  против  метафизиков,  рассматривавших  понятия  вне  связи  друг  с  другом,  абсолютизировавших 
анализ, Гегель выдвинул диалектическое положение о том. что понятия взаимосвязаны между собой. Таким
образом, Гегель обогатил философию разработкой диалектического метода. В его идеалистической диалектике
заключалось глубокое рациональное отражение. Рассматривая основные понятия философии и естествознания,
он в известной мере диалектически подходил к истолкованию природы, хотя в своей системе он и отрицал
развитие природы во времени.
В  Гегелевской  философии  существует  противоречие  между  метафизической  системой  и  диалектическим
методом. Метафизическая система отрицает развитие в природе, а его диалектический метод признает развитие,
смену одних понятий другими, их взаимодействие и движение от простого к сложному.
Развитие общественной жизни Гегель видел лишь в прошлом. Он считал, что история общества завершится
конституционной  сословной  прусской  монархией,  а  венцом  всей  истории  философии  он  объявил  свою
идеалистическую систему объективного идеализма.
Так система Гегеля возобладала над его методом. Однако в гегелевской идеалистической теории общества
содержится много ценных диалектических идей о развитии общественной жизни. Гегель высказал мысль о
закономерностях  общественного  прогресса.  Гражданское  общество,  государство,  правовые,  эстетические,
религиозные,  философские  идеи,  согласно  гегелевской  диалектике,  прошли  длинный  путь  исторического
развития.
Если  идеалистическая  система  взглядов  Гегеля носила  консервативный  характер, то  диалектический метод
Гегеля  имел  огромное  положительное  значение  для  дальнейшего  развития  философии,  явился  одним  из
теоретических источников диалектико-материалистической философии.
Таким образом, историческая роль философских учений немецких философов конца 18 - начала 19 века, в
особенности  Гегеля,  состояла  в  развитии  этими  выдающимися  мыслителями  диалектического  метода.  Но
идеалистические  умозрительные  системы  немецких  философов  в  интересах  дальнейшего  развития
философской мысли требовалось преодолеть, удержав то ценное, что в них содержалось. Это в значительной
степени было достигнуто Л. Фейербахом (1804-1872).

\newpage
\section{Иррационализм Шопенгауэра. Мир как представление и воля}
Для Артура Шопенгауэра, который наряду с Кьеркегором является основоположником иррационализма XIX
века, Гегель также был живым и едва ли не главным врагом.
Система Шопенгауэра изложена в 4 книгах его основного труда "Мир как воля и представление".
У Шопенгауэра основой и животворящим началом всего является не познавательная способность и активность
человека, а в о л я как слепая, бессознательная жизненная сила. Тем самым в "человеке разумном", в homo
sapiens, разум перестал считаться его родовой сущностью; ею становилась неразумная воля, а разум начинал
играть второстепенную, служебную роль.
Жизнь, жизненная сила, волевое напряжение - вот что вышло на авансцену, оттеснив интеллект, рацио на
задний план.
Наука, по Шопенгауэру, никогда не может обрести конечной цели, однако есть сфера, которая рассматривает
"единственную действительную сущность мира" - это ИСКУССТВО. Шопенгауэр говорит, что"обыкновенный
человек, этот фабричный товар природы" не способен на незаинтересованное созерцание так же, как и ученый,
и только гений способен на это. Искусство есть создание гения, и гений возможен только в искусстве. Искусство
воспроизводит постигнутые чистым созерцанием вечные идеи.
Согласно Шопенгауэру, высшее из искусств - это МУЗЫКА, имеющая своей целью уже не воспроизведение
идей, а непосредственное отражение самой воли.
Шопенгауэр  является  прямым  предшественником  философии  жизни  -  иррационалистического  течения
философии конца XIX - начала XX века, основными представителями которого являются Ницше, Дильтей,
Зиммель, Шпенглер, Бергсон и др.
Он не очень любил людей, трудно сходился, был самолюбив. Полная не признанность при жизни (чем-то похож
на Гераклита, хотя в горы Шопенгауэр не пошел ?). Получил большое наследство, занялся «мыслительством».
Последние 10 лет (1845-1846) к Шопенгауэру начинает приходить знаменитость. Произошло следующее: 40-е
годы – годы революций (Французской и Немецкой), которые закончились поражением. Народ приуныл, а тут
третье  издание  книги  Шопенгауэра.  Прямо  в  унисон  настоящему  времени,  кризиса  и  безысходности,
пессимизма. Шопенгауэр говорит, что это естественное состояние жизни. «Великий пессимист», «Философ
мировой скорби».  Мировоззрение Шопенгауэра формируется из трех источников (трех мировоззрений):
Грек Платон (с его миром идей и миром вещей, которая проникнута духом аристократизма, презрения к толпе)
Философия Канта
Философия древней Индии (философия Упанишады)
Общая  их  идея:  утверждается,  что  разум  (наше  мышление)  постигает  только  поверхность  (часть)  бытия.
Подлинные причиныпроисходящего всегда скрыты от нас. Есть иная (высшая) реальность, которая определяет
нашу жизнь (поэтому мы и терпим войны, неудачи и т.д.) Мы никогда не сможем вычислить и усмотреть всех
последствий, так как есть то, что нельзя сосчитать ложкой, …, разумом.
Фихте, Гегель, Шеллинг – три прославленных софиста (~болтуна) послекантовского учения, которые извратили
сущность его учения. Эти три глупца, помешавшиеся на разуме, завели философию в тупик. Но они остаются
правыми, ибо они призваны к философии министерством, а я – природой.
Кант ограничил познавательные возможности человека только разумом, только рациональностью. /* «Кто ясно
мыслит, тот ясно излагает» */. Такая философия (такой подход) лишает нас понимания жизни, так как она
(жизнь) не только рационально познаваема.
Шопенгауэр утверждает: существует иная способность, иной путь познания, интуиция.
Интуиция – реальная способность познавательной человеческой деятельности. /* Спиноза: везде логика, но
основные  положения  я  беру  в  виде  неких  аксиом,  очевидности.  Форма  их  данности  –  интуиция.  Она
нерациональна. */
Что же позволяет интуиция увидеть за завесой? За нашими представлениями (интуитивно открывается, что
мир) есть то, что в интуитивной данности он мне дан, как воля. «Вещь в себе» Канта и есть воля.
У А.Шопенгауэра термин воля относится, прежде всего, не к человеку и его психологии (сознанию) /* У нас
воля – психологическая характеристика человека */. Под волей он понимает некую основу нашего бытия.
Мировая воля: рационально мы о ней ничего не знаем. 
«Мир есть проявление мировой воли, или ее действие (динамика)». Мировая воля дана нам как динамика
(вечное движение и изменение) мироздания. О чем бы мы не стали думать, много у вещей не совпадающих
свойств, но у всего в мире есть одно общее свойство – они прибывают в вечном движении. Эта динамика есть
мировая воля в своем проявлении. Это изменение называется мировой волей.
У человека воля впервые приобретает осознанную форму.
Три ступени воли:
\begin{enumerate}
\item Силы природы (тяготение, текучесть, электричество, магнетизм)
\item Силы жизни (в жизни мировая воля проявляется как воля к жизни). Она самопротиворечива, так как пожирает
сама себя и в разных видах служит своею собственной пищей. /* В камне есть сила природы, но нет силы жизни
*/
\item Человек – это есть  тоже воля к жизни (самосохранение, самовоспроизведение),– но она уже всеядна и
пользуется (пожирает) уже все. Человек есть осознанная воля к жизни.
\end{enumerate}
Вершина всего – разум или интеллект человека (которая тоже есть воля).
?: Похоже на теорию эволюции???
Шопенгауэр: Фундаментальное свойство мировой воли в том, что она бесцельна => она бессмысленна!
«Воля есть бесконечное стремление самоуничтожения и самовоспроизведения вновь».
Человек с его интеллектом – это крайний случай в движении мировой воли. Не было у природы задачи создать
человека, просто в хаосе воли сошлись 1000 причин.
В мировоззрении Шопенгауэра не предполагается никакого высшего разума, бога, плана. Человек не продукт
эволюции, так как все случайно. Тот факт, что, планируя что-либо, мы, зачастую, терпим неудачи, служит
подтверждением отсутствия какого-то ни было направления.
Человек выше ее… Но он не выражает ее сущности, ее идеи.
Шопенгауэр: Что такое мир, в котором мы живем? Во-первых, мир есть мое представление, образованное по
законам моего сознания; и во-вторых, мир, на самом деле, есть беспричинная, бессознательная воля.
Что указывает человеку на существование мировой воли? Шопенгауэр приводит четыре аргумента:
\begin{enumerate}
\item Ссылается на Шеллинга (один из бездарных пачкунов). Он говорил о бессознательной основе бытия, а это
рационалист Шеллинг. 
\item Логический: сознание  – есть продукт естественного развития  организма,  оно  формируется жизненными
потребностями, но если это так, то тогда … но не из этих потребностей, не из сознания. Сформированного ими
нельзя объяснить стремление человека к духовному творчеству, ибо если  сознание есть приспособительный
механизм, то как объяснить избыточные способности? Здесь проявляется мировая воля (творчество из хаоса). 
\item Наблюдение  жизни.  Все  мы  стремимся  в  жизни  к  определенным  целям.  Но  все  планы  и  проекты
заканчиваются неудачей. Проекты разные, а результат отрицателен. => основа жизни не рациональна. 
\item Существование музыки. Музыка есть наиболее непосредственное выражение мировой воли.
\end{enumerate}
Вопрос: В чем сущность и положение человека? 
Шопенгауэр прибегает к метафоре дерева: воля – это корни; наша жизнь (с нашим интеллектом) – это то, что
над землей, ствол и крона. Мы, как дерево, не подозреваем о своих корнях. Участие воли или ее постоянное
присутствие, в конечном счете, деформирует и искажает работу интеллекта. => (в итоге) наш разум (интеллект)
дают временный успех. Интеллект слаб и беспомощен на фоне этой стихии – мировой воли.
Наше научное познание носит частичный и относительный характер, такое познание все равно, в конечном
итоге, не даст власти над обстоятельствами. Человек одновременно несчастен и безнравственен. Он несчастен,
ибо не понимает, не видит, цели и смысла своих поступков; одновременно он безнравственен, ибо увеличивает
страдание в мире, так как, в конечном счете, каждый из нас подчинен слепому безумию мировой воли.
«Мы здесь никому ненужные. Мы появились случайно, и ничего у нас не получится».
Позиция Шопенгауэра: Раз мы появились, то поскольку мы люди, постольку мы должны стремиться быть
людьми.  Все  свои  способности  направить  хоть  и  на  бесцельное,  но  самоутверждение  в  этом  мире.  Есть
культурные механизмы, которые помогают существовать в этом безумии. Надо находить тихие гавани.
\begin{itemize}
\item 1 путь – эстетический. Это путь творчества. Способ, который изобрело человечество – гавань, которую мы 
строим сами. Это дом защищает нас на некоторое время от мирового хаоса. Необязательно быть художником,
так как тот, кто воспринимает произведения искусства, тот тоже пребывает в акте творчества.
\item 2 путь – этический. Это путь, на котором человек стремиться максимально уединиться от мира, в котором много
проблем. В частности, это путь религиозного отшельника или одинокого философа. На таком пути вся суета
отдаляется,  но  он  (этот  путь)  для  избранных.  Для  большинства  жизнь  навсегда  остается  наполненной
неурядицами и т.п. А впереди – только смерть.
\end{itemize}
Философия и религия создают иллюзию жизненных ценностей и иллюзию смысла жизни, а потому,– спасибо
им. Философия и религия облегчают наше существование.
Философия ничего нам не разъясняет. Цель любой философии сделать жизнь выносимой. Философия есть
стремление познать  сквозь завесу  представления то,  что не есть  представление,  но ведь  не философия,  в
конечном счете, заставляет нас жить в любой ситуации?! Жить нас заставляет Воля, проявляющая себя как воля
к жизни. 
Воля – есть первая реальность или наше исходное состояние.
Шопенгауэр ярко проявляет свой иррационализм. «Человеческая жизнь начинается с желания и стремления. /*
Основатель  рационализма  нового  времени  –  Декарт:  «Я  мыслю,  значит  существую».  Иррационализм
Шопенгауэра: «Я желаю, хочу, стремлюсь. Значит, существую». */
А.Шопенгауэр – философ мировой скорби. Но это не унылая, не безысходная скорбь, скорее она напоминает
позицию античных стоиков.
Люди  страдают.  Страдание  порождают  вражду  и  злобу.  Чем  больше  люди  страдают,  тем  больше  злобы.
Противопоставить этому можно только одно: наше стремление к духовности и духовной культуре. 
/* Культура – это средство обуздания злобно-эгоистического начала в человеке. */
Парадокс: Шопенгауэр – классик современного европейского иррационализма, но, с другой стороны, в своих
сочинениях он выступает как чистый рационалист. (Его трактаты строги, без поэтических вольностей. Он
понятен.) /* Популярным в России Шопенгауэр становится в конце XIX-го, в начале XX-го века. */
Если мировая воля не имеет никакой логики, то и человеческая история один из случайных векторов. «История
– смутный кошмар человечества» => такой науки, как истории быть не может, ведь не может быть истории
хаоса! То, что читают в Университете есть иллюзия истории, набор сказок.
А.Шопенгауэр – великий скептик, элитарист. Не питает никаких иллюзий на счет человека. Государство –
жесткий намордник, который необходимо надевать, чтобы привести человечество к определенному порядку.
Государство должно быть наиболее жестким.

\newpage
\section{Философия Ницше: нигилизм и проблема переоценки ценностей}
Ницше  выступает  как  "радикальный  нигилист"  и  требует  кардинальной  переоценки  ценностей  культуры,
философии,  религии.  "Европейский  нигилизм"  Ницше  сводит  к  некоторым  основным  постулатам,
провозгласить которые с резкостью, без страха и лицемерия считает своим долгом. Эти тезисы: ничто больше не
является истинным; бог умер; нет морали; все позволено. Надо точно понять Ницше — он стремится, по его
собственным словам, заниматься не сетованиями и моралистическими пожеланиями, а "описывать грядущее",
которое не может не наступить. По его глубочайшему убеждению (которое, к сожалению, никак не опровергнет
история заканчивающегося XX в.), нигилизм станет реальностью по крайней мере для последующих двух
столетий.  Европейская  культура,  продолжает  Ницше  свое  рассуждение,  издавна  развивается  под  игом
напряжения, которое растет от столетия к столетию, приближая человечество и мир к катастрофе. Себя Ницше
объявляет "первым нигилистом Европы", "философом нигилизма и посланцем инстинкта" в том смысле, что он
изображает  нигилизм  как  неизбежность,  зовет  понять  его  суть.  Нигилизм  может  стать  симптомом
окончательного упадка воли, направленной против бытия. Это "нигилизм слабых". "Что дурно? — Все, что
вытекает  из  слабости"  ("Антихрист".  Афоризм  2).  А  "нигилизм  сильных"  может  и  должен  стать  знаком
выздоровления, пробуждения новой воли к бытию. Без ложной скромности Ницше заявляет, что по отношению
к "знакам упадка и начала" он обладает особым чутьем, большим, чем какой-либо другой человек. Я могу,
говорит о себе философ, быть для других людей учителем, ибо знаю оба полюса противоречия жизни; я и есть
само это противоречие... А то, что его философия, не понятая эпохой, принадлежит к числу "несвоевременных
размышлений", никого  не должно  смущать, ибо нет ничего  более  своевременного,  чем умение мыслителя
преодолеть свое время, диктат его ценностей. К переоценке ценностей Ницше звал своих читателей уже в
ранних работах. Так, "Человеческое, слишком человеческое" он начинает в Искренней, исповедальной манере.
Ницше рассказывает о своем духовном становлении, о страстном увлечении Вагнером и Шопенгауэром и столь
же страстном отказе от их (и других мыслителей) идей и доктрин. А это порождает вопрос, который Ницше
обращает к себе и к своим читателям: "...сколько лживости мне еще нужно, чтобы сызнова позволить себе
роскошь  моей  правдивости?".  В  чем  же  удел  мыслителя,  отказавшегося  от  лжи,  фальши  устаревших,
догматизированных  воззрений?  Стать  из-за  переоценки  ценностей  унылым,  лишенным  чувства  юмора
философом и морализатором-одиночкой? Нет, отвечает Ницше. Везде рождаются, хотя и в великих муках и
постепенно, "свободные умы" и обновленные души. Они движутся навстречу друг другу. "Какие узы крепче
всего?  Какие  путы  почти  неразрывны?  У  людей  высокой  избранной  породы  то  будут  обязанности  —
благоговение,  которое  присуще  юности,  и  нежность  ко  всему,  издревле  почитаемому  и  достойному, 
благодарность  почве,  из  которой  они  выросли,  руке,  которая  их  вела,  храму,  в  котором  они  научились
поклоняться...". Но потом приходит тяготение к "великому разрыву", выраженному в виде тревожного вопроса:
"...Нельзя ли перевернуть все ценности? и, может быть, добро есть зло? а Бог — выдумка и ухищрение дьявола?
И может быть, в последней основе все ложно? И если мы обмануты, то не мы ли, в силу того же самого, и
обманщики?". Намеченная здесь идея переоценки ценностей духовной аристократией нового типа развита в
последующих произведениях, особенно в "Заратустре". 

\newpage
\section{Принципы марксистской философии}
Сложность и особый интерес к философии К. Маркса определены тем обстоятельством, что существовал и
существует марксизм - массовая идеология, сыгравшая огромную роль в Х1Х-ХХ вв. Как и всякая другая, эта
идеология вобрала в себя значительные идейные фрагменты философско-мировоззренческого порядка, причем
их авторство с известной долей справедливости обычно приписывали Марксу. Найти в этом переплетении чисто
философское содержание, различить Маркса-философа и Маркса-идеолога было и остается непростой задачей.
Философское  творчество  Маркса,  независимо  ни  от  каких  обстоятельств,  обладает  историко-философской
ценностью и тем самым требует изучения в русле истории немецкой философской мысли XIX в.
В последние годы в нашей стране от официально неумеренного восхищения творчеством Маркса перешли к
резко негативным оценкам, что объясняется, конечно, общей политико-идеологической ситуацией. Историку
философии, однако, не пристало разделять и пафос былых восторгов, и энтузиазм продолжающихся проклятий.
Авторитет  Маркса-философа  основан  на  несомненности  оригинального  вклада  в  гегелевское  движение.  В
области философии Маркс сам всегда считал себя учеником и последователем Гегеля, претендуя лишь на
относительную самостоятельность, и это тот именно случай, когда нужно прислушатьс к авторской самооценке.
Принадлежность Маркса-философа к гегелевской школе не вызывает сомнений. С университетских лет Маркс
близко контактировал с наиболее видными гегельянцами - Б. Бауэром, Ф. Кеппеном, затем с А. Руге, М. Гессом,
Ф. Энгельсом, состоял в переписке с Л. Фейербахом.
Маркс участвовал в гегельянских периодических изданиях, был в 1842-1843 гг. редактором "Новой Рейнской
газеты", преимущественно младогегельянского органа. Вместе с другими членами школы Маркс постепенно
перешел от увлечения "философией самосознания" Б. Бауэра к гуманистической антропологии Л. Фейербаха,
когда в школе шла "смена лидера".
Некоторые тексты молодого Маркса написаны в сотрудничестве с Б. Бауэром, А. Руге, Ф. Энгельсом, М. Гессом.
В то же время Маркс-гегельянец проявил высокую степень самостоятельности, что в итоге только обогатило
достижения школы. Вступив в движение позже других, Маркс смог более критически подойти к гегелевской
традиции.
Он по-своему оригинально реализовал некоторые потенции антропологии Л. Фейербаха. Так, Маркс принял
фейербаховское  толкование  принципа  тождества  бытия  и  мышления,  но  конкретизировал  абстрактный
философский принцип первичности бытия, обратив внимание на социальное бытие, на историю человечества.
Своеобразным развитием фейербаховской концепции религии как иллюзорного самосознания стало у Маркса
истолкование идеологии как ложного, превращенного сознания, выражающего реальность в "перевернутом"
виде.
Маркс,  сочетая  гегельянский  критицизм  с  его  гегелевским  прообразом,  сформировал  оригинальный  метод
интерпретации социальной истории. Называя этот метод диалектическим, Маркс отдавал пальму первенства
Гегелю, оставляяза собой приоритет лишь в приспособлении метода для своих, по преимуществу социально-философских  и  социально-экономических  исследовательских  задач.  Наиболее  оригинальны,  влиятельны  и
интересны идеи Маркса, развитые в сфере философии истории и философской антропологии. 
Черты  характера  Маркса:  эгоизм,  самовлюбленность,  крайняя  властность,  нетерпимость  к  другим  точкам
зрения. Раздражение от религии; земная власть аристократов поддерживается религией => религия – главный
враг.
Отчужденный труд М рассм в 4 аспектах.
1.Рабоч использует матер, взятые у природы и получает в итоге нужные для жизни предметы, продукты труда.
Ни исх материал, ни прод ему не принадлежат - они ему чужие. Чем больше р. работает, тем больше мир предм,
не принадл ему. Природа делается для раб только средством труда, а предметы, кот создаются в производстве -средством жизни, физ существования. Раб полностью от них зависит.
2. Процесс труда для р принудителен. Но такой труд - это не удовлетвор потребности в труде, а только средство
для удовлетвор др потребностей. Только вне труда р. распоряжается собой - т.е. свободен. Т.о он свободен
только осущ. жизненные функции, общие у чел с животными. А труд - форма деят, специф для чел, для раб
представляется унижением в себе человека.
3. Труд подневольный отнимает у чел его "родовую" жизнь. Род челов. живет в природе. Жизнь чел неразрывно
связана с прир. Эта связь - деятельный контакт с прир, в кот главное - труд, производство: "...производственная
жизнь и есть родовая жизнь". Но для раб труд - лишь средство для поддержания собственной жизни, а не рода.
Р относится к прир и производству не как своб человек, а как рабочий, т.е отчужденно. Это и значит,что у раб
отобраны и родовая жизнь и чел сущность.
4.Подневольный труд порождает отчежд между людьми. Раб чужды друг другу, поскольку они конкурируют за
возм трудиться.
Не только р. но и все люди являются отчужденными. Отнош между людьми тоже отчужденные и различия
только в видах и уровнях отч. М указывает на сущ первичных и вторичных уровней отч. Почему же чел
становится отчужденным?
Отч труд равнозначен сущ частной собственности. Ч собств - основа экон жизни. На частнособственической
экономике держится вся история. Это значит, что эконом история - ключ к пониманию челов жизни как таковой.
"Религия, семья, гос-во, право, мораль, наука, искусство... суть лишь особые виды производства и подчиняются
его всеобщему закону". Жизнь людей в усл отчуждения калечит их, делает "частичными индивидами" или
неразвитыми, недочелов существами. "Чатсная собств сделала нас настолько глупыми и односторонними, что
какой-нибудь предет явл нашим лишь когда мы обладаем им... когда мы им непосредственно владеем, едим его,
пьем - употребляем... Поэтому на место всех физ и духовных чувств стало простое отчуждение всех этих чувств
- чувство обладания".
Устранение  отчуждения.  Универсальный  человек.  Процесс,  обратный  отчеждению,  -  присвоение  чел
собственной подлинной сущности. М связывает его с общ преобразованиями, с освобождением кот в основе
имеет уничтожение отчужд труда. Что будет , если чел начнет производить как человек, т.е. не подневольно. В
этом случае труд станет средством саморазвития человека, в реализацию человеком своих лучших сторон.
Характеристика  присвоения  челов  собственной  сущности,  или  превращ  труда  из  принуд  в  человеческий
рассматр М по тем же параметрам, что и процесс отчуждения: 1. по присвоению предм труда и его результата 2.
по  освобождению  самой  деятельности  3.присв  человеком  труда  общей  родовой  сущност  4.гармонизации
отношений между людбми.
Здесь  М  создает  грандиозную  по  своему  пафосу  картину  челов,  живущего  в  единстве  с  природой,
преобразующего прир в соответствии с ее законами. Гармония с внешней прир осущ в деятельности, в кот
человек реализует свои цели не по законам утилитарной пользы, а по зак красоты. Внутренняя прир чел такжк
преобразуется.  Вместо  отчежденного  недочеловека  появляется  чеовек,  само  природное  развитие  есть
гармоничный  р-т  всей  истории  чел  общества.  В  человеке  начнут  реализовываться  способности,  пока  еще
реализующиеся не у всех (музык ухо, художественно развитый глаз). - творческие способности.
Универсально развитый, жив в единстве и гармонии с внешней и внутренней природой чел - таков идеальный
фил образ, рисующийся М в качестве ядра коммунист идеала. Уничтожение ч собственности необходимо, но
недостаточно для присвоения людьми челов сущности.

\newpage
\section{Основные принципы и эволюция позитивизма}
Позитивизм  (лат.  positivus  —  положительный)  в  качестве  главной  проблемы  рассматривает  вопрос  о
взаимоотношении  философии  и  науки.  Главный  тезис  позитивизма  состоит  в  том,  что  подлинное
(положительное) знание о действительности может быть получено только лишь конкретными, специальными
науками.
Первая историческая форма позитивизма возникла в 30-40 г. XIX века как антитеза традиционной метафизике в
смысле философского учения о началах всего сущего, о всеобщих принципах бытия, знание о которых не может
быть дано в непосредственном чувственном опыте. Основателем позитивистской философии является Огюст
Конт (1798-1857), французский философ и социолог, который продолжил некоторые традиции Просвещения,
высказывал убеждение в способности науки к бесконечному развитию, придерживался классификации наук,
разработанной энциклопедистами.
Кант утверждал, что всякие попытки приспособить «метафизическую» проблематику к науке обречены на
провал, ибо наука не нуждается в какой-либо философии, а должна опираться на себя. «Новая философия»,
которая должна решительно порвать со старой, метафизической («революция в философии») своей главной
задачей должна считать обобщение научных данных, полученных в частных, специальных науках.
Вторая историческая форма позитивизма (рубеж XIX-XX вв.) связана с именами немецкого философа Рихарда
Авенариуса (1843-1896) и австрийского физика и философа Эрнста Маха (1838-1916). Основные течения —
махизм и эмпириокритицизм. Махисты отказывались от изучения внешнего источника знания в противовес
кантовс-кой идеи «вещи в себе» и тем самым возрождали традиции Беркли и Юма. Главную задачу философии
видели не в обобщении данных частных наук (Конт), а в создании теории научного познания. Рассматривали
научные  понятия  в  качестве  знака  (теория  иероглифов)  для  экономного  описания  элементов  опыта  —
ощущений.
В 10-20 гг. XX века появляется третья форма позитивизма — неопозитивизм или аналитическая философия,
имеющая несколько направлений.
Логический позитивизм или логический эмпиризм представлен именами Мори-ца Шлика (1882-1936), Рудольфа
Карнапа  (1891-1970)  и  других.  В  центре  внимания  проблема  эмпирической  осмысленности  научных
утверждений.  Философия,  утверждают  логические  позитивисты,  не  является  ни  теорией  познания,  ни
содержательной наукой о какой-либо реальности. Философия — это род деятельности по анализу естественных
и  искусственных  языков.  Логический  позитивизм  основывается  на  принципе  верификации  (лат.  verus  —
истинный; facere — делать), который означает эмпирическое подтверждение теоретических положений науки
путем  сопоставления  их  с  наблюдаемыми  объектами,  чувственными  данными,  экспериментом.  Научные
утверждения,  не  подтвержденные  опытом,  не  имеют  познавательного  значения,  являются  некорректными. 
Суждение о факте называется протоколом или протокольным предложением. Ограниченность верификации
впоследствии выявилась в том, что универсальные законы науки не сводимы к совокупности протокольных
предложений. Сам принцип проверяемости также не мог быть исчерпаем простой суммой Какого-либо опыта.
Поэтому сторонники лингвистического анализа—другого влиятельного направления неопозитивизма Джордж
Эдуард Мур (1873-1958) и Людвиг Витгенштейн (1889-1951), принципиально отказались от верификационной
теории значения и некоторых других тезисов.
Четвертая  форма  позитивизма  —  постпозитивизм  характеризуется  отходом  от  многих  принципиальных
положений  позитивизма.  Подобная  эволюция  характерна  для  творчества  Карла  Поппера  (1902-1988),
пришедшего к выводу, что философские проблемы не сводятся к анализу языка. Главную задачу философии он
видел в проблеме демаркации— разграничении научного знания от ненаучного. Метод демаркации основан на
принципе фальсификации, т.е. принципиальной опровержимости любого утверждения, относящегося к науке.
Если утверждение, концепция или теория не могут быть опровергнуты, то они относятся не к науке, а к
религии. Рост научного знания заключается в выдвижении смелых гипотез и их опровержении.

\newpage
\section{Особенности развития и характерные черты русской философии}
Фил. мысль в Р. формировалась под влиянием общемировой фил. Однако специф Р фил во многом складывалась
под влиянием социально културных процессов, происходивших на Руси. Христианизация Р. сыграла огромную
роль  в  становлении  рус.  фил  мысли.  Поиски  рус  ф-й  мысли  (16-18в)  проходили  в  противоборстве  2-х
тенденций: 1)акцентировала вним-е на самобыт с неповторимым своеобраз-м рус дух жизни 2) выражала стрем-е вписать Р в процесс раз-я европ культуры, представит считали что поскольку Р встала на путь культ раз-я
позже др стран то она должна уч-ся у запада и пройти тот же истор путь. Своеобр направлением в Р фил
явились воззрения славянофилов Хомякова и Киреевского.  В центре их внимания судьба Р и ее роль в мир
истор процессе. В самобытности истор прошлого они видели залог всечеловеч. призвания Р., тем более, что по
их мнению, зап культура уже завершила круг своего развития и клонится к упадку, что выраж в порожденном
ею  чувстве  обманутой  надежды  и  безотрадной  пустоты.  Славяноф.  развивали  основанное  на  религиозных
представлениях учение о чел и обществе. Достижение  целостности чел и связанное с этим обновление общ
жизни они видели в идее общины, дух основа которой - церковь. 
Основные черты и проблемы русской философии 19 - начала 20 в.
- Начало: Достоевский, Толстой, Соловьев.
Эти писатели выдвинули и сформулировали идею о связи прогресса с христианской нравственностью. Только
то может быть принято обществом что прошло чистилище нравственности.
- Предельный гуманизм.
Знаменитое непротивление злу насилием. / Толстой /
Все достижения прогресса за слезу ребенка. / Достоевский/ 
- Вето на любой прогресс, если он связан с насилием над человеком.
Насилие понимается как превращение человека в производственную машину за счет удушения уникальной,
божественной души.
- Литература: жанр антиутопии.
Будущее основано на любви и искусстве и потому прекрасно. Пример: сны Веры Павловны. Чернышевский
"Что делать"
- Сверхнационалистический характер философствования. 
Включение всего человечества в ожидаемый прогресс прекрасного будущего.
- Постижение сущего дается лишь цельной жизнью духа, лишь в полноте жизни.
Полностью православно-христианский подход к человеку и его месту в природе. Есть Бог, есть человек и
божественное  предназначение  последнего.  Выполнить  это  предназначение  его  задача.  Задача  решается
обращением к Богу, вере, христианской нравственности.
- Антипод русской философии - европейский рационализм - Гегель.
- Русский иррационализм (вера, православие) VS западный (воля) -> деление на славянофилов и западников.
-  Славянофилы  (Киреевский,  Хомяков)  -  самобытный  путь  развития  России.  Земельная  община  и  артель.
Западные приобретения - скорее вред, так как оплачены потерей целостности человеческой личности.
- Западники (Герцен, Грановский, Боткин) - Россия с петровских времен привязана к западу. Прогресс не может
обойти Россию, прогресс поможет нам выбраться из отсталости.
- Примат христианского откровения, веры над рациональным знанием.
- Философия всеединства (Соловьев). Теория цельного знания. Сверхрационализм. Цельность - характеристика
и свойство человеческой души, отличающая человека от животных. 

\newpage
\section{Диалог славянофилов и западников в русской философии и культуре XIX века}
Первыми представителями «органической русской философии» были западники и славянофилы.
К западникам относятся: П.Л. Чаадаев, А.Л. Герцен, Т.М. Грановский, Н.Г. Чернышевский, В.П. Боткин и др.
Основная  идея  западников  заключается  в  признании  европейской  культуры  последним  словом  мировой
цивилизации, необходимости полного культурного воссоединения с Западом, использования опыта его развития 
для процветания России.
Особое место в русской философии XIX в. вообще, а в западничестве в частности занимает П.Я. Чаадаев,
мыслитель,  сделавший  первый  шаг  в  самостоятельном  философском  творчестве  в  России  XIX  столетия,
положивший начало идеям западников. Свое философское миропонимание он излагает в «Философических
письмах» и в работе «Апология сумасшедшего».
По-своему  понимал  Чаадаев  и  вопрос  о  сближении  России  и  Запада.  Он  видел  в  этом  сближении  не
механическое  заимствование  западноевропейского  опыта,  а  объединение  на  общей  христианской  основе,
требующей реформации, обновления православия. Это обновление Чаадаев видел не в подчинении православия
католицизму,  а  именно  в  обновлении,  освобождении  от  застывших  догм  и  придании  религиозной  вере
жизненности и активности, чтобы она могла способствовать обновлению всех сторон и форм жизни. Эта идея
Чаадаева позже была глубоко разработана виднейшим представителем славянофильства А. Хомяковым.
Второе направление в русской философии первой половины XIX в. — славянофильство. О сторонниках этого
направления сложилось устойчивое мнение как о представителях либерального дворянства, провозглашающих
особое историческое предназначение России, особые пути развития ее культуры и духовной жизни. Такое
одностороннее толкование славянофильства нередко приводило к тому, что это направление трактовалось как
реакционное  или,  в  лучшем  случае,  как  консервативное,  отсталое.  Подобная  оценка  далека  от  истины.
Славянофилы  действительно  противопоставляли  Восток  Западу,  остава-  46  ясь  в  своих  философских,
религиозных историко-философских воззрениях на русской почве. Но противопоставление Западу проявлялось
у  них  не  в  огульном  отрицании  его  достижений,  не  в  замшелом  национализме.  Напротив,  славянофилы
признавали и высоко ценили достоинства западноевропейской культуры, философии, духовной жизни в целом.
Они творчески восприняли философию Шеллинга, Гегеля, стремились использовать их идеи.
Славянофилы  отрицали  и  не  воспринимали  негативные  стороны  западной  цивилизации:  социальные
антагонизмы,  крайний  индивидуализм  и  меркантильность,  излишнюю  рациональность  и  т.п.  Истинное
противостояние  славянофильства  Западу  заключалось  в  различном  подходе  к  пониманию  основ,  «начал»
русской  и  западноевропейской  жизни.  Славянофилы  исходили  из  убеждения,  что  русский  народ  должен
обладать самобытными духовными ценностями, а не воспринимать огульно и пассивно духовную продукцию
Запада. И это мнение сохраняет свою актуальность и поныне.
В развитии славянофильства особую роль сыграли И.В. Киреевский, А.С. Хомяков, К.С. и И.С. Аксаковы, Ю.Ф.
Самарин.  Многообразие  их  взглядов  объединяет  общая  позиция:  признание  основополагающего  значения
православия,  рассмотрение  веры  как  источника  истинных  знаний.  В  основе  философского  мировоззрения
славянофильства лежит церковное сознание, выяснение сущности церкви. Наиболее полно эта основа раскрыта
Л.С. Хомяковым. Церковь для него не является системой или организацией, учреждением. Он воспринимает
Церковь как живой, духовный организм, воплощающий в себе истину и любовь, как духовное единство людей,
находящих в ней более совершенную, благодарную жизнь, чем вне ее. Основным принципом Церкви является
органическое, естественное, а не принудительное единение людей на общей духовной основе: бескорыстной
любви к Христу.
Итак,  западничество  и  славянофильство  —  две  противоположные,  но  и  вместе  с  тем  взаимосвязанные
тенденции в развитии русской философской мысли, наглядно показавшие самобытность и большой творческий
потенциал русской философии XIX в. 

\newpage
\section{Религиозная метафизика и философия всеединства Соловьева}
В философии Соловьева, как и в учении Гегеля, онтология и гносеология, бытие и познание неразделимы и
опираются на единую основу. 
Идея всеединства является центральной в философии В. Соловьева, поэтому всю его систему часто называют
философией всеединства.
В философии всеединства речь шла о единении Бога и человека; идеальных и материальных начал; единого и
множественного; рационального, эмпирического и религиозно-мистического знания; нравственности, науки,
религии, эстетики.
Создавая новую синтетическую философию, Соловьев обратился к анализу предшествующей философской.
Философия, по мнению Соловьева, возникает в период напряженного кризиса, когда религиозная социальная
роль не разрывает человеческое общество, сознание. Идея всеединства есть та цементирующая основа, которая
предает целостность всей философии, несмотря на ее бесконечную вариативность и разнообразие.
Его  философия  начинается  с  понятия  не  бытия,  а  сущего.  Абсолютном  сущем,  по  мысли  Соловьева,
содержаться два центра – абсолютное начало, как таковое, и первоматерия. Для первоматерии, выражающей
начало многообразия, вводится понятие София (мудрость). В Философии Соловьева человек “совечен” Богу, он
говорит о человеке как идее бытия, которая заложена в самой основе мира в целом. Софийный идеальный
человек принадлежит вечности, а она ему, поэтому он едино с Богом. 
Принципы  онтологии,  которые  лежат  в  основе  философской  концепции  Владимира  Соловьева  неразрывно
вязаны  с  его  гносеологическим  учением.  В  своей  основе  единство  онтологии  и  гносеологии  у  Соловьева
базируется  на  платоновской  идее  единства  истины,  добра  и  красоты.  На  основе  этой  идеи  Соловье
разрабатывает концепцию целостного знания, которое предполагает постепенный синтез религии, философии,
науки.
Познание  у  Соловьева  связано  с  этикой,  с  эстетическими  чувствами,  но,  главным  образом,  с  реальным
“собирательным творчеством”. В реальном творчестве преобразуются общество, земная природа, универсум.
Средством для решения этих колоссальных проблем Соловьев предлагает единение свободно-нравственного
человечества,  развивающегося  благодаря  нравственному  совершенствованию  каждой  личности  и  всего
общества.

\newpage
\section{Тема свободы в философии Бердяева}
Философия Бердяева впитала в себя множество разнообразных источников. Ранний Бердяев пытался сочетать
гуманизм Маркса с антропологическим социализмом Михайловского и метафизикой неокантианства. Зрелые
философские воззрения Бердяева представляют собой одну из первых в Европе разновидностей христианского
экзистенциализма.  Согласно  экзистенциализму,  задача  философии  -  заниматься  не  проблемами  науки,  а
вопросами сугубо человеческого бытия (существования). Человек помимо своей воли заброшен в этот мир, в
свою судьбу, и он живет в чуждом ему мире: его бытие со всех сторон окружено таинственными знаками,
символами. Страх важнейшее понятие философии экзистенциализма. 
Большое место в экзистенциализме занимает проблема свободы, определяемая как выбор человеком самого
себя: человек таков, каким он себя свободно выбирает. “Чувство вины за все совершающееся вокруг него -чувство свободного человека” (Бердяев).
Экзистенциализм различают религиозный и атеистический. Именно к религиозному - и относится Бердяев.
Философия Бердяева антропоцентрична - проблема духовности, свободы и творчества, судьбы, смысла жизни и
смерти всегда были в центре его философских размышлений. По Бердяеву “личность вообще первичнее бытия”,
бытие  -  воплощение  причинности,  необходимости,  пассивности,  духовное  начало  свободное,  активное,
творческое. Понятие объективного мира Бердяев заменяет термином “объективированный мир”, интерпретируя
его  как  “объективацию  реальности”,  порожденную  субъективным  духом.  Частично  признавая  социальную
обособленность  бытия  личности,  он  вместе  с  тем  считает  главным  в  человеке  то,  что  определяется  его
внутренним миром, а не внешним окружением. Личность, по Бердяеву прежде всего категория религиозного
сознания,  и  поэтому  проявление  человеческой  сущности,  ее  уникальности  и  неповторимости  может  быть
понято лишь в ее отношении к богу.
Бердяев  рассматривая  три  типа  времени  (космическое,  историческое  и  экзистенциальное  или  мета
историческое),  он  главным  образом  озабочен  предсказанием  того,  как  “мета  история  входит  в  историю”,
обоснованием приближения конца истории. Эти мотивы особенно сильно проявились в его последних работах.
Бердяев считал, что философия хочет не только познания мира но и улучшения его. Мораль и нравственность
Бердяев строит на христианских заповедях.
«Свобода первична, по отношению к бытию». 
/*Творческое развитие христианства Бердяева стало в противоречие с догмами христианства*/
Проблема Теодицея [тео=бог, дицей=оправдывать] – проблема оправдания бога: как это абсолютно доброе,
всеблагое начало создало такой мир.
Решение  Бердяева:  Если  бог  абсолют,  то  непонятно  как  решать  проблему.  «Свобода  есть  изначальная
координата бытия, или свобода предшествует бытию и предшествует ему в абсолютном смысле». Бог, прежде
всего, творец, но тогда, по определению, он изначально нуждается в свободе (нет свободы – нет творчества).
/*Версия свободы навеяна идеями Шеллинга.*/ Бог, таким образом, не является абсолютом и всемогущим. Бог
тоже некогда рождается или возникает. Бог и свобода, которая всегда рядом с ним, появляются из неизвестной
нам  бездны  (под  бездной  понимается  не  пространственно-временное  нечто,  а  поле,  сфера,  куда  не  может
проникнуть наше познание).
Свобода существует наряду с богом!
Возникновение мира – процесс, связанный с рождением и становлением самого (мира) бога. Бог развивается
через  становление  мира,  а  становление  мира  идет  медленно  и  проблематично.  В  несовершенстве  мира
отражается несовершенство бога. Это один процесс. Насколько мир и человек заинтересованы в боге, настолько
бог заинтересован в человеке, мире. Два начала заинтересованы друг в друге.
Встреча человека и Иисуса Христа – точка первой встречи бога и человека. Кардинальный момент в развитии
мира. Человек понял выразителем какой силы он является, кем он ведомый, на кого он должен положиться. И
тогда, бог не несет ответственности за зло в мире, ибо он действует у условиях свободы. 
Свобода – это не хорошо и не плохо, потому что дорогой свободы можно прийти куда угодно. Свобода – это не
гарант удач, успехов, она может привести куда угодно. Свобода – это просто наличие выбора, который мы
должны делать всегда сами. Выбор всегда таков, каковы мы. Пути, которые мы выбираем, зависят от нашей
духовности. 
Быть свободным гораздо труднее и опаснее, чем рабом. Западный мир идет по пути «несвободы», так как там,
главным образом, заботятся о комфорте и благополучии. Человек должен развивать свое духовное начало,
только это ведет по пути свободы.


\newpage
\section{Философия экзистенциализма: основные направления и проблемы}
Экзистенциализм – философия существования. Иррационалистическая философия.
Наиболее крупные представители: М. Хейдеггер, религиозный (К. Ясперс, Г. Марсель), атеистический (Ж.П. 
Сартр, А. Камю), Н. Аббаньяно.
Экзистенциалисты.  поставили  вопрос  о  смысле  жизни,  о  судьбе  человечества,  о  выборе  и  личной
ответственности в условиях исторических катастроф и противоречий.
Исходный пункт философии экзистенциализма – изолированный, одинокий индивид, все интересы которого
сосредоточены  на нем же  самом,  на его  собственном ненадежном  и  бренном  существовании.  Отчуждение
человека от общества.
Экзистенциальные проблемы – это проблемы, которые возникают из самого факта существования человека. Для
Э. имеет значение только его собственное существование и его движение к небытию.
Хотя бытие вещей совершенно непонятно, но есть 1 вид бытия отлично нам знакомый – это наше собственное
бытие. Здесь то и открывается доступ к бытию как таковому, он идет через наше существование. Но это
существование  –  нечто  внутреннее  и  невыразимое  в  понятиях:  "существование  есть  то,  что  никогда  не
становится объектом", ибо мы никогда не можем взглянуть на себя со стороны.
Экзистенциализм  –  это  философия,  единственный  предмет  которой  –  человеческое  существование,  точнее
переживание  существования.  Среди  всех  способов  бытия  существования  Э  ищут  такой,  в  котором
существование раскрылось бы наиболее полно – это страх. Страх – это исходное переживание, лежащее в
основе всего существования. В конечном счете, это страх перед смертью.
Экзистенциалисты  объявляют  предметом  философии  бытие.  «Современная  философия,  как  и  в  прошлые
времена, занята бытием" (Сартр). Они утверждают, что понятие бытия является неопределимым, и что никакой
логический анализ его невозможен. Поэтому философия не может быть наукой о бытии и должна искать иных,
ненаучных, иррациональных путей для проникновения в него. Противопоставляя науку философии, говорят, что
наука  занимается  сущим,  а  философия  –  бытием.  Бытие  постигается  не  через  рассудочное  мышление,  а
непосредственно открываясь человеку через его экзистенцию.
Экзистенция  представляет  собой  центральное  ядро  человеческого  Я,  благодаря  чему  Я  выступает  не  как
отдельный  мыслящий  индивид  и  не  как  мыслящее  всеобщее,  а  как  отдельная  неповторимая  личность.
Экзистенция – не сущность человека, а открытая возможность. Важнейшее определение экзистенции – ее
необъективируемость. Можно объективировать способности и знания через материальный мир, рассматривать
психические акты и деятельность, единственное, что неподвластно объективации – экзистенция. В обыденной
жизни человек не осознает экзистенцию, для этого ему надо оказаться в пограничной ситуации.
Обретая себя как экзистенция, человек обретает свободу.
Свобода. Человек сам свободно выбирает свою сущность, он становится тем, кем он себя сделает. Человек – это
постоянная возможность, замысел, проект. Он свободно выбирает себя и несет полную ответственность за свой
выбор. Свобода составляет само человеческое существование, человек и есть свобода.
Однако  свобода  понимается  ими  как  нечто  неизъяснимое,  не  поддающееся  выражению  в  понятиях,
иррациональное. Свободу они мыслят как свободу вне общества. Это внутреннее состояние, настроенность,
переживание  индивида.  Свобода  противопоставляется  необходимости.  Такая  свобода,  противопоставленная
необходимости и отрешенная от общества, – есть пустой формальный принцип. Свобода – это свобода выбора
отношения  к  окружающей  действительности.  Раб  может  быть  свободным,  соответственно  самоопределяя
отношение к своему бытию. Свобода становится неотвратимым роком. «Человек осужден быть свободным».
Свобода есть мучительная необходимость.
Характерной  чертой  человеческого  существования  является  то,  что  он  не  сам  выбирает  условия  своего
существования, он заброшен в мир и подвластен судьбе. От человека не зависит время его рождения и смерти.
Это  приводит их  к мысли,  что помимо  чел  существования  существует потусторонняя  реальность,  которая
понимается как способ существования человека, состоящий в озабоченности человека, направленной куда-то
вне его. Внешний мир представляет среду, мир заботы человека, окружающий человеческое существование и
находящийся в неразрывной связи с ним. Пространство и время есть способы чел существования. Время – это
переживание  существованием  своей  ограниченности,  временности.  Представление  о  времени  до  моего
рождения  и  после  смерти  –  произвольная  экстраполяция.  Говорить  о  том,  что  будет  после  моей  смерти
бессмысленно.
Личность и общество. Общество – всеобщая безличная сила, подавляющая и разрушающая индивидуальность,
отнимающая у человека его бытие, навязывающая личности трафаретные вкусы, нравы, взгляды... Человек,
преследуемый страхом смерти, ищет прибежища в обществе. Растворяясь в нем, он утешает себя тем, что люди
смертны. Но жизнь в обществе не истинна. В глубине человека скрыто истинное, одинокое существование.
Каждый умирает в одиночку.

\newpage
\section{Проблема человеческого существования в экзистенциализме}
Идейные истоки экзистенциализма — философия жизни, феноменология Гуссерля, религиозно-мистическое
учение  Кьеркегора.  Различают  экзистенциализм  религиозный  (Марсель,  Ясперс,  Бердяев,  Бубер)  и
атеистический  (Хайдеггер,  Сартр,  Камю).  В  философии  существования  нашёл  отражение  кризис
оптимистического  либерализма,  опирающегося  на  технический  прогресс,  но  бессильный  объяснить
неустойчивость,  неустроенность  человеческой  жизни,  присущие  человеку  чувство  страха,  отчаяния,
безысходности.
Экзистенциализм  —  это  иррациональная  реакция  на  рационализм  Просвещения  и  немецкой  классической
философии.  По  утверждениям  философов-экзистенциалистов,  основной  порок  рационального  мышления
состоит в том, что оно исходит из принципа противоположности субъекта и объекта, то есть разделяет мир на
две  сферы:  объективную  и  субъективную.  Всю  действительность,  в  том  числе  и  человека,  рациональное
мышление рассматривает только как предмет, как «сущность», познанием которой можно манипулировать в
терминах  субъекта-объекта.  Подлинная  философия  с  точки  зрения  экзистенциализма  должна  исходить  из
единства  объекта  и  субъекта.  Это  единство  воплощено  в  «экзистенции»,  то  есть  некой  иррациональной
реальности.
Согласно экзистенциалистскому учению, чтобы осознать себя как «экзистенцию», человек должен оказаться в
«пограничной ситуации», например перед лицом смерти. В результате мир становится для человека «интимно
близким». Истинным способом познания, способом проникновения в мир «экзистенции» объявляется интуиция
(«экзистенциальный опыт» у Марселя, «понимание» у Хайдеггера, «экзистенциальное озарение» у Ясперса),
которая являет собой иррационалистически истолкованный феноменологический метод Гуссерля.
Значительное  место  в  экзистенциализме  занимает  постановка  и  решение  проблемы  свободы,  которая
определяется  как  «выбор»  личностью  одной  из  бесчисленных  возможностей.  Предметы  и  животные  не
обладают свободой, поскольку сразу обладают «сущим», эссенцией. Человек же постигает своё сущее в течение
всей жизни и несёт ответственность за каждое совершённое им действие, не может объяснять свои ошибки
«обстоятельствами». Таким образом, человек мыслится экзистенциалистами как самостроящий себя «проект».
В конечном итоге идеальная свобода человека это свобода личности от общества.

\newpage
\section{Феноменология Гуссерля}
Гуссерель  рассматривает  психический  акт  как  самостоятельну  особую  реальность.  Особенности  учения
Гуссереля:
1.Гуссерель пытается преодолеть психологизм.
2.Показывает, что общие понятия, которые Брентано выдаются как языковые фикции, существуют и обладают
логическим реальным бытием.
3.Пытается рассмотреть философию исходя из нового метода: созерцание сущностей.
Феномен – явление.
Гуссерель пытается вявить чистую логику, но как он констатирует: познавательная значимость логических
суждений должна быть в свою очередь фундирована (предварительно) в другого рода значимость, не связанную
с познавательной логикой (допредекативную значимость предмета). Отход от общей значимости суждения к
допредикативной значимости восприятия – это первый и решающий методологический шаг феноменологии
духа. Допредикативная значимость – предзаданное восприятие предметов, явлений.
Предрассудки, присущие психологизму:
1.Значимость мышления должна быть обоснована психологически. С т.з. Гуссерля логические законы имеют
идеальное общезначимое содержание.
2.Психологизм: суждение – это психологический феномен. Гуссерель считает, что в логике мы имеем дело с
идеальным содержанием, освобожденным от его эмпирических составляющих.
3.В логике нужно опереться на очевидность, но с позиции феноменологии (ф) очевидность — это не просто
одно чувство наряду с другими. В очевидности (по Г) феноменологической, истина представляет собой скрыто
присутствующсю  изначальность.  Г.  преодолевает  психологическо-эмпирическую  позицию  Брентано  в
«Логических  исследованиях»,  где  он  исследует  идеальное  бытие  понятий,  называемых  эйдосами.  Эйдосы
станут предметом описания феноменов.
Гуссерель ставит задачу разделить акты восприятия и предмета изнутри сознания средствами самого сознания.
Методологический фундамент феноменологии – это метод очевидности или метод редукии. – метод сведения к
очевидности.
Редукция  меняет  установку  сознания:  от  естественной  направленности  на  вещи  и  явления  как  на
существующее, от «захваченности ими», точнее догматической уверенности в их реальном существовании к
фактам и предметам самого сознания., которые называются феноменами сознания (то есть к фактам самого
сознания, а не природного).
2 этапа редукции:
1.Эйдетическая редукция – освобождение от чувственных явлений и переход к чистым феноменам. Природное
существование вещи отбрасывается и образуется феноменологическая предметность в которой можно прежить.
2.Феноменологическая или трансцедентальная редукция – это освобождение от уверенности в укорененности
«я» в мире вещей. Следовательно я (субъект) должно превратиться в чистый поток сознания.
Чистый поток сознания не должен содержать внетренний опыт, наших ощущений.
Это приводит к понятию «интенциональность» (И.)
Интенциональность – это направленность сознания на другое, чем оно само является, то есть на предмет,
понятый как данность, которая хотя и является данностью сознания. И. Не является выходом сознания за свои
собственные пределы, т.к. нет ничего изначально присущего сознанию. В И. Совпадает чистая предметнсть и
чистая субъективность, причем чистую субъективность Гуссерель определяет как ноэзис.
Ноэзис – интенциональный акт, акт восприятия, направленность на предмет.
Чистая предметность – ноэма – то, на что направлен акт сознания.
Интенциональный акт – это единство субъективности и эйдоса (читсая идея).
Чистая логика – это чистая априорность, которая позволит потом развиться интенциональности.

\newpage
\section{Фундаментальная онтология Хайдеггера}
В конце 30х гг. становится ректором университета и принимает активное участие в фашистском движении.
Гуссерль:  сознание  –  интенционально,  т.е.  сознание  чего-то.  В  сознании  имеем  не  реальные  вещи,  а
конструкторы (интенциональные объекты).
Хайдеггер:  интенциональные  объекты  есть  сами  вещи,  данные  сознанию  так,  как  они  существуют  в
действительности.
Он пишет: истина по гречески «алетейа», что буквально переводится как «несокрытость». Т.е. истинно то, что
для нас не скрыто. Т.е. древние греки считали, что вещи могут непосредственно контактировать с сознанием,
т.е. вещи нам непосредственно даны. Но они даны нам не в научном познании, а в интуиции (в жизненной
интуиции).
Искусственность  схемы  гуссерля  по  Хайдеггеру  в  том,  что  он  познание  отрывает  от  деятельности.  По
Хайдеггеру познание – лишь одна из функций интеллекта, человек – существо прежде всего действующее и
живущее. Мир дан непосредственно, и дан не в познании прежде всего, а в действии.
Вещы  нам  даны  непосредственно  как  феномены.  С  греческого  «феномен»  –  то,  что  показывает  себя.  Но
показывает себя полностью (а не какой-то стороной) ? нет никаких кантовских вещей в себе. Это придумка,
возникающая при отрыве познания от деятельности.
Таким образом, феномены – это вещи в действительности, в их данности нашему сознанию, но не вещи сами по
себе. Т.е. вещи и есть формы бытия (или бытие в различных его формах).
Исходной формой бытия является вещь, которую он называет Dasein – это особая исходная форма бытия,
которую он выражает словами: «бытие здесь» – это мое личное существование. Это я сам, как форма бытия.
(Хайдеггер «Бытие и время»)
Основное свойство dasein – его существование в чем-то, которое мы называем миром. Мир -–это наличные для
dasein вещи, среди которых dasein не является вещью. Но dasein и мир образуют неразделимое единство.
Познание по Хайдеггеру – лишь одна из форм существования dasein. А человек прежде всего не теоретик, а
практик и познание – вторичный элемент в его деятельности, а первичный элемент – практика.
Устройство бытия.
Мир, бытие имеет центр – dasein.
Дальше Хайдеггер вводит понятие Werkwelt – «подручный мир» или «подручные вещи» – мир повседневных,
ближайших интересов и занятий. Umwelt – окружающий мир – то, что входит в область моего зрения, но не
очень меня затрагивающий.
Окружающий мир по Хайдеггеру отличается от физического мира науки. Если физическое пространство не
имеет никакого центра, то Umwelt имеет центр – dasein. Природа по сравнению с живым миром моего “я”
является сухой абстракцией. Природа, как ее видит наука – то, что остается от мира, если воспринимать его
чисто созерцательно. Чтобы получить “подлинную” картину природы, ученые стараются свести к минимуму
фактор наблюдателя в опыте. Т.е. наука ? к «объективности». И тогда природа – это Umwelt, ставший жертвой
нашего стремления я к объективности. Научная картина мира становится неподлинной и непонятной.
Поэтому мир науки – это мир искаженный, поэтому это мир непонятный. Понятно только то, что м.б. встречено.
Понятно, как наука строит свою картину мира, понятны будут ее теории и доказательства, но общий результат
всего этого нам совершенно непонятен (что такое мир в целом, как он устроен). Научная картина мира –
любопытная, сложная абстрактная конструкция, которая что-то проясняет; но она не совпадает с миром, в
котором мы живем.
Хайдеггер ставит вопрос о бытии независимо от познания (вне проблемы субъект-объект).
Хайдеггер оказал огромное влияние на развитие экзистенциализма. Но экзистенциализм исходит из другого
понятия бытия: чувственного бытия.
Мир Хайдеггера все же рационален.

\newpage
\section{Бытие как проблема онтологии. Основные онтологические концепции}
Онтология (новолат. ontologia от др.-греч. — сущее, то, что существует и учение, наука) — раздел философии,
изучающий проблемы бытия; наука о бытии.
Термин «Онтология» был предложен Р. Гоклениусом в 1613 году в его «Философском словаре» («Lexicon
philosophicum, quo tanquam clave philisophiae fores aperiunter. Fransofurti»), и чуть позже И. Клаубергом в 1656
году в работе «Metaphysika de ente, quae rectus Ontosophia», предложившем его (в варианте «онтософия») в
качестве  эквивалента  понятию  «метафизика».  В  практическом  употреблении  термин  был  закреплен  Х.
Вольфом, явно разделившим семантику терминов «онтология» и «метафизика».
Обычно под онтологией подразумевается эксплицитная, то есть явная, спецификация концептуализации, где в
качестве  концептуализации  выступает  описание  множества  объектов  и  связей  между  ними.  Формально
онтология состоит из понятий терминов, организованных в таксономию, их описаний и правил вывода.
Основной вопрос онтологии: что существует?
Основные  понятия  онтологии:  бытие,  структура,  свойства,  формы  бытия  (материальное,  идеальное,
экзистенциальное), пространство, время, движение.
Онтология,  таким  образом,  представляет  собой  попытку  наиболее  общего  описания  универсума
существующего, который не ограничивался бы данными отдельных наук и, возможно, не сводился бы к ним.
Иное понимание онтологии даёт американский философ Уиллард Куайн: в его терминах онтология — это
содержание  некоторой  теории,  то  есть  объекты,  которые  постулируются  данной  теорией  в  качестве
существующих.
Вопросы онтологии — это древнейшая тема европейской философии, восходящая к досократикам и особенно
Пармениду. Важнейший вклад в разработку онтологической проблематики внесли Платон и Аристотель. В
средневековой философии центральное место занимала онтологическая проблема существования абстрактных
объектов (универсалий).
В философии XX века специально онтологической проблематикой занимались такие философы как Николай
Гартман («новая онтология»), Мартин Хайдеггер («фундаментальная онтология») и другие. Особый интерес в
современной философии вызывает онтологические проблемы сознания.
Предмет онтологии
* Основным предметом онтологии является бытие, которое определяется как полнота и единство всех видов
реальности: объективной, физической, субъективной, социальной и виртуальной.
   * Реальность традиционно ассоциируется с материей и подразделяется на косную, живую и социальную
материю.
   * Бытие как то, что можно мыслить, противопоставляется немыслимому ничто (а также ещё-не-бытию
возможности  в  философии  аристотелизма).  Поскольку  мышлением  и  постижением  возможностей  бытия
обладает  только  человек,  то  в  последнее  время  (в  феноменологии  и  экзистенциализме)  именно  он
отождествляется бытием. Однако в классической метафизике под бытием понимается Бог. Человек как бытие
обладает свободой и волей.
Современная онтология
Современная  философия  рассматривает  бытие  как  единую  систему,  все  части  которой  взаимосвязаны  и
представляют собой некую целостность, единство. Вместе с тем мир разделен, дискретен и имеет четкую
структуру. В основе структуры мира 3 слоя реальности: бытие природы, бытие социальное, бытие идеальное.
Бытие природы
Бытие природы — первая форма реальности, универсума.
* Включает все существующее кроме человека.
* Является следствием длительной универсальной эволюции.
   * Системная организация мироустройства. Развитие мира — процесс преобразования и взаимодействия
образующих  его  систем.  Способность  всех  природных,  социальных  систем  к  самоорганизации,
самопроизвольному переходу на более высокий уровень организованности и упорядоченности.
* Подсистемы вещества и поля. Вещество — вид материи, обладающий массой покоя. Поле — основной вид
материи, связывающий частицы и тела. Частицы поля не имеют массы покоя, так как способ их существования
— движение.
   * Подсистемы неживой и живой природы. Неживая природа — движение элементарных частиц и полей,
атомов и молекул. Её уровни: вакуумный-микроэлементный-атомный-молекулярный-макроуровень-мегауровень
(планеты, галактики). Живая природа — биологические процессы и явления, происходит из неживой, включена
в неё, но представляет иной уровень развития. Её уровни: молекулярный-клеточный-микроорганизменный-тканевый-организменно-популяционнный-биогеоценотический-биосферный.
Бытие социальное
Бытие социальное — вторая форма реальности.
* Включает в себя бытие общества и бытие человека (экзистенция).
    *  Структура  социального  бытия  или  социума:  индивид,  семья,  коллектив,  класс,  этнос,  государство,
человечество.  По  сферам  общественной  жизни:  материальное  производство,  наука,  духовная  сфера,
политическая сфера, сфера обслуживания и т. д.
Бытие идеальное, духовное
Бытие идеальное, духовное — третья форма реальности.
* Тесно связано с бытием социальным, на своем уровне повторяет и воспроизводит структуру социума.
    *  Включает  неосознаваемые  духовные  структуры  индивидуального  и  коллективного  бессознательного
(архетипов),  сложившиеся  в  психике  людей  в  доцивилизационный  период.  Роль  этих  структур  признается
существенной и определяющей.
    *  Усиление  взаимодействия  всех  форм  духовной  жизни  с  производством,  практикой  (космонавтика,
биоинженерия и т. д.).
    *  Новые  информационные  технологии  и  средства  связи  сделали  духовное  бытие  более  динамичным,
подвижным.
«Белые пятна» современной онтологии
* Возникновение Вселенной — дата приблизительна, не ясна причина Большого взрыва.
* Появление жизни на Земле — не ясен процесс перехода от неживого к живому, от сложных органических
веществ к простейшим живым организмам, возникновение механизма наследственности.
Фрэнсис Крик, лауреат Нобелевской премии, английский биофизик: «Мы не видим пути от первичного бульона
до естественного отбора. Можно прийти к выводу, что происхождение жизни — чудо, но это свидетельствует
только о нашем незнании».
 * Не ясен механизм перехода от мира биологического к миру социальному, механизм появления человека.
Доказано, что человек — продукт эволюции, однако у человека как биологического вида нет родственников,
которые по прежнему оставались бы «детьми природы» и с которыми могла быть установлена генетическая
связь. Шимпанзе и гориллы связаны с человеком только общим предком, жившим более 7 миллионов лет назад.

\newpage
\section{Материя и формы ее движения}
Уже в древности философы пытались представить  видимое многообразие вещей как проявление видимого
начала.  Это  общее,  несотворимая   и  не  уничтожимая  основа  всех  вещей  получила  название
субстанции.Формирование субстанции - это и формирование научного понимания материи.В древней Греции
под  субстанцией   материалисты  понимали  конкретное  вещество.  Древнегреческие  атомисты  считали,  что
субстанция это атом. Все что состоит из атомов и пустоты. В философии и естествознании нового времени и в
работах Ньютона, Ломоносова анатомические идеи получили развитие.
 С конца 16в.до начала 19в. господствовала механистическая картина мира. Материя рассматривалась  как
совокупность неделимых атомов, которые наделены геометрическими и механическими  свойствами: массой,
протяженностью, формой, непроницаемостью и способностью перемещаться.
Но в это время были высказаны и другие идеи о материи:1) о самодвижении в материи2) материя понималась
как  абстрактное  понятие,  в  котором  отображены  свойства  многообразных   веществ.3)  Материи  присуща
внутренне  мысль.Трактовка  материи  идеалистическими  направлениями   фил.Субъективные  идеалисты
отождествляли материю с совокупностью ощущений. Объективныеидеалисты считали, что материя пассивна,
инертна. Жизнь ей придает идею. Итак и для материализма  и для идеализма характерно отождествление
материи с веществом, с конкретными формами ее проявления; Не проводится различие м/у фил. пониманием
материи  и  естественноисторическими   взглядами  на  существующий  мир.  Маркс  и  Энгельс  понимали  под
материей субстанцию.
  Материя  -  это  всеобщая  носительница  св-в,  отношений,  изменений  конкретных  вещей.   Но  материя  не
существует вне конкретных вещей, а только через них. В начале 19-20вв. в естество знании было сделано ряд
открытий, которые опровергли старую, мех. картину мира. Открытие рентгеновских лучей подрывает идею
непроницаемости  материи.  Явление   дефекта  массы,   открытое  Беккерелем  при  изучении  радиоактивного
распада. Был сделан  вывод, что материя превращается в материю. Материя исчезает. Материализм ложен.
Открытие Томпсоном электрона отвергает идею о том, что атом не делим. Прежняя картина мира рушилась. В
естествознании наступил методологический кризис. Новые открытия не вписывались в старую картину мира.
Перед фил. стала задача уточнения понятия <материя>.
  Это  сделал  Ленин  в  работе  <Материализм  и  эмпириализм>.   Материя    -  философская  категория   для
обозначения объективной реальности, кот. дана ч-ку в ощущениях его, которая копируется, фотографируется,
отображается, нашими ощущениями,  существующая независимо  от  них.   В  этом  определении выделено 2
признака материи:
1) Признание первичности материи по отношении к сознанию (объективность ощущения) 
2) Признание принципиальной познаваемости мира. Ленин разграничивает философское понимание материи и
естественнонаучные  знания   о  существующем  мире.Ленин  способствовал  преодолению  кризиса  в  физике,
связанного с включением принципа структурности материи и делимости атомов в научную картину мира.

\newpage
\section{Движение и развитие, прогресс и регресс}
Характеризуя  развития общества, обществоведы обращаются к понятиям «прогресс» и «регресс». Прогресс
понимается  как  тип  направленного  развития,  связанный  переходом  от  низшего  к  высшему,   от  менее
совершенного  к  более  совершенному.  Представление  о  прогрессе   направленном   изменении  к  лучшему
возникло в древности. Однако поначалу оно носило в основном оценочный характер. Сначала прогресс был
отмечен в сфере научного познания, а затем идея  прогресса была распространена на область социальных
отношений. Одним из наиболее выдающихся  философов, разрабатывавших идею общественного прогресса,
был итальянский философ  Вико (17век). Он утверждал, что все народы проходят 3 этапа своего развития:
божественный,  героический и человеческий. Пройдя эти этапы, человечество осуществляет движение  по
нисходящей  линии.  В  целом  прогресс  характеризуется  нарастанием  темпов  развития   и  принимается  как
объективный закон исторической эволюции. Однако одной из существенных особенностей прогресса является
присутствие в нём регресса отдельных элементов, связей  и функций. С общественным прогрессом всегда
связывали  восходящее  поступательное  развитие   человеческого  общества  от  низших  ступеней  к  высшим.
Фаталистическое  понимание  прогресса,   когда  он  принимался  за  нечто  неизбежное,  настраивало  на
пассивность, созерцательность, безынициативность в отношении к жизни. Волюнтаризм понимал прогресс как
отрицание   объективных  законов  всемирной  истории,  религия   как  теологическое  представление  о
божественном  творчестве  и  конечном  состоянии  мира.  Прогресс   многогранное  явление.   В  качестве
конкретных  критериев  общественного  прогресса  выделяют  экономический,  социально   -  политический,
идеологический и гуманистический. С их помощью определяется прогрессивность тех или иных социальных  
систем. Многие социологи предлагают искать критерии прогресса в сфере сознания. Понятие прогресса носит
относительный характер. Возможность прогресса  определяется наличием конкретно  исторических условий.
Эти условия могут способствовать ускорению прогресса или же тормозить его, а иногда и препятствовать ему.
Типы общественного прогресса:
· Прогресс в антагонистических условиях (классовые, эксплуататорские общества) тип прогресса одинаков для
всех обществ и обусловлен господством частной собственности и эксплуатацией небольшой частью общества
большинства. Происходит развитие одних сторон общественной структуры за счёт других.
· Прогресс в частнособственнических формациях совершается за счёт усиления эксплуатации народных масс.
· Прогресс в феодальном обществе.
· Прогресс в капиталистических условиях «невидимые нити» капиталистической зависимости приковывают
рабочего к капиталисту.
· Прогресс в условиях социализма ликвидация всякого антагонизма (исчезновение классовых сил, тормозящих
прогресс), народные массы ведущая сила общественного прогресса.
В мире все находится в движении, от атомов до вселенной. Все пребывает в вечном стремлении к иному
состоянию,  и  не  по  принуждению,  а  по  собственной  природе.  Поскольку  движ  есть  сущностный  атрибут
материи, то оно, также как и сама материя, несотворимо и неуничтожимо. Движение - это способ сущ материи.
Движ заключено в самой природе материи. Одни формы движ превращ в другие и ни один вид не берется
ниокуда. 
Движение есть единство изменчивости и устойчивости, беспокойства и покоя. В потоке не прекращ движения
всегда  присутствуют  дискретные  моменты  покоя.,  проявляющиеся  прежде  всего  в  сохранении  внутренней
природы каждого данного движения, в виде равновесия движений и их относительно устойчивой формы, т.е.
относительного покоя. Покой, т.о., сущ как характеристика движ в какой-либо устойчивой форме. Каак бы не
изменялся предмет, но пока он сущ, он сохраняет свою определенность. Река остается рекой.
Абсолютный покой невозможен.
Существует несколько качественно различных форм движ материи: механ, физическая, хим, биологическая,
социальная... Качественное разнообразие одного уровня не м.б. объяснено кач разнообразием другого. Точное
описание движ частиц воздуха не может объяснить смысл чел речи. Однако необходимо иметь в виду и общие
закономерности, свойственные вскм уровням, а также их взаимодействие. Эта связь выраж в том, что высшее
включает низшее. (ДНК - хим соединение) Однако высшие формы не включены в низшие. (нет жизни в хим
соединениях)
Прослеживание связей между различными формами движ материи позволяет создать картину их развития во
вселенной.  На  его  разных  этапах  возникают  все  новые  уровни  организации  материи  и  соотв  им  формы
движения, причем появление каждой новой формы связано с сосотоянием Вселенной как целого. Сразу после Б.
Взрыва не было ни атомов ни соотв им форм движения. Хим и физ формы движения возникли на опред уровне
развития  Вселенной.  Также  на  опр  этапе  косм  эволюции  сформировались  планетные  системы,  возникли
условия  для  возникнов  жизни,т.е  биол  формы  движения.  В  этом  смысле  жизнь  надо  расматривать  как
космическое явление. В свою очередь только пройдя длшит этап эволюции, жив природа смогла породить
социально организованную материю., и тогда возникла социальная форма движ.
Совр наука показывает, что наша астрономическая вселенная, мир, в кот мы живем, по-видимому, явл только
одним  из  возможных  миров.  Причем  уже  в  особенностях  взаимод.  элементарных  частиц  заложены  опр
предпосылки, возможности для развертывания более сложных форм движения. (мировые константы)
В соврем космологии указанные идеи входят в содержание так называемого антропного принципа, согласнокот
наш мир устроен таким образом, что допускает возможность появл человека как закономерного итога эволюции
материи. Но возможны и др миры, с другими мир константами. Эти миры возможно бедны, пусты и допускают
только низшие формы движ материи, а возможно и наоборот. В этом смысле человек и чел общество предстают
как  такая  форма  организации  материи,  кот  обусловлена  свойствами  целого  нашей  Вселенной,
фундаментальными характеристиками космоса. 
Мировое развитие являет собой заклномерный поступательный процесс, противоречия которого представляют
собой источник, движ силу общ прогресса.Всемирная история постоянно выдвигала пробл противоречий общ
прогресса, и каждая ее эпоха свидетельствовала о катаклизмах, прерворотах и вместе с тем она являет собой
необходимый процесс движ человечества от одних форм своей соц организации к другим, более совершенным. 
Критерий  прогресса  -  общественно  историческая  практика,  в  кот  выделяются  два  ее  основных  вида:
производственная и социально-преобразующая. Ядром этой практики выступает развитие производ сил как
высшего критерия общ прогресса. Главное в произв силах - это человек Этим объясняется то, что в данном
критерии воплощаются и достижения науки, принципы управления, и социально-полит состояние общества, и
уровень образования, и образ жизни вплоть до мировоззрения, кот опосредовано влияют на эффективность
производства. Вот почему "развитие производ сил человечества означает прежде всего развитие богатства чел
природы как самоцель". Действительным ядром общ прогресса выступают способ производства.
Для  определения  подлинно  прогрессивного  есть  критерий,  выработанный  самой  историей  человечества.
Критерий этот, выраженный словом гуманизм, обозначает как специф свойства чел природы так и оценку этих
свойств как высшего начала общ жизни. Прогрессивно то, что способствует возвышению гуманизма.

\newpage
\section{Пространство и время. Субстанциональная и реляционная концепции}
Материальный  мир  состоит  из  структурных  объектов,  которые  находятся  в  движении  и  развитии,
представляющие собой процессы, которые развертываются по определенным этапам.
Наиболее общая характеристика пространства — свойство объекта быть протяженным, занимать место среди
других, граничить с другими объектами.
Сравнение  различных  длительностей,  выражающих  скорость  развертывания  процессов,  их  ритм  и  темп
является понятием времени.
Категории  пространства  и  времени  выступают  как  формы  бытия  материи.  Существует  две  концепции
пространства и времени:
* субстанциальная — рассматривает пространство и время как особые сущности, которые существуют сами
по себе, независимо от материальных объектов (Демокрит, Эпикур, Ньютон);
* реляционная — рассматривает пространство и время как особые отношения между объектами и процессами
и вне их не существуют (Лейбниц).
Всеобщие  свойства  пространства  и  времени:  объективность  и  независимость  от  сознания  человека;
абсолютность как атрибутов материи; неразрывная связь друг с другом и с движением материи; зависимость от
структурных отношений и процессов развития в материальных системах; единство прерывного и непрерывного
в их структуре; количественная и качественная бесконечность.
Различают метрические (т.е. связанные с измерениями) и топологические (например, связность, симметрия
пространства и непрерывность, одномерность, необратимость времени) свойства пространства и времени.
Топологические  характеристики  описывают:  прерывность  и  непрерывность,  размерность,свяэность,
ориентируемость.
Метрические характеристики: кривизну, конечность и бес-конечносгь, изотропность, гомогенность.
Всеобщие  свойства  пространства:  протяженность,  означающая  рядоположенность  и  сосуществование
различных элементов (точек, отрезков, объемов и i:n.), возможность прибавления к каждому данному элементу
некоторого следующего элемента либо возможность уменьшения числа элементов; связность и непрерывность;
трехмерность.
С протяженностью пространства неразрывно связаны его метрические свойства, выражающие особенности
связи пространственных элементов, порядок и количественные законы этих связей.
Специфические (локальные) свойства пространства: симметрия и асимметрия, конкретная форма и размеры,
местоположение,  расстояние  между  телами,  пространственное  распределение  вещества  и  поля,  границы,
определяющие различные системы.
Всеобщие свойства времени: объективность; неразрывная связь с материей и ее атрибутами; длительность;
одномерность и ассиметричность; необратимость и направленность от прошлого к будущему.
Специфическими свойствами времени являются конкретные периоды существования тел от возникновения до
перехода в качественно иные формы, одновременность событий, которая всегда относительна, ритм процессов,
скорость изменения состояний, темпы развития, временные отношения между различными циклами в структуре
систем.
Теория А. Эйнштейна доказала, что в реальном физическом мире пространственные и временные интервалы
меняются при переходе от одной системы отсчета к другой.
Теория относительности вывела глубокую связь между пространством и временем, показав, что в природе
существует единое  пространство —  время,  а  отдельно  пространство и отдельно  время выступают  как  его
своеобразные проекции, на которые оно по-разному расщепляется в зависимости от характера движения тел. 

\newpage
\section{Принцип детерминизма и проблема причинности}
Принцип  детерминизма - все реальные явления, процессы детерминированы, т.е. возникают,  развиваются и
уничтожаются в результате действия определённых причин.Причина - явление,  повлёкшее за собой другое
явление.  Следствие  -  явление,  возникшее  в  результате   действия  причины.  Типы  причинных  связей:I.
однонаправленные  причинно-следственные   связи.II.  взаимодействиеВзаимодействие  -  это,  когда  причина
испытывает обратное влияние со стороны следствия.Причинные основания - совокупность всех обстоятельств,
при которых наступает следствие:- причина- условие- мотивы и т.д.Детерминизм - концепция мира, которая
основывается  на  принципах  причинности  и  закономерности.Формы  детерминизма:-   механистический
(лапласовский)  -  причинная  связь  понимается  как  однозначная,  т.е.   определённое  состояние  системы
определяет  последующее  состояние  системы.Причинность   -  необходимость.-  статистический  -  признание
многозначного  соотношения  между  причиной   и  следствием.В  причине  заключается  ряд  возможностей  и
вариантов развития.Индетерминизм - отрицание закономерностей и причинной обусловленности явлений.
Детерминизм (от лат. determine — определяю) — учение о первоначальной определяемости всех происходящих
в мире процессов, включая все процессы человеческой жизни, со стороны Бога (теологический детерминизм,
или учение о предопределении), или только явлений природы (космологический детерминизм), или специально
человеческой  воли  (антропологическо-этический  детерминизм),  для  свободы  которой,  как  и  для
ответственности, не оставалось бы тогда места. Под определяемостью, здесь подразумевается философское
утверждение, что каждое произошедшее событие, включая, и человеческие поступки и поведение однозначно 
определяется  множеством  причин,  непосредственно  предшествующим  данному  событию.  В  таком  свете
детерминизм может быть также определен, как тезис утверждающий, что имеется только одно, точно заданное,
возможное будущее.
Все  определено  в  этом  мире  и  ничто  не  в  состоянии  этого  изменить.  Однако,  всякое  действие  вызывает
следствие  подобно  всему  тому,  что  происходит  в  этой  жизни.  На  принципе  детерминизма  построена  вся
классическая физика, за исключением термодинамики и молекулярной физики. Детерминизм подразумевает
выполнение обратимости времени, т. е. частица прийдет в исходное состояние, если обратить время. Каждая
траектория единственным образом определяется начальными условиями. Всё это находится в замечательном
согласии с экспериментальными данными макромира.
Между детерминизмом и индетерминизмом имеются также переходы, например в учениях Лютера, Цвингли и
Канта: так, если учение детерминизма распространяется на эмпирическую (естественную) природу человека, то
его моральная сторона становится объектом разновидности индетерминизма.

\newpage
\section{Сущность сознания и его структура}
Сознание — высшая, свойственная лишь человеку форма отражения обьективной действительности. Сознание
представляет  собой  единство  психических  процессов,активно  участвующих  в  осмыслении  человеком
обьективного  мира  и  своего  собственного  бытия.  С  самого  рождения  человек  попадает  в  мир  предметов,
созданных предыдущими поколениями и формируется как таковой лишь в процесе обучения целенаправленому
их использользованию, которое происходит лишь в процессе общения. Имено потому, что человек относится к
объектам с пониманием, со знанием, способ его отношения к миру называется сознанием. Любое ощущение
или чувство является частью сознания так как обладает значением и смыслом. Однако сознание не есть только
знание  или  языковое  мышление.  С  другой  стороны  нельзя  отождествлять  сознание  и  психику,  т.к.  не  все
психические  процессы  включаются  в  данный  момент  в  сознание.  Оно  возникло  в  процессе  общественно-производственной деятельности человека и неразрывно связанно с языком. Сознание существует в 2 формах —
общественном  и  индивидуальном.  Основными  подходами  к  происхождению  сознания  были  следующие:
Платон: «тело  человека  — вместилище  бессмертной  души и ее  раб,  бестелесная душа управляет всем  во
вселенной».  Христианство:  «разум  человека,  его  мышление  —  искорка  божественного  разума.  Именно  он
мыслит, желает, чувствует в человеческом сознании». Декарт: ’’Сознание — внепространственная субстанция,
впервые  расnматривает  проблему  самосознания’’.  Гегель:  С.  —  одно  из  воплощений  всемирного  разума’’.
Впервые рассматривает социально — историческую природу сознания, говорит о принципе историзма. В ХХ
веке возникает теория отражения. Согласно этой теории сознание это высщая форма отражения. Таким образом
сознание формируется деятельностью чтобы затем влиять на эту деятельность, определяя и регулируя ее.
Сознание  структурно  организованно,  представляет  систему  элементов,  находящихся  между  собой  в
закономерных  отношениях.  В  структуре  сознания  наиболее  отчетливо  выделяются  такие  элементы  как
осознание вещей, а также переживание, т.е. отношение к содержанию того что отражается. Развитие сознания
предполагает  прежде  всего  обогащение  его  новыми  знаниями.  Познание  вещей  имеет  разные  уровни
проникновения и степень ясности понимания. Отсюда обыденное, философское, научное, и.т.д. осознание мира
а также чувственный и рациональный уровень сознания. Ощущения, понятия, восприятия, мышление образуют
ядро  сознания.  Но  они  не  исчерпывают  всей  структурной  полноты  сознания:  оно  включает  в  себя  и  акт
внимания как свой необходимый компонет. Именно благодаря сосредоточенности внимания определенный круг
объектов находится в фокусе сознания. Воздействующие на нас предметы, события вызывают у нас не только
познавательные  образы,  но  и  эмоции.  Богатейшая  сфера  эмоциональной  жизни  человека  включает  в  себя
собственно чувства, настроение, или эмоциональное самочувствие и аффекты (ярость, ужас и.т.д.). Чувства,
эмоции  суть  компоненты  сознания  Сознание  не  ограничивается  познавательными  процессами,
направленностью на обьект, эмоциональной сферой. Наши намерения претворяются в жизнь благодаря усилиям
воли. Однако сознание — это не сумма множества составляющих его элементов, а их интегральное сложно
структурированное целое.
Наиболее  общеизвестное  определение  сознания  можно  сформулировать  следующим  образом.  Сознание
является комбинацией следующих аспектов:
1.Восприятие и ощущения 
2.Абстрактное мышление
3.Память
4.Воображение
5.Эмоции 
6.Воля (свобода выбора)
Личность – целостная комбинация этих категорий. Для человека характерно наличие и развитость абстрактного
мышления  и  воображения.  Существуют  два  различных  подхода  к  изучению  сознания:  физиологический  и
психический. 
Совершенно очевидно, что мозг – материальный субстрат сознания. И возникает извечный вопрос – в каком
отношении  находятся  сознание  и  мозг?  Всегда  существовали  материалистическая  и  идеалистическая
тенденции. 
7.Демокрит.  Первая  материалистическая  тенденция.  Он  говорил,  что  сознание  –  это  сферические  атомы, 
сдерживаемые  в  теле  человека  внешним  воздухом.  Бесспорно,  чистый  материализм.  Мысли  объяснялись
движением атомов. Потом, гораздо позже, в эпоху Возрождения появилась теория, что мысли – выделения мозга
(типа желчи). Еще позже появилась концепция биотоков и т.п. Например, Павлов говорил, что если бы череп
был прорачный,  а возбужденные  клетки  светились,  то мы видели  бы чередование  мерцание  –  мысли.  Он
отождествлял психические и физиологические процессы 
8.Идеалитисческая (?) модель. Гласит, что психические процессы протекают параллельно физиологическим, что
есть душа человека и она бессмертная. Мысль – не функция мозга, и носителем психики является идеальная
субстанция – наподобие духовной субстанции Спинозы, которая есть в человеке помимо материальной. Но
психические процессы связаны с физиологическими т.к. по мере усложнения нервной системы усложняются
психические процессы (имеется в виду переход ребенок - взрослый)
Как бы там ни было нельзя ни отождествлять физиологию с психикой, не разделять их. Их надо рассматривать в
тесной взаимосвязи, природа которой до сих пор не ясна.
Можно вести критерии сознания, свидетельствующие о его наличии. Критерии сознания следующие:
9.Способность отражения окружающего мира (но для этого должен быть эталон, для сравнения)
10.Способность самоотражения
11.Оценка окружающего мира 
12.Самооценка (самооценкой называется применение моральных законов к самому себе. Это присуще только
человеку.)
13.Свобода выбора (на основе оценки окружающего мира и самооценки)
Таким образом мы видим, что сознание человека по меньшей мере пятигранно. Ганди принадлежат слова:
“Душа человека подобна драгоценному камню, на каждой грани которого горят огни…”
Происхождение сознания – не только продукт биологической эволюции но и социальной в том числе.

\newpage
\section{Сознание и бессознательное (Фрейд и Юнг)}
С  вопросами  биологического  и  социального,  сущности  и  существования  тесно  связана  и  проблема
бессознательного и сознательного в философской антропологии, отражающая важную сторону существования
человека.
Длительное  время  в  философии  доминировал  принцип  антропологического  рационализма  —  человек,  его
мотивы поведения и само бытие рассматривались только как проявление сознательной жизни. Этот взгляд
нашел свое яркое воплощение в знаменитом картезианском тезисе «cogito ergo sum» («мыслю, следовательно,
существую»). Человек в этом плане выступал лишь как «человек разумный». Но, начиная с Нового времени, в
философской антропологии все большее место занимает проблема бессознательного.
Лейбниц, Кант, Кьеркегор, Гартман, Шопенгауэр, Ницше с разных сторон и позиций начинают анализировать
роль и значение психических процессов, не осознающихся человеком.
Но определяющее влияние на разработку этой проблемы оказал 3. Фрейд, открывший целое направление в
философской антропологии и утвердивший бессознательное как важнейший фактор человеческого измерения и
существования. Он представил бессознательное как могущественную силу, которая противостоит сознанию.
Согласно его концепции, психика человека состоит из трех пластов. Самый нижний и самый мощный слой —
«Оно» (Id) находится за пределами сознания. По своему объему он сравним с подводной частью айсберга. В
нем  сосредоточены  различные  биологические влечения и страсти,  прежде  всего сексуального  характера, и
вытесненные из сознания идеи. Затем следует сравнительно небольшой слой сознательного — это «Я» (Ego)
человека. Верхний пласт человеческого духа — «Сверх-Я» (Super Ego) — это идеалы и нормы общества, сфера
долженствования  и  моральная  цензура.  По  Фрейду,  личность,  человеческое  «Я»  вынуждено  постоянно
терзаться и разрываться между Сциллой и Харибдой — неосознанными осуждаемыми «Оно» и нравственно-культурной цензурой «Сверх-Я». Таким образом, оказывается, что собственное «Я» — сознание человека — не
является  «хозяином  в  своем  собственном  доме».  Именно  сфера  «Оно»,  всецело  подчиненная  принципу
удовольствия и наслаждения, оказывает, по Фрейду, решающее влияние на мысли, чувства и поступки человека.
Человек — это прежде всего существо, управляемое и движимое сексуальными устремлениями и сексуальной
энергией (либидо).
Драматизм  человеческого  существования  у  Фрейда  усиливается  тем,  что  среди  бессознательных  влечений
имеется и врожденная склонность к разрушению и агрессии, которая находит свое предельное выражение в
«инстинкте смерти», противостоящем «инстинкту жизни». Внутренний мир человека оказался, следовательно,
еще и ареной борьбы между двумя этими влечениями. В конце концов Эрос и Танатос рассматриваются им как
две наиболее могущественные силы, определяющие поведение человека.
Таким образом, фрейдовский человек оказался сотканным из целого ряда противоречий между биологическими
влечениями и сознательными социальными нормами, сознательным и бессознательным, инстинктом жизни и
инстинктом смерти. Но в итоге биологическое бессознательное начало оказывается у него определяющим.
Человек, по Фрейду, — это прежде всего эротическое существо, управляемое бессознательными инстинктами.
Проблема бессознательного интересовала и швейцарского психиатра К.-Г. Юнга. Однако он выступил против
трактовки человека как существа эротического и попытался более глубоко дифференцировать фрейдовское
«Оно». В частности, Юнг выделил в нем помимо личностного бессознательного, как отражение в психике
индивидуального  опыта,  еще  и  более  глубокий  слой  —  коллективное  бессознательное,  которое  является 
отражением опыта предшествующих поколений. Содержание коллективного бессознательного составляют, по
Юнгу, общечеловеческие первообразы — архетипы (например, образ матери-родины, народного героя, богатыря
и  т.д.).  Совокупность  архетипов  образует  опыт  предшествующих  поколений,  который  наследуется  новыми
поколениями.  Архетипы  лежат  в  основе  мифов,  сновидений,  символики  художественного  творчества.
Сущностное  ядро  личности  составляет  единство  индивидуального  и  коллективного  бессознательного,  но
основное  значение  имеет  все-таки  последнее.  Человек,  таким  образом,  —  это  прежде  всего  существо
архетипное.
Проблема  бессознательного  и  сознательного  развивалась  и  другими  представителями  психоанализа  —
последователями  Фрейда,  которые  уточняли  и  развивали  его  учение,  внося  в  него  свои  коррективы.  Так,
австрийский  психиатр  А.  Адлер  подверг  критике  учение  Фрейда,  преувеличивающего  биологическую  и
эротическую  детерминацию  человека.  По  Адлеру,  человек  —  не  только  биологическое,  но  и  социальное
существо,  жизнедеятельность  которого  связана  с  сознательными  интересами,  поэтому  «бессознательное  не
противоречит сознанию», как это имеет место у Фрейда. Таким образом, Адлер в определенной степени уже
социологизирует  бессознательное  и  пытается  снять  противоречие  между  бессознательным  и  сознанием  в
рассмотрении человека.
Американский неофрейдист, социальный психолог и социолог Э. Фромм выступил против биологизации и
эротизации бессознательного и подверг критике теорию Фрейда об антагонизме между сущностью человека и
культурой.  Но  вместе  с  тем  он  отверг  и  социологизаторские  трактовки  человека.  По  его  собственному
признанию,  его  точка  зрения  является  «не  биологической,  и  не  социальной».  Одним  из  наиболее  важных
факторов  развития  человека,  по  Фромму,  является  противоречие,  вытекающее  из  двойственной  природы
человека,  который  является  частью  природы  и  подчинен  ее  законам,  но  одновременно  это  и  субъект,
наделенный разумом, существо социальное. Это противоречие он называет экзистенциальной дихотомией. Она
связана  с  тем,  что  ввиду  отсутствия  сильных  инстинктов,  которые  помогают  в  жизни  животным,  человек
должен принимать решения, руководствуясь своим сознанием. Но получается так, что результаты при этом не
всегда оказываются продуктивными, что порождает тревогу и беспокойство. Поэтому «цена, которую человек
платит за сознание», — это неуверенность его.

\newpage
\section{Субъект познания, основные концепции субъекта}
Под  субъектом  познания  следует  понимать  наделенного  сознанием  человека,  включенного  в  систему
социокультурных связей, чья активность направлена на постижение тайн противостоящего ему объекта. 
Основные концепции субъекта
   * Психологический субъект познания (изолированный субъект). В данной концепции субъект напрямую
отождествляется с человеческим индивидом, осуществляющим познавательный акт. Такая позиция близка к
нашему  повседневно-реалистическому  опыту  и  наиболее  распространена.  В  рамках  данной  концепции
познающий чаще всего рассматривается как пассивный регистратор внешних воздействий, с той или иной
степенью адекватности отражающий объект. Такой подход не учитывает активный и конструктивный характер
поведения субъекта — того, что последний способен не только отражать, но и формировать объект познания.
   * Трансцедентальный субъект познания. Данная концепция утверждает, что существует инвариантное и
устойчивое "познавательное ядро" в каждом человеке, которое обеспечивает единство познания в контексте
различных  эпох  и  культур.  Его  выявление  составляет  важную  часть  всей  теоретико-познавательной
деятельности. Такая трактовка субъекта восходит к И.Канту
    *  Коллективный  субъект  познания  практически  реализуется  посредством  совместных  усилий  многих
индивидуальных психологических субъектов. Такой субъект не сводится к простой сумме индивидуальных
субъектов  и  относительно  от  них  автономен.  Примером  подобного  субъекта  может  служить  научно-исследовательский коллектив, профессиональное сообщество или даже человеческое общество в целом.

\newpage
\section{Объект познания, основные концепции объекта}
Объект — это та сторона действительности, на которую направлено познание.
Объект  познания  обычно  определяется  путем  выделения  части  объективной  реальности,  вовлеченной  в
человеческую производственную и познавательную деятельность. Однако понятия “объект” и “объективная
реальность” не вполне правомерно рассматривать на основе включения одного в другое. Если принять во
внимание,  что  человек,  человечество,  которые  обычно  характеризуются  как  субъект,  сами  включаются  в
процесс  познания  в  качестве  объекта,  то  правомерность  указанной  выше  логической  операции  станет
сомнительной.  Не  всякая  объективная  реальность  есть  объект  познания,  но  и  не  всякий  объект  познания
является объективной реальностью.
Далее.  Категория  объекта  часто  рассматривается  как  соотносительная  категории  субъекта.  Эта
соотносительность  не  обозначает  зависимости  существования  объективной  реальности  от  существования
субъекта. Но от существования и деятельности субъекта зависит вовлеченность части объективной реальности в
качестве объекта в его практику и познание. Следовательно, объективная реальность и объект познания не
тождественны.
Несовпадение объекта познания с объективной реальностью не только в том, что он представляет собой лишь
ее  часть,  но  и  в  том,  что  логическое  познание,  составляющее  необходимую  сторону  познания  вообще,
направлено на исследование идеализированных систем, построенных с помощью абстрагирования отдельных
сторон, свойств и отношений реальности. Это так называемые вторичные объекты. Е. К. Войшвилло дает
характеристику такого вторичного объекта: “Объектом мысли могут быть и сама мысль (понятие, суждение),
представление, образ вообще, и даже отдельные стороны, свойства мысли, образа (логическая форма мысли,
знаковая форма ее выражения)...”.30 В результате образования таких идеализированных систем происходит
определенное расхождение объекта познания и объективной реальности. Известно, что геометрия, механика и
др. науки изучают свойства объектов, которых как таковых в данном виде в самой действительности нет.
Исходя  из  сказанного,  следует  заключить,  что  объективная  реальность  лишь  в  конечном  счете  является
объектом познания. Если попытаться более полно и конкретно очертить объект познания, то следует выделить
относительно самостоятельные его стороны. Сюда входят: 1) предметы и явления природы, вовлеченные в
сферу субъективной деятельности в широком смысле, как практической, так и теоретической, познавательной;
2) человеческие отношения, общество во всех его аспектах; 3) отношение знания к явлениям материального
мира; 4) отношение средств выражения знания (знаковых систем) к знанию; 5) отношение средств выражения
знания (знаковых систем) к явлениям материального мира; 6) отношение элементов знания между собой; 7)
отношение  средств  выражения  знания  (знаковых  систем)  между  собой;  8)  отношение  средств  познания  к
явлениям материального мира; 9) отношение средств познания к знанию и т.п. Разумеется, это неполная и самая
общая градация, нуждающаяся в развитии и дальнейшей конкретизации. Таким образом, лишь в конечном
счете, в пределе объектом познания выступают предметы и явления Материального мира.
Сфера  объекта  познания  по  мере  развития  практической  и  познавательной  деятельности  постоянно
расширяется.  Познание,  особенно  научное,  в  не  меньшей  степени  является  фактором  расширения  сферы
объекта, нежели практика, хотя последняя и является исходом, началом и основой выделения объекта и для
познания. Выделение объекта познания, превращение субъекта в объект познания произошло одновременно с
возникновением общественного сознания, когда человек стал отделять себя от окружающей среды. Касаясь этой
особенности  человека,  К.  Маркс  писал:  “Животное  непосредственно  тождественно  со  своей
жизнедеятельностью. Оно не отличает себя от своей жизнедеятельности. Оно и есть эта жизнедеятельность.
Человек же делает сам свою жизнедеятельность предметом своей воли и своего сознания”.31
Объект  и  субъект  рассматриваются  в  познании  как  некие  противоположности.  Но  они  определены  и
отграничены не раз и навсегда. В разных отношениях, в разных аспектах, в разное историческое время они
могут выступать и в качестве субъекта, и в качестве объекта, конечно, исключая объективную реальность,
материальный мир, который не может быть субъектом в той своей части, которая не относится к человеку, к
обществу.
Идея противоречивого тождества субъекта и объекта, в абстрактной форме заявленная Гегелем и чрезвычайно
важная для понимания истины как процесса, получила свое онтологическое обоснование в категории практики,
выдвинутой диалектическим материализмом. Термин “практика” имеет емкое содержание. Это прежде всего
материальный  процесс  специфического  “обмена  веществ”  общества  и  природы,  предметно-чувственная
деятельность человека. Особенность данного процесса в том, что он включает в себя идеальный момент, ибо
осуществляется  людьми,  обладающими  сознанием,  имеющими  свои  цели,  планы,  представляющими
технологию  их  реализации,  прогнозирующими  результаты.  Наконец,  практика  —  процесс  социально-исторический,  не  только  потому,  что  его  субъектом  выступает  не  отдельное  существо,  а  организованное
человеческое  сообщество,  но  и  потому,  что  данное  сообщество  носит  исторический  характер.  Практика
воплощает в себе единство живого и овеществленного труда, выражающего преемственность развивающейся
материальной и идеальной культуры, субъектом которой выступает определенным образом обобществленное
человечество.
Категория практики имеет сложную структуру. Она включает в себя целый ряд необходимых соотношений —
идеального  и  материального,  индивидуального  и  социального,  прошлого  и  будущего,  преемственности  и
новизны. Указанные соотношения полностью теряют свой смысл и значение без основной диалектической
связи — субъекта и объекта.
В  современной  философии  эта  связь  оказалась  предметом  всеобъемлющей  критики.  В  постмодернизме
возникла тенденция размывания субъект-объектной оппозиции, на том якобы основании, что она перестает
выполнять роль ведущей оси, организующей мыслительное пространство. Теряется значение крайних терминов
оппозиции,  особенно  субъекта.  По  свидетельству  французского  философа  М.  Фуко,  понятие  “субъект”
употребляется разве что как своеобразная дань классической традиции. На самом деле можно говорить только о
том, что в определенных условиях некий индивид выполняет функцию субъекта. Постмодернистская культура,
по сути, декларирует бессубъектную философию. Однако “новизна”, выраженная в метафоре “смерть субъекта”,
представляет  тупиковую  ветвь  философской  эволюции.  Не  случайно  “After  постмодернизм”  (поздний
постмодернизм) заговорил о “возрождении субъекта”.
Категория практики, выступающая в своей онтологической функции, выражающая основу и способ бытия
человека в мире, указывает на немыслимость игнорирования философией понятия субъект. И именно практика,
представляющая  собой  систему  определенного  множества  деятельностей,  обусловливает  относительную
самостоятельность  Развития  ее  (практики)  идеальных  и  материальных  сторон,  индивидуального  и
общественного  в  реальном  процессе  преобразования  природы,  общества  и  человека,  субъективного  и
объективного.

\newpage
\section{Познание как «отражение» и как «конструирование» действительности}
Чувственное и эмпирическое познание - не одно и то же. Чувственное знание - это знание в виде ощущений и
восприятии свойств вещей, непосредственно данных органам чувств
Эмпирическое знание может быть отражением данного не непосредственно, а опосредованно. Например, я
вижу  показание  прибора  или  кривую  электрокардиограммы,  информирующие  меня  о  состоянии
соответствующего  объекта,  которого  я  не  вижу.  Иначе  говоря,  эмпирический  уровень  познания  связан  с
использованием  всевозможных  приборов;  он  предполагает  наблюдение,  описание  наблюдаемого,  ведение
протоколов, использование документов, например историк работает с архивами и иными источниками. Словом,
это более высокий уровень познания, чем просто чувственное познание.
Исходным  чувственным  образом  в  познавательной  деятельности  является  ощущение  -  простейший
чувственный образ, отражение, копия или своего рода снимок отдельных свойств предметов. Многообразие
ощущений верно отображает объективный характер качественного многообразия мира и вызвано им. Потеря
способности ощущать неизбежно влечет за собой потерю сознания.
Специфичность органов чувств не только не препятствует правильному познанию внешнего мира, как это
пытались представить “физиологические” идеалисты, но, напротив, она обеспечивает наиболее полное и точное
отражение объективных свойств предметов.
Положение  об  ощущении  как  субъективном  образе  объективного  мира  направлено  своим  острием  против
механического деления качеств на первичные и вторичные. С этой точки зрения первичные качества (форма,
объем и т.д.) являются отражением объективно существующих особенностей предметов, а вторичные (цвет, звук
и т.д.) носят чисто субъективный характер. Отсюда ложный вывод: цвет, запах - это не свойства предметов, а
наши ощущения (Э.  Мах);  словом  “цвет” обозначается  определенный  класс психических переживаний  (В.
Оствальд). Мир же беззвучен, лишен красок, запахов.
Тот  или  иной  предмет  воздействует  на  органы  чувств  человека  какое-то  определенное  время.  Затем  это
воздействие  прекращается.  Но  образ  предмета  не  исчезает  сразу  же  бесследно.  Он  запечатлевается  и
сохраняется в памяти. Следовательно, мыслить что-то можно и по его исчезновении: ведь о нем остается
определенное представление. Душа обретает возможность оперировать образами вещей, не имея их в поле
чувственного восприятия.
Можно ли рассуждать о познании, игнорируя память? Конечно, нет: душа без памяти - что сеть без рыбы.
Никакое познание немыслимо без этого чудесного феномена. Процессы ощущения и восприятия оставляют
после себя “следы” в мозгу, суть которых состоит в способности воспроизводить образы предметов, которые в
данный  момент  не  воздействуют  на  человека.  Память  играет  очень  важную  познавательную  роль.  Она
объединяет прошедшее и настоящее в одно органическое целое, где имеется их взаимное проникновение.
Человек может творчески комбинировать и относительно свободно создавать новые образы. Представление -это  промежуточное  звено  между  восприятием  и  теоретическим  мышлением.  Познание  невозможно  без
воображения:  оно  есть  свойство  человеческого  духа  величайшей  ценности.  Воображение  восполняет
недостаток наглядности в потоке отвлеченной мысли. Сила воображения не только снова вызывает имеющиеся
в опыте образы, но и связывает их друг с другом и, таким образом, поднимает их до общих представлений.
Воспроизведение образов осуществляется силой воображения произвольно и без помощи непосредственного
созерцания.
Люди стремятся познать то, чего они еще не знают. Но для начала они должны, хотя бы в самом общем виде,
знать, чего же они не знают и что они хотят знать. “Не всякий знает, как много надо знать, чтобы знать, как мало
мы знаем”, - гласит восточное изречение.
Правильная  постановка  проблемы  направляет  поиски  ее  решения.  Пытаться  найти  решение  поставленной
проблемы можно двумя путями: искать нужную информацию в существующей литературе или самостоятельно
исследовать проблему с помощью наблюдений, экспериментов и теоретического мышления.
Важными методами исследования в науке, особенно в естествознании, являются наблюдение и эксперимент.
Наблюдение представляет собой преднамеренное, планомерное восприятие, осуществляемое с целью выявить
существенные свойства  и отношения  объекта  познания.  Эксперимент  это  метод  исследования, с  помощью
которого объект или воспроизводится искусственно, или ставится в определенные условия, отвечающие целям
исследования.  Особую  форму  познания  составляет  мысленный  эксперимент,  который  совершается  над
воображаемой моделью. Для него характерно тесное взаимодействие воображения и мышления. Основным
методом эксперимента является метод изменения условий, в которых обычно находится исследуемый предмет.
Он дает возможность вскрыть причинную зависимость между условиями и свойствами исследуемого объекта, а
также характер изменения этих свойств в связи с изменением условии.
В  ходе  и  в  результате  наблюдения  и  эксперимента  осуществляется  описание  или  протоколирование.  Оно
производится  и  в  виде  отчета  с  использованием  общепринятых  терминов,  и  наглядным  образом  в  виде
графиков, рисунков, фото- и кинопленок, и символически в виде математических, химических формул и т.п.
Основное научное требование к описанию - это достоверность, точность воспроизведения данных наблюдений
и эксперимента. 

\newpage
\section{Формы чувственного и рационального знания}
Итак, ощущение есть приблизительно верное отражение действительности. Ощущение предполагает наличие
ощущаемого, образ вещи не может быть без самой вещи, отражение без отражаемого. Ощущение есть первая
форма чувственной ступени познания. Но чувственное познание развивается диалектически от простого к более
сложному , не заканчивается на стадии ощущений.
На  основе  ощущений  возникают  восприятия,  являющиеся  более  высокой  и  более  сложной  формой
чувственного познания. Если ощущение отражает отдельные свойства вещей, то восприятие отражает предмет
как совокупность его объективных свойств - форму, цвет, запах, вкус и т. п. В восприятии различные ощущения
находятся  в  единстве,  в  обобщенном  виде.  Восприятие  является  результатом  совокупности  отражательной
деятельности нескольких органов чувств, давая нам поэтому более полное знание о вещах, чем ощущение.
От восприятия чувственных познаний идем к представлениям. Здесь уже нет непосредственной связи с вещами,
так  как  представления  возникают  на  основе  прошлых  ощущений  и  восприятий.  Представление-это
воспроизведенные памятью образы предметов или процессов, основанные на прошлом опыте. Так, например,
мы  не  можем  в  данный  момент  непосредственно  ни  ощущать,  ни  воспринимать  новый  электровоз,  но
представить себе его можем, поскольку видели его раньше. Представления создаются также на основе изучения
снимков, чертежей и т. д. Представление, оставаясь формой чувственного познания, является в то же время
первым  шагом  по  пути  к  абстрактному  мышлению.  Таким  образом,  в  ощущениях,  восприятиях  и
представлениях человек приблизительно верно отражает предметы и явления окружающего мира и тем самым
получает достоверные знания.
Познание есть поступательный процесс. Оно развивается по восходящей линия от простого к сложному, от
низшего к высшему. В процессе познания человек идет от явления к сущности, от познания поверхностных
явлений к пониманию глубинных процессов и внутренне присущих им законов. Однако познание сущности и
законов возможно только на ступени, абстрактного, логического мышления.
Процесс познания есть бесконечный процесс восхождения от живого созерцания к абстрактному мышлению.
Чувственное  и  рациональное  познание  находятся  в  неразрывном  единстве.  Оба  эти  моментах  познании
являются  отражением  внешнего  мира  в  сознании  человека.  Содержанием  чувственного  и  рационального
познания является материальный мир. Живое созерцание и абстрактное мышление есть лишь различные формы
познания одной и той же объективной реальности.
Без  живого  созерцания  немыслимо  абстрактное  мышление.  В  свою  очередь  без  абстрактного  мышления
познание  материального  мира-невозможно.  Органы  чувств  дают  нам  достоверные  знания  о  материальных
вещах  только  в  неразрывной  связи  с  мышлением  и  практикой.  Наоборот,  абстрактное  мышление  без
чувственного познания и практики не может дать истинных знаний.
Признавая неразрывное единство чувственного и рационального познания, диалектический материализм вместе
с  тем  считает,  что  оба  эти  момента  познания  качественно  различны.  Это  различие  состоит  в  том,  что
чувственное  познание  дает  нам  поверхностные  сведения  о  вещах,  а  рациональное  познание  раскрывает
сущность и закономерности их развития. Переход от живого созерцания .к абстрактному мышлению является
качественным скачком. Этот скачок знаменует собой переход от познания явления к познанию сущности.
Процесс познания и получаемые в нем знания в ходе исторического развития практики и самого познания все
более  дифференцируется  и  воплощается  в  различных  своих  формах.  Последние  хотя  и  связаны,  но  не
тождественны одна другой, каждая из них имеет свою специфику. Выделим основные формы познания в их
историческом поступательном процессе.
На  ранних  этапах  истории  существовало  обыденно-практическое  познание,  поставлявшее  элементарные
сведения о природе, а также о самих людях, условиях их жизни, общении и т. д. Основой данной формы
познания  был  опыт  повседневной  жизни,  практики  людей.  Полученные  на  эти  базе  знания  носят  хотя  и
прочный, но хаотический, разрозненный характер, представляя собой простой набор сведений, правил и т. п.
Сфера  обыденного  познания  многообразна.  Она  включает  в  себя  здравый  смысл,  верования,  приметы,
первичные  обобщения  наличного  опыта,  закрепляемые  в  традициях,  преданиях,  назиданиях  и  т.  п.,
интуитивные убеждения, предчувствия и пр.
Одна из исторически первых форм - игровое познание как важный элемент деятельности. В ходе игры индивид
осуществляет активную познавательную деятельность, приобретает большой объем новых знаний, впитывает в
себя богатство культуры -деловые игры, спортивные игры, игра актеров и т. п.
В настоящее время понятие игры широко используется в математике, экономике, кибернетике и других науках.
Здесь все чаще применяются специальные игровые модели и игровые сценарии, где проигрываются различные
варианты течения сложных процессов и решения научных и практических проблем.
Важную  роль,  особенно  на  начальном  этапе  истории  человечества,  играло  мифологическое  познание.  Его
специфика в том, что оно представляет собой фантастическое отражен»» реальности, является бессознательно-художественной переработкой природы и общества народной фантазией. В рамках мифологии вырабатывались
определенные  знания  о природе,  космосе,  о самих  людях, их условиях бытия,  формах  общения и т. д.  В
последнее время было выяснено (особенно в философии структурализма), что мифологическое мышление - это
не  просто  безудержная  игра  фантазии,  а  своеобразное  моделирование  мира,  позволяющее  фиксировать  и
передавать опыт поколений. Так, Леви-Строс указывал на конкретность и метафоричность мифологического
мышления, его способность к обобщению, классификациям и логическому анализу.
Некоторые  современные  исследователи  полагают,  что  в  наше  время  значение  мифологического  познания
отнюдь нс уменьшается. Так, П. Фейерабенд убежден, что достижения мифа несравненно более значительны
чем научные: изобретатели мифа, по его мнению, положили начало культуре, в то время как рационалисты
только изменяли ее, причем не всегда в лучшую сторону.
Уже  в  рамках  мифологии  зарождается  художественно-образная  форма  познания,  которая  в  дальнейшем
получила наиболее развитое выражение в искусстве. Хотя оно специально и не решает познавательные задачи,
но содержит в себе достаточно мощный гносеологический потенциал. Более того, например, в герменевтике
искусство считается важнейшим способом раскрытия истины. Хотя, конечно, художественная деятельность
несводима целиком к познанию, но познавательная функция искусства посредством системы художественных
образов -одна из важнейших для него. Художественно осваивая действительность в различных своих видах
(живопись, музыка, театр и т. д.), удовлетворяя эстетические потребности людей, искусство одновременно
познает мир, а человек творит его - в том числе и по законам красоты. В структуру любого произведения
искусства всегда включаются в той или другой форме определенные знания о разных людях и их характерах, о
тех или иных странах и народах, их обычаях, нравах, быте, об их чувствах, мыслях и т. д.
Одними  из  древних  форм  познания,  генетически  связанными  с  мифологией,  являются  философское  и
религиозное познание. Особенности последнего определяются тем, что оно обусловлено непосредственной
эмоциональной  формой  отношения  людей  к  господствующим  над  ними  земными  силами  (природными  и
социальными). Будучи фантастическим отражением последних, религиозные представления содержат в себе
определенные  знания  о  действительности,  хотя  нередко  и  превратные.  Достаточно  мудрой  и  глубокой
сокровищницей  религиозных  и  других  знаний,  накопленных  людьми  веками  и  тысячелетиями,  являются,
например, Библия и Коран. Однако религия (как и мифология) не воспроизводили знание в систематической и,
тем  более,  теоретической  форме.  Она  никогда  не  выполняла  и  не  выполняет  функции  производства
объективного знания, носящего всеобщий, целостный, самоценностный и доказательный характер.
Говоря о формах знания, нельзя обойти вниманием достаточно известную (особенно в современной западной
гносеологии)  концепцию  личностного  знания.  Знание  по  этой  концепции  -  это  активное  постижение
познаваемых вещей, действие, требующее особого искусства и особых инструментов. Поскольку науку делают
люди,  то  получаемые  в  процессе  научной  деятельности  знания  (как  и  сам  этот  процесс)  не  могут  быть
деперсонифицированными. Личностное знание необходимо предполагает интеллектуальную самоотдачу. В нем
запечатлена не только познаваемая действительность, но сама познающая личность, ее заинтересованное (а не
безразличное) отношение к знанию, личный подход к его трактовке и использованию, собственное осмысление
его  в  контексте  специфических,  сугубо  индивидуальных,  изменчивых  и,  как  правило,  неконтролируемых
ассоциаций.
Личностное знание - это не просто совокупность каких-то утверждений, но и переживание индивида. Личность
живет в нем «как в одеянии из собственной кожи», а не просто констатирует его существование. Тем самым в
каждом акте познания присутствует .страстный вклад познающей личности, и что эта добавка не свидетельство
несовершенства,  но  насущно  необходимый  элемент  знания.  Но  такая  добавка  не  делает  последнее  чисто
субъективным.
В  настоящее  время усиливается  интерес к  проблеме  иррационального, т. е.  того, что лежит  за  пределами
досягаемости разума и недоступно постижению с помощью известных рациональных средств, и вместе с тем
все более укрепляется убеждение в том, что наличие иррациональных пластов в человеческом духе порождает
ту  глубину,  из  которой  появляются  все  новые  смыслы,  идеи,  творения.  Взаимопереход  рационального  и
иррационального - одно из фундаментальных оснований процесса познания. Однако значение внерациональных
факторов не следует преувеличивать, как это делают сторонники иррационализма.
Типологизация форм самого знания может быть проведена по самым различным основаниям (критериям). В
этой связи выделяют,  например,  знания рациональные и  эмоциональные, феноменолистские  (качественные
концепции)  и  эссенцианалистские  (сооруженные  в  основном  количественными  средствами  анализа),
эмпирические и теоретические, фундаментальные и прикладные, философские и чистнонаучные, естественно-научные и гуманитарные и т. д.
Специфическими формами или методами познания являются наблюдение, описание и т. д. На этой ступени
познавательного процесса происходит сбор фактов, фиксирующих внешние проявления, свойства предметов.
Теоретическая форма познания- это углубление человеческой мысли в сущность явлений действительности.
При этом научное познание пользуется такими методами, как моделирование, создание гипотез, теорий и т. д.
Действительность отражается человеком с помощью различных форм познавательной деятельности.
Таким образом человек ничего не может знать о предметах и явлениях внешнего мира без того материала,
который  он  получает  от  органов  чувств  в  формах  чувственного  созерцания  (ощущение,  восприятие,
представление).  Они  дают  чувственно  наглядное  воспроизведение  предметов,  их  отдельных  свойств.
Существенные  связи  между  предметами,  закономерности  их  развития  отражаются  в  формах  абстрактного
мышления - понятиях, суждениях, умозаключениях. В процессе познания человек использует также различные
логические приемы (анализ и синтез, дедукцию, индукцию и т. д.), позволяющие ему теоретически воссоздавать
изучаемый предмет.

\newpage
\section{Мышление и язык. Основные функции языка}
Важно отметить, что связи мыслительных процессов с лингвистическими структурами, широко обсуждается
сегодня  представителями  различных  школ  и  учений  философии  -  структурализмом,   постпозитивизмом
(лингвистический позитивизм), герменевтикой и др. Представители постпозитивизма обсуждают, как правило,
отношения  между  мышлением  и  языком  в  рамках  проблемы  духовного  и  телесного  (“ментального”  и
“физического”).  Одним  из  наиболее  активных  защитников  идеи  “экстралингвистического  знания”  является
К.Хуккер.  Он  исходит  из  того,  что  лингвистические  структуры  -  это  подкласс  информационных  структур,
поэтому недопустимо, по его мнению, отождествлять мысль и речь. Справедливо отмечая, более широкий
характер информационных структур по сравнению с лингвистическими К.Хуккер склонен к абсолютизации их,
придания  им  статуса  бытийности.  Из  этой  идеи  исходит  и  другая  идея  постпозитивизма  -  о  тождестве
“ментального” и “физического”, эту идею пропагандируют  “элининативныке материалисты”. Они полагают,
что “ментальные термины” теории языка и мышления должны быть элиминированы, как ненаучные и заменены
терминами нейрофизиологии. Чтобы решить эту задачу, нужно, прежде всего, как они полагают, отвергнуть
“миф  данного”,  т.е.  утверждение  о  том,  что  мы  располагаем  некоторым  непосредственным  и  мгновенным
знанием  о  собственных  “ментальных”  процессах.  Пожалуй,  самым   решительным  образом  отрицает
“непосредственно  данное”  П.Фейерабент.  По  его  убеждению, “непосредственно  данное”  является  вовсе не
фактом природы, а “результатом того способа, которым любой род занятий (или мнения) относительно сознания
воплощен  и  воплощается  в  языке”.  Этот  “якобы  факт  природы”  есть  типичная  кажимость,  обусловленная
“бедностью содержания ментальных терминов по сравнению с физическими терминами”/
Заметим, что отрицание “непосредственного данного” означает, что знание существует только тогда, когда оно
вербализовано, т.е. выражено словами. Если этот факт “непосредственно данного” признается, то вопрос о его
отношении  языку  и  речи  решается  по-разному.  Встречается  точка  зрения,  что  непосредственное  знание  о
собственных сознательных состояниях всегда так или иначе вербализовано.  Например, Г.Фейгл говорит о
наличии  сугубо  личного  языка,  с  помощью  которого  субъект  выражает  для  себя  указанное  знание.
Непосредственное знание, прямой опыт он называет “сырыми чувствами”. Последние и выступают в форме
“личного  языка”,  который  в  процессе  общения  переводится  на  интерсубъективный,  обыденный  язык.  Эти
примеры  свидетельствуют  о  междисциплинарном  характере  проблемы  соотношения  языка  и  мышления  и
возможности различных трактовок этого взаимодействия.
Язык - главная из знаковых систем человека, важнейшее средство человеческого общения. К. Маркс, например,
назвал язык “непосредственной действительностью мысли”. С помощью слов можно интерпретировать другие
знаковые системы (например, можно описать картину). Язык - универсальный материал, который используется
людьми при объяснении мира и формировании той или иной его модели. Хотя художник может это сделать и
при помощи зрительных образов, а музыкант - при помощи звуков, но все они вооружены, прежде всего,
знаками универсального кода - языка. 
Язык - это особая знаковая система. Любой язык состоит из различных слов, то есть условных звуковых знаков,
обозначающих  различные  предметы  и  процессы,  а  также  из  правил,  позволяющих  строить  из  этих  слов
предложения.  Именно  предложения  являются  средством  выражения  мысли.  С  помощью  вопросительных
предложений люди спрашивают, выражают свое недоумение или незнание, с помощью повелительных - отдают
приказы, повествовательные предложения служат для описания окружающего мира, для передачи и выражения
знаний о нем. Совокупность слов того или иного языка образует его словарь. Словари наиболее развитых
современных языков насчитывают десятки тысяч слов. С их помощью благодаря правилам комбинирования и
объединения слов в предложения можно написать и произнести  неограниченное количество осмысленных
фраз, заполнив ими сотни миллионов статей, книг и файлов. В силу этого язык позволяет выражать самые
разные мысли, описывать чувства и переживания людей, формулировать математические теоремы и т.д.
Язык может быть устным и письменным, он возникает в человеческом сообществе, выполняя  важнейшие
функции:  выражения  мысли  или  сознания;хранения  и  передачи  информации;средства  общения  или
коммуникативную.

\newpage
\section{Проблема понимания [EMPTY]}
\section{Проблема истины. Основные концепции и критерии}
Проблема истины является ведущей в гносеологии. Все проблемы теории познания касаются либо средств и
путей достижения истины, либо форм существования истины.В философии существуют следующие концепции
истины:1)  классическая  теория  истин.  Истина  -  это  правильное  отражение  предмета,   процесса  в
индивидуальном познании.- самая древнейшая из концепций, её поддерживали  Аристотель, Платон, Гегель,
Фейербах.2)  когерентная  концепция  -  рассматривает  истину   как  соответствие  одних  знаний  другим.3)
прагматическая концепция. Истиной считается  то, что полезно для человека.4) конвенциальная концепция.
Истина - это то, что считает большинство.5) экзистенциальная концепция (Хайдеггер). Истина есть свобода,
это процесс, с одной стороны, в котором мир открывается нам с одной стороны, а с другой - человек сам волен
выбирать, каким способом и чем можно познать этот мир.6) неатомистическая концепция. Истина - это божье  
откровение.Характерной чертой истины является наличие в ней объективной по содержанию и субъективной
по форме сторон.
Истина объективна - это значит, что содержание человеческих представлений не зависит  от субъекта, не
зависит  ни  от  человека,  ни  от  человечества.  В  идеалистических  системах   истина  понимается  как  вечно
неизменное  и  абсолютное  свойство  идеальных  объектов.   Точка  зрения  сторонников  субъективно-идеалистического  эмпиризма  состоит  в  понимании   истинности  как  соответствие  мышления  ощущениям
субъекта или как соответствие идей  стремлениям личности к достижению успеха. На каждом историческом
этапе человечество располагает относительной истиной - приблизительно адекватным, неполным, содержащим
заблуждения знанием. Истинное знание каждой эпохи содержит элементы абсолютной  истины. Абсолютная
истина - такое знание, которое полностью исчерпывает предмет познания и не может быть опровергнуто при
дальнейшем  развитии  познания.  Всякая  относительная   истина  содержит  элемент  абсолютного  знания.
Абсолютная истина складывается из суммы относительных истин. Критерий истины находится не в мышлении
самом по себе и не в действительности, взятой вне субъекта, а заключается в практике.
Истина  –  это  адекватная  информация  об  объекте,  получаемая  посредством  его  чувственного  или
интеллектуального постижения либо сообщения о нем и характеризуемая с точки зрения ее достоверности.
Таким  образом,  истина  существует  не  как  объективная,  а  как  субъективная,  духовная  реальность  в  ее
информационном и ценностном аспектах.
Истина относительна, ибо она отражает объект не полностью, не целиком, не исчерпывающим образом, а в
изветсных  пределах,  условиях, отношениях,  которые  постоянно  изменяются  и  развиваются.  Относительная
истина есть ограниченно верное знание о чем-либо. Гора познания не имеет вершины. Истины, познанные
наукой на том или ином историческим этапе, не могут считаться окончательными. Они по необходимости
являются относительными, т.е. истинами, котрые нуждаются в дальнейшем развитии, углублении, уточнении.
Абсолютная истина – это такое содержание знания, которое не опровергается последующим развитием науки, а
обогащается  и  постоянно  подтверждается  жизнью. Под  абсолютной  истиной  в  науке  имеют  в  виду
исчерпывающее, предельное знание об объекте, как бы достижение тех границ, за которыми уже больше нечего
познавать.  Процесс  развития  науки  можно  представить  в  виде  ряда  последовательных  приближений  к
абсолютной истине, каждое из которых точнее, чем предыдущие.
Конкретность – это свойство истины, основанное на знании реальных связей, взаимодействия всех сторон
объекта,  главных,  существенных  свойств,  тенденций  его  развития.  Принцип  конкретности  истины  требует
подходить к фактам не с общими формулами и схемами, а с учетом конкретной обстановки, реальных условий.
Что дает людям гарантию истинности их знаний, служит основанием для отличения истины от заблуждения и
ошибок?
Р. Декарт, Б. Спиноза, Г. Лейбниц предлагали в качестве критерия истины ясность и отчетливость мыслимого.
Ясно то, что открыто для наблюдающего разума и с очевидностью признается таковым, не возбуждая сомнений.
Пример такой истины – «квадрат имеет четыре стороны». Подобного рода истины – результат «естественного
света разума». Однако, этот критерий не гарантирует надежности.
Выдвигался  и  такой  критерий  истины  как  общезначимость:  истинно  то,  что  соответствует  мнению
большинства. Однако еще Декарт заметил, что вопрос об истинности не решается большинством голосов. Из
истории науки мы знаем, что первооткрыватели, отсаивая истину, как правило, оказывались в одиночестве.
В некоторых философских системах существует такой критерий истины, как принцип прагматизма, т.е. теории
узкоутилитарного понимания истины, игнорирующего ее предметные основания и ее объективную значимость.
Истиной прагматизм признает то, что лучше всего «работает» на нас.
Один из фундаментальных принципов научного мышления гласит: некоторое положение является истинным в
том случае, если  можно  доказать, применимо ли оно в той или  иной  конкретной ситуации. Это принцип
выражается  термином  «реализуемость».  Посредством  реализации  идеи  в  практическом  действии  знание
соизмеряется,  сопоставляется  со  своим  объектом,  выявляя  тем  самым  настоящую  меру  объективности,
истинности своего содержания. В знании истинно то, что прямо или косвенно подтверждено на практике, т.е.
результативно осуществлено в практике.

\newpage
\section{Проблема человека в истории философии}
Философия Древнего Востока о человеке.
Первые  представления о человеке  возникают задолго до  самой  философии.  На начальных этапах истории
людям  присущи  мифологические  и  религиозные  формы  самосознания.  В  преданиях,  сказаниях,  мифах
раскрывается  понимание  природы,  предназначения  и  смысла  человека  и  его  бытия.  Кристаллизация
философского понимания человека происходит как раз на базе заложенных в них представлений, идей, образов
и понятий и в диалоге между формирующейся философией и мифологией. Именно таким образом и возникают
первые учения о человеке в государствах Древнего Востока.
Древнеиндийская философия человека представлена прежде всего в памятнике древнеиндийской литературы —
Ведах,  в  которых  выражено  одновременно  мифологическое,  религиозное  и  философское  мировоззрение.
Повышенный интерес к человеку и в примыкающих к Ведам текстах — упанишадах. В них раскрываются
проблемы нравственности человека, а также пути и способы освобождения его от мира объектов и страстей.
Человек  считается  тем  совершеннее  и  нравственнее,  чем  больше  он  достигает  успеха  в  деле  такого 
освобождения. Последнее, в свою очередь, осуществляется посредством растворения индивидуальной души
(атмана) в мировой душе, в универсальном принципе мира (брахмане).
Человек  в  философии  Древней  Индии  мыслится  как  часть  мировой  души.  В  учении  о  переселении  душ
(сансаре)  граница  между  живыми  существами  (растениями,  животными,  человеком)  и  богами  оказывается
проходимой  и  подвижной.  Но  важно  заметить,  что  только  человеку  присуще  стремление  к  свободе,  к
избавлению от страстей и пут эмпирического бытия с его законом сансары-кармы. В этом пафос упанишад.
Упанишады оказали огромное влияние на развитие всей философии человека в Индии. В частности, велико их
влияние  на  учения  джайнизма,  буддизма,  индуизма,  санкхьи,  йоги.  Это  влияние  сказалось  и  на  взглядах
известного индийского философа М.К. Ганди.
Философия Древнего Китая создала также самобытное учение о человеке. Один из наиболее значительных ее
представителей — Конфуций разработал концепцию «неба», которое означает не только часть природы, но и
высшую духовную силу, определяющую развитие мира и человека. Но в центре его философии находится не
небо,  не  природный  мир  вообще,  а  человек,  его  земная  жизнь  и  существование,  т.е.  она  носит
антропоцентристский характер.
Обеспокоенный  разложением  современного  ему  общества,  Конфуций  обращает  внимание  прежде  всего  на
нравственное поведение человека. Он писал, что наделенный небом определенными этическими качествами,
человек обязан поступать в согласии с моральным законом — дао и совершенствовать эти качества в процессе
обучения. Целью обучения является достижение уровня «идеального человека», «благородного мужа» (цзюнь-цзы), концепцию которого впервые разработал Конфуций. Чтобы приблизиться к цзюнь-цзы, каждый должен
следовать целому ряду этических принципов. Центральное место среди них принадлежит концепции жэнь
(человечность, гуманность, любовь к людям), которая выражает закон идеальных отношений между людьми в
семье и государстве в соответствии с правилом «не делай людям того, чего не пожелаешь себе». Это правило в
качестве нравственного императива в разных вариантах будет встречаться потом и в учениях «семи мудрецов» в
Древней Греции, в Библии, у Канта, у Вл. Соловьева и других. Особое внимание Конфуций уделяет принципу
сяо (сыновняя почтительность и уважение к родителям и старшим), являющемуся основой других добродетелей
и самым эффективным методом управления страной, рассматриваемой как «большая семья». Значительное
внимание он уделял также таким принципам поведения, как ли (этикет), и (справедливость) и др.
Наряду с учением Конфуция и его последователей в древнекитайской философии следует отметить и другое
направление — даосизм. Основателем его является Лао-цзы. Исходной идеей даосизма служит учение о дао
(путь, дорога) — это невидимый, вездесущий, естественный и спонтанный закон природы, общества, поведения
и. мышления отдельного человека. Человек должен следовать в своей жизни принципу дао, т.е. его поведение
должно  согласовываться  с  природой  человека  и  вселенной.  При  соблюдении  принципа  дао  возможно
бездействие, недеяние, приводящее тем не менее к полной свободе, счастью и процветанию.
Характеризуя древневосточную философию человека, отметим, что важнейшей чертой ее является ориентация
личности на крайне почтительное и гуманное отношение как к социальному, так и природному миру. Вместе с
тем эта философская традиция ориентирована на совершенствование внутреннего мира человека. Улучшение
общественной жизни, порядков, нравов, управления и т. д. связывается прежде всего с изменением индивида и
приспособлением его к обществу, а не с изменением внешнего мира и обстоятельств. Человек сам определяет
пути  своего  совершенствования  и  является  своим  богом  и  спасителем.  Нельзя  при  этом  забывать,  что
характерной чертой философского антропологизма является трансцендентализм — человек, его мир и судьба
непременно связываются с трансцендентным (запредельным) миром.
Философия человека Древнего Востока оказала огромное влияние на последующее развитие учений о человеке,
а также на формирование образа жизни, способа мышления, культурных образцов и традиций стран Востока.
Общественное и  индивидуальное сознание  людей в этих странах до  сих  пор находится под  воздействием
образцов, представлений и идей, сформулированных в тот далекий период.
Проблема человека в философии Древней Греции.
Античная  Греция  положила  начало  западноевропейской  философской  традиции  вообще  и  философской
антропологии в частности. древнегреческой философии первоначально человек не существует сам по себе, а
лишь в системе определенных отношений, воспринимаемых как абсолютный порядок и космос. Со всей своей
природной  и  социальной  средой,  соседями  и  полисом,  неодушевленными  и  одушевленными  предметами,
животными и богами он живет в едином, нераздельном мире. Даже боги, также находящиеся внутри космоса,
являются  для  людей  реальными  действующими  лицами.  Само  понятие  космоса  здесь  имеет  человеческий
смысл,  вместе  с  тем  человек  мыслится  как  часть  космоса,  как  микрокосм,  являющийся  отражением
макрокосмоса,  понимаемого  как  живой  организм.  Именно  таковы  взгляды  на  человека  у  представителей
милетской школы, стоящих на позициях гилозоизма, т.е. отрицавших границу между живым и неживым и
полагавших всеобщую одушевленность универсума.
Поворот  к  собственно  антропологической  проблематике  связан  с  критической  и  просветительской
деятельностью софистов и создателем философской этики Сократом.
Исходный принцип софистов, сформулированный их лидером Протагором, следующий: «Мера всех вещей —
человек, существующих, что они существуют, а несуществующих, что они не существуют».
В концепции софистов следует обратить внимание прежде всего на три момента:
- релятивизм и субъективизм в понимании таких этических феноменов, как благо, добродетель, справедливость  
и т.д.;
- в бытие как главное действующее лицо они вводят человека;
- впервые процесс познания они наполняют экзистенциальным смыслом и обосновывают экзистенциальный
характер истины.
Для Сократа основной интерес представляет внутренний мир человека, его душа и добродетели. Он впервые
обосновывает принцип этического рационализма, утверждая, что «добродетель есть знание». Поэтому человек,
познавший что такое добро и справедливость, не будет поступать дурно и несправедливо. Задача человека как
раз и состоит в том, чтобы всегда стремиться к нравственному совершенству на основе познания истины. И
прежде всего она сводится к познанию самого себя, своей нравственной сущности и ее реализации.
Всей своей жизнью Сократ старался реализовать нравственный пафос своей философии человека, а сама его
смерть, когда он ради утверждения справедливости отказался от жизни, явилась апофео¬зом его нравственной
философии.
Демокрит — представитель материалистического монизма в учении о человеке. Человек, по Демокриту, — это
часть природы, и, как вся природа, он состоит из атомов. Из атомов же состоит и душа человека. Вместе со
смертью  тела уничтожается  и душа.  В отличие  от такого вульгарно-материалистического  взгляда  на душу
человека его этическая концепция носит более деликатный характер. Цель жизни, по нему, — счастье, но оно не
сводится к телесным наслаждениям и эгоизму. Счастье — это прежде всего радостное и хорошее расположение
духа — эвтюмия. Важнейшее условие ее — мера, соблюсти которую помогает человеку разум. Как утверждал
Демокрит, «желать чрезмерно подобает ребенку, а не мужу», мужественным же человеком является тот, кто
сильнее своих страстей.
В отличие от Демокрита Платон стоит на позиции антропологического дуализма души и тела. Но именно душа
является субстанцией, которая делает человека человеком, а тело рассматривается как враждебная ей материя.
Поэтому от качества души зависит и общая характеристика человека, его предназначение и социальный статус.
На первом месте в иерархии душ находится душа философа, на последнем — душа тирана. Это объясняется
тем, что душа философа наиболее мудра и восприимчива к знанию, а это и является главным в характеристике
сущности человека и его отличия от животного.
Человеческая душа постоянно тяготеет к трансцендентному миру идей, она вечна, тело же смертно. Это учение
о двойственном характере человека оказало влияние на средневековое религиозное учение о нем. В единстве и
противоположности  души  и  тела  заключен,  по  Платону,  вечный  трагизм  человеческого  существования.
Телесность ставит человека  в  животный  мир,  душа  возвышает его  над этим  миром, тело  — это материя,
природа, душа же устремлена в мир идей. Позднее этот трагизм станет одним из существенных моментов
русской религиозной философской антропологии.
В концепции Аристотеля человек рассматривается как существо общественное, государственное, политическое.
И эта социальная природа человека отличает его и от животного, и от «недоразвитых в нравственном смысле
существ», и от «сверхчеловека». По этому поводу он пишет, что «тот, кто не способен вступать в общение или,
считая  себя  существом  самодовлеющим,  не  чувствует  потребности  ни  в  чем,  уже  не  составляет  элемента
государства, становясь либо животным, либо божеством».
Еще один отличительный признак человека — его разумность, «человек и есть в первую очередь ум». Таким
образом,  человек,  по  Аристотелю,  —  это  общественное  животное,  наделенное  разумом.  Социальность  и
разумность — две основные характеристики, отличающие его от животного.
К этому следует добавить, что Аристотель вплотную подходит к формулировке положения о деятельностной
сущности  человека.  Он,  в  частности,  пишет,  что  добродетельная  жизнь  человека  имеет  проявление  в
деятельности, в которой заключена и единственная возможность самореализации личности.
Новая сторона философского антропологизма обнаруживается в эпоху разложения древнегреческого общества.
На  первый  план  здесь  выступают  проблемы  человека,  связанные  с  социальным  и  нравственным  упадком,
утратой экзистенциальных ценностей и смысла жизни людей. В этой ситуации на передний план выдвигается
интеллектуально-терапевтическая  функция  философии,  т.е.  та  функция,  которую  В.  Франкл  назвал
логотерапевтической. Особенно ярко она выражена в учении Эпикура, который утверждал, что подобно тому,
как  медицина  помогает  лечить  тело  человека,  философия  должна  помогать  лечить  его  душу.  В  плане
соотношения  индивида  и  общества  Эпикур  стоит  на  позициях  методологического  и  социально-этического
индивидуализма. Исходный пункт рассмотрения общества и человека — это индивид. Социум — это лишь
средство для удовлетворения потребностей отдельного человека, его желаний и блага.
В заключение отметим, что древнегреческая философская антропология, как и древневосточная, несет на себе
печать мифологии и религии и развивается в непосредственном диалоге с ними.
Так же, как древневосточная философия человека оказала огромное влияние на все последующее ее развитие в
рамках  восточной  традиции,  древнегреческая  философская  антропология  является  начатом  и  источником
западноевропейской традиции в философии человека.
Средневековая христианская концепция человека.
В средние века человек рассматривается прежде всего, как часть мирового порядка, установленного Богом. А
представление о нем самом, как оно выражено в христианстве, сводится к тому, что человек есть «образ и
подобие Бога». Но согласно этой точки зрения в реальности этот человек внутренне раздвоен вследствие его
грехопадения,  поэтому  он  рассматривается  как  единство  божественной  и  человеческой  природы,  которое 
находит свое выражение в личности Христа. Поскольку каждый изначально обладает божественной природой,
он  имеет  возможность  внутреннего  приобщения  к  божественной  «благодати»  и  тем  самым  сделаться
«сверхчеловеком».  В  этом  смысле  концепция  сверхчеловека  часто  развивается  и  в  русской  религиозной
философии.
В социальном плане в Средние века человек провозглашается пассивным участником божественного порядка и
является  существом  тварным  и  ничтожным  по  отношению  к  Богу.  В  отличие  от  античных  богов,  как  бы
родственных  человеку,  христианский  бог  стоит  над  природой  и  человеком,  является  их  трансцендентным
творцом и творческим началом. Главная задача для человека состоит в том, чтобы приобщиться к богу и
обрести спасение в день страшного суда. Поэтому вся драма человеческой истории выражается в парадигме:
грехопадение — искупление. И каждый человек призван реализовать это, соизмеряя свои поступки с Богом. В
христианстве каждый сам за себя отвечает перед Богом.
Видным представителем средневековой христианской философии является Августин Блаженный. Не только его
онтология и учение о боге как абсолютном бытии, но и учение о человеке многое берет от Платона. Человек —
это противоположность души и тела, которые являются независимыми. Однако именно душа делает человека
человеком. Это собственная, имманентная субстанция его. То, что Августин вносит нового по этому вопросу, —
развитие  человеческой  личности,  которое  он  рассматривает  в  «Исповеди».  Она  представляет
автобиографическое  исследование,  описывающее  внутреннее  становление  автора  как  личности.  Здесь  мы
находим и психологический самоанализ, и показ противоречивого характера развития личности, и указание на
темные  бездны  души.  Учение  Августина  повлияло  на  последующее  формирование  экзистенциализма,
представители которого рассматривают его как своего предшественника.
В  отличие  от  Августина  Фома  Аквинский  использует  для  обоснования  христианского  учения  о  человеке
философию  Аристотеля.  Человек  —  это  промежуточное  существо  между  животными  и  ангелами.  Он
представляет  единство  души  и  тела,  но  именно  душа  является  «двигателем»  тела  и  определяет  сущность
человека.  В  отличие  от  Августина,  для  которого  душа  является  не  зависимой  от  тела  и  тождественной  с
человеком, для Фомы Аквинского человек есть личностное единство того и другого. Душа — нематериальная
субстанция, но получает свое окончательное осуществление только через тело.
Человек в философии эпохи Возрождения.
Философская антропология эпохи Возрождения формируется под влиянием зарождающихся капиталистических
отношении, научного знания и новой культуры, получившей название гуманизм.
Если религиозная философия Средневековья решала проблему человека в мистическом плане, то философия
эпохи Возрождения (Ренессанса) ставит человека на земную основу и на этой почве пытается решить его
проблемы. В противоположность учению об изначальной греховности человека она утверждает естественное
стремление  его  к  добру,  счастью  и  гармонии.  Ей  органически  присущи  гуманизм  и  антропоцентризм.  В
философии этого периода Бог не отрицается полностью. Но, несмотря на пантеизм, философы делают своим
знамением  не  его,  а  человека.  Вся  философия  оказывается  проникнута  пафосом  гуманизма,  автономии
человека, верой в его безграничные возможности.
Так, согласно Пико делла Мирандоле (1463—1494), человек занимает центральное место в мироздании. Это
происходит потому, что он причастен всему земному и небесному. Астральный детерминизм он отвергает в
пользу свободы воли человека. Свобода выбора и творческие способности обусловливают то, что каждый сам
является  творцом  своего  счастья  или  несчастья  и  способен  дойти  как  до  животного  состояния,  так  и
возвыситься до богоподобного существа.
В  философской  антропологии  этого  периода  уже  достаточно  отчетливо  слышны  мотивы  приближающего
индивидуализма, эгоизма и утилитаризма, связанные с нарождающимися капиталистическими общественными
отношениями и господством частного интереса. Так, Лоренцо Балла (1406 — 1457) со всей определенностью
заявляет, что благоразумие и справедливость сводятся к выгоде индивида, на первом месте должны стоять свои
собственные интересы, а на последнем — родины. И вообще, по его мнению, сохраняет «свою силу славнейшее
изречение «там для меня родина, где хорошо».
Человек Нового времени в европейской философии.
Влияние  господства  частного  интереса  на  Представления  о  человеке,  мотивы  его  ведения  и  жизненные
установки  со всей  очевидностью  выражены в  концепции  Т.  Гоббса.  В противоположность  Аристотелю он
утверждает, что человек по природе своей — существо не общественное. Напротив, «человек человеку — волк»
(homo  homini  lupus  est),  а  «война  всех  против  всех»  является  естественным  состоянием  общества.  Его
методологический  индивидуализм  и  номинализм  тесно  связаны  с  социологическим  и  этическим
индивидуализмом. Глубинной же основой такого состояния является всеобщая конкуренция между людьми в
условиях новых экономических отношений. Сам он в этой связи пишет:
Человеческая жизнь может быть сравнима с состязанием в беге... единственная цель и единственная награда
каждого из участников, это — оказаться впереди своих конкурентов.
Влияние развития науки на представления о человеке и обусловленный им антропологический рационализм
ярко обнаруживаются в философских взглядах Б. Паскаля, который утверждал, что все величие и достоинство
человека «в его способности мыслить».
Однако основателем новоевропейского рационализма вообше и антропологического рационализма в частности 
по праву считается Р. Декарт. Согласно ему, мышление является единственно достоверным свидетельством
человеческого существования, что вытекает уже из его основополагающего тезиса: «мыслю, следовательно,
существую» («cogito ergo sum»). Кроме того, у философа наблюдается антропологический дуализм души и тела,
рассмотрение  их  как  двух  разнокачественных  субстанций,  имевших  большое  значение  для  разработки
психофизической  проблемы.  Согласно  Декарту,  тело  является  своего  рода  машиной,  тогда  как  сознание
воздействует на него и, в свою очередь, испытывает на себе его влияние.
Этот  механистический  взгляд  на  человека,  рассматриваемого  в  качестве  машины,  получил  широкое
распространение в тот период. Знаменем такой концепции может служить название работы Ж. Ламетри —
«Человек-машина»,  в  которой  представлена  точка  зрения  механистического  материализма  на  человека.
Согласно  ему,  существует  лишь  единая  материальная  субстанция,  а  человеческий  организм  —  это
самостоятельно заводящаяся машина, подобная часовому механизму.
Подобный взгляд характерен для всех французских материалистов XVIII в. (Гольбах, Гельвеции, Дидро).
Другая, отличительная черта их философской антропологии — рассмотрение человека как продукта природы,
абсолютно детерминированного ее законами, так что он «не может — даже в мысли выйти из природы». Стоя на
принципах  последовательного  механистического  детерминизма,  они,  конечно,  не  могли  ни  в  какой  мере
признать свободу воли человека. Еще одна характерная черта этих мыслителей состояла в том, что, критикуя
христианскую догматику об изначальной греховности человека, они утверждали, что человек по своей природе
изначально добр и не греховен.
Немецкая классическая философия
Основоположник  немецкой  классической  философии  И.  Кант  ставит  человека  в  центр  философских
исследований. Для него вопрос «Что такое человек?» является основным вопросом философии, а сам человек
— «самый главный предмет в мире». Подобно Декарту, Кант стоит на позиции антропологического дуализма,
но его дуализм — это не дуализм души и тела, а нравственно-природный дуализм. Человек, по Канту, с одной
стороны,  принадлежит  природной  необходимости,  а  с  другой  —  нравственной  свободе  и  абсолютным
ценностям.  Как  составная  часть  чувственного  мира  явлений  он  подчинен  необходимости,  а  как  носитель
духовности — он свободен. Но главная роль отводится Кантом нравственной деятельности человека.
Кант  стремится  утвердить  человека  в  качестве  автономного  и  независимого  начала  и  законодателя  своей
теоретической  и  практической  деятельности.  При  этом  исходным  принципом  поведения  должен  быть
категорический императив — формальное внутреннее повеление, требование, основанное на том, что всякая
личность является самоцелью и самодостаточна и поэтому не должна рассматриваться ни в коем случае как
средство осуществления каких бы то ни было даже очень благих задач.
Человек, пишет Кант, «по природе зол», но вместе с тем он обладает и задатками добра. Задача нравственного
воспитания и состоит в том, чтобы добрые задатки смогли одержать верх над изначально присущей человеку
склонностью ко злу. Хотя зло изначально преобладает, но задатки добра дают о себе знать в виде чувства вины,
которое овладевает людьми. Поэтому нормальный человек, по Канту, «никогда не свободен от вины», которая
составляет основу морали. Человек, который всегда прав и у которого всегда спокойная совесть, такой человек,
не может быть моральным. Основное отличие человека от других существ — самосознание. Из этого факта
вытекает и эгоизм как природное свойство человека, но философ выступает против эгоизма, в каких бы формах
он не проявлялся.
Антропологическая концепция Гегеля, как и вся его философия, проникнута рационализмом. Само отличие
человека от животного заключается прежде всего в мышлении, которое сообщает всему человеческому его
человечность. Он с наибольшей силой выразил положение о человеке как субъекте духовной деятельности и
носителе общезначимого духа и разума. Личность, в отличие от индивида, начинается только с осознания
человеком себя как существа «бесконечного, всеобщего и свободного». В социальном плане его учение ярко
выражает методологический и социологический коллективизм, то есть принцип приоритета социального целого
над  индивидом.  В  отличие  от  немецкого  идеализма  материалист  Л.  Фейербах  утверждает  самоценность  и
значимость  живого,  эмпирического  человека,  которого  он  понимает,  прежде  всего,  как  часть  природы,
чувственно-телесное  существо.  Антропологический  принцип,  являющийся  стержнем  всей  его  философии,
предполагает  именно  такое понимание человека.  Антропологический монизм  Фейербаха направлен  против
идеалистического понимания человека и дуализма души и тела и связан с утверждением материалистического
взгляда на его природу. Но самого человека Фейербах понимает слишком абстрактно. Его человек оказывается
изолированным  от  реальных  социальных  связей,  отношений  и  деятельности.  В  основе  его  философской
антропологии  лежат  отношения  между  Я  и  Ты,  при  этом  особенно  важными  в  этом  плане  оказываются
отношения между мужчиной и женщиной.
Антропологическая проблема в русской философии.
В истории русской философии можно в русской философии выделить два основных направления, касающихся
человека:
1) материалистические учения революционных демократов (Белинского, Герцена, Чернышевского и др.);
2) концепции представителей религиозной философии (Федорова, Вл. Соловьева, Бердяева и др.).
В развитии философских взглядов В.Г. Белинского проблема человека постепенно приобретает первостепенное
значение. В письме к Боткину от 1 марта 1841 г. он отмечает, что «судьба субъекта, индивидуума, личности  
важнее  судеб  всего  мира».  При  этом  достижение  свободы  и  независимости  личности  он  связывает  с
социальными преобразованиями, утверждая, что они возможны только в обществе, «основанном на правде и
доблести». Обоснование и утверждение необходимости развития личности и ее защиты приводят Белинского к
критике капитализма и религии и защите идей утопического социализма и атеизма.
Защиту идей «русского социализма» исходя из необходимости освобождения трудящегося человека, прежде
всего «мужика», предпринял А.И. Герцен. Его антропология рационалистична: человек вышел из «животного
сна» именно благодаря разуму. И чем больше соответствие между разумом и деятельностью, тем больше он
чувствует себя свободным. В вопросе о формировании личности он стоял на позиции ее взаимодействия с
социальной  средой.  В  частности,  он  писал,  что  личность  «создается  средой  и  событиями,  но  и  события
осуществляются личностями и носят на себе их печать; тут взаимодействие».
В  работе  «Антропологический  принцип  в  философии»  Н.Г.  Черны¬шевский  утверждает  природно-монистическую сущность человека. Человек — высшее произведение природы. На взгляды Чернышевского
оказало влияние учение Фейербаха, и многие недостатки последнего свойственны также и Чернышевскому.
Хотя,  в  отличие  от  Фейербаха,  он  вводит  в  учение  о  человеке  социальные  аспекты  человеческого
существования,  в  частности  связывает  решение  проблемы  человека  с  преобразованием  общества  на
социалистических началах. Как и всем представителям натуралистического направления философии человека,
ему присуща и натуралистическая трактовка духовной жизнедеятельности человека.
В концепциях русских религиозных философов антропологическая проблематика занимает центральное место.
Это особенно относится к периоду развития русской философии, начиная с Ф.М. Достоевского, являющегося
мыслителем экзистенциального склада и внесшего в развитие этого направления значительный вклад. И хотя
представители  этого  направления  постоянно  обращаются  к  Богу,  однако  в  центре  их  внимания  находится
человек, его предназначение и судьба. Слова Бердяева о Достоевском: «Его мысль занята антропологией, а не
теологией», можно отнести ко многим представителям русской религиозной философии.
В  основе  учения  о  человеке  в  русской  религиозной  философии  находится  вопрос  о  природе  и  сущности
человека. Его решение часто видится на пути дуализма души и тела, свободы и необходимости, добра и зла,
божественного и земного. Так, антропологические взгляды Достоевского зиждятся на той предпосылке, что
человек в своей глубинной сущности содержит два полярных начала — бога и дьявола, добро и зло, которые
проявляются особенно сильно, когда человек «отпущен на свободу».
Это  трагическое  противоречие  двух  начал  в  человеке  лежит  и  в  основе  философской  антропологии  Вл.
Соловьева.
Человек, — пишет он, — совмещает в себе всевозможные противоположности, которые все сводятся к одной
великой  противоположности  между  безусловным  и  условным,  между  абсолютною  и  вечною  сущностью  и
преходящим явлением или видимостью. Человек есть вместе и божество и ничтожество.
В не меньшей степени эта проблема души и тела отражена и в философии Н.А. Бердяева, который отмечает:
Человек есть микрокосм и микротеос. Он сотворен по образу и подобию Бога. Но в то же самое время человек
есть существо природное и ограниченное. В человеке есть двойственность: человек есть точка пересечения
двух миров, он отражает в себе мир высший и мир низший... В качестве существа плотского он связан со всем
круговоротом мировой жизни, как существо духовное он связан с миром духовным и с Богом».
В силу этой изначальной раздвоенности и дуализма человека его судьба оказывается трагичной по самой своей
сути.
Весь трагизм жизни, — пишет Бердяев, — происходит от столкновения конечного и бесконечного, временного и
вечного,  от  несоответствия  между  человеком,  как  духовным  существом,  и  человеком,  как  природным
существом, живущим в природном мире.
С  точки  зрения  представителей  этого  направления,  главное  для  человека  имеет  духовная,  божественная
субстанция, а подлинный смысл человека и его существования заключается в том, чтобы соединить человека с
Богом.  В  русской  религиозной  философии  вопрос  о  человеке  органически  превращается  в  божественный
вопрос, а вопрос о Боге — в человеческий. Человек раскрывает свою подлинную сущность в Боге, а Бог
проявляется в человеке. Отсюда одна из центральных проблем этого направления — проблема богочеловека,
или сверхчеловека. В отличие от концепции Ницше, у которого сверхчеловек — это человекобог, в русской
философии  сверхчеловек  —  это  богочеловек.  Ее  антропология  носит  сугубо  гуманистический  характер,
утверждая превосходство добра над злом и бога над дьяволом.

\newpage
\section{Основные факторы и этапы антропосоциогенеза}
Антропосоциогенез - процесс происхожнения человека, становление его как биологического вида в процессе
формирования  общества,  которое,  в  свою  очередь  обозначается  социальным  развитием  или  социогенезом.
Методологической основой антропогенеза является естественно-материалистическая теория развития, генетика
и  дарвинизм,  диалектически  объясняющие  взаимоотношения  биологического  и  социального  факторов
эволюции человека как замещение более высокой формой материи - социальной более низких - биологических,
которые не отменяются, а лишь подчиняются и преобразуются первой.
Важную роль в процессе антропогенеза играла осознанная целенаправленная трудовая деятельность, повлекшая
совершенствование головного мозга, развитие конечностей, формирование сознания. Роль труда как основного
фактора  антропогенеза  была  неодинаковой  на  разных  этапах  его  развития,  поскольку  в  ранней  стадии 
первобытного  общества  (стаде)  прогресс  в  социальной  организации  в  значительной  мере  зависел  от
биологических изменений человека; в целом процесс антропогенеза сопровождался постепенным сужением
сферы действия естественного отбора в сторону возникновения общественных закономерностей и создания
социальной и культурной Среды . 
Согласно  археологическим  открытиям  последних  десятилетий  20  в.  на  африканском  континенте,  удалось
доказать, что прямохождение у древних предков современного человека функционально развилось около 6-5
млн. лет тому назад . Этот исторический период и стал периодом возникновения древнейших представителей
семейства гоминид (человекообразных обезьян) африканских двуногих приматов австралопитеков афарского
Australopithecus afarensis и южно-африканского австралопитека Australopithecus africanus, найденных в Южной
Африке, Кении (район озера Рудольф) и Танзании (ущелье Олдовай). Строение их скелетов свидетельствует о
прямохождении, зубная система близка к человеческой, а общая масса мозга составляет 380-500 г при общей
массе  тела  от  25  до  65  кг.  В  местах  обнаружения  остатков  австралопитеков  имеется  множество  костей,
расколотых  тяжелыми  предметами.  Большое  число  черепов  животных  расколоты  с  левой  стороны,  что
свидетельствует о австралопитеки были в основном правшами. Некоторые австралопитеки, видимо, начали
осваивать огонь .
Австралопитеки имеют сходство с человеком не по объему и строению головного мозга, а в основном по
способу передвижения. Изучение австралопитековых показало, что именно двуногость, а не большой объем
мозга, явилась ключевой адаптацией ранних гоминид. В 1964 г. по находкам, сделанным в Танзании, был
выделен  новый  вид  Homo  habilis  -человек  умелый,  имеющий  абсолютный  возраст  2-1.7  млн.  лет,
отличительными особенностями которого от австралопитеков являются двуногость, в целом прогрессивное
строение  кисти,  зубной  системы,  объем  мозговой  коробки  от  540-700  см3,  что  примерно  в  полтора  раза
превышает  объем  мозга  австралопитеков.  На  внутренней  поверхности  черепа  обнаруживаются  признаки
прогрессивных  нейроморфологических  изменений,  определяющиеся  по  отпечаткам  головного  мозга:
выраженная  ассиметрия  полушарий  и  развитие  двух  речевых  центров  как  условие  для  возникновения
членораздельной  речи.  Большой  палец  стопы  не  отведен  в  сторону,  что  свидетельствует  о  том,  что
морфологические  перестройки,  связанные  с  прямохождением,  у  него  полностью  завершились.  Вместе  с
остатками Homo habilis найдены орудия труда со следами целенаправленной обработки, свидетельствующие о
наличии ранних форм трудовой деятельности .
Перечисленные признаки, ведущим из которых является прогрессивное развитие мозга характеризует уже иной
уровень  морфофункциональной  организации,  характерной  для  рода  человека   Homo.  Сопоставление
морфологии африканского и афарского австралопитеком с современным человеком Homo sapiens позволяет
предположить,  что  общим  предком  человека  разумного  является  австралопитек  афарский  Australopithecus
afarensis. Африканский австралопитек Australopithecus africanus является в этой схеме представителем боковой
ветви  эволюции,  приведший  к  узкой  специализации  и  образованию  форм  типа  Australopithecus  robustus,
вымершего около 1 млн. лет назад. Таким образом, на протяжении 1-1.5 млн лет представители двух близких
родов и, возможно, нескольких видов семейства гоминид сосуществовали, причем не только во времени, но и на
перекрывающихся природных территориях. В основе дивергенции (разделения признаков) различных линий
ранних  гоминид  и  австралопитеков  могли  лежать  разного  рода  механизмы  изоляции,  в  первую  очередь
генетические  мутации  в  виде  хромосомных  перестроек.  Это  означает,  что  эволюция  австралопитековых,
ведущим фактором которой являлся естественный отбор, шла постепенно, приводя благодаря дивергенции к
морфологическому и экологическому разнообразию.
Ведущими  факторами  эволюции  на  прегоминидной  стадии  антропогенеза  являлись,  несомненно,  факторы
биологической  эволюции,  главным  из  которых  является  естественный  отбор.  Об  этом  свидетельствует,  в
частности, большое видовое разнообразие австралопитековых, обитавших в различных условиях практически
на всей территории Южной, Центральной и Северо-Восточной Африки. В то же время в происхождении рода
Homo имело место скачкообразное изменение наследственного материала.
В  разных  органах  и  системах  прогоминид  обнаруживалась  ассинхронность  исторического  развития  -филогенеза.  Есть  предположение  о  том,  что  эволюция  коры  больших  полушарий  мозга  состоит  из  двух
компонентов,  разобщенных  по  времени:  соматического,  обеспечивающего  сенсорно-моторные  функции,  и
несоматического, связанного с высшими психическими функциями. Если локомоторный комплекс подвергался
длительным  постоянным  изменениям,  то  головной  мозг  эволюционировал  скачкообразно.  Элементы
скачкообразности в
эволюции  некоторых  структур  ранних  гоминид  могли  быть  обусловлены  генными  мутациями  (реверсии,
транзиции, транслокации и т. п.), что могло повлечь за собой развитие других морфофизиологических свойств в
результате накопления мутаций под контролем естественного отбора. Но именно в период становления Homo
habilis возникла, вероятно, часть хромосомных перестроек в геноме человека, о которых говорилось выше. 
Следущим этапом эволюции после появления Homo habilis считается возникновение около 1.5 млн. лет назад
архантропов,  представителем  которого  является  вид  Homo  erectus.  Трудовая  деятельность,  материальная
культура и ярко выраженная социальность позволили ему быстро расселиться на территории Африки и Азии и
освоить  обширный  ареал,  разнообразный  в  климатическом  отношении.  Орудия  труда  Homo  erectus  более
прогрессивны, чем у Homo habilis, а масса мозга (800-1000 г) превышает минимальную массу (750 г), при
которой возможно существование речи. Наличие при этом речевых центров, возникших впервые у Homo habilis,
предполагает и развитие второй сигнальной системы.
Выделяют  три  группы  Homo  erectus,  обитающие  в  Европе,  Азии  и  Африке.  Долгое  время  древнейшими
архатропами считались азиатские представители их Индонезии и Восточного Китая - питекантроп и синантроп.
Однако находки последних лет на территории Израиля (1982) и Кении (1984), датирующиеся соответственно 2.0
и  1.6  млн. лет,  сопровождающиеся  элементами  материальной культуры и  признаками использования огня,
показали,  что  эволюция  гоминид  происходила  на  Африканском  контитенте  и  на  Ближнем  Востоке.  Это
доказательство позволило связать Homo erectus с восточноафриканскими формами Homo habilis.
Наличие большого количества находок архантропов древностью 1.5-0.1 млн. лет в отдаленных от Африки
регионах - в Юго-Восточной и Восточной Азии, в Центральной Европе и даже на Британских островах -свидетельствует об активных адаптациях их к разнообразным условиям существования. В связи с тем что
небольшое  различие  ископаемых  остатков  Homo  erectus  не  соответствует  разнообразию  природно-климтических условий указанных территорий, можно заключить, что в этих адаптациях значительную роль
играли наряду с факторами биологической эволюции также и социальные факторы: совместное изготовление
орудий труда и использование огня.
Роль  Homo  erectus  в  качестве  этапа  антропогенеза  никогда  не  подвергалась  сомнению.  Что  же  касается
палеонтропа, или неандертальского человека, то его роль в процессе эволюции в настоящее время оспаривается.
Это  связано  в  первую  очередь  с  обнаружением  большого  количества  ископаемых  остатков  человека  с
промежуточными  чертами  между  Homo  erectus  и  человеком  современного  типа.  Кроме  того,
палеонтологические находки последних лет позволяют судить о недооценке интеллектуальных способностей
неандертальцев. На всех стоянках неандертальца обнаружены следы кострищ и обгоревшие кости животных,
что свидетельствует об использовании огня при приготовлении пищи. Орудия труда неандертальца гораздо
совершеннее, чем у предковых форм. Масса головного мозга составляет приблизительно 1500 г, причем сильное
развитие получили отделы, ответственные за логическое мышление. Костные остатки неандертальца из Сен-Сезер (Франция) были найдены вместе с орудиями труда, свойственными верхнепалеолитическому человеку,
что  свидетельствует  об  отсутствии  резкой  интеллектуальной  грани  между  неандертальцем  и  современным
человеком.  Также  имеются  данные  о  ритуальных  захоронениях  неандертальцев  на  территории  Ближнего
Востока.
Эти и ряд других находок позволили в конце 60-х годов выделить палеонтропов в отдельный подвид Homo
sapiens neanderthalensis в отличие от палеонтропа Homo sapiens sapiens, который, таким образом, также получил
видовую классификацию. Наиболее древние ископаемые остатки его возрастом 100 тыс. лет обнаружены также
на  территории  Северо-Восточной  Африки.  Многочисленные  находки  палеонтропов  и  неоантропов  на
территории Европы, датирующиеся 37-25 тыс. лет, свидетельствуют о существовании обоих подвидов человека
в течение нескольких тысячелетий.
В тот же период неоантропы обитали уже не только в Европе и Африке, но и в отдаленных районах Азии (о.
Тайвань, о. Окинавы) и даже в Америке. Эти данные указывают на необычайно быстрый процесс разделения
современного  человека,  что  может  быть  доказательством  скачкообразного  характера  антропогенеза  в  этот
период как в биологическом, так и в социальном смысле. Homo sapiens neanderthalensis в виде ископаемых
остатков не обнаруживается позже рубежа в 25 тыс. лет. Быстрое исчезновение палеонтропов может быть
объяснено  вытеснением  их  людьми  с  более  совершенной  техникой  изготовлений  труда  и  возможной
скрешиваемости (метисации) с ними.
Данные антропологии и археологии доказывают роль целенаправленной трудовой деятельности как основного
фактора антропосоциогенеза в процессе социального развития человека. Однако она была неодинаковой на
разных этапах его развития, поскольку в ранней стадии первобытного общества (стаде) прогресс в социальной
организации в значительной мере зависел от биологических изменений человека; однако процесс антропогенеза
сопровождался  постепенным  сужением  сферы  действия  естественного  отбора  в  сторону  возникновения
общественных закономерностей и создания социальной и культурной среды.

\newpage
\section{Понятия «индивид», «индивидуальность», «личность»}
Индивид (от лат. Individuum - неделимое), первоначально - лат. Перевод греческого понятия 'атом' (впервые у
Цицерона), в дальнейшем - обозначение единичного в отличие от совокупности, массы; отд. Живое существо,
особь, отд. Человек - в отличие от коллектива, социальной группы, общества в целом.
Индивидуальность - неповторимое своеобразие какого-либо явления, отделяющее существа, человека. В самом
общем плане И. В качестве особенного, характеризующего данную единичность в ее качественных отличиях,
противопоставляется типичному как общему, присущему всем элементам данного класса или значительной
части их.
Индивидуальность  не  только  обладает  различными  способностями,  но  еще  и  представляет  некую  их
целостность.  Если  понятие  индивидуальности  подводит  деятельность  человека  под  меру  своеобразия  и
неповторимости,  многосторонности  и  гармоничности,  естественности  и  непринужденности,  то  понятие
личности поддерживает в ней сознательно-волевое начало. Человек как индивидуальность выражает себя в
продуктивных действиях, и поступки его интересуют нас лишь в той мере, в какой они получают органичное
предметное воплощение. О личности можно сказать обратное, в ней интересны именно поступки.
Личность - общежитейский и научный термин, обозначающий:
 1. человечность индивида как субъекта отношений и сознательной деятельности (лицо, в широком смысле
слова) или
  2. устойчивую систему социально-значимых черт, характеризующих индивида как члена того или иного
общества или общности. 
Жизнеспособность  человека  покоится  на  воле  к  жизни  и  предполагает  постоянное  личностное  усилие.
Простейшей,  исходной  формой  этого  усилия  является  подчинение  общественным  нравственным  запретам,
зрелой и развитой - работа по определению смысла жизни.
Человек - совокупность всех общественных отношений.
1. Идеалистическое и религиозно-мистическое понимание человека;
2. натуралистическое (биологическое) понимание человека;
3. сущностное понимание человека;
4. целостное понимание человека. 
Человека  философия  понимает  как  целостность.  Сущность  человека  связана  с  обществ.  Условиями  его
функционирования и развития, с деятельность, в ходе которой он оказывается и предпосылкой и продуктом
истории. 

\newpage
\section{Свобода и ответственность личности в обществе}
Все общественные отношения разделяются на первичные (материальные) и вторичные (духовно-практические).
В общественной жизни объективное и субъективное, практическое и духовное неразделимы.
Свободно-волевая деятельность и закономерности в обществе:
   * М.Вебер (субъективный идеализм): абсолютизировал неповторимость исторических событий и на этой
основе отверг наличие какой-либо закономерности в обществе. Но если в обществе нет объективной тенденции,
то невозможно прогнозирование событий, а стало быть, теряет смысл и цели социальное бытие.
    *  Марксизм:  в  обществе  имеют  место  объективная  необходимость,  причинная  обусловленность  и
повторяемость (т.е. общественная жизнь детерминирована) но существуют особенности общественных законов.
Человечество  всегда должно  будет корректировать  линию своего поведения, считаясь с законами  живой  и
неживой природы. Маркс: "люди не свободны в выборе своих производительных сил, которые образуют основу
всей их истории, потому что всякая производительная сила есть приобретенная сила, продукт предшествующей
деятельности". Свобода есть деятельность на основе познанной необходимости. Необходимость отражает нечто
устойчивое, упорядоченное, что и отражается в законах сохранения. Свобода же отражает развитие, появление
нового, разнообразного, новых возможностей. Необходимость выражает наличное, показывает, каков мир есть,
а свобода отражает будущее - каким мир должен быть. Развитие общества и есть постоянный процесс перехода
необходимости в свободу.
* Гегель: мировая история - процесс возрастания свободы. Свобода многолика, но сущность ее одна - наличие
разнообразных возможностей, следовательно, она - наибольшая ценность. Маркс: лишь при условии обретении
социальной свободы начинается развитие человеческих сил, которое является самоцелью.

\newpage
\section{Ценности человеческого бытия (труд, творчество, любовь, игра)}
Разломленность человеческого бытия на фрагментарные формы жизни, мужскую и женскую,
есть нечто большее, нежели случайные биологические состояния, нежели чисто внешняя
обусловленность  психофизической  организации:  двойственность  полов  относится  в
бытийному строю нашего конечного существования и является фундаментальным моментом
нашей конечности как таковой.
Каждый из нас выступает одновременно личностью и носителем пола, индивидом лишь в пространстве рода,
каждый  из  нас  лишен  другой  половины  человеческого  бытия,  лишен  в  такой  степени,  что  именно  эта
лишенность и порождает величайшую и могучую страсть, глубочайшее чувство, смутную волю к восполнению
и томление по непреходящему бытию -- загадочное стремление обреченных на смерть людей к некоей вечной
жизни.  О  том,  как  Эрос  в  своей  последней  смысловой  глубине  отнесен  к  бессмертию  смертных,  Платон
высказывает в "Пире" устами пророчицы Диотимы: тайна всякой человеческой любви -- воля к вечности во
времени, влечение к устоянию, к длительности именно конечного во времени человека, гонимого раздирающим
потоком времени, знающего о своей бренности [1]. К тому, что без труда дается бессмертным богам в их
самодостаточности, стремятся смертные люди, которые не в состоянии уберечь свое бытие от разрушительной
силы времени, -- и они почти обретают вечность в объятии. Возможно, доставляемое Эросом переживание
вечности содействовало выработке человеческого представления о вечности и бессмертии богов, содействовало
возникновению понятия бытия, разделившего смертное и бессмертное: бытие во времени и бытие по ту сторону
всякого времени. Возможно, в человеческой любви коренится та поэтическая сила, что создала миф, и тогда
Эрос  на  самом  деле  оказался  бы  старейшим  из  богов.  Все  рассмотренные  до  сих  пор  основные
экзистенциальные феномены суть не только существенные моменты человеческого бытия, но также и источник
человеческого  понимания  бытия, не  только онтологические  структуры человека,  но  и  смысловой  горизонт
человеческой онтологии. Тот род и способ, каким мы понимаем бытие, как мы рассматриваем многообразное
сущее, как мыслим себе очертания вещи, делаем различие между безжизненным и одушевленным бытием, 
между видами и родами разнооформленных вещей, как мы толкуем сущность и существование, различаем
действительность  и  возможность,  необходимость  и  случайность  и  тому  подобное  --  все  это  определено  и
обусловлено своеобразием нашего разума, структурой познавательной способности. Но ведь наш разум есть
разум  открытого  смерти  и  смерти  предуготовленного  существа,  разум  действующего,  трудящегося  и
борющегося создания, разум преимущественно практический, наконец -- разум творения, раздвоенного на две
полярные формы жизни и томящегося по единению, исцелению и восполнению. Наш разум не безразличен по
отношению  к  основным  феноменам  нашего  существования,  неизбежно  он  является  разумом  конечного
человека,  определенного  и  обусловленного  в  своем  бытии  смертью,  трудом,  гocподством  и  любовью.
Конечность  человеческого  разума  постигается  недостаточно,  когда  ее  истолковывают  в  качестве
ограниченности,  суженности,  стесненности,  то  есть  пытаются  определить  через  дистанцию,  отделяющую
человеческий  разум  от  некоего  гипотетического  разума  божества  или  мирового  духа.  Измеренный
божественной меркой, человеческий разум оказывается несущественным, убогим, жалким, тусклым огоньком,
изгнанным в дальние дали от сияния, озаряющего вселенную. Разум бога не знает ни смерти, ни труда, ни
господства над равным, ни любви как стремления по утраченной другой половине своего бытия. Считается, что
божественный разум безграничен, закончен, завершен и блаженно покоится в себе. Для нас непостижимо,
каким образом бог понимает бытие, исходя из своего всемогущества, всеприсутствия и всезнания. Но поэтому
он и не может быть меркой для конечного человеческого разума. Всякая попытка уподобить себя богу есть
высокомерие.  Неоднократно  в  истории  западной  метафизики  создавалась  трагическая  ситуация,  в  которой
истолкование бытия человеком связывалось с желанием поставить себя на место божественного разума или хотя
бы по аналогии снять "дистанцию", перебросить мостик между конечным и бесконечным бытием с помощью
analogia entis [2]. С этой традицией следует порвать, если мы готовы вступить в истину нашего конечного
существования и адекватно воспринять нашу антропологическую реальность.
Какие  же  имеются  человеческие  основания  для  того,  чтобы  человек  постоянно  перескакивал  через  свое
"condition humaine" [3], казался способным отринуть свою конечность, мог овладевать сверх-человеческими
возможностями,  грезить  об  абсолютном  разуме  или  абсолютной  власти,  мог  измыслить  действительное  и
примыслить недействительное, был в состоянии освободиться от тягот нашей жизни -- бремени труда, остроты
борьбы,  тени  смерти  и  мук  любовного  томления?  Пожалуй,  не  следует  торопиться  с  психологическим
объяснением и указывать на особую душевную способность -- способность фантазии. Невозможно оспаривать
существование этой способности. Всякий знает ее и бесчисленные формы ее выражения. Несомненно, сила
воображения относится к основным способностям человеческой души; она проявляется в ночном сновидении, в
полуосознанной дневной грезе, в представляемых влечениях нашей инстинктивной жизни, в изобретательности
беседы, в многочисленных ожиданиях, которые сопровождают и обгоняют, прокладывая ему путь, процесс
нашего восприятия. Фантазия действует почти повсеместно: она гнездится в нашем самосознании, определяя
тот образ, который складывается у нас о себе, или же тот, в котором нам хотелось бы видеться ближним, она
ловко  сопротивляется  беспощадному  самопознанию,  приукрашивает  или  искажает  для  нас  образ  другого,
определяет отношение человека к смерти, наполняет нас страхом или надеждой, она -- в качестве творческого
озарения -- направляет и окрыляет труд, она открывает возможность политического действия и просветляет
друг  для  друга  любящих.  Тысячью  способов  фантазия  проницает  человеческую  жизнь,  таится  во  всяком
проекте будущего, во всяком идеале и всяком идоле, выводит человеческие потребности из их естественного
состояния к роскоши; она присутствует при всяком открытии, разжигает войну и кружит у пояса Афродиты.
Фантазия открывает нам возможность освободиться от фактичности, от непреклонного долженствования так-бытия, освободиться хотя бы не в действительности, а "понарошку", забыть на время невзгоды и бежать в более
счастливый  мир  грез.  Она  может  обратиться  в  опиум  для  души.  С  другой  стороны,  фантазия  открывает
великолепный доступ к возможному как таковому, к общению с быть-могущим, она обладает силой раскрытия,
необычайной по значению. Фантазия -- одновременно опасное и благодатное достояние человека, без нее наше
бытие оказалось бы безотрадным и лишенным творчества. Проницая все сферы человеческой жизни, фантазия
все же обладает особым местом, которое можно счесть ее домом: это игра.
Так называем пятый из основных феноменов человеческого существования. Если он назван последним, то не
потому, что является "последним" в иерархическом смысле -- менее значительным и весомым, нежели смерть,
труд, господство и любовь. Игра столь же изначальна, как и эти феномены. Она охватывает всю человеческую
жизнь до самого основания, овладевает ею и существенным образом определяет бытийный склад человека, а
также  способ  понимания  бытия  человеком.  Она  пронизывает  другие  основные  феномены  человеческого
существования,  будучи  неразрывно  переплетенной  и  скрепленной  с  ними.  Игра  есть  исключительная
возможность человеческого бытия. Играть может только человек. Ни животное, ни бог играть не могут. Лишь
сущее, конечным образом отнесенное к всеобъемлющему универсуму и при этом пребывающее в промежутке
между действительностью и возможностью, существует в игре. Эти "тезисы" нуждаются в пояснении, так как
на первый взгляд противоречат привычному жизненному опыту. Каждый знает игру, это совершенно знакомое
явление. Но, по Гегелю, знакомое еще не есть познанное [4]. Как раз то, что кажется нам привычным и само
собой разумеющимся, порой наиболее упрямо ускользает от какого бы то ни было понятийного постижения.
Каждый  знает  игру  по  своей  собственной  жизни,  имеет  представление  об  игре,  знает  игровое  поведение
ближних,  бесчисленные  формы  игры,  знает  общественные  игры,  цирцеевские  массовые  представления,
развлекательные игры и несколько более напряженные, менее легкие и привлекательные, нежели детские игры,
игры взрослых; каждый знает об игровых элементах в сферах труда и политики, в общении полов друг с 
другом, игровые элементы почти во всех областях культуры. Home ludens неотделим от homo faber и homo
politicus. Игра есть такое измерение существования, которое многочисленными нитями сплетено с другими
измерениями.  Всякий человек  играл и  может  высказаться  об  игре,  опираясь на собственный опыт. Чтобы
сделать  игру  предметом  размышления,  ее  не  нужно  привносить  откуда-либо  извне:  сообразно  с
обстоятельствами мы обнаруживаем, что вовлечены в игру, мы накоротке с этой ключевой возможностью даже
тогда, когда на самом деле не играем или полагаем, что давно оставили позади игровую стадию своей жизни.
Каждому  известно  несчетное  число  игровых  ситуаций  в  частной,  семейной  и  общественной  сферах.  Они
изобилуют игровыми действиями, которые суть повседневные события и происшествия в человеческом мире.
Никому игра не чужда, всякий знает ее по свидетельству собственной жизни. Будничная привычность игры,
однако, зачастую препятствует более глубокой постановке вопроса о сущности, бытийном смысле и статусе
игры. Такая привычность совершенно заслоняет вопрос о том, действительно ли и в какой мере игровое начало
человека определяет и оформляет его понимание бытия в целом. Будничная привычность игры чаще всего
остается без вопросов благодаря будничному толкованию игры. В качестве основного феномена, игра обладает
структурой истолкованности. И это толкование не сводится к примеси частного или общественного сознания,
которая могла бы и отсутствовать. Основные Экзистенциальные феномены -- не просто бытийные способы
человеческого существования: они также и способы понимания, с помощью которых человек понимает себя как
смертного, как трудящегося, как борца, любящего и игрока и стремится через такие смысловые горизонты
объяснить одновременно бытие всех вещей.
Что же характеризует будничное толкование человеческой игры? Не что иное, как попытку вытеснить игру из
сущностного центра человеческого бытия, лишить ее сути, понять ее как "пограничный феномен" нашей жизни,
забрать у нее весомость и подлинное значение. Хотя очевидны частота игровых действий, интенсивность, с
какой предаются игре, ее растущая оценка в связи с возрастанием свободного времени в технизированном
обществе,  по-прежнему  в  игре  принято  усматривать  прежде  всего  "отдых",  "расслабление",
времяпрепровождение  и  радостную  праздность,  благотворную  "паузу",  прерывающую  рабочий  день  или
присущую дню праздничному. Там, где толкование игры исходит из ее противопоставления труду или вообще
серьезности жизни, там мы имеем дело с наиболее поверхностным, но преобладающим в повседневности
пониманием  игры.  Игра  при  этом  считается  неким  дополнительным  феноменом,  чем-то  несерьезным,
необязательным,  произвольно-самовольным.  Даже  признавая,  что  игра  имеет  власть  над  людьми  и  своим
очарованием прельщает их, игру все же не рассматривают с точки зрения ее позитивного значения и неверно
толкуют  как  некую  интермедию  между  серьезными  жизненными  занятиями,  как  паузу,  как  наполнение
свободного времени. Сказанное о будничном толковании игры, которое ее умаляет, относится прежде всего к
жизни взрослых. Играют -- да ведь только между делом, шутки ради, для разлечения, времяпрепровождения,
ради  того,  чтобы  на  время  выпрячься  из  кабалы  труда,  а  может  даже  и  с  терапевтическими  целями:
расслабиться, восстановиться, отстраниться от серьезности жизни -- игрой пользуются как сном. Считается, что
реальность взрослой жизни -- решения, решения моральные и политические, тягость труда, острота борьбы,
ответственность  за  себя  и  за  близких.  Будто  бы  только  ребенку  пристало  жить  игрой,  проводить  часы  в
радостной  беззаботности,  попусту  расточать  время.  Счастье  детства,  блаженство  игры  --  мимолетны,  как
мимолетен этот период времени нашей жизни, когда мы еще имеем время, потому что еще не знаем о нем, еще
не видим в "теперь" "уже", "никогда больше" и "еще не", когда наша жизнь мчит в глубоком и неосознанном
настоящем, когда жизненный поток увлекает нас, не ведающих о течении, стремящемся к нашему концу. Чистое
настоящее детства и считается обычно временем игры. Играет ли по-настоящему и в подлинном смысле слова
только  дитя,  а  во  взрослой  жизни  присутствуют  лишь  какие-то  реминисценции  детства,  неосуществимые
попытки  "повторения",--  или  же  игра  остается  основным  феноменом  и  для  других  возрастов?  Понятие
"основной феномен" не подразумевает требования, чтобы явленный образ человеческой жизни непременно и
непрестанно  выказывал  какой-то  определенный  признак.  Вопрос  о  том,  является  ли  игра  основным
экзистенциальным феноменом, не зависит от того, играем ли мы постоянно или же только иногда. Основным
феноменам  вовсе  не  обязательно  проявляться  всегда  и  во  всех  случаях  в  виде  какой-то  постоянной
документации. Да это и не необходимо -- чтобы они "могли" проявляться непрестанно. То, что определяет
человека как существо временное в самом его основании, вовсе не должно происходить в каждый момент
"теперь" его жизни. Смерть все же расположена в конце времени жизни, любовь -- на вершине жизни, игра (как
детская  игра)  --  в  ее  начале.  Подобная  фиксация  и  датировка  во  времени  упускает  то,  что  основные
экзистенциальные феномены захватывают человека всецело. Смерть -- не просто "событие", но и бытийное
постижение  смертности  человеком.  Так  и  игра:  не  просто  калейдоскоп  игровых  актов,  но  прежде  всего
основной способ человеческого общения с возможным и недействительным. Мы начинаем с краткого анализа
игрового поведения, то есть занятия игрой. Из-за своей краткости и сжатости этот анализ может показаться
абстрактно-формальным, но выводимые структуры каждый может, учитывая определенные единичные случаи,
проверить на самом себе. При различении "структуры" и "единичного случая" последний принято обозначать
как  пример  (Bei-Spiel)  структуры.  Многоразличное,  в  котором  утверждается  структура,  понимается  как
случайное, привнесенное игрой случая. Отношение постоянного к изменчивому, необходимого к случайному,
достаточно примечательно характеризуется метафорой игры, причем поначалу должен оставаться открытым
вопрос  о  том,  является  ли  применение  идеи  игры  к  онтологическим  отношениям  неосмотрительным
"антропоморфизмом" или же оно выводимо из самого предмета размышления. Каковы же существенные черты
человеческой игры? Мы начинаем с формы исполнения. Игра -- это импульсивное, спонтанное протекающее  
вершение,  окрыленное  действование,  подобное  движению  человеческого  бытия  в  себе  самом.  Но  игровая
подвижность  не  совпадает  с  обычной  формой  движения  человеческой  жизни.  Рассматривая  обычное
действование,  во  всем  сделанном  мы  обнаруживаем  указание  на  конечную  цель  человека,  на  счастье,
эвдаймонию. Жизнь принимается в качестве урока, обязательного задания, проекта; у нас нет места для отдыха,
мы воспринимаем себя "в пути" и обречены вечно быть изгнанными из всякого настоящего, увлекаемыми
вперед силой внутреннего жизненного проекта, нацеленного на эвдаймонию. Мы все неустанно стремимся к
счастью, но не едины во мнении, в чем же оно заключается. В напряжении нас держит не только беспокойный
порыв к счастью, но и неопределенность в толковании "истинного счастья". Мы пытаемся заработать, завоевать,
за-любить себе счастье и полноту жизни, но нас постоянно влечет за пределы достигнутого, всякое доброе
настоящее мы жертвуем неведомому "лучшему" будущему. Хотя игра как играние есть импульсивно подвижное
бытие,  она  находится  в  стороне  от  всякого  беспокойного  стремления,  проистекающего  из  характера
человеческого бытия как "задачи": у нее нет никакой цели, ее цель и смысл -- в ней самой. Игра -- не ради
будущего  блаженства,  она  уже  сама  по  себе  есть  "счастье",  лишена  всеобщего  "футуризма",  это  дарящее
блаженство настоящее, непредумышленное свершение. Никоим образом это не исключает того, чтобы игра
содержала в себе моменты значительного напряжения, как, например, игра-состязание. Но игра, со своими
волнениями, со всей шкалой внутреннего напряжения и проектом игрового действия, никогда не выходит за
свои пределы и остается в себе самой. Глубокий парадокс нашего существования состоит в том, что в своей
продолжающейся всю жизнь охоте за счастьем мы никогда не настигаем его, никого нельзя перед смертью
назвать счастливым в полном смысле слова, и что мы тем не менее, оставив на мгновение свое преследование,
нежданно  оказываемся  в  "оазисе"  счастья.  Чем  меньше  мы  сплетаем,  игру  с  прочими  .жизненными
устремлениями,  чем  бесцельней  игра,  тем  раньше  мы  находим  в  ней  малое,  но  полное  в  себе  счастье.
Дионисийский  дифирамб  Ницше  "Среди  дочерей  пустыни"  [5],  зачастую  недооцениваемый  и  неправильно
толкуемый, воспевает как раз чары и оазисное счастье игры в пустыне и бессмысленности современного бытия,
порождаемых обесцениванием некогда высших ценностей. Игра не имеет "цели", она ничему не служит. Она
бесполезна, и никчемна: она не соотнесена с какой-то конечной целью -- конечной целью человеческой жизни, в
которую верят или которую провозглашают. Подлинный игрок играет ради того, чтобы играть. Игра -- для себя
и в себе, она более, нежели в одном смысле, есть "исключение". Часто утверждают, что игра целедостаточна в
самой себе, что она несет в себе цели, которые, однако, не выходят за пределы игровой структуры. Но ведь и
всякое  законченное  трудовое  действие  несет  цели  в  себе,  единичные  приемы  согласованы  друг  с  другом,
происходят по единому плану, направляются единым замыслом. Однако трудовое действие в целом служит
выходящим за его пределы целям, вплетено в более широкий смысловой контекст. Игровому действию присущи
лишь имманентные ему цели. Если мы играем ради того, чтобы за счет игры достичь какой-то, иной цели, если
мы играем ради закалки тела, ради здоровья, приобретения военных навыков, играем, чтобы избавиться от
скуки и провести пустое, бессмысленное время,-- тогда мы упускаем из виду собственное значение игры.
Считается,  что  игре  воздается  сполна,  если  ей  приписывается  биологическое  значение  какой-то  еще  пока
безопасной, лишенной риска тренировки и отработки будущих серьезных дел нашей жизни. Игра в этом случае
служит  для  подготовки  --  сначала  посредством  ни  к  чему  не  обязывающих  проб-поступков  и  способов
поведения, которые позднее станут обязательными и неотменимыми. Именно в педагогике обнаруживается
значительное число теорем, низводящих игру до предварительной пробы будущего серьезного действия, до
маневренного  поля  для  опытов  над  бытием.  При  таком  понимании  игры  ее  польза  и  целительная  сила
усматриваются в том, чтобы в направляемой и контролируемой детской игре предвосхитить будущую взрослую
жизнь и плавно, через игровой маскарад, подвести питомца ко времени, когда лишнего времени у него не
останется: все поглотят обязанности, дом, заботы и звания. Оставляем открытым вопрос, исчерпывается ли
подобным  пониманием  игры  ее  педагогическая  значимость  и  вообще  --  ухватывается  ли  хотя  бы
приблизительно. Мы скептически относимся к широко распространенному мнению, будто бы игра принадлежит
исключительно детскому возрасту. Конечно, дети играют более открыто, притворяясь и маскируясь меньше,
нежели это делают взрослые, но игра есть возможность не только ребенка, но человека вообще. Человек как
человек  есть  игрок.  Игровому  свершению  присуща  особая  настроенность,  настроение  окрыленного
удовольствия,  которое  больше  простой  радости  от  свершения,  сопровождающего  спонтанные  поступки,
радость, в которой мы наслаждаемся своей свободой, своим деятельным бытием. Игровое удовольствие -- не
только удовольствие в игре, но и удовольствие от игры, удовольствие от особенного смешения реальности и
нереальности. Игровое удовольствие объемлет также и печаль, ужас, страх: игровое удовольствие античной
трагедии  охватывает  и  страдания  Эдипа.  И  игра-страсть,  переживаемая  как  удовольствие,  влечет  за  собой
катарсис души, который есть нечто большее,
 нежели разрядка застоявшихся аффектов. Далее, игра связана с правилами. То, что ограничивает произвол в
действиях играющего человека, -- не природа, не ее сопротивление человеческому вторжению, не враждебность
ближних, как в сфере господства, -- игра сама полагает себе пределы и границы, она покоряется правилу,
которое сама же и ставит. Играющие связаны игровым правилом, будь то соревнование, карточная игра или игра
детей. Можно отменить "правила", договориться о новых. Но пока человек играет и осмысленно понимает
процесс игры, он остается связанным правилами. Первым делом играющие договариваются о правилах -- пусть
это даже будет условленная импровизация. Конечно, не все время изобретаются "новые" игры -- готовые игры с
твердыми,  известными  правилами  существуют  в  любой  социальной  ситуации.  Но  есть  и  творческое
изобретение  новых  игр,  возникающих  из  спонтанной  деятельности  фантазии  и  затем  "фиксируемых"  во 
взаимной договоренности. Однако мы играем не потому, что в окружающем социуме имеются игры: игры
наличны и возможны лишь потому, что мы играем в сущностной основе своей.
Чем мы играем? На этот вопрос нельзя ответить сразу и недвусмысленно. Всякий игрок играет прежде всего
самим собой, принимая на себя определенную смысловую функцию в смысловом целом общественной игры: он
играет средствами игры (игралищами), вещами, признанными подходящими для игры или специально для нее
изготовленными.  К  таким  средствам  относятся:  игровое  поле,  обозначения  границ,  отметки,  необходимые
инструменты, вспомогательные средства вещественного характера. Не все игралища есть игрушки в строгом
смысле слова. Там, где игра является в чистой двигательной форме (спорт, соревнования и т.д.), она нуждается в
разнообразном игровом инвентаре. Но чем больше игра приобретает черты игры-представления, тем больше в
игровом инвентаре от настоящей игрушки. Кажется, что об игрушке может рассказать любой ребенок, и, однако,
природа игрушки -- темная, запутанная проблема. Само название двусмысленно: мы зовем какую-либо вещь
игрушкой, когда считаем, что можно приспособить ее для игры. Мы говорим сейчас как бы со стороны; с точки
зрения  неиграющего,  не  вовлеченного  в  игру.  Какие-то  чисто  природные  вещи  могут  показаться  нам
пригодными для чужой игры, например ракушки на берегу для детской игры. С другой стороны, нам известно
об искусственном производстве и изготовлении игрушек для определенной игровой потребности. Значит, люди
не  производят  игрушки  в  игре:  они  производят  их  в  труде,  серьезном  трудовом  действии,  снабжающем
игрушками  рынок?  Человеческий  труд,  таким  образом,  производит  не  только  средства  пропитания  и
инструменты для обработки природного материала, он производит жизненно необходимые вещи и для других
измерений бытия, производит оружие воина, украшения женщин, культовый инвентарь для богослужения и --игрушку, насколько игрушкой могут быть искусственные вещи. С этой точки зрения, игрушка есть один из
предметов в общем контексте единой мировой реальности, бытующий иначе, но все же не менее реально, чем,
например,  играющий  ребенок.  Кукла  --  чучело  из  пластмассы,  приобретаемое  за  определенную  цену.  Для
девочки, играющей в куклу, кукла -- "ребенок", а сама она -- его "мама". Конечно же, девочка не становится
жертвой заблуждения, она не путает безжизненную куклу с живым ребенком.
Играющая девочка живет одновременно в двух царствах: в обычной действительности и в сфере нереального,
воображаемого.  В  своей  игре  она  называет  куклу  ребенком:  игрушка  обладает  магическими  чертами,  она
возникает, в строгом смысле, не благодаря промышленному производству, она возникает не в процессе труда, но
в игре и из игры, насколько последняя является проектом особого смыслового измерения, не включающаяся в
действительность, но, скорее, парящего над нею в качестве некоей неуловимой видимости. Здесь раскрывается
сфера возможного, не связанная с течением реальных событий, область, которая, хоть и нуждается в месте и
использует его, занимает пространство и время, но сама по себе не является частью реального пространства и
времени:  нереальное  место  в  нереальном  пространстве  и  времени.  Игрушка  возникает  тогда,  когда  мы
перестаем рассматривать ее в качестве фабричного изделия, извне, и начинаем смотреть на нее глазами игрока,
в рамках единого смыслового контекста игрового мира. Творческое порождение игрового мира -- особенная
продуктивность игры-представления -- чаще всего имеет место в рамках коллективного действия, сыгранности
игрового сообщества. Созидая игровой мир, играющие не остаются в стороне от своего создания, они не
остаются  вовне,  но  сами  вступают  в  игровой  мир  и  играют  там  определенные  роли.  Внутри  созданного
фантазией творческого проекта игрового мира играющие маскируют себя как "творцов", некоторым образом
теряются в своих созданиях, погружаются в свою роль и встречаются с партнером по игре, которые также
играют определенные роли. Конечно, вещи игрового мира никоим образом не перекрывают реальные вещи
реального мира: они лишь преобразуют их в атмосфере продуцированного смысла, но не меняют их реально в
их бытии. Сила игровой фантазии в реальности, разумеется, есть бессилие. Если говорить об изменении бытия,
то здесь игра, очевидно, не может сравниться с человеческим трудом или борьбой за власть. Что же, значит
ничтожное свидетельство нашей творческой силы, которая едва набрасывает очертания воздушных замков в
податливом  материале  фантазии,  несущественно?  Или  оно  свидетельствует  об  исключительном  умении
вступать в контакт с возможностями посреди прочно установленной реальности, к которой мы привязаны
самыми различными способами? Не есть ли это освобождающее, вызволяющее общение с возможностями
также и общение с первоистоком, откуда вообще только и произошло прочное, устойчивое и неизменное бытие?
Является  ли  такая  изначальность  игры  человеческим,  слишком  человеческим  заблуждением,  чрезмерной
оценкой совершенно бессильного что-либо изменить способа поведения или же в человеческой игре нам явлено
указание  на  то,  что  более  всего  остального  может  быть  названным  первоистоком?  Бытийный  строй
человеческой игры совсем не легко прояснить, еще труднее указать присущий игре особый род понимания
бытия. Человек втянут в игру, в трагедию и комедию своего конечного бытия, из которого он никак не может
ускользнуть в чистое, нерушимое самостояние божества. "Вокруг героев", -- говорил Ницше, -- все обращается
в трагедию, вокруг полубогов -- в сатиру, а вокруг богов -- как? -- вероятно, в мир" [6].
Игра как фундаментальная особенность нашего бытия
Игра, которую знает всякий, знает по собственному опыту задолго до того, как вообще научится надежно
управлять своим разумом, игра, в которой всякий свободен задолго до того, как сможет различать понятия
свободы и несвободы, -- эта игра не есть пограничный феномен нашей жизни или преимущество одного только
детства. Человек как человек играет -- и лишь он один, один среди всех существ. Игра есть фундаментальная
особенность нашего существования, которую не может обойти вниманием никакая антропология. Уже чисто
эмпирическое изучение человека выявляет многочисленные феномены явной и замаскированной игры в самых 
различных сферах жизни, обнаруживает в высшей степени интересные образцы игрового поведения в простых
и  сложных  формах,  на  всех  ступенях  культуры  --  от  первобытных  пигмеев  до  позднеиндустриальных
урбанизированных  народов.  Все  возрасты  жизни  причастны  игре,  все  опутаны  игрой  и  одновременно
"освобождены",  окрылены,  осчастливлены  в  ней  --  ребенок  в  песочнице  точно  так  же,  как  и  взрослые  в
"общественной  игре"  своих  конвенциональных  ролей  или  старец,  в  одиночестве  раскладывающий  свой
"пасьянс". Подлинно эмпирическому исследованию следовало бы когда-нибудь собрать и сравнить игровые
обычаи всех времен и народов, зарегистрировать и классифицировать огромное наследие объективированной
фантазии, запечатленное в человеческих играх. Это была бы история "изобретений" -- другого рода, конечно,
чем изобретения орудий труда, машин и оружия, изобретений, которые могут показаться менее полезными, но
которые в основе своей были чрезвычайно необходимыми. Нет ничего необходимее избытка, ни в чем человек
не  нуждается  столь  остро,  как  в  "цели"  для  своей  бесцельной  деятельности.  Естественные  потребности
понуждают нас к действию, нужда учит трудиться и бороться. Затруднение ясно дает нам понять, что нам
следует делать в том или ином случае. А как обстоит дело тогда, когда потребности на время утихают, когда их
неумолимый бич не подгоняет нас, когда у нас есть время, которое буйно для нас разрастается, растягивается и
угрожает вовсе опустеть. Без игры человеческое бытие погрузилось бы в растительное существование. Игра к
тому  же  вливает  многие  смысловые  мотивы  в  жизненные  сферы  труда  и  господства:  как  говорится,  игра
оборачивается серьезностью. Иной раз сделанные в игре изобретения внезапно получают реальное значение.
Человеческое общество  многообразно экспериментирует  на  игровом поле  прежде,  чем  испробованные там
возможности станут твердыми нормами и. обычаями, обязательными правилами и предписаниями. Игра как
испытание  возможностей  занимает  в  системе  экономии  социальной  практики  громадное  место,  хотя  ее
экзистенциальный смысл никогда не исчерпывается этой функцией. Философская антропология обязана выйти
за пределы эмпирического понимания игры и прежде всего разработать концепцию принципиальной структуры,
бытийного строя и имманентного бытийного понимания игры.
Человеческую игру сложно разграничить с тем, что в
биолого-зоологическом  исследовании  поведения  зовется  игрой  животных.  Разве  не  бесспорно  наличие  в
животном  царстве  многочисленных  и  многообразных  способов  поведения,  которые  мы  совершенно  не
задумываясь должны назвать "играми"? Мы не можем найти для этого никакого другого выражения. Поведение
детей и поведение детенышей животных кажется особенно близкими одно другому. Взаимное преследование и
бегство, игра в преследование добычи, проба растущих сил в драках и притворной борьбе, беспокойное, живое
проявление энергии и радости жизни -- все это мы замечаем как у животного, так и у человека. Внешне --прямо-таки  поразительное  сходство.  Но  сходство  между  животными  и  человеком  сказывается  не  только  в
поведении человеческих детей и детенышей животных. Человек -- живое существо, "animal": бесчисленные
черты сближают и роднят его с животными, и близость эта столь велика, что тысячелетиями человек ищет все
новые формулы, чтобы отличить себя от животного. Вероятно, один из сильнейших стимулов антропологии --стремление к подобному различению. Животное избегает человека. По крайней мере дикое животное со своим
ненарушенным инстинктом старается обойти нас стороной, оно чуждается нарушителя спокойствия в природе,
но не "различает" себя от нас. Человек есть природное создание, которое неустанно проводит границы, отделяет
самого себя от природы, от природы вокруг и внутри себя -- обездоленное животное, не управляемое уже
надежными  инстинктами,  обреченное  отстранять  себя,  --  оно  уже  не  существует  просто  так,  но,  скорее,
отброшено назад на свое бытие, отражено к нему, оно относится к самому себе и к бытию всего сущего,
неустанно ищет потерянные тропы и нуждается в определениях самого себя, чувствует себя "венцом творения",
"подобием  бога",  местом,  где  все,  что  есть,  обращается  в  слово,  или  же  вместилищем  мирового  духа.
Человеческий  дух  уже  разработал  многочисленные  формулы  для  того,  чтобы  утвердиться  в  своей
исключительности  и  необыкновенной  весомости,  чтобы  дистанциироваться  от  всех  прочих  природных
созданий. Возможно, трудным делом окажется отобрать среди подобных различений те, которые идут от нашей
гордости и высокомерия, и те, которые на самом деле истинны. Пусть некоторые из этих формул ложны --несомненно то, что мы различаем и существуем в подобных различениях. Акт постижения человеком самого
себя  имеет  предпосылкой  противопоставление  себя  всему  остальному  сущему.  Животное  не  знает  игры
фантазии как общения с возможностями, оно не играет, относя себя к воображаемой видимости. С точки зрения
науки  о  поведении,  специфически  человеческое  в  игре  выявлено  быть  не  может.  Неотложной  задачей
философского осмысления остается утверждение понятия игры, означающего основной феномен нашего бытия,
вопреки широкому и неясному использованию слова "игра" в рамках зоологического исследования поведения.
Задача эта тем неотложней, чем обширнее материалы о психологии животных. То обстоятельство, что человек
нуждается в "антропологии", в понятийном самопонимании, что он живет с им самим созданным образом
самого себя, с видением своей задачи и определением своего места, постоянно пеленгуя свое положение в
космосе, что он может понимать себя, лишь отделив себя от всех остальных областей сущего и в то же время
относя  себя  к  совокупному  целому,  ко  вселенной,  уже  само  это  есть  антропологический  факт  огромного
значения.
У животного нет никакой "зоологии", и она ему не нужна, тем более -- как бы с противоположной стороны -- у
него нет "антропологии". Конечно, домашнее животное знает человека, собака -- своего хозяина, дикий зверь --своего врага. Но подобное знание инакового сущего не составляет момента самопознания. Антропология -- не
какая-то случайная наука в длинном ряду прочих человеческих наук. Никогда мы не становимся для себя
"темой", предметом обсуждения, как природное вещество, безжизненная материя, растительное и животное  
царства. Человек действительно бесконечно интересуется собой и именно ради себя исследует предметный мир.
Всякое познание вещей в конечном счете -- ради самопознания. Все обращенные вовне науки укоренены в
антропологическом  интересе  человека  к  самому  себе.  Субъект  всех  наук  ищет  в  антропологии  истинное
понимание самого себя, понимание себя как существа, которое понимает. Особое положение антропологии -- не
только  в  системе  наук,  которым  предается  человек,  но  и  в  совокупности  всех  человеческих  интересов  и
устремлений основывается на изначальной самоозабоченности человеческого существования. Труд есть явное
выражение подобной самозаботы; только потому, что в "теперь" человек предвидит "позже", в "сегодня" --"завтра", он может позаботиться, спланировать, потрудиться, принять на себя теперешние тяготы ради будущего
удовольствия. В сфере же господства, борьбы за власть людей над людьми возможно обеспечение будущего,
стабилизация  отношений  насилия  институционально  закрепленными  правовыми  отношениями.  Труд  и
господство свидетельствуют об отнесенной к будущему самозаботе человеческого бытия.
А как обстоит дело с игрой? Не является ли ее именно глубокая беззаботность, ее радостное, пребывающее в
себе настоящее, ее бесцельность и бесполезность, ее блаженное парение и удаленность от всех насущных
жизненных нужд тем, что придает ей волшебную силу, пленительное очарование и способность осчастливить?
Разве  игра  не  противоречит  тому,  что  мы  только  что  назвали  центральной  антропологической  структурой
человеческого  интереса  --  "заботой?"  Разве  теперь  не  могут  нам  возразить,  что  беззаботность  игры  есть
указание  на  то,  что  игра  изначально  есть  нечто  нечеловеческое,  что  она,  скорее,  принадлежит  к  еще  не
потревоженной,  не  нарушенной  никакой  рефлексией  животной  жизни  природного  создания,  что  человек
обладает естественной способностью к игре преимущественно лишь в детском возрасте, в состоянии, ближе
всего  находящемся  к  растительной  и  животной  природной  жизни,  что  он  все  больше  утрачивает
непринужденность игры, когда начинается серьезность жизни? Подобное возражение упустило бы из виду,
сколь  велико  отличие  человеческой  беззаботности  от  всякого  лишь  по  видимости  сходного  поведения
животного.  Животное  не  "заботится"  и  не  бывает  "беззаботным"  в  нашем  смысле  слова.  Лишь  сущее,  в
существе своем определенное "заботой", может также и быть "беззаботным". В строгом смысле животное -- ни
"свободно", ни "несвободно", ни "разумно", ни "неразумно". Лишь у человека есть возможность прожить жизнь
по-рабски и неразумно. Беззаботность игры по существу своему не имеет негативного характера, подобно
неразумию  или  рабскому  сознанию.  Здесь  как  раз  все  наоборот:  именно  бесполезная  игра  аутентична  и
подлинна, а не такая,  которая служит  каким-то  внеигровым  целям, как  то:  тренировка тела, установление
рекорда,  времяпрепровождение  как  средство  развлечься.  В  новейших  теориях  игры  сделана  попытка
представить игру как феномен, который присущ не только живому, но известным образом встречается повсюду.
Как утверждают сторонники этих теорий, отражения лунного света на волнующейся водной поверхности есть
игра света; череда облаков в небесах отбрасывает игру теней на леса и луга. Определенная замкнутость места
действия, движение, производимое на фоне ландшафтной декорации лунным светом, тенью от облаков и тому
подобным,  будто  бы  позволяют  предположить,  что  где-то  посреди  реального,  опытно  постигаемого  мира
является  некий  игровой  феномен,  "парящий"  над  реальными  вещами  в  качестве  прекрасной  эстетической
видимости. Игра есть прежде всего якобы свободно парящий эпифеномен, прекрасное сияние, скольжение
теней. Подобные игры можно обнаружить во всем просторе открывающейся нам природы. Это нечто вроде
эстетического  творчества  природы,  и  тогда  с  полным  правом  можно  говорить,  например,  об  игре  волн;
оказывается, что это вовсе не человеческая метафора, не перенос человеческих отношений на явления природы.
Напротив, природа играет в самом изначальном смысле, а игры природных созданий, животных и людей,
производны. На первый взгляд в этом утверждении содержится нечто подкупающее. Можно увязать его с
красочной, образной повседневной речью, которая постоянно подхватывает игровую модель, чтобы выразить в
языке по-человечески переживаемую, трогающую нас своей красотой и очарованием природу. Игра выводится
из теснины
только-человеческого  явления  в  качестве  оптического  события  огромного  диапазона.  Очевидно,  подобные
"игры", которым не нужен никакой игрок-человек, возможны повсюду: человек, в крайнем случае, может быть
вовлечен  в  такую  игру.  Итак,  человеческие  игры  представляются  частными  случаями  всеобщей,
распространенной на всю природу "игры".
Нам  кажется,  что  такое  понимание  игры  неправильно.  Здесь  основанием  анализа  делается  определенное
эстетическое или даже эстетизирующее отношение к природе, но это основание остается в тени и явно не
признается. Световые эффекты и скользящие тени столь же реальны, как и вещи, которые они освещают или
затемняют. Природные  вещи  окружающего нас мира  всегда выступают при  определенных обстоятельствах
своего об-стояния: на рассвете, под бросающим тень облачным небом, в сумеречной ночной тьме, полной
лунного сияния. И каждая вещь на берегу водоема бросает свое зеркальное отражение на поверхность воды. Так
что так называемые игры света и тени -- не более чем лирическое описание тех способов, какими даны нам
вещи окружающего мира. Естественно, мы не случайно используем подобные "метафоры", говорим об игре
волн  или  об  игре  световых  бликов  на  водной  зыби.  Однако  не  сама  природа  играет,  поскольку  она  есть
непосредственный феномен, а мы сами, по существу своему игроки, усматриваем в природе игровые черты, мы
используем  понятие  игры  в  переносном  смысле,  чтобы  приветствовать  вихрь  прекрасного  и  кажущегося
произвольным  танца  света  на  волнующейся  водной  поверхности.  На  деле  танец  света  на  тысячегранных
гребешках  волн  никогда  не  "произволен",  никогда  не  свободен,  никогда  он  не  бывает  исходящим  из  себя
творческим движением. Нерушимо и недвусмысленно здесь властвуют оптические законы. Световые эффекты
--  "игра"  в  столь  же  малой  степени,  в  какой  гребешки  волн,  с  их  ломающимися  пенными  гребнями,  --  
белогривые кони Посейдона. Поэтические метафоры здесь с наивным правом может использовать грезящая,
погруженная в прекрасную видимость душа -- но не человек, который мыслит, постигает и занимается наукой
или который занят выработкой философского понятия игры. Мы не хотели этим сказать, что не может и не
должно  быть  осмысленного  переноса  идеи  игры  на  внечеловеческое  сущее.  Там,  где  метафорическое  или
символическое  понимание  игры  оказывается  шире  человеческой  сферы,  необходимо  просто  критически
выверить и разъяснить оправданность, смысл и пределы подобного перехода границ. Но ни в коем случае не
следует  отдаваться  полупоэтической  манере  эстетизирующего  созерцания  природы.  Проблема
"антропоморфизма" столь же стара, как и стремление европейской метафизики выработать онтологические и
космологические понятия. То понимание бытия и мира, которого мы можем достичь, всегда и неизбежно будет
человеческим, то есть пониманием бытия и мира конечным созданием, которое рождается, любит, зачинает и
рожает, которое трудится и борется, играет и умирает. Элеат Парменид сделал попытку помыслить бытие в
чистом  виде,  исходя  из  него  самого,  а  с  другой  стороны,  представить  понимание  бытия  человеком  как
ничтожное и иллюзорное: он попытался мысленно взглянуть глазами бога. Но его мышление осталось вместе с
тем связанным с неким путем, hodos dizesios (фрагм. 2), путем исследования. То же можно сказать и о Гегеле,
который  переосмыслил  путь  человеческого  мышления  в  путь  бытия,  самопознающего  себя  в  человеке  и
благодаря человеку. Антропоморфизм не преодолевается просто отказом от наивного языка образов и заменой
его строгими понятиями. Наша голова, мыслящий мозг не менее человечны, чем наши органы чувств. Для
проблемы  игры  из  этого  следует,  что  игра  есть  онтологическая  структура  человека  и  путь  человеческой
онтологии. Содержащиеся здесь соотношения между игрой и пониманием бытия могут быть замечены лишь
тогда, когда феномен человеческой игры будет достаточным образом разъяснен в своей структуре. Наш анализ с
самого начала отказался от того поэтизирующего способа рассмотрения, который надеется обнаружить феномен
игры  повсюду,  где  вольная,  наивная  в  своем  антропоморфизме  речь  метафорически  говорит  об  "игре"  --например, о серебристых лунных лучах на волнуемой ветром морской поверхности. По видимости "более
широкое" понятие игры, включающее в себя "игры" луны, воды и света наравне с играми колышащейся нивы,
детеныша животного или человеческого ребенка, а то и вовсе -- ангела и бога, в действительности не дает
ничего, кроме эстетического впечатления, впечатления витающей необязательности, прекрасного, произвола и
сценической замкнутости. Мы настаиваем на том, что игра в основе своей определяется печатью человеческого
смысла. Выше мы упомянули в нашем изложении моменты протекания игры: настроение удовольствия, которое
может охватывать и свою противоположность -- печаль, страдание, отчаяние, пребывающее в себе "чистое
настоящее", не перебиваемое футуризмом нашей повседневной жизни; затем мы перешли к разъяснению правил
игры как самополагания и самоограничения игроков и коснулись коммуникативного характера человеческой
игры, играния-друг-с-другом, игрового сообщества, чтобы наконец наметить тонкое различие между "средством
игры"  и "игрушкой". Особо  важным нам представляется различение  внешней перспективы чужой игры,  в
которой  зритель  не  принимает  участия,  и  внутренней  перспективы,  в  которой  игра  является  игроку.
Деятельность игрока есть необычное производство -- производство "видимости", воображаемое созидание и все
же не ничто, а, скорее, порождение какой-то нереальности, которая обладает чарующей и пленяющей силой и не
противостоит  игроку,  но  втягивает  его  в  себя.  Понятие  "игрок"  столь  же  двусмысленно,  как  и  понятие
"игрушка". Подобно тому как игрушка является реальной вещью в реальном мире и одновременно вещью в
воображаемом мире видимости с действующими только в нем правилами, так и игрок есть человек, который
играет, и одновременно человек согласно его игровой "роли". Играющие словно погружаются в свои роли,
"исчезают" в них и скрывают за разыгранным поведением свое играющее поведение.
"Игровой мир" -- ключевое понятие для истолкования всякой
игры-представления. Этот игровой мир не заключен внутри самих людей и не является полностью независимым
от их душевной жизни, подобно реальному миру плотно примыкающих друг к другу в пространстве вещей.
Игровой  мир  --  не  снаружи  и  не  внутри,  он  столь  же  вовне,  в  качестве  ограниченного  воображаемого
пространства, границы которого знают и соблюдают объединившиеся игроки, сколь и внутри: в представлениях,
помыслах  и  фантазиях  самих  играющих.  Крайне  сложно  определить  местоположение  "игрового  мира".
Феномен, с которым легко обращается даже ребенок, оказывается почти невозможно зафиксировать в понятии.
Маленькая девочка, играющая со своей куклой, уверенно и со знанием дела
движется по переходам из одного "мира" в другой, она без труда снует из воображаемого мира в реальный и
обратно  и  даже  может  одновременно  находиться  в  обоих  мирах.  Она  не  становится  жертвой  обмана  или
самообмана, она знает о кукле как игрушке и одновременно об игровых ролях куклы и себя самой. Игровой мир
не существует нигде и никогда, однако он занимает в реальном пространстве особое игровое пространство, а в
реальном времени -- особое игровое время. Эти двойные пространство и время не обязательно перекрываются
одно другим: один час "игры" может охватывать всю жизнь. Игровой мир обладает собственным имманентным
настоящим. Играющее Я и Я игрового мира должны различаться, хотя и составляют одно и то же лицо. Это
тождество есть предпосылка для различения реальной личности и ее "роли". Определенная аналогия между
игрой и картиной поможет нам несколько это прояснить. Когда мы рассматриваем предметное изображение,
представляющее какую-то вещь, мы совершенно свободно можем различить: висящая на стене картина состоит
из  холста,  красок и рамки,  а также изображенного  на ней пейзажа.  Мы одновременно  видим реальные  и
представленные на картине  вещи. Краска холста  не  заслоняет от  нас цвет неба  в  изображенном пейзаже,
напротив: сквозь цвет холста мы просматриваем краски вещей, изображенных на картине. Мы может также
различить реальные краски и представленный ими цвет, место в пространстве и размеры единого предмета 
"картина" и пространственность внутри картины, изображенную на ней величину вещей. Стоя у изображающей
пейзаж картины, мы словно смотрим на простор за окном -- сходно с этим, но все же не точно так же. Картина
позволяет нам заглянуть в "образный мир", мы вглядываемся через узко ограниченный кусок пространства,
охваченный рамой, в некий "пейзаж", но при этом знаем, что он не раскинулся за стеной комнаты, что действие
картины сходно с действием окна, но на деле не является таковым. Окно позволяет выглянуть из замкнутого
пространства на простор, картина -- вглядеться в "образный мир", который мы видим фрагментарно. Свободное
пространство за окном переходит в пространство комнаты не прерываясь. Напротив, пространство комнаты не
переходит  непрерывно  в  пейзажное  пространство  картины,  оно  определяет  только  то,  что  есть  в  картине
"реального":  изрисованное  полотно.  Пространство  образного  мира  --  не  часть  реального  пространства,  в
котором занимает какое-то определенное место и картина как вещь. Находясь в каком-то определенном месте
реального пространства, мы вглядываемся в "нереальное" пространство пейзажа, принадлежащего к образному
миру. Изображение нереального пространства использует пространство реальное, но они не совпадают. Важно
не  то,  на  чем  основывается  иллюзорное,  лучше  --  воображаемое,  явление  пейзажа  образного  мира,
действительно  ли  и  каким  образом  осознанные  и  освоенные  иллюзионистические  эффекты  стали
использоваться как художественные средства искусства для создания идеальной видимости. В нашем контексте
важно отметить ту свободу и легкость, с какой мы принимаем различение картины и изображенного на ней
"образного мира" (не употребляя никаких понятийных различений). Мы не смешиваем две области: область
реальных вещей и область вещей внутри картины. Если же случится подобное смешение, то мы вовсе не
заметим никакой "картины". Восприятие картины (не касаясь здесь художественных проблем) относится к
объективно наличной "видимости", представляющей собой медиум, тот, в котором мы видим пейзаж образного
мира. В самом образном мире -- опять же реальность, но не та, в которой мы живем, страдаем и действуем, не
подлинная реальность, а как бы "реальность". Мы можем представить себя внутри пейзажа образного мира и
людей, для которых образный мир означал бы их "реальное окружение"; они оказались бы субъектами внутри
мирообраза данного образного мира, мы же -- субъектами восприятия изображения. Мы находимся в иной
ситуации, чем изображенные на картине люди. Мы одновременно видим картину и видим внутри картины,
находимся в реальном сосуществовании с другими наблюдателями картины и в как-бы-сосуществовании с
лицами внутри образного мира. Положение дел, однако, еще более запутано следующими обстоятельствами.
Поскольку всякое изображение, отвлекаясь от присущего ему "образного мира", нуждается также и в реальных
носителях изображения (полотно, краски, зеркальные эффекты и т.д.) и оказывается в этом отношении частью
простой  реальности,  картина  снова  может  быть  изображена  на  другой  картине,  и  так  мы  сталкиваемся  с
повторениями (итерацией) образов. Например, картина изображает "интерьер", культивированное внутреннее
пространство с зеркалами и картинами на стенах. Тогда декорация образного мира относится к имеющимся
внутри него картинам как наша реальность -- ко всей картине как таковой. Модификация "как бы" образной
"видимости" может быть воспроизведена -- нам легко представить себе картины внутри картин образного мира.
Но разгадать итерационные отношения не так-то легко. Лишь в воображаемом медиуме образной видимости
кажется возможным сколь угодно частое воспроизведение отношения реальности к образному миру; в строгом
смысле, образность высшего порядка ничего не прибавляет к воображаемому характеру картины. Изображение
внутри  изображения  не  является  более  воображаемым,  чем  исходное  изображение.  Усиление,  которое  мы,
наверное, интенционально понимаем, само есть только "видимость". Указание на эти сложные соотношения в
картине, которые, правда, всегда известны нам, но едва ли могут быть изложены с понятийной строгостью,
послужит  путеводной  нитью  для  структуро-аналитического  понимания  игры.  В  игре  мы  производим
воображаемый игровой мир. Реальными поступками, которые, однако, пронизаны магическим действием и
смысловой  мощью  фантазии,  мы  создаем  в  игровом  сообществе  с  другими  (иногда  в  воображаемом
сосуществовании с воображаемыми партнерами) ограниченный игровыми правилами и смыслом представления
мир игры. И мы не остаемся перед ним как созерцатели картины, но сами входим в него и берем внутри этого
игрового мира определенную роль. Роль может переживаться с различной интенсивностью. Есть такие игры, в
которых  человек  до  известной  степени  теряет  себя,  идентифицирует  себя  со  своей  ролью  почти  до
неразличимости, погружается в свою роль и ускользает от самого себя. Но подобные погружения нестабильны.
Всякой игре приходит конец, и мы просыпаемся от пленившего нас сна. А есть игры, в которых играющий
обращается со своей ролью суверенно легко, наслаждается своей свободой в сознании, что в любой момент он
может отказаться от роли. Игру можно играть с глубокой, почти неосознаваемой творческой активностью, а
можно -- с порхающей легкостью и грациозной элегантностью. Игровое представление не охватывает одних
только играющих, закуклившихся в свои роли: оно соотнесено и со зрителями, игровым сообществом, для
которого поднят занавес. Об этом ясно свидетельствует зрелищная игра. Зрители здесь не случайные свидетели
чужой игры, они небезучастны, к ним с самого начала обращена игра, она дает им что-то понять, завлекает в
сети своих чар. Даже не действуя, зрители оказываются околдованными. Представление в его традиционной
форме, с окружающей его декорацией, подобно картине. Зрители видят раскрывающийся перед ними игровой
мир. Пространство, в котором они себя ощущают, не переходит в сценическое пространство -- или же переходит
только  в  пространство  сцены,  поскольку  оно  есть  все  же  лишь  игровой  реквизит,  а  не  дорога  в  Колон.
Пространство игрового мира использует реальное место, действие игрового мира -- реальное время, и все же
его невозможно определить и датировать в системе координат реальности. Раскрытая сцена -- словно окно в
воображаемый  мир.  И  этот  необычный  мир,  открывающийся  в  игре,  не  только  противостоит  привычной
реальности,  но  обладает  возможностью  воспроизвести  внутри  себя  это  противостояние  и  свой  контраст  с 
реальностью. Подобно картинам в картинах существуют и игры в играх. И здесь итерация многоступенчата по
интенции, но удерживается в одном и том же медиуме видимости игрового мира. По своему воображаемому
содержанию игра третьей ступени не более воображаема, чем игра второй или первой ступени. И все же такая
итерация не лишена значения. Когда долго колебавшийся принц Датский велит поставить внутри игрового мира
еще одну игру, изображающую цареубийство, и этим разоблачающим представлением ставит в безвыходное
положение причастную к убийству мать и ее любовника, то при этом игровое сообщество взирает в игре на
другое игровое сообщество, становится свидетелем ужасной завороженности -- и само подпадает под власть
колдовских чар.
Двоякое самопонимание человеческой игры: непосредственность жизни и рефлексия
Игра принадлежит к элементарным экзистенциальным актам человека, которые знакомы и самому неразвитому
самосознанию и, стало быть, всегда находятся в поле сознания. Игра неизменно ведет с собой самотолкование.
Играющий человек понимает себя и участвующих в игре других только внутри общего игрового действа; ему
известна разрешающая, облегчающая и освобождающая сила игры, но также и ее колдовская, зачаровывающая
сила. Игра похищает нас из-под власти привычной и будничной "серьезности жизни", проявляющейся прежде
всего в суровости и тягости труда, в борьбе за власть: это похищение порой возвращает нас к еще более
глубокой серьезности, к бездонно-радостной, трагикомической серьезности, в которой мы созерцаем бытие
словно  в  зеркале.  Хотя  человеческой  игре  неизменно  присуще  двоякое  самопонимание,  при  котором
"серьезность" и "игра" кажутся
противоположностями  и  в  качестве  таковых  вновь  снимают  себя,  играющий  человек  не  интересуется
мыслительным самопониманием,  понятийным расчленением своего окрыленного,  упоительно настроенного
действования. Игра любит маску, закутывание, маскарад, "непрямое сообщение", двусмысленно-таинственное:
она бросает завесу между собой и точным понятием, не выказывает себя в недвусмысленных структурах, каком-то  одном простом  облике.  Привольная переполненность  жизнью,  радость от воссоединения противоречий,
наслаждение  печалью,  сознательное  наслаждение  бессознательным,  чувство  произвольности,  самоотдача
поднимающимся из темной сердцевины жизни импульсам, творческая деятельность, которая есть блаженное
настоящее, не приносящее себя в жертву далекому будущему, -- все это черты человеческой игры, упорно
сопротивляющиеся с самого начала всякому мыслительному подходу. Выразить игру в понятии -- разве это не
противоречие  само  по  себе,  невозможная  затея,  которая  как  раз  усложняет  постановку  интересующей  нас
проблемы?  Разрушает  ли  здесь  рефлексия  феномен,  являющийся  чистой  непосредственностью  жизни?
Осмыслить игру  -- разве это  не все равно что грубыми пальцами схватить крыло бабочки?  Возможно.  В
напряженном  соотношении  игры  и  мышления  парадигматически  выражается  общее  противоречие  между
непосредственностью жизни и рефлексией, между в-себе-бытием и понятием, между экзистенцией и сознанием,
между мышлением и. бытием, и именно у того самого существа, которое существует в качестве понимающего
бытие  существа.  Игра  есть  такой  основной  экзистенциальный  феномен,  который,  вероятно,  более  всех
остальных отталкивает от себя понятие.
Но разве не относится это в еще большей степени к смерти? Человеческая смерть ускользает от понятия совсем
на иной  манер: она непостижима для нас,  как  конец  сущего,  которое  было  уверено  в  своем  бытии; уход
умирающего,  его  отход  из  здешнего,  из  пространства  и  времени  немыслимы.  Зафиксировать  пустоту  и
неопределенность царства мертвых, помыслить это "ничто" оказывается для человеческого духа предельной
негативностью,  поглощающей  всякую  определимость.  Но  смерть  --  это  именно  темная,  устрашающая,
пугающая сила, неумолимо выводящая человека перед самим собой, устраивающая ему очную ставку со всей
его судьбой, пробуждающая размышление и смущающие души вопросы. Из смерти, затрагивающей каждого,
рождается  философия  (хотя  не  только  и  не  исключительно  из  этого).  Страх  перед  смертью,  перед  этой
абсолютной  властительницей есть, по  существу, начало  мудрости.  Человек  будучи смертным,  нуждается в
философии. Это melete thanatou -- "забота о смерти", -- как звучит одно из величайших античных определений
философствования. И Аристотель, понявший источник мышления как удивление, изумление, thaumazein, -- и он
говорит о, том, что философия исходит из "меланхолии" -- не из болезненной тоски, но, наверное, из тоски
естества.  Человеческий  труд  изначально  открыт  для  понятия,  он  не  отгорожен  от  него  подобно  игре.  Он
направлен на самопрояснение, на рациональную ясность, его эффективность возрастает, когда он постигает
себя, методически рефлектирует над собой: познание и труд взаимно повышают свой уровень. Пусть античная
theoria  и  основывается,  по-видимому,  на  каком-то  ином  опыте,  и  развивается  в  первую  очередь  в  рамках
внетрудового досуга: наука нового времени своей прагматической структурой указывает все же на тесную связь
труда и познания. Не случайно метафоры, характеризующие познавательный процесс, взяты из области труда и
борьбы, не случйно мы говорим о "работе понятия", о борьбе человеческого духа с потаенностью сущего.
Познание  и  постижение  того,  что  есть,  часто  понимается  как  духовная  обработка  вещей,  как  ломающий
сопротивление натиск, и философию зовут гигантомахией, борьбой гигантов. "Прежде потаенная и замкнутая
сущность вселенной, -- говорит Гегель в гейдельбергском "Введении в историю философии", -- не имеет сил,
чтобы  оказать  сопротивление  дерзновению  познания;  она  должна  раскрыться  перед  ним,  показать  свои
богатства и глубины, предоставив ему все это для его наслаждения" [7]. Наверное, больше нельзя разделять этот
триумфальный пафос, осмысленный лишь на почве абсолютной философии тождества. Но едва ли есть повод
отрицать  близость  познания  и борьбы  или  же  познания  и труда.  Напротив,  когда  понятийные объяснения
характеризуются  метафорой  "игры",  когда  говорят  об  игре  понятия,  это  воспринимается  почти  как 
девальвирующее  возражение.  Ненависть  к  "произвольному",  "необязательному"  и  "несерьезному"  крепко
связана  с  игривым  или  наигранным  мышлением:  это  некая  бессмысленная  радость  от  дистинкций,
псевдопроблем,  пустых  упражнений  в  остроумии.  Мышление  считается  слишком  серьезной  вещью,  чтобы
можно было  допустить его  сравнение  с  игрой. Игра якобы изначально не  расположена  к  мышлению,  она
избегает понятия, она теряет свою непринужденность, свою нерушимую импульсивность, свою радостную
невинность, когда педант и буквоед хотят набросить на нее сеть понятий.
Пока человек играет, он не мыслит, а пока он мыслит, он не играет. Таково расхожее мнение о соотношении
игры и мышления. Конечно, в нем содержится нечто истинное. Оно высказывает частичную истину. Игра чужда
понятию, поскольку сама по себе не настаивает на каком-то структурном самопонимании. Но она никоим
образом не чужда пониманию вообще. Напротив. Она означает и позволяет означать: она представляет. Игра как
представление  есть  преимущественно  оповещение.  Этот  структурный  момент  оповещения  трудно
зафиксировать  и  точно  определить.  Всякое  представление  содержит  некий  "смысл",  который  должен  быть
"возвещен". Соотношение между игрой и смыслом отличается от соотношения словесного звучания и значения.
Игровой смысл не есть нечто отличное от игры: игра -- не средство, не орудие, не повод для выражения смысла.
Она сама есть собственный смысл. Игра осмысленна в себе самой и через себя самое. Играющие движутся в
смысловой атмосфере своей игры. Но, может быть, смысл имеет место лишь для них?
Понятие зрителей игры двояко, и его следует различать. Иногда подразумевается безучастный, равнодушный
зритель,  который  воспринимает  игровое  поведение  Других,  понимает,  что  эти  другие  "играют",  то  есть
разряжаются,  отдыхают,  рассеиваются,  развлекаются  прекрасным  времяпрепровождением.  Мы  видим
играющих на улице детей, картежников в трактире, спортсменов на спортивной арене. Мы наблюдаем это
издалека, нас разделяет дистанция равнодушия, мы не принимаем в этом никакого участия. Игра -- известная
нам по своему типу форма поведения, одна среди прочих. Как безучастные зрители мы наблюдаем также за
трудом других людей, их политическими действиями, за прогулкой влюбленных. Мы можем неожиданно стать
свидетелями какого-нибудь несчастного случая, чужой смерти. Все подобные свидетельства мы совершаем
"проходя мимо". Прохождение-мимо вообще есть преимущественный способ человеческого сосуществования.
Мы высказываем это без примеси сожаления или обвинения. Совершенно невозможно вести диалог со всеми
людьми окружающего нас мира, поддерживать с ними подлинные отношения интенсивного сосуществования.
Люди, с которыми мы можем действительно сосуществовать в позитивном сообществе, -- это всегда лишь узкий
круг близких, приверженцев, друзей. Эмфатическое "обнимитесь, миллионы, в поцелуе слейся, свет" можно
представить  себе  лишь  в  предельно  абстрактной  всеобщности,  в  качестве  универсальной  филантропии.
Изначальные,  подлинные  переживания  общности  всегда  избирательны.  Люди,  которые  нас  затрагивают,  к
которым привязано наше сердце, выделены из огромной безликой массы безразличных нам людей. Неверно,
когда социальный мир представляют в виде концентрических колец, полагая внутри наиболее узкого кольца
жизненное соединение мужчины и женщины и располагая вокруг семью, родственников, друзей, друзей друзей
и т.д.  вплоть  до чужих и безразличных нам  людей.  Чужие,  мимо которых "проходят", образуют  столь же
элементарную  структуру  совместного  бытия,  как  и  сфера  интимности.  Последняя  сама  по  себе  вовсе  не
первичнее  окружающей  все  близкое  сферы  "чужого".  Человек,  становящийся  важным  для  другого  --  как
возлюбленный, как дитя, как друг, -- всегда находится "на людях". "Люди", то есть открытое, неопределенное
множество сосуществующих других, которыми мы не интересуемся, наблюдая их лишь со стороны, образуют
непременный  фон  всей  нашей  избирательной  коммуникации.  Безучастно  и  равнодушно  созерцаем  мы  не
трогающую нас игру чужих людей: именно "проходя мимо".
Совсем другое дело, когда зрители принадлежат к игровому сообществу. Игровое сообщество охватывает как
играющих, так и заинтересованных, соучаствующих и затронутых их игрой свидетелей. Играющие -- "внутри"
игрового мира, зрители -- "перед" ним. Мы уже говорили о том, что структурные особенности представления
особенно  хорошо  проясняются  на  примере  зрелища  (Schau-Spiel).  Участвующие  здесь  зрители  созерцают
чужую игру не просто так, "проходя мимо", но соучаствуя в игровом сообществе и ориентируясь на то, что
возвещает им игра. К игре они относятся иначе, нежели играющие-актеры, у них нет роли в игре, воображаемые
маски не утаивают и не скрывают их. Они созерцают игру-маскарад ролей, но при этом сами они не являются
персонажами игрового мира, они смотрят в него как бы извне и видят видимость игрового мира "перед собой".
Вместе с тем они не впадают в обманчивое заблуждение, не путают события в игровом мире с происшествиями
в реальной действительности. Само собой разумеется, игровые действия всегда остаются также и реальными
происшествиями,  однако  происшествия  играния  не  идентичны  с  событиями,  сыгранными  или,  лучше,
разыгранными на показных, воображаемых подмостках игрового мира. Зрители знают, что на подмостках --реальный актер, но видят его именно "в этой роли". Они знают: падет занавес, и он смоет грим, отложит в
сторону реквизит, снимет маску и из героя обратится в простого гражданина. Иллюзорный характер игрового
явления известен. Это знание, однако, не есть самое существенное. Не бутафория субстанция игры. И актер
может различать себя как актера от того человека, каким он является в показном медиуме игрового мира, он не
заблуждается относительно самого себя и не станет обманывать зрителей -- не станет, как порой неверно
утверждают, пробуждать в них обманчивую иллюзию и стараться, чтобы они приняли представленный им
"театр" за чистую действительность. Величие актера не в том, чтобы зрители были ослеплены и предположили
себя  свидетелями  исключительных  событий  в  будничной  действительности.  Имагинативный,  подчеркнуто
показной характер игрового мира вовсе не должен исчезнуть для игрового сообщества зрителей, они не должны
быть застигнуты врасплох мощным гипнотизирующим, ослепляющим воздействием и утратить сознание того, 
что они присутствуют при игре. Возвещающая сила представления достигает кульминации не тогда, когда
игровое  событие  смешивается  с  повседневной  действительностью,  когда  страх  и  сострадание  так  же
захватывают зрителей, как это может случиться, окажись мы свидетелями какого-нибудь ужасного дорожного
происшествия. При виде подобной катастрофы, затрагивающей чужих нам людей, мы ощущаем в себе порывы,
общечеловеческого чувства солидарности, мы жалеем несчастных, страшимся опасности, столь впечатляюще
продемонстрированной катастрофой, -- опасности, которая подстерегает всех и когда-нибудь может настигнуть
и нас. Мы не идентифицируем себя с потерпевшими, настаиваем на чуждости нам жертв катастрофы. Элемент
"показного", ирреального и воображаемого, характеризующий игровой мир, еще более, очевидно, отстраняет
деяния и страдания выступающих в игре фигур: ведь они только "выдуманы", вымышлены, сочинены, просто
сыграны. Это "как бы" деяния и "как бы" страдания, совершенно лишенные весомости и принадлежащие к
воздушному  царству  парящих  в  эфире  образов  фантазии.  Когда  же  речь  заходит  о  воспроизведении  этой
модификации "как бы", об играх внутри игр, о грезах внутри грез, тогда мы, как зрители уж вовсе не уязвимы
для свирепствующей в игровом мире судьбы: с надежного расстояния мы наслаждаемся зрелищем счастья и
гибели Эдипа. Надежно ли наше построение? Что же тогда столь сильно трогает, потрясает и поражает зрителей
в самое сердце? На чем основывается чарующая, волшебная сила представления, почему мы смотрим затаив
дыхание, как представление цареубийства -- игра в игре -- накладывает в шекспировском "Гамлете" чары на
игровое сообщество, которое само принадлежит к игровому миру, почему и сами мы оказываемся под властью
этих чар? Ведь считается, что игра есть нечто "нереальное", но при этом не имеется в виду, конечно, что игра
как играние не существует вовсе. Скорее, именно потому, что играние как таковое существует, разыгранного им
игрового мира нет. Нет в той простой реальности, где развертываются игровые действия, но игровой мир --вовсе не "ничто", не иллюзорный образ, он обладает приданным ему содержанием, это сценарий со множеством
ролей. Нереальность игрового мира есть предпосылка для того, чтобы в нем мог сказаться некий "смысл",
затягивающий нечто такое, что "реальнее" так называемых фактов. Чтобы сохранить выглядывающий из-за
фактов  смысл,  игровой  мир  должен  казаться  "ничтожнее"  фактов.  В  нереальности  игры  выявляется
сверхреальность сущности. Игра-представление нацелена на возвещение сущности. Мы слишком привыкли
полагать отнесенность к сущности преимущественно в мышлении, в качестве мыслительного отношения к
"идее", ко "всеобщему" как инвариантной структуре, виду, роду и т.д. Единичное лишь несовершенным образом
участвует во всеобщности сущности, единичное связано с идеей через участие, methexis, participatio, отделено
от непреходящей и постоянной идеи своей бренностью. Фактическая вещь относится к сущности как экземпляр
-- к виду или роду. Все совершенно иначе в сфере игры, особенно представления. Здесь налицо следующая
проблема: каким образом в определенных ролях выявляются сущностные возможности человечности. Всегда
перед  нами  здесь  "этот"  человек, который действует и страдает, утверждает себя и погибает, сущностный
человек в своем бессилии перед судьбой, опутанный виной и страданием, со своими надеждами и со своей
погибелью.  Нереальность  сплетенного  из  видимости  игрового  мира  есть  дереализация,  отрицание
определенных единичных случаев, и в то же время -- котурн, на котором получает репрезентацию фигура
игрового мира. В своем возвещении игра символична. Структура понятийного высказывания не пронизывает ее,
она не оперирует с различением единичного и всеобщего, но возвещает в символе, в совпадении единичного и
всеобщего, возвещает парадигматической фигурой, которая "не-реальна", потому что не подразумевает какого-то определенного реального индивида, и "сверх-реальна", потому что имеет в виду сущностное и возможное в
каждом. Смысл представления, одновременно воображаемый и сущностный, нереален и сверхреален. Зритель
игрового сообщества становится свидетелем события, которое не произошло в повседневной действительности,
которое кажется удаленным в некую утопию и все же открыто для зрящего -- то, что он видит в игровом
медиуме, не есть какой-то произвольный вымысел, который затрагивает чужих ему людей и, по существу, не
может коснуться его самого. Свидетель представления, который действительно вовлечен в игровое сообщество,
а  не  просто  "проходит  мимо"  него,  не  может  больше  делать  расхожего  различия  между  собой  и  своими
близкими,  с  одной  стороны,  и  безразличными  ему  другими  --  с  другой.  Нет  больше  противопоставления
человека и людей. Зрящий познает, узревает сущностно бесчеловечное -- его потрясает понимание того, что он
сам в своей сущностной глубине идентичен с чужими персонажами, что разбитый горем сын Лайя, несущий
бремя проклятия Орест, безумный Аякс -- все в нем, в качестве жутких, страшных и страшащих возможностей.
Страх и сострадание лишены здесь какой бы то ни было рефлексивной структуры, указывающей на примере
чужого страдания сходную угрозу собственному бытию и, таким образом, подводящей отдельного человека под
всеобщее понятие. Сострадание и страх обычно считаются
разнонаправленными порывами души: можно сказать, что один нацелен на других людей, второй есть забота о
собственном устоянии. Потрясение, вызываемое трагической игрой, в известной мере снимает различие между
мною  как  отдельным  существом  и  другими  людьми  --  столь  же  обособленными  экзистенциями.  Страх
оказывается  не  заботой  о  моем  эмпирическом,  реальном,  находящемся  под  угрозой  Я,  но  заботой  о
человеческом существе, угрозу которому мы видим в зеркале зрелища. И страдание обращено не вовне, к
другим, но уводит внутрь, туда, где всякий индивид соприкасается с до-индивидуальной основой. Если мы
хотим  получить  удовольствие  от  парадоксального  способа  выражения,  можно  сказать  --  в  противовес
аристотелевскому учению о том, что действие трагедии основано на эффектах "страха" и "сострадания" и их
катарсиса, -- что это верно, лишь когда учитывают обращение обоих аффектов, их структурное изменение: оба
меняют, так сказать, свое интенциональное направление, страх принимает структуру, прежде принадлежавшую
состраданию, и наоборот. Метафизика искусства европейской традиции отправлялась от платоновской борьбы 
против поэтов и его истолкования игры, а также от аристотелевской поэтики. Вновь обратить эту традицию в
открытую проблему, поставить под вопрос верность проторенных путей, критически перепроверить чрезмерное
сближение  понимания  игры  с  пониманием,  присущим  понятийному  мышлению,  поддержать  различение
символа и эйдетического понятия -- все это неотложные вопросы, стоящие перед философской антропологией
при истолковании ею одной из сфер человеческой жизни -- игры. Естественно, подобная задача не может быть
решена достаточно полно в узких рамках данной работы. Она лишь обозначена здесь как указание на более
широкий проблемный горизонт.
Указав  на  "страх"  и  "сострадание",  мы  попытались  пояснить  ситуацию  зрителя,  включенного  в  игровое
сообщество, не созерцающего игру с безучастным равнодушием и не "проходящего мимо" нее, захваченного и
затронутого миром игры. Зрелищу нужны отнюдь не только актеры и их роли, но и игровое сообщество.
Конечно, попытка охарактеризовать зрителя игры-представления лишь через взволнованную затронутость была
бы односторонней. Ведь есть совершенно иное поведение зрителей -- например, зрителей комедии и сатиры,
вызывающих  смех.  И  в  этом  случае  верно,  что  смеемся  мы  над  самими  собою,  не  над  эмпирическими
недостатками,  слабостями  и  дурачествами  тех  или  иных  индивидов,  но  над  слабостью  и  глупостью
человеческого существа. Комедия не менее символична, нежели трагедия. Она освобождает нас, устанавливая
ироническую дистанцию между человеком и человеческим. Шутка, юмор, ирония -- эти основные элементы
игровой веселеет" прокладывают путь временному освобождению человека в смеховом возвышении над самим
собой. Комедия снимает с нас бремя гнетущей кабалы труда, подчиненности чьему-то господству, любовной
страсти и мрачной тени смерти. Смех как смех-над-самим-собой свойственен лишь существу, существующему
как  конечная  свобода.  Ни  одно  животное  смеяться  не  может.  Бергсон  написал  знаменитое  сочинение  под
заглавием "Le Rire", где он представил смех сущностным отличием человека. Правда, греки верили, что их боги,
поскольку им не нужно трудиться для поддержания своей бессмертной жизни, либо -- по аналогии с человеком
-- проводят дни в веселой игре, как говорил Гомер, либо заняты непрестанным мышлением и управлением
миром, как говорили философы. Олимпийские боги изображались игроками, люди -- их игрушками, которыми
они распоряжались по своему усмотрению. И то, что для смертного было сокрушающей сердце трагедией,
бессмертному вполне могло показаться комедией. Наверное, их смех имел недобрый оттенок злорадства -- и
тогда,  когда  раскат  гомерического  хохота  потряс  Олимп,  когда  Гефест,  искусный  и  "рогатый"  бог  поймал
Афродиту в объятиях Ареса в нервущуюся сеть. Лишь исполненный достоинства неизменный бог метафизики
далек от смеха. Образу христианского бога также чужды смех, юмор, ирония, обращенная на себя самого,
совершенное существо не знает никакого смеха, никакого радостного игрового самоосвобождения.
Отсюда  вытекает  со  строгой  логической  последовательностью,  что  первый  истинно  безбожный  человек,
ницшевский Заратустра, упоенно славит смех: "Этот венец смеющегося, этот венок из роз: я сам возложил на
себя этот венец, я сам освятил свой хохот. Никого другого я не нашел сегодня достаточно сильным для этого" .
Игровое сообщество, принадлежащее к игровому представлению, еще недостаточно полно охарактеризовано в
своем отношении к игровому миру тем, что выделяется символическое понимание, использующее нереальность
сцены  в  качестве  условия  для  явления  сверхреальной  сущности.  Обстоятельства  человеческого  существа,
однако, совсем не так просты. Человек, как мы выяснили в ходе рассмотрения основных экзистенциальных
феноменов,  не  обладает  твердо  определенной  сущностью,  которая  затем  сопровождалась  бы  множеством
случайных обстоятельств: человек есть смертный и он есть трудящийся, борец, любящий и игрок. Эти сферы
жизни  никогда  не  изолированы  друг  от  друга,  ни  по  бытию,  ни  по  пониманию.  Труд  и  господство  в
бесчисленных формах переплетаются в истории человеческого рода, любовь и смерть смыкаются друг с другом,
как  мы  попытались  показать.  Игра  стоит  в  оппозиции  к  тем  феноменам  жизни,  которые  принято  считать
тягостной серьезностью жизни. Игра -- иная, она есть колеблющееся в элементе "нереального" активное и
импульсивное  общение  с  воображаемым,  туманным  царством  возможностей.  Игрой,  вполне  реальным
действием, мы создаем "нереальный" игровой мир и глубоко рады этому созданию, мы в восторге от его
фантастичности, которой, впрочем, меньше, чем пены, выбрасываемой на берег волнами. Хотя в ходе анализа
игры мы и объяснили, что для игрока, в игровом мире вещи "реальны", все же следует уточнить, что это
реальность в кавычках: ее не путают с подлинной реальностью. Как игрок в своей роли, так и зритель внутри
игрового сообщества -- оба знают о фиктивности реальности в игровом мире. Они сохраняют это знание, когда
речь идет уже об игре внутри игры, -- как сохраняется различие реальной вещи и "картины", когда в картинах в
свою  очередь  встречаются  картины  и  т.д.  Итерационные  возможности  игры  родственны  итерационным
структурам образности. Игра может воспроизводиться внутри игры. Попросту говоря, дети, играющие в самую
древнюю подражательную игру, выступающие в своем игровом мире "отцами" и "матерями", у которых есть
"дети",  вполне  могут  послать  этих  детей  из  игрового  мира  "поиграть"  на  улице,  пока  дома  не  будут
приготовлены куличики из песка. Уже на этом примере можно узнать нечто значительное. Игре известна не
простая  однородная  итерация,  повторение  игры  в  игре,  в  своей  воображаемой  зоне  представления  игра
объемлет и внеигровое поведение людей. Как один из пяти основных феноменов, игра охватывает не только
себя, но и четыре других феномена. Содержание нашего существования вновь обнаруживается в игре: играют в
смерть, похороны, поминовение мертвых, играют в любовь, борьбу, труд. Здесь мы имеем дело вовсе не с
какими-то искаженными, неподлинными формами данных феноменов человеческого бытия, их розыгрыш --вовсе не обманчивое действие, с помощью которого человек вводит других в заблуждение, притворяется, будто
на самом деле трудится, борется, любит. Эту неподлинную модификацию, лицемерную симуляцию подлинных
экзистенциальных актов, часто, но неправомерно, зовут "игрой". В столь же малой мере это игра, в какой ложь  
является  поэзией.  Ведь  произвольным  все  это  оказывается  только  для  обманывающих,  но  никак  не  для
обманутых.  В  игре  же  не  бывает  лживой  подтасовки  с  намерением  обмануть.  Игрок  и  зритель  игрового
представления знают о фиктивности игрового мира. Об игре в строгом смысле слова можно говорить лишь там,
где воображаемое осознано и открыто признано как таковое. Это не противоречит тому, что игроки иногда
попадают под чары собственной игры, перестают видеть реальность, в которой они играют и имагинативно
строят свой игровой мир. От погруженности в игру можно очнуться. Сыгранная борьба, сыгранный труд и т.д.--опять-таки весьма двусмысленные понятия. Иной раз имеют в виду притворную борьбу, притворный труд, в
другой  раз  --  подлинную  борьбу  и  подлинный  труд,  но  в  качестве  событий  внутри  игрового  мира.  У
человеческой игры нет каких-то иных возможностей для своего выражения, кроме жизненных сфер нашего
существования. Отношение игры к другим основным феноменам -- не просто соседство и соотнесенность, как в
случае с трудом и борьбой  или любовью  и  смертью.  .Игра  охватывает  и  объемлет все другие феномены,
представляет их в непривычном элементе воображаемого и тем самым дает человеческому бытию возможность
самопредставления и самосозерцания в зеркале чистой видимости. Следует еще поразмыслить над тем, что это
означает.
Всеобъемлющая структура человеческой игры.
Игра  объемлет  все.  Она  вершится  человеческим  действованием,  окрыленным  фантазией,  в  чудесном
промежуточном  пространстве  между  действительностью  и  возможностью,  реальностью  и  воображаемой
видимостью и представляет на учиненной ею идеальной сцене -- в себе самой -- все другие феномены бытия, да
вдобавок самое себя. Подобная всеобъемлющая структура необыкновенно сложна в своем интенциональном
строе  и  предполагает  не  только  сообразную  переживанию  классификацию  пережитых  игровых  миров
различных ступеней, но и взаимопроникновение "возможного и действительного", становящееся прямо-таки
проблемой  калькуляции.  К  сказанному  следует  еще  добавить,  что  игровые  элементы  присутствуют  во
множестве форм неигровой жизни, зачастую в виде маленьких увеселений в лощинах серьезного жизненного
ландшафта,  так  что  посреди  сурового,  мрачного  и  отягченного  страданиями  человеческого  существования
всплывают "острова" игрового блаженства. Что было бы с влюбленными с их поистине бесконечной задачей без
разыгранной шутки, без радостных сердечных арабесок? Чем была бы война без авантюры, без игровых правил
рыцарственности, чем был бы труд без игрового гения, чем была бы политическая сцена без добровольного или
недобровольного фарса властителей? Иногда выказываемая во всех этих сферах серьезность есть не более чем
хорошо сидящая маска скрытой игры. Именно потому, что игра способна менять облачения, ее присутствие не
всегда  легко  установить.  Порой  люди  застают  друг  друга  за  игровыми  действиями,  которые  совсем  и  не
выглядят  таковыми.  Феноменология  шутки  как  конституирующего  социальность  фактора  все  еще  не
разработана. От всякой игры, открытой и скрытой, как бы замаскированной, следует строго отличать лицемерие
с  целью  обмана,  подложную  "как  бы"  модификацию  чувств,  умонастроений  и  действий,  в  которой  люди
"представляются" друг перед другом, обманывают не только словами, но и образом поведения, поступками,
когда, например, "играют в любовь", не ощущая ее, когда, как говорится, устраивают "спектакль". Ложь, которая
может быть не только словесной, но и ложью жестов, мимики, даже "молчания", есть жуткая, зловещая тень,
ложащаяся  на  межчеловеческие  отношения  и  угрожающая  им.  Человеческая  ложь  --  это  не  мимикрия
животных,  но  хитрость,  притворство,  коварство  зверей  в  борьбе  за  добычу,  человеческая  ложь  лишена
невинности хищника. По сравнению с животным человек оснастил поведение, имеющееся уже в животном
мире,  средствами  своего  интеллекта,  когда  из  доисторического  собирателя  превратился  в  охотника,
преследующего дичь с помощью всевозможных уловок -- ям, приманок и тому подобного. Рафинированные
обманные  средства  охоты  затем  были  перенесены  в  сферу  охоты  на  людей  --  в  военную  борьбу  и  ее
продолжение в политической риторике, вторглись во взаимное приманивание и вечную войну полов. Охота как
область хитрости и обмана есть поведение, характеризующее животный и человеческий мир, это область своего
рода естественной лжи. Но человек отличается от животного среди прочего и тем, что понимает истину как
таковую, что он открыт смыслу и способен разделять понятый смысл с другими людьми. Межчеловеческое
сообщение -- будь то в мимике, жесте или слове -- есть нечто большее, чем сигнал, и существенно отличается от
предупреждающего свиста серны или же призывного рева оленя. Так как человек знает об истине, об истине
смыслового  и  о  смысле  истины,  он  знает  и  о  мучительной  потаенности  сущего,  может  спрашивать  и
высказываться,  знает  о  неистинности  как  проходящей  замкнутости  вещей,  о  неистинности  как  следствии
притворства, способность к которому он постигает как власть. Именно потому, что человек определяется через
свое отношение к истине, он обладает возможностью лжи. Это опасная, злая возможность -- злая не потому, что
непрестанно учиняет вред, отравляя ядом недоверия отношения между людьми, но потому, что делает само
отношение человека к истине двусмысленным, неверным и ненадежным. Притворное знание хуже незнания,
ложь хуже заблуждения. Легче примириться с тем, что -- как существо конечное -- мы можем познать лишь
весьма  ограниченный  круг  вещей,  чем  вынести  ложь  и  обман  со  стороны  других  людей.  Лицемерная
неподлинность в нагруженных смыслом словах и поступках часто зовется "игрой", а игра противопоставляется
подлинному и правдивому, истинному. Конечно, это неправильное толкование зла и злоупотребление понятиям
игры.  И  все  же  между  обманом  и  игрой  есть  связь.  Сама  игра  --  это  не  обман,  но  она  пользуется
иллюзионистскими эффектами, которыми обычно оперирует и обман, игра воспринимает элементы показного --не для того, чтобы выдать показное за реальность, а чтобы использовать его в качестве средства выражения.
Маски  в  игре  не  должны  вводить  в  заблуждение,  они  должны  зачаровывать,  это  --  реквизит  практики 
волшебства.  Игра  развертывается  внутри  условной  "видимости",  она  ее  не  отрицает,  но  и  не  выдает  за
неподдельно реальное. Всякая игра связана с иллюзорной, воображаемой "видимостью", но не затем, чтобы
обмануть,  а  с  целью  завоевать  измерение  магического.  Когда  в  игровом  мире  представления  "являются"
внеигровые феномены бытия, когда в игре борются, трудятся, любят, а то и вовсе умирают, это не значит, что
игра, с целью обмана, устроила неподлинный спектакль. Это как раз подлинный театр, подмостки зрелища,
выводящего человеческую жизнь перед ней самой. Игра -- исключительный способ для-себя-бытия. Это не для-себя-бытие, происходящее от рассудочной рефлексии, не сознательное обращение представляющей жизни на
себя самое. Ведь игра есть действие, практика общения с воображаемым. В человеческой игре наше бытие
действенно отражается в себе самом, мы показываем себе, чем и как мы являемся. Игровое для-себя-бытие
человека прагматично. Оно существенно отличается от чисто интеллектуального для-себя-бытия, идущего от
рефлексии. Игра принадлежит к элементарному, дорефлективному бытию. Однако она не "непосредственна".
Она обладает структурой "опосредствования", она проста, пока остается игрой, двойственна -- когда выступает
как действие в реальном мире и одновременно -- в мире игровом. Деятельный характер игрового действия,
выделяющий игру в сравнении с рефлексией сознания, очевидно, не дан в игровом сообществе. Не есть ли это
характерная черта "созерцания" в широком смысле слова? Как мы сказали, игровое сообщество включает в себя
игроков  и  свидетелей  игры,  ограниченную  сцену  игрового  мира  и  людей  перед  подмостками.  Последние
включены  в  игру  постольку,  поскольку  они  очарованы  игрой.  При  этом  они  не  "действуют"  сами,  они
погружены в созерцание, которое их захватывает или забавляет. Но "со-зерцатели" игрового представления
"идентифицируют"  себя  с  игроками  имагинативным  образом.  На  этой  идентификации,  наверное,  и
основывается в значительной степени зачаровывающая силы игры. В обесцвеченной и ослабленной форме этот
момент  "идентификации  зрителей  и  игроков"  сказывается  и  на  всяких  цирцеевских  представлениях,  на
развлекательных играх, устраиваемых для масс большого города. Наверное, неправильно насмехаться над тем,
что в современных футбольных состязаниях 22 человека бегают по полю, а сотни тысяч наблюдают за ними.
Оставив в стороне значение подобных грандиозных представлений, определяемое социологически (например,
как "содержание сознания" или "тема разговора" масс), укажем на то, что быть зрителем -- это само по себе есть
род  сильной  эмоциональной  причастности,  способ  идентифицирующего  соучастия  в  игре,  поднимающий
множество интересных  проблем.  Игровое  сообщество зрелища  объединено и собрано  в  общеколлективной
иллюзии, которую могут сознавать и признавать "нереальной" и в то же время понимать как место явления
"сверхреальной"  сущности.  Сцена  представляет,  подмостки  ее  --  весь  мир.  Прежде  всего  это,  наверное,
"человеческий мир", совокупность человеческого бытия, а сверх того, вероятно, и совокупность всего бытия, к
которому  человек  постоянно  себя  относит.  Зрелище,  по  существу,  есть  пример  (Bei-Spiel),  образец,
парадигматическое представление того, что мы есть и каковы мы. Этот имеющийся в человеческой игре образец
заключается в осмысленном представлении бытия во всех его жизненных измерениях для него самого. Владея
способностью играть, человек может созерцать себя, обретать образ собственной жизни во всей его высоте и
глубине, задолго до того, как он начинает размышлять и понятийно постигать истину своего существования.
Игровая рефлексия образна, игра выводит сущность в явление до всякого явного размышления. Человек как
человек по своему бытию есть отношение. Он не сходен с вещью, которая сначала есть в самой себе и только
потом  относится  к  чему-либо.  Категориальная  модель  "субстанции"  не  подходит  к  человеку.  Человек
существует как отношение, отношение к себе, вещам и миру, он существует в пространстве и относит себя к
родине и чужбине, существует во времени и относит себя к собственному прошлому, обусловлен родом и полом
и относит себя к собственному полу в стыде и институциональных отношениях (брак, семья). Культ умерших,
труд, господство, любовь -- все это ключевые способы самоотнесения человека. Игра же есть отношение к
относительному бытию человека. Все основные феномены бытия сплетены друг с другом, игра отражает в себе
их все, в том числе и саму себя. Это и придает игре исключительный статус. С представляющей способностью
игры  связано  и  то,  что  она  может  иначе  наполнять  время,  нежели  остальные  феномены  человеческого
существования.  Предстояние  смерти  бросает  тень  на  всякий  человеческий  поступок.  Мера  времени  нам
установлена, пусть даже мы и не знаем о ней. В свете последнего мгновения все часы, дни и годы как бы даны
нам взаймы. В труде и борьбе человеческое время заполнено нужными, необходимыми делами: время всегда у
нас отнимают. То же можно сказать и о любви, которая, наверное, есть самое трудное, менее всего осваиваемое
занятие нашего бытия, постоянно терпящее крушение.
Лишь у игры есть "свободное время". Слишком часто проблему игры и свободного времени рассматривали
поверхностно,  брали  преимущественно  как  проблему  заполнения  имеющегося  в  распоряжении  свободного
времени всевозможными играми, чтобы  оно  оказалось  осмысленным  и  дарило бы  счастье.  Играем ли мы
потому, что у нас есть свободное время, или же у нас есть свободное время как раз потому, что мы играем?
Такая формулировка проблемы -- не простое переворачивание. В том и другом случае слова "свободное время"
имеют разный смысл. В первом случае свободное время идентично с жизненным временем, не заполненным и
не блокированным неотложными задачами. Время нужно нам, чтобы спать, есть, производить средства к жизни,
поддерживать  общественные  установления  исполнением  самых  различных  служебных  обязанностей.
Свободное время -- тот промежуток, который "остается" после того, как мы сделали все насущные для жизни
дела. Треть 24-часового дня используется на сон, треть -- на труд, треть на жизнь, ради которой трудятся, то
есть на семью, исполнение гражданских обязанностей, развлечения и наслаждение жизнью, занятия любовью,
отдых  и  т.д.  Свободным  временем  человек  обладает  в  той  степени,  в  какой  он  освобожден  от  нужды
естественных  потребностей  и  давления  общественных  необходимостей.  Можно,  например,  по  своему 
усмотрению и произвольно распоряжаться свободным временем, можно, говоря вместе с ранним Марксом,
"охотиться,  рыбачить",  быть  "критическим  критиком",  человек  извлекается  из-под  гнета  необходимости  и
беспрепятственно  передвигается  в  "царстве  свободы".  Свободное  время  --  комплементарная  оппозиция
"рабочего времени" (или служебного). Когда рабочее время сокращается, свободное время расширяется. В
прежние времена большинство людей должно было трудиться большую часть времени своего бодрствования,
отдавая все силы для производства средств жизни для себя и
привилегированного  слоя  господ,  которые  не  "трудились",  а,  скорее,  "правили",  владели  землей,  оружием,
военной мощью. Вместе с
индустриализацией человеческая рабочая сила и прежде всего рабочее время высвобождаются за счет машин в
никогда  не  предполагавшихся  прежде  масштабах:  свободное  время  во  всех  индустриальных  странах
значительно  возрастает.  Возникли  совершенно  новые,  насущные  и  требующие  немедленного  решения
проблемы. Урбанизация сконцентрировала в разных местах земного шара огромные массы людей, чья занятость
в свободное время не формируется "естественно" врожденными склонностями и интересами, но, скорее, ее
необходимо организовывать, направлять, ею нужно руководить. В век техники и человеческое свободное время
приобретает технократический оттенок. Было бы интеллигентской спесью романтически-анахронистического
толка разделять людей на две категории: тех, которые сами могут распорядиться своим свободным временем,
потому что ими движут собственные и самостоятельные интересы, и тех, которые растерянно и беспомощно
стоят перед данным им свободным временем, не зная, что им предпринять, которые остаются
"несовершеннолетними" в формировании собственной жизни, которые нуждаются в руководстве, в поводе и
месте для развлечения. Значительный размах индустрии развлечений подверг наличную здесь потребность
инфляции нормированию и нивелировке. Гигантский аппарат современного технического оснащения жизни не
мог  бы  развиваться  и  функционировать,  если  бы  речь  шла  только  об  удовлетворении  элементарных
естественных потребностей в пище, одежде, жилье. Человеку нужно не столько "необходимое", сколько как раз
"избыточное". Индустриальная экспансия неизбежно должна была распространиться и на поведение человека в
свободное время; и, как я полагаю, она все больше будет формировать содержание сознания людей.
"Идеологическая  война",  крупномасштабная  обработка  сознания  делается  в  преуспевающей  экономике
условием  "полной  занятости".  Другими  словами,  технизация  не  может  ограничиться  воздействием  на
"экономическое" и "военное" поведение людей, как его понимали до сих пор, она все больше будет вторгаться в
резервацию  индивидуального  произвола,  производя  "промышленно  изготовленные  patterns  of  life".  Этим
процессом вряд ли будут затронуты только не-, полу- и едва образованные люди: так называемая элита также не
избежит вовлечения в этот "trend". В прежние времена люди, эпизодически сбрасывавшие с себя тяготы труда,
не нуждались для своего услаждения ни в чем другом, кроме природы, ее мира и красоты. Однако эта мирная и
прекрасная  природа  была  все  же  "Аркадией  человеческой  души",  как  культ  природы,  исполненный
пантеистического  чувства  и  во  многом  обязанный  антикизирующей  учености,  она  возникла  в  эпоху
Возрождения. В отношении к природе современного человека меньше этого возвышающего душу почитания,
дивящегося вселенской гармонии, математической и органической красоте, толкующего природные законы как
истечение  сверхъестественной  мудрости.  Человек  нашего  времени  относится  к  природе  практически-технически,  подходит  к  ней  как  завоеватель  или  по  крайней  мере  как  разведчик.  Туризм,  который  стал
возможным благодаря современным транспортным средствам и был ими вызван, во много раз превосходит по
своим  масштабам  великое  переселение  народов.  То,  что  преподносится  человеческому  любопытству  из
увиденного и услышанного благодаря кино, радио и телевидению, -- вовсе не суррогат естественного опыта, не
предложение  консервированной  духовной  пищи,  но  совершенно  новые  и  оригинальные  источники
переживания,  которые  нацелены  на  планетарную  тотальность  информации,  подобно  тому  как  экономика
развивается с расчетом на мировой рынок. Следует оставить излюбленный "критиками культуры" плач по
"утрате почвы". Техническая регламентация свободного времени -- вовсе не зло априори, даже если учитывать и
признавать, что часто она оборачивается уродством. Пока свободное время рассматривается как время, не
занятое ни трудом, ни политическими делами и обязанностями, оно продолжает контрастировать с трудом и
политической деятельностью. Чем в большей степени труд берут на себя машины, тем больше человек получает
времени для себя самого, однако оно оказывается "опустошенным", незаполненным временем, которое может
быть употреблено для любых целей и занято всем, чем угодно. Пустое время легко обращается в пустыню
скуки, которую приходится разгонять. Таким образом, когда у нас появляется свободное время просто потому,
что мы должны трудиться, после удовлетворения потребности в отдыхе оказывается, что этот промежуток
времени совершенно стерилен и может быть произвольно заполнен каким угодно содержанием. Иначе обстоит
дело, когда мы говорим, что у нас есть свободное время, поскольку и пока мы играем. Свобода времени теперь
означает не "пустоту", а творческое исполнение жизни, а именно -- осуществление воображаемого творчества,
смысловое представление бытия, в известной мере освобождающее нас от свершившихся ситуаций нашей
жизни. Такое освобождение, конечно, не реально и не истинно, мы не избегаем последствий своих поступков.
Человеческая свобода не в силах перескочить свои последствия. Но у нас есть выбор, в сделанном выборе
соустановлена цепочка следований. В игре у нас нет реальной возможности действительно возвратиться к
состоянию перед выбором, но в воображаемом игровом мире мы можем все еще или снова быть тем, кем мы
давно  и  безвозвратно  перестали  быть  в  реальном  мире.  Всякий  акт  свободного  самоосуществления
осуществляет горизонт заранее готовых возможностей. Играя, человек может отстранить от себя ("как бы") все
свое прошлое и вновь начать с точки отсчета. Прошлое, которым мы не располагаем, вновь оказывается в 
нашем  распоряжении.  Возможна  аналогичная  позиция  по  отношению  к  будущему:  реальные  шансы  не
взвешиваются,  не  питается  никаких  ограниченных  надежд,  в  игре  мы  способны  к  самому  свободному
предвосхищению, для нас нет никаких препятствий, мы можем из-мыс-лить их прочь, убрать все оказывающее
сопротивление, можем создать себе на арене игрового мира желанную декорацию. Задержанный временем
человек теряет связь с течением времени, в которое он обычно неизменно вступает или в которое он вовлечен.
Нельзя против этого возразить, что речь все же идет только об иллюзорном, утопическом "освобождении": эта
игровая  свобода  есть  свобода  для  "не-реального"  и  в  нереальном.  В  игровой  "видимости"  упраздняется
историчность человека, игра уводит его из состояний, закрепленных необратимыми решениями, в простор
вообще никогда не  фиксированного бытия, где  все возможно. В игре жизнь представляется нам "легкой",
лишенной тягостей: с нас словно сваливается бремя обязанностей, знаний и забот, игра приобретает черты
грезы, становится общением с "возможностями", которые скорее были из-обретены, нежели обретены. Если
свести  воедино  все  указанные  характеристики  игры  --  магическое  созидание  видимости  игрового  мира,
завороженность  игрового сообщества,  идентификацию  зрителей  с игроками,  самосозерцание человеческого
бытия  в  игре  как  "зерцале  жизни",  до-рациональную  осмысленность  игры,  ее  символическую  силу,
парадигматическую  функцию  и  освобождение  времени  ввиду  обратимости  всех  решений  в  игре,  игровое
облегчение  бытия  и  способность  игры  охватывать  в  себе  все  другие  основные  феномены  человеческого
существования, включая самое себя, то есть способность играть не только в труд, борьбу, любовь и смерть, но и
в игру, -- если мы сведем это воедино, то раскроется праздничный характер игры как общий ее строй. Человек
играет тогда, когда он празднует бытие. "Праздник" прерывает череду отягченных заботами дней, он отграничен
от серого однообразия будней, отделен и возвышен как нечто необычное, особенное, редкое. Но совершенно
недостаточно определять праздник только через противопоставление его будням, ибо праздник имеет значение
и  для  будней,  которым  необходимы  возвеселение,  радость  и  про-яснение.  Праздник  извлечен  из  потока
будничных событий, чтобы служить им маяком, чтобы озарять их. Он обладает репрезентативной, замещающей
функцией. В архаическом обществе яснее видна сущность праздника, нежели в нивелированной временной
последовательности нашей действительности. Там, где верх берут часы, хронометры, точные механизмы для
измерения времени, там человечеству остается все меньше времени для настоящего праздника. Там, где дни и
годы все еще измеряются по ходу солнца и звезд, там празднуют солнцевороты, времена года, различные
космические события, от которых зависит земная жизнь, там празднуют также урожай, который принесло
обработанное  поле,  победу  над  врагом  отчизны,  брачные  торжества  и  роды,  даже  смерть  празднично
справляется как поминовение предков. В праздничном хороводе переплетаются музыка и танец, в хороводе,
который есть нечто большее, нежели непосредственное выражение радости. С музыкой и танцем смыкается
мимический жест -- все это на праздничном игрище, где сообщество празднующих преображается в сообщество
созерцающих,  которые  осмысленно  созерцают  отраженный  образ  бытия  и  приходят  к  предчувственному
прозрению того, что есть. Как коллективное действие игра, наверно, изначально существует в виде праздника.
На заре истории праздник украшался боевыми играми воинов, благодарением за урожай земледельцев, жертвой,
приносимой мертвым, танцевальной игрой юношей и девушек и маскарадом, который ставил все бытие в
зримое присутствие сценического представления. Украшение праздника, которое могло далеко превосходить
будничную потребность в украшении, стало существенным импульсом для возникновения искусства. Конечно,
есть много веских оснований выводить искусство и из мастерства ремесленного умения. Но праздник был
могучим  прорывом  творческих  игровых  сил  человеческого  существа.  Праздничная  игра  стала  корнем  и
основанием человеческого искусства. Игра и искусство внутренне связаны. Конечно, не все игры -- искусство,
но  искусство  есть  наиболее  оригинальная  форма  игры,  она  есть  высочайшая  возможность  посредством
"видимости" явить сущность. Могут возразить: разве искусство не оканчивается на произведении искусства, на
реальном образе, который лучится собственной непотаенной красотой? Нельзя этого отрицать, но способ бытия
произведения  искусства  как  такового  все  же  остается  проблемой.  Что  есть  произведение  искусства:
неподдельно реальная вещь или же вокруг этой вещи всегда есть некая аура -- как бы незримая сцена? Является
ли микельанджеловский "Давид" мастерски высеченной мраморной глыбой на одной из площадей Флоренции
или же он стоит в собственном "воображаемом" мире -- спокоен, в сознании своей силы, праща закинута через
плечо, холодный испытующий взгляд устремлен на превосходящего мощью врага? "Давид" -- и то, и другое;
искусно высеченная мраморная глыба, но и юноша, приготовившийся к борьбе не на жизнь, а на смерть.
Произведения  искусства  озаряются собственным сиянием,  они  стоят словно в "просвете",  к которому мы,
созерцающие и рассматривающие, направляем свой взор и который "раскрывается" нам. Здесь мы не пытаемся
дать  философскую  теорию  произведению  искусства.  Речь  идет  исключительно  о  сжатом  указании  на
соотношение  игры  и  искусства.  Игра  есть  корень  всякого  человеческого  искусства.  Ребенку  и  художнику
наиболее  ясно  открывается  контур  игры  как  творчески-созидательного  общения  с  раскрывающимися
возможностями. Праздник как собирание и представление всех бытийных отношений имеет также и еще одно
важное значение. В архаическом обществе праздник понимался как волшебное заклятие сверх-человеческих
сил, как призыв добрых демонов, изгнание злобных кобольдов, как исключительно благоприятная возможность
для эпифании богов. Праздничное пиршество превращается в культовое жертвенное пиршество, на котором
смертные смешиваются с бессмертными, вкушают в хлебе земную плоть, в вине -- земную кровь. Зрелище
объединяет  культовое  сообщество  в  мироозначающей  и  мироистолковывающей  игре,  в  замаскированных
персонажах сцены игрового мира указывает воочию богов и полубогов, обычно ускользающих от человеческого
глаза. Репрезентирующая функция игры исполняется тут двояко: фигура игрового мира замещает нечто, что 
обладает сверхреальностью сущности, а декорация замещает всю вселенную. По отношению к богам человек
здесь выступает не так, как по отношению к себе или себе подобным: он относит, себя, веруя, к существу,
которому  принадлежит  управление  миром.  В  человеческой  игре,  очевидно,  легче  символизировать
человеческое, нежели то умозрительное существо, которое не трудится и не борется, не любит и не умирает, как
мы. Бог всегда сведущ во всем, без усилий проникает он своим ясновидящим взором от одного конца света до
другого и силой своего помысла, по Анаксагору, сотрясает все. Есть ли подобное существо, мы не можем знать
с уверенностью и подлинной надежностью. То, что в человеческой игре, с тех пор, вероятно, как человек играет,
имеются роли и образы богов, еще не доказывает, что они есть на самом деле. В явление игрового мира игра
может выводить не только то, что имеется вне игры в жизненных сферах труда, господства и т.д. Игра -- не
всегда и не исключительно осмысленное зерцало действительности. Не все, что может быть сыграно, обязано
поэтому  и  существовать.  Играя,  игра  может  воспроизводить  собственную  силу  вымысла  и  выводить  в
воображаемое присутствие творения грезы. Фантазия поэтов создала вымышленные существа различного рода,
сирен  и  лемуров;  химера  существует  в  воображении.  Человеческая  игра,  --  нечто  большее,  нежели
"самоизмышление"  различных  химер,  больше,  нежели  только  представляемое  поведение.  В  своем
прагматическом и опредмечивающем видении сцен игрового мира игра открывает возможности, которые мы
созерцаем именно в качестве являющей себя видимости. Боги приходят в человеческую игру и "пребывают" в
ней, захватывая и завораживая нас. Культ, миф, религия, поскольку они человеческого происхождения, равно
как  и  искусство,  уходят  своими  корнями  в  бытийный  феномен  игры.  Но  кто  сможет  недвусмысленно
утверждать, что религия и искусство лишь отражаются в игре или что они как раз произошли из игровой
способности  человеческого  рода?  Как  бы  то  ни  было,  человеческую  игру,  это  глубоко  двусмысленное
экзистенциальное состояние, кажется, озаряет милость небожителей и, уж конечно, -- улыбки муз.

\newpage
\section{Понятие культуры. Культура и природа. Культура и цивилизация}
Культура (лат. cultura, от корня colere — «возделывать») — обобщающее понятие для форм жизнедеятельности
человека, созданных и создаваемых нами в процессе эволюции. Культура — это нравственные, моральные и
материальные ценности, умения, знания, обычаи, традиции. 
Во многом, современное понятие «культуры» как цивилизации сформировалось в XVIII — начале XIX веков в
Западной  Европе.  В  дальнейшем,  это  понятие,  с  одной  стороны,  стало  включать  отличия  между  разными
группами людей в самой Европе, а с другой стороны — различия между метрополиями и их колониями по
всему миру. Отсюда то, что в данном случае понятие «культуры» является эквивалентом «цивилизации», то есть
антипода понятию «природы». Используя такое определение, можно с лёгкостью классифицировать отдельных
людей и даже целые страны по уровню цивилизованности. Отдельные авторы даже определяют культуру просто
как «всё лучшее в мире, что было создано и сказано» (Мэтью Арнольд), а всё что не попадает в это определение
— хаос и анархия. С этой точки зрения, культура тесно связана с социальным развитием и прогрессом в
обществе.  Арнольд  последовательно  использует  своё  определение:  «…культура  является  результатом
постоянного совершенствования, вытекающего из процессов получения знаний обо всём, что нас касается, её
составляет всё лучшее, что было сказано и помыслено» (Арнольд, 1882).
На практике, понятие культуры относится ко всем лучшим изделиям и поступкам, в том числе в области моды,
искусства и классической музыки. С этой точки зрения, в понятие «культурный» попадают люди, каким-либо
образом  связанные  с этими  областями. При  этом  люди, причастные к  классической  музыке, находятся  по
определению  на  более  высоком  уровне,  чем  любители  панк-рока  из  рабочих  кварталов  или  аборигены
Австралии.
Люди, поддерживающие такую точку зрения, зачастую отвергают множественное понимание культуры. Они не
верят в существование различных культур, каждой со своей логикой и своими ценностями. Фактически, для них
существует только один стандарт, который необходимо применять ко всем без исключения. Таким образом,
согласно  этому  мировоззрению,  люди,  не  укладывающиеся  в  общие  рамки,  сразу  причисляются  к
«некультурным», у них отбирается право на наличие «своей» культуры.
Однако,  в  рамках  такого  мировоззрения,  существует  своё  течение  —  где  менее  «культурные»  люди
рассматриваются,  во  многом,  как  более  «естественные»,  а  «высокой»  культуре  приписывается  подавление
«человеческой природы». Такая точка зрения встречается в работах многих авторов уже начиная с XVIII-го
века. Они, например, подчёркивают, что народная музыка (как созданная простыми людьми) честнее выражает
естественный образ  жизни,  в то  время как  классическая  музыка выглядит  поверхностной и  декадентской.
Следуя такому мнению, люди за пределами «западной цивилизации» — «благородные дикари», не испорченные
сильно расслоённым капитализмом Запада.
Сегодня большинство исследователей отвергают обе крайности. Они не принимают и понятие «единственно
правильной»  культуры,  так  и  полное  противопоставление  её  природе.  В  данном  случае  признаётся,  что
«неэлитарное» может обладать столь же высокой культурой, что и «элитарное», а «незападные» жители могут
быть  столь  же  культурными,  просто  культура  которых  выражается  другими  способами.  Однако  в  данной
концепции проводится различие между «высокой» культурой, как культурой элит, и «массовой» культурой,
подразумевающей  товары  и  произведения,  направленные  на  потребности  простых  людей.  Следует  также
отметить, что в некоторых работах оба вида культуры, «высокая» и «низкая», относятся просто к различным
субкультурам.
Одна из основных оппозиций философии, культурологии, социологии культуры. Оппозиция «К. — П.» является
предпосылкой подразделения всех наук на науки о культуре (К.) и науки о природе (П.), она лежит в основе
определения  предмета  культурологии  и  предмета  социологии  К.  Самым  общим  образом  К.  может  быть
охарактеризована  как  все  то,  что  создано  руками  и  умом  человека  в  процессе  его  исторической
жизнедеятельности, соответственно, П. — это все существующее и не созданное человеком. К. представляет
собой то, что не имело бы места и не способно было бы удерживаться в дальнейшем без постоянных усилий и
поддержки человека; П. — то, что не является результатом человеческой деятельности и может существовать
независимо от нее. Очевидно, что с расширением сферы К. область П. соответствующим образом сужается;
если К. хиреет, съеживается и гаснет, сфера природного расширяется.
К.включает материальную и духовную части. К К. относятся как созданные человеком здания, машины, каналы,
предметы повседневной жизни и т.п., так и созданные им идеи, ценности, религии, научные теории, нормы,
традиции, правила грамматики и ритуала, и т.п. К. находится в постоянной динамике, она радикально меняется
от эпохи к эпохе. Это означает, что постоянно меняется и понятие П., противопоставляемое понятию К. 
Широкое понимание К. как противоположности П. необходимо при обсуждении общих проблем становления и
развития К. и, в частности, проблемы видения П. конкретными К. Это понимание является в известном смысле
классическим, хотя оно не единственно возможное.

\newpage
\section{Культура элитарная и массовая}
Массовая культура формирует иную, ту, что называют высокой, или лучше — элитарной. Причем по разным
оценкам потребителями элитарной культуры в Европе на протяжении нескольких веков остается примерно одна
и та же доля населения — что-то около одного процента. Именно массовая культура — индикатор многих
сторон жизни общества и одновременно коллективный пропагандист и организатор его, общества, настроений.
Внутри  массовой  культуры  существует  своя  иерархия  ценностей  и  иерархия  персон.  Взвешенная  система
оценок и, наоборот, скандальные потасовки, драка за место у престола.
Массовая  культура  —  это  часть  общей  культуры,  отделенная  от  элитарной  лишь  большим  количеством
потребителей и социальной востребованностью. Эта определенность не строга, более того, объекты переходят
через эту условную границу довольно часто. Все остальные признаки подобного отделения только следуют из
количественного фактора.
Музыка Моцарта в зале филармонии остается явлением элитарной культуры, а та же мелодия в упрощенном
варианте, звучащая как сигнал вызова мобильного телефона,— явление культуры массовой.
Итак,  в  отношении  субъекта  творчества  -  восприятия  можно  выделить  народную  культуру,  элитарную  и
массовую (см.: массовая культура). При этом народная культура находится практически в стадии музеефикации
- консервации или превращается в сувенирный бизнес.
Элитарность и массовость имеют равное отношение как к феноменам Культуры. В самой массовой культуре
можно выделить, например, стихийно складывающуюся культуру под воздействием массы внешних факторов:
культуру  тоталитарную,  навязанную  массам  тем  или  иным  тоталитарным  режимом  (советским  в  СССР,
нацистским в Германии) и всячески поддерживаемую им. Искусство социалистического реализма является
одной из главных разновидностей такого искусства.
Возможна также фиксация внимания на функционировании и модификации традиционных видов искусства и
появлении  новых.  К  последним  относятся  фотоискусство,  кино,  телевидение,  видео-,  различные  виды
электронных искусств, компьютерное искусство и их всевозможные взаимо-соединения и комбинации. 

\newpage
\section{Социальные общности: поколение, пол, семья, профессия, этнос, народ, нация}
Социальная  общность  —  широкое  понятие,  объединяющее  различные  совокупности  людей,  для  которых
характерны некоторые одинаковые черты жизнедеятельности и сознания.
Общности  различного  типа  —  это  формы  совместной  жизнедеятельности  людей,  формы  человеческого
общежития. Они складываются на различной основе и крайне многообразны. Это общности, формирующиеся в
сфере общественного производства (классы, профессиональные группы и т. п.), вырастающие на этнической
основе (народности, нации), на основе демографических различий (половозрастные общности) и др.
Исторически первой формой социальной общности была семья и такие, основанные на кровнородственных
отношениях, социальные общности, как род и племя. В дальнейшем социальные общности формируются также
и на других основаниях и несут на себе отпечатки конкретного социально-экономического строя.
Для социальных общностей характерно не только наличие общих объективных характеристик, но и осознание
единства своих интересов по сравнению с др.общностями, более или менее развитое чувство «мы». Именно на
этой основе происходит превращение простой (статистической) совокупности людей, обладающих общими
объективными характеристиками, в реальную социальную общность (в частности, «класса в себе» в «класс для
себя»).
Люди  одновременно  являются  членами  различных  общностей,  с  разной  степенью  внутреннего  единства.
Поэтому часто единство в одном (напр., в национальной принадлежности) может уступить место различию в
другом (например, в классовой принадлежности).
Функционирование социальных отношений , институтов контроля и организаций порождает сложную систему
социальных связей , управляющую потребностями , интересами и целями людей . Эта система сплачивает 
индивидов и их группы в единое целое - социальную общность и через нее - в социальную систему . Характер
социальных связей определяет как внешнюю структуру социальных общностей , так и ее функции . Внешняя
структура  общности  может  быть  определена  ,  например  ,  ее  объективными  данными  :  сведениями  о
демографической структуре общности , профессиональной структуре , об образовательной характеристике ее
членов и т. п.
Функционально социальные общности направляют действия своих членов на достижения групповых целей .
Социальная  общность  обеспечивает  координацию  этих  действий  ,  что  ведет  к  повышению  ее  внутренней
сплоченности  .  Последняя  возможна  благодаря  образцам  поведения  ,  нормам  ,  определяющим  отношения
внутри  этой  общности  ,  а  также  социально-психологическим  механизмам  ,  направляющим  поведение  ее
членов .
Среди многих видов социальных общностей особое значение с точки зрения влияния на поведение имеют
такие, как семья , трудовой коллектив , группы совместного проведения досуга , а также различные социально-территориальные общности ( поселок, небольшой город , крупные города , регион и т. д. ) . Скажем , семья
осуществляет социализацию молодежи в ходе освоения ею нормативов общественной жизни, формирует у нее
чувство  безопасности  ,  удовлетворяет  эмоциональную  потребность  в  совместных  переживаниях  ,
предотвращает психологическую неуравновешенность , помогает преодолеть состояние изолированности и т. д.
Территориальная общность , ее состояние также влияют на характер поведения ее членов , в особенности в
сфере  неформальных  контактов  .  Профессиональные  группы  кроме  возможности  решения  чисто
профессиональных  вопросов  формируют  у  членов  чувство  трудовой  солидарности  ,  обеспечивают
профессиональный  престиж  и  авторитет  ,  контролируют  поведение  людей  с  позиций  профессиональной
морали. 

\newpage
\section{Происхождение, сущность и функции государства}
Пристального внимания заслуживает вопрос о происхождении, сущности и функциях государства,т.к. именно
государство является ядром политической системы, самым древним и развитым политическим институтом. Вот
почему во всех крупных философских системах, начиная с античности, присутствует учение о государстве. Это
теория идеального государства Платона, "Политика" Аристотеля, идеи Т.Гоббса и Дж.Локка об общественном
договоре, диалектика гражданского общества и государства Гегеля, классовая концепция государства К.Маркса
и В.И.Ленина.
Предметом особых дискуссий является вопрос о происхождении государства. Рассмотрим некоторые точки
зрения на эту проблему. Античная философия в лице Платона и Аристотеля рассматривала возникновение
государства  как  проявление  естественных  потребностей,  присущих  человеку.  Формирование  сословий
философов,  воинов  и  работников  и  было  проявлением  этих  потребностей.  У  Аристотеля  государство
отождествлялось  с  обществом,  поэтому  политическая  сфера  охватывала  все:  семью,  религию,  культуру,
искусство. Быть членом общества означало быть членом государства, следовательно действовать в соответствии
с  его  законами.  Философия  нового  времени  выдвигает  договорную  теорию  происхождения  государства,
представителями  которой  были  Т.Гоббс,  Ж.Руссо,  А.Радищев.  Они  полагали,  что  государство  возникло  в
результате общественного договора, сознательно заключенного людьми. Это - сила, служащая всему обществу:
и богатыми и бедными, оно формируется для обеспечения мира и безопасности граждан. Положительным
моментом  этой  теории  было  то,  что  впервые  исследователи  подчеркнули  земное,  человеческое,  а  не
божественное происхождение государства.
Теория насилия, представленная Е.Дюрингом и Л.Гумпловичем, объясняла появление государства в результате
войн и политического насилия, которые углубляют социальное неравенство, приводят к образованию классов и
эксплуатации.
Существует так же психологическая теория, которая рассматривает возникновение государства как результат
того, что у одних людей существует потребность подчинять, а у других подчиняться. Представителем этой
точки зрения был, например, Шопенгауэр.
С середины XIX века возникает новая, классовая теория происхождения государства, которая получает свое
логическое  завершение  в  марксистской  теории  государства.  Согласно  К.Марксу  и  Ф.Энгельсу  государство
возникает  в  результате  раскола  общества  на  антагонистические  классы.  Развивая  именно  эту,  классовую
особенность государственного господства, В.И.Ленин пришел к выводу о том, что государство - это машина для
подавления  одних  классов  другими,  аппарат  насилия,  который  возникает  в  силу  того,  что  классовые
противоречия объективно не могут быть примирены. Оно появляется, возникает не в силу внешних причин
(войны  и  политическое  насилие),  хотя  и  они  тоже  играют  свою  роль,  но  под  воздействием  внутренних
экономических причин.
Решение  проблемы  происхождения  государства  определяет  и  ответ  на  вопрос  о  его  сущности.  Следует
подчеркнуть,  что  в  настоящее  время  постепенно  преодолевается  свойственный  марксизму  узкоклассовый
подход к сути государства, как только органа одного, господствующего класса, противостоящего другим, как
аппарата  насилия,  давления  на  общество.  Государство  остается  институтом,  при  помощи  которого
экономически господствующий класс становится господствующим и политически. Но при этом подчеркивается
и  всеобщность  ряда  функций  государства  как  органа,  управляющего  делами  всего  общества,  а  также
выражающего  в  той  или  иной  степени  интересы  и  защищающего  всех  и  каждого  конкретного  человека. 
Государство,  таким  образом,  представляет  систему  организованной  политической  власти,  направленной  на
регулирование  как  классовых,  так  и  других  социально-политических  отношений.  Оно  является  формой
общественного  сосуществования  всех  членов  общества,  выражая  и  охраняя  права,  обязанности  и  свободу
каждого.  Если  партии  и  другие  общественные  организации  представляют  интересы  отдельных  классов  и
социальных групп в политической системе, то государство выражает всеобщий интерес.
Все это дает право дать следующее его определение: государство есть социальный институт, осуществляющий
всю полноту политической власти и управления на определенной территории.
Изначально любое государство выполняло триединую задачу: 
-управлять хозяйством и обществом; 
-защищать власть класса эксплуататоров и подавлять сопротивление эксплуатируемых; 
-оборонять собственную территорию и (если имеется возможность) грабить чужую. 
По  мере  развития  общественных  отношений  появилась  возможность  более  цивилизованного  поведения
государства.
Природа государства и его положение в политической системе предполагают наличие ряда специфических
функций, отличающих его от других политических институтов. Функциями государства называются основные
направления его деятельности, связанные с суверенитетом государственной власти. От функций отличаются
цели и задачи государства, отражающие основные направления избираемой тем или иным правительством или
режимом политической стратегии, средства её реализации.
Функции государства классифицируются:
* по сфере общественной жизни: на внутренние и внешние,
* по продолжительности действия: на постоянные (осуществляемые на всех этапах развития государства) и
временные (отражающие определённый этап развития государства),
* по значению: на основные и неосновные,
* по влиянию на общество: на охранительные и регулятивные.
Основной классификацией является деление функций государства на внутренние и внешние. К внутренним
функциям государства относятся:
* Правовая функция — обеспечение правопорядка, установление правовых норм, регулирующих общественные
отношения и поведение граждан, охрана прав и свобод человека и гражданина.
* Политическая функция — обеспечение политической стабильности, выработка программно-стратегических
целей и задач развития общества.
*  Организаторская  функция  —  упорядочивание  всей  властной  деятельности,  осуществление  контроля  за
исполнением законов, координация деятельности всех субъектов политической системы.
* Экономическая функция — организация, координация и регулирование экономических процессов с помощью
налоговой и кредитной политики, планирования, создания стимулов экономической активности, осуществления
санкций.
* Социальная функция — обеспечение солидарных отношений в обществе, сотрудничества различных слоев
общества, реализации принципа социальной справедливости, защита интересов тех категорий граждан, которые
в  силу  объективных  причин  не  могут  самостоятельно  обеспечить  достойный  уровень  жизни  (инвалиды,
пенсионеры, матери, дети), поддержка жилищного строительства, здравоохранения, системы общественного
транспорта.
*  Экологическая  функция  —  гарантирование  человеку  здоровой  среды  обитания,  установление  режима
природопользования.
* Культурная функция — создание условий для удовлетворения культурных запросов людей, формирования
высокой духовности, гражданственности, гарантирование открытого информационного пространства.
* Образовательная функция — деятельность по обеспечению демократизации образования, его непрерывности
и качественности, предоставлению людям равных возможностей получения образования.
К внешним функциям государства относятся:
* Функция обеспечения национальной безопасности — поддержание достаточного уровня обороноспособности
общества, защита территориальной целостности, суверенитета государства.
*  Функция  поддержания  мирового  порядка  —  участие  в  развитии  системы  международных  отношений,
деятельность по предотвращению войн, сокращению вооружений, участие в решении глобальных проблем
человечества.
* Функция взаимовыгодного сотрудничества в экономической, политической, культурной и других сферах с
другими государствами.
Также проводится разделение между:
* деятельностью по выработке политических решений и
* деятельностью по выполнению этих решений — государственному управлению.

\newpage
\section{Массовое общество и феномен «восстания масс»}
"Массовое  общество"  (англ.  mass  society),  понятие,  употребляемое  немарксистскими  социологами  и
философами  для  обозначения  ряда  специфических  черт  современного  общества.  В  области  социально-экономической "М. о."  связывается с индустриализацией и урбанизацией,  стандартизацией  производства  и 
массовым  потреблением,  бюрократизацией  общественной  жизни,  распространением  средств  массовой
коммуникации и "массовой культуры".
"ВОССТАНИЕ МАСС" ("La Rebelion de las masas", 1930) — работа Ортеги-и-Гассета. Философ констатирует,
что в современной Европе происходит явление "полного захвата массами общественной власти". "Масса", как
полагает Ортега-и-Гассет, есть "совокупность лиц, не выделенных ничем". По его мысли, плебейство и гнет
массы даже в традиционно элитарных кругах — характерный признак современности: "заурядные души, не
обманываясь насчет собственной заурядности, безбоязненно утверждают свое право на нее и навязывают ее
всем и всюду". Новоявленные политические режимы оказываются результатом "политического диктата масс". В
то же время, согласно убеждению Ортеги-и-Гассета, чем общество "аристократичней, тем в большей степени
оно  общество, как и  наоборот".  Массы, достигнув  сравнительно высокого  жизненного уровня, "вышли  из
повиновения, не подчиняются никакому меньшинству, не следуют за ним и не только не считаются с ним, но и
вытесняют его и  сами его замешают". Автор  акцентирует  призвание  людей "вечно быть  осужденными на
свободу, вечно решать, чем ты станешь в этом мире. И решать без устали и без передышки". Представителю же
массы  жизнь  представляется  "лишенной  преград":  "средний  человек  усваивает  как  истину,  что  все  люди
узаконенно равны". "Человек массы" получает удовлетворение от ощущения идентичности с себе подобными.
Его  душевный  склад  суть  типаж  избалованного  ребенка.  По  мысли  Ортеги-и-Гассета,  благородство
определяется  "требовательностью  и  долгом,  а  не  правами".  Личные  права  суть  "взятый  с  бою  рубеж".
"Всеобщие" же права типа "прав человека и гражданина", "обретаются по инерции, даром и за чужой счет,
раздаются  всем  поровну  и  не  требуют  усилий...  Всеобщими  правами  владеют,  а  личными  непрестанно
завладевают". Массовый человек полагает себя совершенным, "тирания пошлости в общественной жизни, быть
может,  самобытнейшая  черта  современности,  наименее  сопоставимая  с  прошлым.  Прежде  в  европейской
истории чернь никогда не заблуждалась насчет собственных идей касательно чего бы то ни было. Она ...не
присваивала себе умозрительных суждений — например, о политике или искусстве — и не определяла, что они
такое и чем должны стать... Никогда ей не взбредало в голову ни противопоставлять идеям политика свои, ни
даже  судить  их,  опираясь  на  некий  свод  идей,  признанных  своими...  Плебей  не  решался  даже  отдаленно
участвовать  почти  ни  в  какой  общественной  жизни,  по  большей  части  всегда  концептуальной.  Сегодня,
напротив,  у  среднего  человека  самые  неукоснительные  представления  обо  всем,  что  творится  и  должно
твориться во Вселенной". Как подчеркивает Ортега-и-Гассет, это "никоим образом" не прогресс: идеи массового
человека не есть культура, "культурой он не обзавелся": в Европе возникает "тип человека, который не желает
ни признавать, ни доказывать правоту, а намерен просто-напросто навязать свою волю". Это "Великая Хартия"
одичания: это агрессивное завоевание "права не быть правым". Человек, не желающий, не умеющий "ладить с
оппозицией", есть "дикарь, внезапно всплывший со дна цивилизации". 19 в. утратил "историческую культуру":
большевизм и фашизм... отчетливо представляют собой, согласно Ортеге-и-Гассету, движение вспять. Свою
долю исторической истины они используют "допотопно", антиисторически. Едва возникнув, они оказываются
"реликтовыми": "произошедшее в России исторически невыразительно, и не знаменует собой начало новой
жизни". Философ пишет: "Обе попытки — это ложные зори, у которых не будет завтрашнего утра". Ибо
"европейская история впервые оказалась отданной на откуп заурядности... Заурядность, прежде подвластная,
решила  властвовать".  "Специалисты",  узко  подготовленные  "ученые-невежды",  —  наитипичнейшие
представители "массового сознания". "Суть же достижений современной Европы в либеральной демократии и
технике. Главная же опасность Европы 1930-х, по мысли Ортеги-и-Гассета, "полностью огосударствленная
жизнь, экспансия власти, поглощение государством всякой социальной самостоятельности". Человека массы
вынудят жить для государственной машины. Высосав из него все соки, она умрет "самой мертвой из смертей —
ржавой смертью механизма".

\newpage
\section{Основные концепции философии истории}
Философия истории анализирует проблемы смысла и цели существования общества, перспективы его развития.
С  возникновением  христианства  оказалось  возможным  иное  понимание  истории.  Оно  предполагало
поступательное прогрессивное развитие в истории человечества: от акта творения Богом до ее "финала" -второго пришествия Христа и Страшного суда. В новое время земная история перестала восприниматься как
священная история. В XVIII веке в работах просветителей постоянно звучала тема мощи человеческого разума,
который следовало лишь освободить от пут религии и предрассудков. История оценивалась теперь как история
разума. Наиболее системно представление о неизбежности прогресса в движении человечества было дано в
философии Гегеля. В марксистской концепции общества прогресс рассматривался как результат неуклонного
развития производительных сил. С конца XX века понятие прогресса общества и истории все более связывается
с развитием телесных и духовных характеристик самого человека. С особой силой философия истории XX века
поставила проблему коммуникации как основания человеческого существования. История возможна лишь в той
мере, в какой люди открыты миру и друг другу. История реализуется через общение.
Философы и социологи используют различные подходы к изучению общественного прогресса. В марксистской
философии,  например,  был  разработан  формационный  подход.  С  точки  зрения  формационного  подхода,
исторический  прогресс  понимается  как  смена  общественно-экономических  формаций.  Общественно-экономическая формация - это исторически конкретное общество на данном этапе его развития. Она включает в  
себя все явления, которые имеются в обществе: материальные, духовные, политические, социальные, семейно-бытовые. Основу общественно-экономической формации составляет способ производства материальной жизни
в единстве производительных сил и производственных отношений. В развитии, исторического процесса Маркс
выделял  пять  формаций:  первобытнообщинную,  рабовладельческую,  феодальную,  капиталистическую  и
коммунистическую. Формационный подход к развитию общества существует наряду с культурологическим и
цивилизационным подходами.
Культурологический подход к истории использовал Освальд Шпенглер (1880- 1936). Он исходил из того, что
каждая  культура  существует  изолированно  и  замкнуто.  Появляясь  на  определенном  этапе  исторического
процесса,  она  переживает  возрасты  отдельного  человека  (детство,  юность,  зрелость  и  старость)  и  затем
погибает. Смерть культуры, по Шпенглеру, начинается с возникновения цивилизации.
Понятие  "культура"  относится  к  числу  фундаментальных  в  современном  обществознании.  Культуру  часто
определяют как "вторую природу", т.е. культура есть природа, обработанная человеком в целях удовлетворения
тех  или  иных  потребностей.  Однако  культуру  нельзя  свести  только  к  вещам,  произведенным  человеком.
Понятие культуры охватывает собой и продукты духовного производства, распространяется на общественные
отношения. Суть культуры заключается в том, что она несет в себе систему и природных, и социальных качеств.
Культура обнаруживает себя в истории. Одним из главных критериев жизнестойкости и прогресса культуры
является способность одной культуры вбирать и осваивать достижения других культур. Культура обусловлена
потребностью общества в закреплении и передаче совокупного духовного опыта.
Категория  цивилизации  охватывает  природу  и  уровень  развития  материальной  и  духовной  культуры.
Цивилизация воплощает в себе технологический аспект культуры. Главное в цивилизации - это непрерывная
смена  технологий  для  удовлетворения  столь  же  непрерывно  растущих  потребностей  и  возможностей
человечества.

\newpage
\section{Проблема социального прогресса, его факторы и критерии}
Социальный  прогресс  –  направленный  процесс,  который  неуклонно  приводит  систему  все  ближе  к  более
предпочтительному,  лучшему  состоянию  (по  мнению  большинства  исследователей  –  к  реализации
определенных ценностей этического порядка: счастью, свободе, процветанию, знаниям).
Идея  прогресса  лежит  в  фундаментальной  особенности  человеческого  бытия  –  противоречии  между
реальностью  и  желаниями,  жизнью  и  мечтами.  Концепция  прогресса  смягчает  возникающее  при  этом
напряжение, порождая надежду на лучший мир в будущем и уверяя, что его приход гарантирован или, по
крайней  мере,  возможен  ("Мир  сегодня  верит  в  прогресс,  потому  что  единственной  альтернативой  будет
всеобщее отчаяние" (С. Поллард)).
Современное толкование социального прогресса основывается на нескольких фундаментальных идеях: 1) о
необратимом времени, текущем линейно и обеспечивающем непрерывность прошлого, настоящего и будущего
(прогресс  есть  положительно  оцениваемая  разница  между  прошлым  и  настоящим);  2)  о  направленном
движении, в котором ни одна стадия не повторяется; 3) о кумулятивном процессе, протекающем либо по
возрастающей, шаг за шагом, либо революционным путем; 4) о различии между типичными, необходимыми
стадиями, которые проходит процесс; 5) об эндогенных причинах, вызывающих самодвижение (саморазвитие)
процесса; 6) о признании неизбежного, необходимого, естественного характера процесса, который не может
быть остановлен или отвергнут; 7) об улучшении, усовершенствовании, отражающими тот факт, что каждая
последующая стадия лучше предыдущей.
Кульминацией прогресса должна стать полная реализация таких ценностей, как счастье, изобилие, свобода,
справедливость, равенство. Отсюда следует, что прогресс – ценностная категория. И каждая историческая эпоха
оценивает его исходя из своего понимания ценностей (в XIX в. критериями прогресса были индустриализация,
урбанизация, модернизация; в начале XXI в. они таковыми уже не считаются).
В  ХХ  в.  социальные  процессы  шли  крайне  противоречиво,  их  оценка  исследователями  и  общественным
мнением была амбивалентной (как позитивной, так и негативной). В конце концов, это привело к кризису идеи
социального прогресса.
Проявления  кризиса  идеи  прогресса:  1)  идея  прогресса  сменилась  распространением  мистицизма,  бунтом
против  рассудка  и  науки,  всеобщим  пессимизмом  в  образе  дегенерации,  разрушения  и  упадка;  2)  идея  о
необходимости постоянного экономического и технологического роста сменилась идеей пределов роста; 3) вера
в  рассудок  и  науку  сменилась  убеждением  в  доминирующей  роли  эмоций,  интуиции,  подсознательного  и
бессознательного, утверждении иррационализма; 4) утверждение о важности, высочайшей ценности жизни на
земле  сменилось  чувством  бессмысленности,  аномии  и  отчуждения;  5)  крушение  идей  утопизма
(окончательный  удар  по  утопическому  мышлению  нанесло  падение  коммунистической  системы);  6)
лейтмотивом конца ХХ – начала XXI в. стало повсеместное распространение идеи кризиса. При этом люди
склонны  рассматривать  социальный  кризис  как  хронический,  всеобщий  и  не  предвидят  его  будущего
ослабления.
Однако  не  все  исследователи  настроены  столь  пессимистично.  По  мнению  П.  Штомпки,  в  современном
обществе существуют реальные возможности и условия для обеспечения социального прогресса сегодня и в
обозримом  будущем.  Для  этого  есть  все  необходимые  предпосылки:  1)  наличие  в  обществе  творческих,
независимых, адекватно осознающих реальность деятелей; 2) богатые и гибкие общественные структуры; 3) 
благоприятные  и  активно  воспринимаемые  естественные  условия;  4)  долгая  и  уважаемая  традиция;  5)
оптимистичный, долгосрочный взгляд на будущее и его планирование. 

\newpage
\section{Общественное сознание и его основные формы}
Общественное  сознание  представляет  собой  многогранный  динамический  процесс,  поддерживаемый
активностью  индивидуальных  сознаний.  В  общественном  сознании  содержатся  устойчивые  представления,
связанные с некоторой системой норм и принципов, теории, пытающиеся обобщить особенности различных
сторон общественной жизни. Когда говорят об общественном сознании в собственном смысле слова, имеют в
виду,  прежде  всего  то,  чем  сознание  людей,  объединенных  в  некоторые  группы,  отличается  от  сугубо
индивидуального сознания человека, направленного, скажем, на решение его личных проблем, на организацию
индивидуальной  жизни.  В  этом  смысле  общественное  сознание  —  это  сознание,  всегда  направленное  на
решение общих проблем устройства общественной жизни в целом и на изучение таких свойств окружающего
мира, которые имеют общее значение.
Благодаря наличию у людей общего сознания и закреплению в сознании устойчивых образов лишь таких идей,
которые  оказываются  перспективными  в  практическом  смысле,  общество  функционирует  как  целостный
организм, то есть оно представляет не просто стихийно сложившиеся в процессе производства отношения, но
содержит сознательно упорядочиваемые людьми связи.
Идеи, получающие закрепление в общественном сознании — это не просто отражение действительности, это
еще и реорганизация действительности, практическое приспособление человека к миру. Такое приспособление
осуществляется  за  счет  того,  что  вырабатываются  новые  формы  социальной  связи,  утверждаются  новые
социальные нормы и те идеи, которые оказываются необходимыми для их воспроизводства.
Формы общественного сознания
Общественное  сознание  представлено  в  различных  формах,  в  которых  выражена  специфическая
направленность отражения действительности. Она зависит от объекта отражения и его целей. Среди форм
общественного сознания можно выделить:
* Экономическое сознание;
* Политическое сознание;
* Правовое сознание;
* Нравственное сознание;
* Эстетическое сознание;
* Религиозное сознание;
* Научное сознание;
* Философское сознание;
Во всех формах общественного сознания происходит объединение каждого отдельного человека с некоторой
общностью людей  или со всем  обществом в  целом, при  чем строится такое объединение  на базе  общего
решения  специфических  вопросов  организации  жизни,  устройства  социальных  институтов,  организации
процесса познания и т. д. Формы общественного сознания, поэтому, всегда тесно связаны с определенного типа
общественными  отношениями:  экономическими,  политическими,  нравственными,  эстетическими,
отношениями между членами научного сообщества и др.

\newpage
\section{Этика и нравственное сознание}
ЭТИКА (греч. ethika: от ethos — нрав, обычай, характер, образ мысли) — 1) на уровне самоопределения —
теория морали, видящая свою цель в обосновании модели достойной жизни; 2) практически — на протяжении
всей истории Э. — обоснование той или иной конкретной моральной системы, фундированное конкретной
интерпретацией  универсалий  культуры,  относящихся  к  субъектному  ряду  (см.  Универсалии,  Категории
культуры): добро и зло, долг, честь, совесть, справедливость, смысл жизни и т.д. В силу этого в традиционной
культуре Э. как теоретическая модель морали и Э. как моральное поучение дифференцировались далеко не
всегда  (от  восточных  кодексов  духовной  и  телесной  гигиены  до  Плутарха);  для  классической  культуры
характерна  ситуация,  когда  этики-теоретики  выступали  одновременно  и  моралистами  —  создателями
определенных этических систем (см. Сократ, Эпикур, Спиноза, Гельвеций, Гольбах, Дидро, Руссо, Кант, Гегель);
неклассическая  культура  конституирует  постулат  о  том,  что  Э.  одновременно  выступает  и  теорией
нравственного  сознания,  и  самим  нравственным  сознанием  в  теоретической  форме  (см.  Марксизм).
Фундаментальная презумпция практической морали (так называемое "золотое правило поведения": поступай по
отношению к другому так, как ты хотел бы. чтобы он поступал по отношению к тебе) в то же время выступает и
предметом обоснования для самых различных этических систем — в диапазоне от конфуцианства и вплоть до
категорического императива Канта, Э. ненасилия Л.Н.Толстого (см. Толстой), этической программы Мартина
Лютера Кинга и др. Согласно ретроспективе Шопенгауэра, "основное положение, относительно содержания
которого согласны ... все моралисты, таково: neminem laede, immo omnes, quantum potes, juva /лат. "никому не
вреди и даже, сколь можешь, помогай" — M.M./,— это, собственно, и есть... собственный фундамент этики,
который  в  течение  целых  тысячелетий  разыскивают,  как  философский  камень".  Термин  "этос"  исходно
употреблялся (начиная с древнегреческой натурфилософии) для фиксации комплекса атрибутивных качеств: от
"этоса праэлементов" у Эмпедокла до расширительного употребления термина "Э." в философской традиции: 
"Э." как название общефилософских произведений у Абеляра, Спинозы, Н.Гартмана. Вместе с тем (также
начиная с античной философии) сфера предметной аппликации данного термина фокусируется на феномене
человеческих  качеств,  в  силу  чего  по  своему  содержанию  Э.  фактически  совпадает  с  философской
антропологией  (дифференциация  философии  на  логику,  физику  и  Э.  у  стоиков,  впоследствии
воспроизводящаяся  в  философской  традиции  вплоть  до  трансцендентализма,—  см.  Стоицизм,  Юм,
Возрождение, Философия Возрождения, Просвещение). На основе дифференциации добродетелей человека на
"этические" как добродетели нрава и  "дианоэтические" как добродетели разума  Аристотель конституирует
понятие "Э." как фиксирующее теоретическое осмысление проблемного поля, центрированного вопросом о том,
какой  "этос"  выступает  в  качестве  совершенного.  Нормативный  характер  Э.  эксплицитно  постулируется
кантовской рефлексией над теорией морали, — Э. конституируется в качестве учения о должном, обретая
характер "практической философии" (см. Кант). Содержательная сторона эволюции Э. во многом определяется
конкретными историческими конфигурациями, которые имели место применительно к оппозиции интернализма
и экстернализма в видении морали (которым соответствуют зафиксированные Кантом трактовки Э. в качестве
"автономной" и "гетерономной"). Если в контексте историцизма мораль рассматривалась как сфера автономии
человеческого  духа,  то  в  рамках  традиций  социального  реализма  и  теизма  она  выступала  как
детерминированная  извне  (в  качестве  внешних  детерминант  морали  рассматривались  —  в  зависимости  от
конкретного содержания этических систем — Абсолют как таковой (см. Предопределение, Провиденциализм);
традиции  национальной  культуры  (этноэтика);  сложившиеся  социальные  отношения  (от  Э.Дюркгейма  до
неомарксизма); корпоративный (классовый) интерес (классический марксизм); уровень интеллектуального и
духовного  развития,  характерный  для  субъекта  морали  и  социального  организмов  целом  (практически  вся
философия  Просвещения,  исключая  Ж.Ж.Руссо,  и  отчасти  философия  Возрождения);  специфика
доминирующих воспитательных стратегий (от Д.Дидро до М.Мид) и пр. Однако в любом случае — в системе
отсчета субъекта — Э. конституируется в концептуальном пространстве совмещения презумпций интернализма
и экстернализма: с одной стороны, фиксируя наличие нормативной системы должного, с другой же — оставляя
за индивидуальным субъектом прерогативу морального выбора. — В этом отношении свобода как таковая
выступает  в  концептуальном  пространстве  Э.  в  качестве  необходимого  условия  возможности  моральной
ответственности (в системах теистической Э. именно в предоставлении права выбора проявляется любовь
Господа к человеку, ибо дает ему возможность свободы и ответной любви — см., например, у В.Н.Лосского).
Содержание конкретных систем Э. во многом вторично по отношению к фундирующим его общефилософским
презумпциям (которые при этом могут выступать для автора этической концепции в качестве имплицитных):
эмотивная  Э.  как  выражение  презумпций  позитивизма,  Э.  диалога  как  предметная  экземплификация
диалогической  философии  в  целом,  аналогичны  эволюционная  Э.,  аналитическая  Э.,  Э.  прагматизма,
феноменологическая Э., Э. экзистенциализма и т.п. К центральным проблемам традиционной Э. относятся
проблема соотношения блага и должного (решения которой варьируются в диапазоне от трактовки долга как
служения  благу  —  до  понимания  блага  как  соответствия  должному);  проблема  соотношения  мотивации
нравственного  поступка  и  его  последствий  (если  консеквенциальная  Э.  полагает  анализ  мотивов
исчерпывающим для оценки нравственного поступка, то альтернативная позиция сосредоточивает внимание на
оценке  его  объективных  последствий,  возлагая  ответственность  за  них  на  субъекта  поступка);  проблема
целесообразности морали (решения которой варьируются от артикуляции нравственного поступка в качестве
целерационального  до  признания  его  сугубо  ценностно-рациональным)  и  т.п.  В  рамках  неклассической
традиции статус Э. как универсальной теории морали подвергается существенному сомнению даже в качестве
возможности: согласно позиции Ницше, Э. как "науке о нравственности ...до настоящего времени недоставало,
как  это  ни  покажется  ...странным,  проблемы  самой  морали:  отсутствовало  даже  всякое  подозрение
относительно  того,  что  тут  может  быть  нечто  проблематичное.  То,  что  философы  называли  "основанием
морали...,  было  ...  только  ученой  формой  доброй  веры  в  господствующую  мораль,  новым  средством  ее
выражения".  В  этом  контексте  любое  суждение  морального  характера  оказывается  сделанным  изнутри
определенной  моральной  системы,  что  обусловливает  фактическую  невозможность  мета-уровня  анализа
феноменов нравственного порядка. В силу этого "само слово "наука о морали" в применении к тому, что им
обозначается,  оказывается  слишком  претенциозным  и  не  согласным  с  хорошим  вкусом,  который  всегда
обыкновенно  предпочитает  слова  более  скромные"  (Ницше).  В  качестве  альтернативы  традиционным
претензиям  на  построение  Э.  как  аксиологически  нейтральной  теоретической  модели  морали  Ницше
постулирует создание "генеалогии морали": "право гражданства" в этом контексте имеет лишь реконструкция
процессуальности  моральной  истории,  пытающаяся  "охватить  понятиями  ...  и  привести  к  известным
комбинациям огромную область тех нежных чувств ценности вещей и тех различий в этих ценностях, которые
живут, растут, оставляют потомство и гибнут", — реконструкция, являющая собой "подготовительную ступень
к учению о типах морали", но не претендующая при этом на статус универсальной теории, обладающей правом
и  самой  возможностью  якобы  нейтральных  аксиологически  суждений.  Таким  образом,  радикальный  отказ
неклассической философии от Э. в ее традиционном понимании фундирует собой идею "генеалогии морали",
т.е.  реконструкцию  ее  исторических  трансформаций,  вне  возможности  конституирования  универсальной
системы Э. на все времена (см. Ницше). (Позднее предложенный Ницше в этом контексте генеалогический
метод выступит основой конституирования генеалогии как общей постмодернистской методологии анализа
развивающихся систем — см. Генеалогия, Фуко.) Что же касается культуры постнеклассического типа, то она не
только  углубляет  критику  в  адрес  попыток  построения  универсально-нейтральной  Э.:  в  семантико-
аксиологическом  пространстве  постмодернизма  Э.  в  традиционном  ее  понимании  вообще  не  может  быть
конституирована как таковая. Тому имеется несколько причин: 1). В контексте радикального отказа постмодерна
от  ригористических  по  своей  природе  "метанарраций"  (см.  Закат  метанарраций)  культурное  пространство
конституирует себя как программно плюралистичное (см. Постмодернистская чувствительность) и ацентричное
(см. Ацентризм), вне какой бы то ни было возможности определения аксиологических или иных приоритетов.
Э. же не просто аксиологична по самой своей сути, но и доктринально-нормативна, в силу чего не может быть
конституирована в условиях мозаичной организации культурного целого (см. Номадология), предполагающего
принципиально  внеоценочную  рядоположенность  и  практическую  реализацию  сосуществования  различных
(вплоть до альтернативных и взаимоисключающих) поведенческих стратегий. 2). Современная культура может
быть  охарактеризована  как  фундированная  презумпцией  идиографизма,  предполагающей  отказ  от
концептуальных  систем,  организованных  по  принципам  жесткого  дедуктивизма  и  номотетики:  явление  и
(соответственно)  факт  обретают  статус  события  (см.  Событие,  Событийность),  адекватная  интерпретация
которого предполагает его рассмотрение в качестве единично-уникального, что означает финальный отказ от
любых универсальных презумпций и аксиологических шкал (см. Идиографизм). — В подобной системе отсчета
Э., неизменно предполагающая подведение частного поступка под общее правило и его оценку, исходя из
общезначимой нормы, не может конституировать
 свое содержание. 3). Необходимым основанием Э. как таковой является феномен субъекта (более того, этот
субъект, как отмечает К.Венн, является носителем "двойной субъективности", ибо интегрирует в себе субъекта
этического рассуждения и морального субъекта как предмета этой теории), — между тем визитной карточкой
для современной культуры может служить фундаментальная презумпция "смерти субъекта", предполагающая
отказ от феномена Я в любых его артикуляциях (см. "Смерть субъекта", "Смерть Автора", "Смерть Бога", Я). 4).
Э. по своей природе атрибутивно метафизична (см. Метафизика): роковым вопросом для Э. стал вопрос о
соотношении конкретно-исторического и общечеловеческого содержания морали, и несмотря на его очевидно
проблемный  статус  (см.  "Скандал  в  философии")  история  Э.  на  всем  своем  протяжении  демонстрирует
настойчивые  попытки  конституирования  системы  общечеловеческих  нравственных  ценностей.  Между  тем
современная культура эксплицитно осмысливает себя как фундированную парадигмой "постметафизического
мышления",  в  пространстве  которого  осуществляется  последовательный  и  радикальный  отказ  от  таких
презумпций  классической  метафизики,  как  презумпция  логоцентризма  (см.  Логоцентризм,  Логотомия,
Логомахия), презумпция имманентности смысла (см. Метафизика отсутствия) и т.п. (см. Постметафизическое
мышление). 5). Все уровни системной организации Э. как теоретической дисциплины фундированы принципом
бинаризма: парные категории (добро/зло, должное/сущее, добродетель/порок и т.д.), альтернативные моральные
принципы (аскетизм/гедонизм, эгоизм/коллективизм, альтруизм/утилитаризм и т.д.), противоположные оценки и
т.п. — вплоть до необходимой для конституирования Э. презумпции возможности бинарной оппозиции добра и
зла,  между  тем  культурная  ситуация  постмодерна  характеризуется  программным  отказом  от  самой  идеи
бинарных  оппозиций  (см.  Бинаризм),  в  силу  чего  в  ментальном  пространстве  постмодерна  в  принципе
"немыслимы  дуализм  или  дихотомия,  даже  в  примитивной  форме  добра  и  зла"  (Делез  и  Гваттари).  6).
Современная культура осуществляет рефлексивно осмысленный поворот к нелинейному видению реальности
(см. Нелинейных динамик теория, Неодетерминизм). В этом контексте Фуко, например, решительно негативно
оценивает историков морали, выстраивавших "линейные генезисы". Так, в концепции исторического времени
Делеза (см. Делез, Событийность, Эон) вводится понятие "не-совозможных" миров, каждый из которых, вместе
с тем, в равной мере может быть возведен к определенному состоянию, являющемуся — в системе отсчета как
того,  так  и  другого  мира  —  его  генетическим  истоком.  "Не-совозможные  миры,  несмотря  на  их  не-совозможность,  все  же  имеют  нечто  общее  —  нечто  объективно  общее,  —  что  представляет  собой
двусмысленный знак генетического элемента, в отношении которого несколько миров являются решениями
одной и той же проблемы" (Делез). Поворот вектора эволюции в сторону оформления того или иного "мира"
объективно  случаен,  и  в  этом  отношении  предшествовавшие  настоящему  моменту  (и  определившие  его
событийную специфику) бифуркации снимают с индивида ответственность за совершенные в этот момент
поступки (по Делезу, "нет больше Адама-грешника, а есть мир, где Адам согрешил"), но налагают на него
ответственность  за  определяемое  его  поступками  здесь  и  сейчас  будущее.  Эти  выводы  постмодернизма
практически  изоморфны  формулируемым  синергетикой  (см.  Синергетика)  выводами  о  "новых  отношениях
между человеком и природой и между человеком и человеком" (Пригожин, И.Стенгерс), когда человек вновь
оказывается в центре мироздания и наделяется новой мерой ответственности за последнее. В целом, таким
образом, Э. в современных условиях может быть конституирована лишь при условии отказаот традиционно
базовых своих характеристик: так, если Й.Флетчер в качестве атрибутивного параметра этического мышления
фиксирует его актуализацию в повелительном наклонении (в отличие, например, от науки, чей стиль мышления
актуализирует себя в наклонении изъявительном), то, согласно позиции Д.Мак-Кенса, в сложившейся ситуации,
напротив, "ей не следует быть внеконтекстуальной, предписывающей ... этикой, распространяющей вполне
готовую всеобщую Истину". Если Э. интерпретирует регуляцию человеческого поведения как должную быть
организованной по сугубо дедуктивному принципу, то современная философия ориентируется на радикально
альтернативные стратегии: постмодернизм предлагает модель самоорганизации человеческой субъективности
как автохтонного процесса — вне навязываемых ей извне регламентации и ограничений со стороны тех или
иных  моральных кодексов, —  "речь идет об образовании себя через разного  рода  техники жизни,  а  не о
подавлении  при  помощи  запрета  и  закона"  (Фуко).  По  оценке  Кристевой,  в  настоящее  время  "в  этике 
73
неожиданно возникает вопрос, какие коды (нравы, социальные соглашения) должны быть разрушены, чтобы,
пусть  на  время  и  с  ясным  осознанием  того,  что  сюда  привлекается,  дать  простор  свободной  игре
отрицательности".  С  точки  зрения  Фуко,  дедуктивно  выстроенный  канон,  чья  реализация  осуществляется
посредством механизма запрета, вообще не является и не может являться формообразующим по отношению к
морали. Оценивая тезис о том, что "мораль целиком заключается в запретах", в качестве ошибочного, Фуко
ставит "проблему этики как формы, которую следует придать своему поведению и своей жизни" (см. Хюбрис).
Соответственно постмодернизм артикулирует моральное поведение не в качестве соответствующего заданной
извне норме, но в качестве продукта особой, имманентной личности и строго индивидуальной "стилизации
поведения"; более того, сам "принцип стилизации поведения" не является универсально необходимым, жестко
ригористичным и требуемым от всех, но имеет смысл и актуальность лишь для тех, "кто хочет придать своему
существованию возможно более прекрасную и завершенную форму" (Фуко). Аналогично Э.Джердайн делает
акцент не на выполнении общего предписания, а на сугубо ситуативном "человеческом управлении собою"
посредством абсолютно неуниверсальных механизмов. В плоскости идиографизма решается вопрос о взаимной
адаптации соучастников коммуникации в трансцендентально-герменевтической концепции языка Апеля. В том
же  ключе  артикулируют  проблему  отношения  к  Другому  поздние  версии  постмодернизма  (см.  Afterpostmodernism).  Конкретные  практики  поведения  мыслятся  в  постмодернизме  как  продукт  особого
("герменевтического") индивидуального опыта, направленного на осознание и организацию себя в качестве
субъекта  —  своего  рода  "практики  существования",  "эстетики  существования"  или  "техники  себя",  не
подчиненные  ни  ригористическому  канону,  ни  какому  бы  то  ни  было  общему  правилу,  но  каждый  раз
выстраиваемые  субъектом  заново  —  своего  рода  "практикование  себя,  целью  которого  является
конституирование  себя  в  качестве  творца  своей  собственной  жизни"  (Фуко).  Подобные  "самотехники"
принципиально идиографичны, ибо не имеют, по оценке Фуко, ничего общего с дедуктивным подчинением
наличному  ценностно-нормативному  канону  как  эксплицитной  системе  предписаний  и,  в  первую  очередь,
запретов:  "владение  собой  ...  принимает  ...  различные  формы,  и  нет  ...  одной  какой-то  области,  которая
объединила бы их". Д.Мак-Кенс постулирует в этом контексте возможность Э. лишь в смысле "открытой" или
"множественной",  если  понимать  под  "множественностью",  в  соответствии  со  сформулированной  Р.Бартом
презумпцией,  не  простой  количественный  плюрализм,  но  принципиальный  отказ  от  возможности
конституирования канона, т.е. "множественность", которая, согласно Кристевой, реализуется как "взрыв".

\newpage
\section{Эстетика и художественное сознание}
ЭСТЕТИКА — термин, разработанный и специфицированный А.Э.Баумгартеном в трактате "Aesthetica" (1750
—1758).  Предложенное  Баумгартеном  новолатинское  лингвистическое  образование,  восходит  к  греч.
прилагательному  "эстетикос"  —  чувствующий,  ощущающий,  чувственный,  от  "эстесис"  —  ощущение,
чувствование,  чувство,  а  также,  перен.  "замечание,  понимание,  познание".  Латинский  неологизм  позволил
Баумгартену  обозначить  "эстетическое"  как  первую,  низшую  форму  познания  (gnoseologia  inpherior),
отличающую ее от чувственности (sensus) — всеобщего условия преданности Универсума (Декарт, Ньютон,
Лейбниц, Вольтер) и его непосредственной обращенности к мыслящей субстанции (sensorium Dei). Тем самым
Э. и эстетическое изначально в понятийном плане определялись в гносеологическом статусе. Э., по понятию,
была заявлена как теоретическая дисциплина, изучающая область смыслообразующих выразительных форм
действительности,  обращенных  к  познавательным  процедурам  на  основе  чувства  прекрасного,  а  также  их
художественных экспликаций. В последнем случае Э. получает статус "философии искусства" (Гегель). Идея
прекрасного определяет всю содержательную суть и направленность классической Э., одновременно становясь
предметом эстетического познания и принципом его организации. Прекрасное, таким образом, предположено
как субстанциальная основа эстетического опыта вообще, через который мир воспринимается в своей свободе и
высшей представленной гармонии, где понятие свободы должно сделать действительной в чувственном мире
заданную его законами цель, и, следовательно, природу должно быть возможно мыслить таким образом, чтобы
закономерность ее формы соответствовала по крайней мере возможности целей, заданных ей законами свободы
(Кант).  Прекрасное  будучи  необходимым  условием,  оформляющим  содержание  эстетического  опыта,
осуществляется  в  последнем  виде  идеала,  обладающего  законодательной  и  нормативной  значимостью
(ценностью).  Идеал,  нормативно  определяющий  (специфицирующий)  содержание  эстетического  опыта,
предполагает  собственный  принцип  обнаружения  его  законов  (законосообразности)  —  рефлектирующую
способность  суждения  или  вкус.  Вкус  в  эстетическом  опыте  обладает  законодательной  способностью
спецификации природы, ее эмпирических законов по принципу целесообразности для наших познавательных
способностей (Кант) и таким образом устанавливает постижимую иерархию родов и видов, переход от одного к
другому  и  высшему  роду  (прекрасному  идеалу),  а,  следовательно,  и  закон  и  порядок  суждения  вкуса  в
специфичной  логике  "эстетических  категорий".  Специфичность  категорий  эстетического  суждения  вкуса
заключается  в  том,  что  они  не  являются  логическими  (рассудочными)  понятиями  и  не  могут  быть
дефиницированы  как  понятия,  а  лишь  выражают  (изображают)  чувство  удовольствия  или  неудовольствия
(характер  благорасположения),  которое  предполагается  наличествующим  у  каждого,  однако  не  является
всеобщей определенностью. Выразительно-изобразительный характер способности эстетического суждения в
его  устанавливающей  (законодательной)  форме,  а  равно  и  в  форме  рассуждения  вкуса  об  установленной
эстетической предметности, эксплицирует Искусство как сферу, не принадлежащую уже абсолютно никакой 
эмпирической действительности (природе вообще), полностью иррелевантную ей. Искусство, постулирующее в
форме  воплощенного  идеала,  понимаемого  как  высшее  выражение  мировой  социальной  гармонии,  акта
свободного  творчества  свободного  народа,  является,  одновременно,  манифестацией  духа  "оживляющего
принципы в душе". Принцип, в свою очередь "есть не что иное, как способность изображения эстетических
идей" (Кант). Именно Искусство, взятое в его эстетическом измерении теоретизирует, т.е. представляет для
рассуждающего разума (духа)  чистую эстетическую  предметность,  извлеченную  художественным  гением  и
обращенную  к  чистому  (лишенному  интереса)  чувству  удовольствия  и  неудовольствия.  Э.  эмансипирует
Искусство в феноменологической форме свободного духовного творчества на основе воображения и по законам
прекрасного идеала, тем самым отделяя его как от праксеологической, так и от гносеологической (научно-теоретической)  сфер  деятельности.  В  концептуальном  плане,  Э.  генерирует  феномен  "классической
художественной  культуры",  "классики",  нормативно  специфицирующей  европейскую  культуру  вообще,  где
искусство представлено идеально, в форме теории ("классицизм", "просвещение") и нормативного образца,
установленного законодательным суждением вкуса и законосозидающим творчеством художественного гения.
Нормативным идеалом в европейской классике становится "Античность", однако не исторический античный
(греко-римский) мир, а его художественная реконструкция, произведенная искусством, коллекционированием,
философией, начиная от Ренессанса вплоть до середины 17 в. и завершенная И.И.Винкельманом ("История
искусства  древности", 1764). Винкельман  предложил  формирующейся  Э. то, в  чем она  крайне  нуждалась:
теоретическим  рассуждениям  о  вкусе,  о  прекрасном.  Была  предложена  аппелирующая  к  истории
художественная  модель,  представляющая  прекрасный  идеал  в  наглядно-чувственной,  воплощенной  форме.
Модель крайне условная, поскольку Винкельман создавал ее не столько опираясь на методику исследования,
атрибуции и систематизации реальных художественных произведений (в число "антиков" попадали и прямые
подделки), сколько на уже разработанную и разрабатываемую новоевропейскую метафизику Античности, ее
теоретический, дискурсивный образ (Буало, Берк, Батте, Вольтер). Такая нормативность классики содержит
известный парадокс историзма мира искусства: в своих прекрасных произведениях (классических шедеврах) он
экстраисторичен (обращен к вечности), но в своем генезисе — историчен. Исторический вектор формирования
эстетической культуры наиболее полно и всесторонне разработан Гегелем в "Феноменологии Духа" и в "Э.". В
"Феноменологии Духа" Гегель рассматривает искусство как художественную религию, которая определяется
исключительно феноменально, как "самосознание" духа, взятое в своей сугубой субъектности (самоположения
для  себя  же):  "дух...  который  свою  сущность,  вознесенную  над  действительностью,  порождает  теперь  из
чистоты самости" ("Феноменология духа"). В таком виде дух является "абсолютным произведением искусства и
одновременно  столь  же  абсолютным  "художником".  Однако  такое  абсолютное  произведение  искусства,  в
котором произведение тождественно художнику, не дано в наличности, как то постулировалось классической
культурой  и,  соответственно,  Э.,  а  есть  результат  становления,  переживающего  различные  этапы  (формы)
своего  осуществления.  И  здесь  Гегель  подвергнет  радикальной  деконструкции  абсолют  воплощенного
прекрасного  идеала,  рассматривая  таковой  только  в  темпорально  связанных  формах  самосозерцания
художественных  индивидуаций  "всеобщей  человечности",  участвующих  в  воле  и  действиях  целого.  Так
возникают  виды  самосозерцающего  свое  становление  духа,  проявляющиеся  в  соответствующих  видах
художественного  произведения:  эпос,  трагедия,  комедия.  Они-то  и  есть  сущностный,  субстанциальный
"эстесис",  который в феноменологическом плане и становится предметом  Э. как  философии искусства.  Э.
Гегеля  завершает  формирование  европейской  классической  художественной  культуры,  возвещая  уже
случившийся  "конец  искусства",  что  художественно  воплощается  в  творческой  индивидуальности  Гёте,
особенно в поздний, "классический" период. Установка Э. относительно полного типа культуры взятого как
завершающий этап Всемирной истории, способствовал и формированию истории Э., когда предшествующие
исторические периоды (эпохи) в своих творческих художественных феноменах стали рассматриваться именно с
позиции эстетической телеологии и заданности. Так появляются "Э. античного мира", "Э. средних веков", "Э.
Возрождения",  художественный  опыт  которых  осмысливается  исключительно  с  позиций  нормативно-эстетической телеологии и заданности. Собственно, такая установка была сформулирована в "Э." (Философии
искусства Гегеля), его последователя Ф.Т.Фишера ("Astentik oder Wissansaft der Schonen" тт. 1—3, 1846—1858) и
заложила  основы  традиции  истории  эстетических  учений.  В  этой  истории  характерна  провиденциально-телеологическая установка на приближение к прекрасному, художественно-выраженному идеалу (эстетической,
т.е.  обращенной  к  восприятию  на  уровне  разума  чувственного  созерцания,  манифестации  художественно
выраженного  духа).  Но  эта  же  история  изначально  оказалась  ограничена  как  в  применимости  своих
методологических  установок,  так  и  в  полноте  охвата  продуктов  художественного  опыта  (праксиса).  За  ее
пределами  оказались  многие  артефакты,  кстати  существенные  для  понимания  исторической  специфики
художественного  творчества  даже  самой  европейской  культуры,  не  говоря  уже  о  иных  культурно-художественных  мирах:  Дальний  и  Ближний  Восток,  Африка,  доколумбова  Америка  и  т.п.,  которые
рассматривались  скорее  этнографически,  чем  эстетически.  Что  же  касается  Европы,  то  целые  периоды
Античности (архаика, этрусское искусство, искусство народов Северной Европы и т.п.) оказались за пределами
не только истории Э., но и истории искусства. В равной степени средневековая культура была представлена
исторически  и  теоретически  лишь  отдельными  нормативными  формами:  философско-теологические
интерпретации  искусства  и  т.п.  проторенессансные  феномены,  либо  некоторые  ретроспекции  античной
классики.  Не  входили  в  таким  образом  сконструированную  модель  исторического  развития  эстетической
культуры  и  вопросы,  связанные  с  региональными,  национально-этническими  спецификациями  творчески-
художественной феноменологии. В связи с тем, что классическая Э. и классическая художественная культура
предполагают творчество как духовное самостановление для самосознания в форме созерцания, то и человек
субъектно  определяется  как  "живое  произведение  искусства"  (Гегель),  формирующееся  лишь  в  процессе
эстетического  воспитания,  педагогика  которого  становится  существеннейшим  элементом  Э.  Теоретико-эстетическая концепция эстетического воспитания была наиболее полно разработана
 Шиллером ("Uber die asthetische Erzichung des Menschen", 1795), где постулируется необходимость творения
"эстетической  реальности",  реальности  высшего  порядка,  в  которой  только  возможно  художественное
сотворение  человеческой  личности,  взятой  в  целостности  (гармонии)  свободного  и  универсального
развертывания своих способностей (сущностных сил). Основной идеей эстетического воспитания становится
игра  как  единственная  форма  свободного  (от  утилитарных  влечений)  проявления  бытия  человека.  В
художественно-классической  форме  идея  эстетического  воспитания  была  раскрыта  Гёте  ("Театральное
призвание  Вильгельма  Мейстера",  "Годы  учений"  и  "Годы  странствий  Вильгельма  Мейстера").  Идея
эстетического воспитания наметила тему европейской художественной культуры, продолжающуюся и в 20 в.,
что же касается педагогики, то эстетическое воспитание оказалось мало востребованным, хотя и вдохновило
ряд педагогических концепций, в частности К.Д.Ушинского ("Человек как предмет воспитания", тт. 1—2, 1868
—1869) в России, Морриса в Англии, а также Бергсона и Дьюи. Однако все эти педагогические устремления не
выходили за рамки эксперимента, подчас, крайне драматического. Парадокс эстетического воспитания был
художественно осмыслен Манном ("Волшебная гора") и особенно остро — Гессе ("Игра в бисер"). Однако
тематизация эстетического воспитания способствовала становлению "Э." как учебной (школьной) дисциплины
и формированию европейской классической модели воспитания и образования. Педагогический модус Э. и
эстетического, тем не менее выводил эстетическую проблематику в более общую, антрополого-гуманитарную
перспективу,  где  эстетическое  стало  все  в  большей  степени  рассматриваться  как  феномен  развертывания
сущностных  аспектов  человеческого  бытия,  а  эстетическая  теория,  соответственно,  как  проработка
возможностей развертывания гуманистической перспективы. Гегелевская концепция человека в виде "живого
произведения искусства" явилась импульсом, весьма существенно преобразующим смысл Э. (в теоретическом и
праксеологическом планах). Суть его в том, что Э. (эстетическая теория) становится одной из парадигмальных
(даже  процедурных)  установок  формирующейся  философской  антропологии  во  всех  ее  многообразных
модификациях. Эстетический и равно художественный опыт начинает интенсивно проникать в сущностные
основы бытия, экзистенциальность человека, которая в значительной степени мыслится аутопойетически. Так
формируется  неклассическая  Э.,  связанная  с  разнообразными  формами  жизнетворчества,  и  имеет  своим
следствием  формирование  инновационных  художественных  течений,  принципиально  антиномичных  и
парадоксально-маргинальных  в  отношении  к  классической,  эстетически-концептуализированной  культуре.
Неклассическая  Э.  в  своем  жизнетворческом  модусе  заявляет  о  себе  как  негативная  по  отношению  к
классической  форме,  прежде  всего  в  отношении  искусства,  которое  понимается  как  художественно-жизнетворческий акт. Художник вступает здесь как автор и исполнитель, непосредственно включенный в бытие
произведения и постоянно осуществляющий его пойесис. Аутопойетический пафос направлен, прежде всего, на
создание  авторизованного  мифопоэтического  мира,  относительно  самодостаточного  и  самодостоверного,
включенного  в  него  творца-художника.  Такой  мир  обладает  собственной  Э.,  чаще  всего  достаточно  полно
разработанной  в  концептуальном  плане.  Цельность  культуры  претерпевает  радикальную  дифференциацию;
художественная  культура  теперь  представляет  настающий  и  калейдоскопический  универсум  множества
автономных  миров.  Возможность  построения  аутопойетического  художественного  мира  была  заявлена
творчеством  Гельдерлина  ("Гиперион"),  но  в  радикальной  форме  проработана  Ницше  ("Происхождение
трагедии из духа музыки", "Казус Вагнера" и "Так говорил Заратустра"). Ницше тем более радикализирует
установку  на  возможность (невозможность)  целостности  культурно-художественного  мира, предлагая тезис
"Смерти  Бога".  "Смерть  Бога"  лишает  мир,  прежде  всего  в  его  творческом  осуществлении  всеобщей
телеологической  задачи,  следовательно  —  всеобщего  нормативного  идеала,  следовательно  —  возможности
континуальной  "всеобщей  человечности".  Мир  теперь  —  дионисийский  экстаз,  трагедия  и  индивидуации
самоудовлетворяющего "Эго", где аполлоническая завершенность представления для созерцающей рефлексии
существует лишь как перманентная целесообразность абсолютной самоутверждающей и законосозидающей
воли  (воли  к  власти).  Художественная  культура,  взятая  в  подобном  ракурсе,  становится  принципиально
культурой модерна, современной и актуальной и одновременно-исключительной и оригинальной. Формируется
возможность  многообразия  "Э.",  где  само  понятие  Э.  теперь  не  ограничивается  ни  статусом  философско-теоретической  рефлексии  по  поводу  прекрасного,  ни  статусом  "рефлесирующей  способности  суждения"
(вкуса), устанавливающей всеобщую сообщаемость систематизированных ценностей, ни даже спецификацией
"гениальности",  с  ее  претензией  на  всеобщее  эстетическое  законодательство  (нормативный  стиль).  Э.
становится  понятием,  характеризующим  целостность  конкретного  "субкультурного"  художественного  мира,
обладающего  своим  авторским  стилем,  рисунком.  Такого  рода  модернизация  неизбежно  приводит  к
разнообразию, эстетически себя определяющих, направлений, течений, вполне суверенных и генетически никак
не связанных. Происходит разрушение Всемирной истории Э. и вместе с этим и Всемирной истории искусств,
как детерминированного процесса сменяющих, однако, зависимых друг от друга исторических стилей, эпох и
т.п.  История  эстетической  культуры  теперь  мыслится  как  трансформация  и  трансмутация  множества
коммуницирующих художественных систем в хронотопе социо-культурного события. Э. получает еще одно
измерение-социологическое (И.Тэн, Я.Буркхардт, Маркс, Лукач, Ортега-и-Гассет). Художественно-эстетические 
феномены  представляются  здесь  как  выражение  социальных  статусов  (интересов,  целей,  представлений)
социальных  групп  в  системах  экономических  и  политических  отношений  (обменов,  коммуникаций,
конфликтов).  Социологизирующая  Э.  развивается  в  двух  конкурирующих  системах  самоописания  и
саморепрезентаций  общества:  политической  и  экономической  и  непосредственно  смыкается  с  социальной
антропологией. Таким образом, по определению Хайдеггера, "искусство вдвигается в горизонт Э. и становится
непосредственным  выражением  жизни  человека",  однако  жизни,  погруженной  в  глобальную  социальную
коммуникацию,  что,  во-первых  требует  описания  эстетического  феномена  в  контексте  этой  коммуникации
(структурализм,  постструктурализм,  лингвистические  и  семиологические  процедуры  описания)  и
контекстуального истолкования эстетических событий (герменевтика). Проблематичной остается классическая
тема  эстетической  культуры  в  модернистской  и  постмодернистской  ситуации.  Здесь  необходимо  отметить
статус  Музея,  удерживающего  в  определенной  степени  классическую  парадигму  как  ориентирующую
координату, в отношении которой модернистские (неомодернистские) проекты опознают и определяют себя как
явления  художественной  культуры.  Присутствие  классической  парадигмы  в  коммуницирующей  системе
эстетических проектов удерживает феноменологичекую суть европейской эстетической культуры, концентрируя
ее на выявление общегуманитарной телеологической установки, которая определяется Хайдеггером термином
"алетейя"  в  смысловом  аспекте  "процветания"  и  "несокрытости"  истины,  открытости  для  обоснования  и
утверждения верховных ценностей человечества.

\newpage
\section{Глобализация как тенденция развития современной цивилизации и ее проблемы}
ГЛОБАЛИЗАЦИЯ - термин для обозначения ситуации изменения всех сторон жизни общества под влиянием
общемировой  тенденции  к  взаимозависимости  и  открытости.  Двигателем  Г.  выступает  современный
капитализм (см.), понимаемый как фаза истории человечества и как несущий определенную (неолиберальную)
геополитическую  и  политическую  программу.  По  мысли  Р.  Робертсона  ("Глобализация".  Лондон,  1992),
"понятие глобализации относится как к компрессии мира, так и к интенсификации осознания мира как целого...
как к конкретной глобальной взаимозависимости, так и осознанию глобального целого в 20 в.". М. Уотерс
("Глобализация". Лондон; Нью-Йорк, 1995) отмечает: Г. - это процесс, "в ходе которого и благодаря которому
определяющее воздействие географии на социальное и культурное структурирование упраздняется и в котором
люди это упразднение все в большей мере осознают".
Впервые  Г.  как  феномен  слияния  рынков  отдельных  видов  продукции  и  услуг,  производимых  крупными
транснациональными корпорациями, была осмысленно зафиксирована, по-видимому, американским ученым Т.
Левиттом в статье, опубликованной в журнале "Гарвард бизнес ревью" в 1983. Более широкое значение новому
термину придали в Гарвардской школе бизнеса, а главным его популяризатором стал японский консультант этой
школы К. Омэ, опубликовавший  в  1990 книгу "Мир без границ".  Полагая,  что  мировая экономика теперь
определяется  взаимозависимостью  трех  центров  -  ЕС,  США,  Япония  -  он  утверждал,  что  экономический
национализм отдельных государств стал бессмысленным, в роли же главных "актеров" на экономической сцене
выступают "глобальные фирмы".
Г. - это признание растущей взаимозависимости современного мира, главным следствием которой является
значительное  ослабление  (некоторые  исследователи  настаивают  даже  на  разрушении)  национального
государственного  суверенитета  под  напором  действий  иных  субъектов  современного  мирового  процесса  -прежде  всего  транснациональных  корпораций  и  иных  транснациональных  образований,  например,
международных компаний, финансовых институтов, этнических диаспор, религиозных движений, мафиозных
групп и т.д. Не связанные, в отличие от национальных государств, условиями международных договоров и
конвенций, эти транснациональные образования оказываются в более выгодном положении, перераспределяя в
свою  пользу  властные  полномочия.  Основной  сферой  Г.  является  международная  экономическая  система
(мировая экономика), т.е. глобальные производство, обмен и потребление, осуществляемые предприятиями в
национальных экономиках и на всемирном рынке. Хотя основная часть глобального продукта потребляется в
странах-производителях,  национальное  развитие  все  более  увязывается  с  глобальными  структурами  и
становится более многосторонним и разноплановым, чем это было в прошлом. Успех глобальных корпораций в
значительной мере опирается на признание их продукции потребителями разных стран. Даже в странах с
высоким  уровнем  развития товарного национализма продукция  глобальных корпораций  находит  успешный
сбыт  по  причине  ее  относительной  дешевизны  и  качества.  Кроме  того,  огромные  прибыли  глобальных
корпораций позволяют вкладывать гигантские средства в рекламу, внушающую потребителям, что это именно
те товары и услуги, которые им нужны. Глобальные производство и сбыт формируют глобального потребителя -"гражданина мира", который во всех странах ищет и находит одно и то же, наслаждается одним и тем же. Г.
представляет  собой  комплексную  тенденцию  в  развитии  современного  мира,  затрагивающую  его
экономические, политические, культурные, но в первую очередь информационно-коммуникационные аспекты.
Так, мощным двигателем процесса Г. стали электронные СМИ. Спутниковое телевидение, получающее все
большее распространение, ломает национальные границы, превращает постоянно растущую часть населения
Земли в одну гигантскую телеаудиторию, которая смотрит одни и те же фильмы, любит и знает одних и тех же
звезд, стремится к одним и тем же символам материального успеха (см. Всемирная деревня). Другой важной
составляющей  процесса  Г.  выступили  современные  информационные  технологии  (Интернет,  компьютерная
связь,  мультимедиа  и  т.п.),  позволившие  транснациональным  организациям  размещать  различные 
составляющие  производственного  процесса  в  разных  странах,  сохраняя  при  этом  тесные  контакты.  Так,
мировой финансовый рынок являет собой в настоящее время интегрированную глобальную систему, четко
скоординированную через мгновенные телекоммуникации. Уменьшение необходимости физических контактов
между  производителями  и  потребителями  позволило  некоторым  услугам,  которые  ранее  невозможно  было
продать на международных рынках, стать объектом специфической торговли. Так, любую деятельность, которая
осуществима на экране или по телефону (от подготовки программного обеспечения до продажи авиабилетов),
стало возможным произвести в любой точке мира, связавшись через спутник или компьютер. Формирующаяся
система глобальной информации формирует потребности и интересы, общие для жителей всех стран. В свою
очередь глобальные потребности ведут к появлению глобальных продуктов. Это проявляется в стандартизации
товаров и унификации торговых марок. К числу товаров, которые приобретают вид стандартных (глобальных),
уже  можно  отнести  мотоциклы,  аудиокассеты,  диски  СD,  стереоаппаратуру,  компьютеры  и  ряд  других.
Следующий уровень Г. - унификация менеджерской стратегии и тактики, разрабатываемой для глобального
рынка без попыток адаптации управления к  местной специфике. Таким  образом, Г.  предполагает также и
тенденцию  к  унификации  мира,  к  жизни  по  единым  принципам,  приверженности  единым  ценностям,
следованию единым обычаям и нормам поведения. (Индивидуальные феноменологии во все большей степени
релятивизируются  в  глобальной  системе  координат.)  Г.  фундирована  социальным  состоянием,  когда
символические  практики  приоритетны  по  отношению  к  материальным,  начинает  доминировать  Г.
символических  обменов,  освобождающих  (уже  в  силу  своей  природы)  социальные  отношения  от
пространственных  референций.  Г.  ведет  к  взаимному  уподоблению  социальных  практик  на  уровне
повседневной  жизни.  Именно  этот  аспект  Г.  является  традиционным  объектом  критики  защитников
национальных культурных традиций от эскалации западной, и прежде всего американской, информационной
продукции.  Они  полагают,  что  Г.  информационно-коммуникационных  процессов  ведет  к  нивелированию
самобытности культур в различных странах и подчинению СМИ не гуманитарным, а коммерческим интересам.
Как  отмечает  французский  ученый  Д.  Коэн,  "прошло  немного  лет,  а  к  понятию  глобализации  уже  стали
относиться с подозрением: одни считают, что оно предполагает фаталистическое отношение к происходящим в
мире  изменениям,  в  то  время  как  другие  призывают  к  защите  нынешнего  дорогой  ценой  завоеванного
общественного порядка от тенденций столь сомнительного качества". По мысли же директора Центра Фернанда
Броделя по изучению экономики, исторической системы и цивилизаций И. Валлерстайна (1999), идея Г. как
универсального  этапа  мирового  развития  есть  "громадная  ошибка  современной  действительности,  обман,
навязанный нам властными группировками и нами же самими, часто в отчаянии... Мы можем думать об этом
переходе как о политической битве между двумя крупными лагерями: тех, кто хочет удержать привилегии
существующей неравноправной системы, и тех, кто хотел бы видеть создание новой системы, которая будет
более справедливой". 

\end{document}
